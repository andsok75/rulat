\documentclass[10pt]{article}
\usepackage[a4paper, margin=2cm]{geometry}
\usepackage{fontspec}
\setmainfont{Linux Libertine O}
\begin{document}

\newcommand{\e}{ë}
\newcommand{\yi}{yı}
\newcommand{\ia}{ıa}
\newcommand{\io}{ıo}

\newcommand{\y}{y̆}
\newcommand{\Y}{Y̆}

\newcommand{\X}{X̹}
\newcommand{\x}{x̹}
\newcommand{\C}{C̹}
\renewcommand{\c}{c̹}

Bukv{\yi} alphabeta a takje ih nazvani{\y}a:

\setlength{\tabcolsep}{2pt}
\begin{tabular}{c c l @{\hspace{1cm}} c c l @{\hspace{1cm}} c c l @{\hspace{1cm}} c c l}
a &---& a  & g &---& ge & n &---& en & v &---& ve \\
b &---& be & h &---& ha & o &---& o  & x &---& xa \\
c &---& ce & i &---& i  & p &---& pe & {\x} &---& {\x}a \\
{\c} &---& {\c}e & j &---& je & r &---& er & y &---& i kratko{\y}e (m{\ia}gki{\y} znak) \\
d &---& de & k &---& ka & s &---& es & {\y} &---& {\y}e \\
e &---& e  & l &---& el & t &---& te & z &---& ze \\
f &---& ef & m &---& em & u &---& u  &  \\
\end{tabular}

\noindent
Bukv{\yi} \textit{q} i \hspace{-2pt}\textit{w} ispolyzu{\y}utsa tolyko v inostrann{\yi}h slovah.
Ime{\y}utsa takje sledu{y}u{\x}i{\y}e dopolnitelyn{\yi}{\y}e cet{\yi}re bukv{\yi}:

\setlength{\tabcolsep}{2pt}
\begin{tabular}{c c c}
    {\e}  &---& o m{\ia}gko{\y}e \\
    {\ia} &---& a m{\ia}gko{\y}e \\
    {\io} &---& u m{\ia}gko{\y}e \\
    {\yi} &---& i tv{\e}rdo{\y}e \\
\end{tabular}

\noindent
no oni upotrebl{\ia}{\y}utsa tolyko posle soglasn{\yi}h 
i oboznaca{\y}ut te je zvuki cto i bukv{\yi} \textit{o}, \textit{a}, \textit{u}, \textit{i}, no izmen{\ia}{\y}ut tv{\e}rdosty 
(ili m{\ia}gkosty v sluca{\y}e \textit{{\yi}}) predidu{\x}e{\y}
soglasno{\y}. Bukva \textit{e} mojet sto{\y}aty i posle m{\ia}gkih a inogda i posle tv{\e}rd{\yi}h soglasn{\yi}h.
Posledne{\y}e v osnovnom v inostrann{\yi}h slovah, naprimer, \textit{testirovaty}.
V nacale slov i posle glasn{\yi}h bukva \textit{e} ispolyzu{\y}etsa redko, naprimer, v slovah \textit{eto} i \textit{poetomu},
a takje v inostrann{\yi}h slovah, naprimer, \textit{effect, echo}.

Sootvetstvi{\y}e s bukvami kirilli{\c}i privedeno s sledu{\y}u{\x}e{\y} tabli{\c}e:

\setlength{\tabcolsep}{2pt}
\begin{tabular}{c c c @{\hspace{1cm}} c c c @{\hspace{1cm}} c c c @{\hspace{1cm}} c c c}
\textit{a} &---& \textit{а} & \textit{g} &---& \textit{г} & \textit{n} &---& \textit{н} & \textit{v} &---& \textit{в} \\
\textit{b} &---& \textit{б} & \textit{h} &---& \textit{х} & \textit{o} &---& \textit{о}  & \textit{x} &---& \textit{ш} \\
\textit{c} &---& \textit{ч} & \textit{i} &---& \textit{и}  & \textit{p} &---& \textit{п} & \textit{{\x}} &---& \textit{щ} \\
\textit{{\c}} &---& \textit{ц} & \textit{j} &---& \textit{ж} & \textit{r} &---& \textit{р} & \textit{y} &---& \textit{ь} \\
\textit{d} &---& \textit{д} & \textit{k} &---& \textit{к} & \textit{s} &---& \textit{с} & \textit{{\y}} &---& \textit{й} \\
\textit{e} &---& \textit{э} & \textit{l} &---& \textit{л} & \textit{t} &---& \textit{т} & \textit{z} &---& \textit{з} \\
\textit{f} &---& \textit{ф} & \textit{m} &---& \textit{м} & \textit{u} &---& \textit{у}  &  \\
\end{tabular}

\noindent
no vo mnogih sluca{\y}ah ono ne {\y}avl{\ia}{\y}etsa odnoznacn{\yi}m.
Naprimer, bukva \textit{{\y}} upotrebl{\ia}{\y}etsa gorazdo ca{\x}e cem \textit{й},
potomu cto ona trebu{\y}etsa v nacale slov i posle glasn{\yi}h tam gde v kirilli{\c}e 
kak pravilo ispolyzu{\y}ut bukv{\yi} \textit{е}, \textit{ё}, \textit{ю}, \textit{я}, \textit{и}:
\textit{{\y}abloko, pri{\y}atn{\yi}{\y}, {\y}olka, {\y}ujn{\yi}{\y}, {\y}esli, hoz{\ia}{\y}in, v zdani{\y}i}.
Posle soglasn{\yi}h primenimo sledu{\y}u{\x}e{\y}e sootvetstvie{\y}e:

\setlength{\tabcolsep}{2pt}
\begin{tabular}{c c c @{\hspace{1cm}} c c c @{\hspace{1cm}} c c c}
    \textit{e}  &---& \textit{е}    & \textit{{\ia}} &---& \textit{я} & \textit{{\yi}} &---& \textit{ы} \\
    \textit{{\e}}  &---& \textit{ё} & \textit{{\io}} &---& \textit{ю} \\
\end{tabular}

\noindent
Bukva \textit{ъ} ne ime{\y}et equivalenta vovse tak kak {\y}e{\y}o functi{\y}a ne trebu{\y}etsa,
naprimer, \textit{ot{\y}ehaty, ob{\y}om, v{\y}uga}.

Bukv{\yi} \textit{{\y}, {\e}, {\ia}, {\io}, {\yi}} v boly{\x}instve textov (za iskluceni{\y}em
textov dl{\ia} dete{\y}) upro{\x}a{\y}utsa:
    \textit{{\y}}  $\rightarrow$ \textit{y},
    \textit{{\e}}  $\rightarrow$ \textit{e},
    \textit{{\ia}} $\rightarrow$ \textit{ia},
    \textit{{\io}} $\rightarrow$ \textit{io},
    \textit{{\yi}} $\rightarrow$ \textit{yi}.
Naprimer, sravnite

\textit{
    Pod kop{\yi}ta mo{\y}e{\y} loxadi brosilsa ni{\x}i{\y}, vop{\ia}, cto gr{\ia}d{\e}t kone{\c} sveta i {\y}a doljen poka{\y}atsa i otdaty {\y}emu vse denygi. Odnako, pon{\ia}v, cto {\y}a ne otlica{\y}usy osobo{\y} nabojnost{\y}u, on tut je zab{\yi}l obo mne i pristal k dvum dorodn{\yi}m kup{\c}am, kotor{\yi}{\y}e b{\yi}li neskolyko perepugan{\yi} tem bezumi{\y}em, cto proishodilo vokrug.
}

\noindent s

\textit{
    Pod kopyita moyey loxadi brosilsa ni{\x}iy, vopia, cto griadet kone{\c} sveta i ya doljen pokayatsa i otdaty yemu vse denygi. Odnako, poniav, cto ya ne otlicayusy osoboy nabojnostyu, on tut je zabyil obo mne i pristal k dvum dorodnyim kup{\c}am, kotoryiye byili neskolyko perepuganyi tem bezumiyem, cto proishodilo vokrug.
}

%Pomimo etovo, posle xip{\ia}{\x}ih \textit{c, {\c}, j, x, {\x}} bukv{\yi} \textit{{\e}, {\yi}} ne ispolyzu{\y}utsa,
%vmesto nih vsegda pixutsa \textit{o, i}. A okoncani{\y}a glagolov pixutsa cerez \textit{tsa} ne zavisimo ot m{\ia}gkosti soglasn{\yi}h.

%case "А", "Б", "В", "Г", "Д", "Е", "Ё", "Ж", "З", "И", "Й", "К", "Л", "М", "Н", "О", "П", "Р", "С", "Т", "У", "Ф", "Х", "Ц", "Ч", "Ш", "Щ", "Ъ", "Ы", "Ь", "Э", "Ю", "Я":
%fallthrough
%case "а", "б", "в", "г", "д", "е", "ё", "ж", "з", "и", "й", "к", "л", "м", "н", "о", "п", "р", "с", "т", "у", "ф", "х", "ц", "ч", "ш", "щ", "ъ", "ы", "ь", "э", "ю", "я":
%return true

\end{document}