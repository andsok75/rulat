\documentclass[10pt]{book}
\usepackage{fontspec}
\setmainfont{Linux Libertine O}
\begin{document}

Pugalo sidelo na kryıxe fligelıa, nablıuday̆a rassvet. Nebo, kak eto castenyko byıvay̆et v fevrale nedaleko ot morıa, neskolyko minut napominalo qvetom morskuy̆u rakovinu, takim nejno-rozovyım ono byılo, a zatem srazu potusknelo, nalilosy svinqom, v glubine kotorovo, kazalosy, raspleskali lucxiy̆e vostocnyıy̆e cernila. Pocti srazu nacal nakrapyıvaty dojdy, i oduxevlënnovo s kryıxi kak vetrom sdulo.

Dojdy Pugalo lıubilo daje menyxe, cem otsutstviy̆e razvleceniy̆. A poslednih ne slucalosy uje dovolyno davno.

Y̆a zanimalsa tem, cto zakancival ranniy̆ zavtrak — varënyıy̆e y̆ay̆qa, jarenay̆a sardina i cesnocnyıy̆ sup, kotoryıy̆ okazalsa porıadkom peresolen.

Vsë telo cesalosy, nocy̆u y̆a otrazil cetyıre ataki klopov, no propustil organizovannyıy̆ udar s levovo flanga i tepery myıslenno proklinal etot postoy̆alyıy̆ dvor i to, cto nastoy̆ka dlıa otpugivaniy̆a nasekomyıh zakoncilasy tak ne vovremıa.

Vladeleq zavedeniy̆a, vidıa moy̆u hmuruy̆u roju, ne rexalsa prosity rascot i toptalsa vozle kladovki, poglıadyıvay̆a to na menıa, to na oblaconnyıh v beloy̆e.[39]

Stranniki v odejdah, kotoryıy̆e za vremıa putexestviy̆a davno uje priobreli seryıy̆ qvet, s prostyımi posohami, y̆eli postnuy̆u grecku. Ustavxiy̆e ot beskonecnoy̆ dorogi bogomolyqi xli ot myısa Del Sur, samoy̆ y̆ujnoy̆ tocki Nararyı, v Kruso.

Ne pervyıy̆e piligrimyı na moy̆om puti. I vse kak odin tverdıat, cto devocka, jivux̨ay̆a v Kruso, uzrela cudo. Mol, priletel k ney̆ kryılatyıy̆ vestnik, i kryıly̆a y̆evo byıli podobnyı dyımu. Soobx̨il on, razumey̆etsa, cto Straxnyıy̆ sud ne za gorami i nado pokay̆atsa, prejde cem vostrubit rog i podnimetsa iz zemli prah.

— Tyı verix v eto? — sprosil u menıa Propovednik, kogda myı tolyko uslyıxali novosty.

— Cto angel sletel s nebes? Posledniy̆ angel, kotorovo, kak govorıat, videli, poy̆avlıalsa, kogda raspıali Christa, i izvestil namestnika, deda imperatora Konstantina, o tom, cto grıadut bolyxiy̆e peremenyı. No v toy̆ istoriy̆i hotıa byı byıl sery̆oznyıy̆ povod.

— Koneq sveta, po-tvoy̆emu, ne povod?

— Y̆a nemnogo ustal ot konqov sveta, Propovednik. Kajdyıy̆ god vsıakiy̆, kto scitay̆et, cto videl angela ili slyıxal boga, zay̆avlıay̆et o tom, cto mir na kray̆u gibeli, cto vot-vot slucitsa tretiy̆ potop, cetvërtay̆a velikay̆a epidemiy̆a y̆ustirskovo pota i v kajdom gorode na meste domov grexnikov vyırastut ognedyıxax̨iy̆e goryı, kotoryıy̆e budut plevatsa seroy̆ i jabami.

— To y̆esty tyı ne jdëx Apokalipsisa?

— Ne somnevay̆usy, cto rano ili pozdno myı dostanem nebesa i te provedut pokazatelynuy̆u cistku parxivyıh oveq, no uveren, eto slucitsa ne pri moy̆ey̆ jizni.

— Y̆esli cestno, y̆a toje ne verıu v etu istoriy̆u. Na koy̆ cort, prosti Gospodi, angelu priletaty k kakoy̆-to desıatiletney̆ devconke, kogda v y̆evo rasporıajeniy̆i kuda boley̆e interesnyıy̆e predstaviteli celovecestva?

Odin iz piligrimov, uje davno poglıadyıvay̆ux̨iy̆ na menıa, otodvinul pustuy̆u misku, nespexno vyıter gubyı rukavom i podoxol k moy̆emu stolu:

— Bog v pomox̨. Napravlıay̆etesy v Kruso?

Y̆a podumal, stoy̆it li delitsa svoy̆imi planami s pervyım vstrecnyım, rexil, cto huje ne budet, i prosto kivnul.

— Nas vosemnadqaty. Myı mirnyıy̆e lıudi, a dorogi vdoly poberejy̆a byıvay̆ut opasnyı. Zaplatim za zax̨itu.

Vot tolyko etovo mne ne hvatalo. Plestisy dva s polovinoy̆ dnıa vmeste s raspevay̆ux̨imi svıatyıy̆e gimnyı bogomolyqami, kogda na loxadi mojno okazatsa v gorode uje k veceru. U menıa prosto net lixnevo vremeni.

— Tyı oxibsa, dobryıy̆ celovek, — lıubezno otvetil y̆a y̆emu. — Y̆a ne nay̆emnik i ne voy̆in.

Ctobyı ne byılo bolyxe voprosov, pokazal rukoy̆atku kinjala:

— U menıa svoy̆i qeli. Y̆esli hocex zax̨ityı, shodi na kupeceskiy̆ post. Oni obyıcno proday̆ut uslugi ohrannikov.

On, kajetsa, udovletvorilsa moy̆im otvetom i vernulsa k sputnikam. Propovednik pokrutil palyqem u viska:

— Otdaty denygi odnomu, ctobyı on ohranıal vosemnadqaty. Voy̆istinu Bojyi lıudi. Takovo idiotizma y̆a ne vstrecal s teh por, kak rexil ostanovity nay̆emnikov na kryılyqe moy̆ey̆ qerkvi. Lucxe byı sideli doma, cem xlıatsa po dorogam.

— Kak tyı surov s utra. — Y̆a s usmexkoy̆ otlojil lojku, okonciv zavtrak. Sledovalo rasplatitsa i otpravlıatsa v dorogu. Pod merzkim dojdem.

— Y̆a istinu govorıu, Lıudvig. A uj y̆esli doma ne siditsa i v zadniqe sverbit, naucisy strelıaty iz arbaleta. Vosemnadqaty celovek s arbaletami. Oni lıubyıh razboy̆nikov udelay̆ut.

— Otpravlıay̆ux̨imsıa k svıatyınıam ne pristalo nosity pri sebe cto-to tıajeley̆e bibliy̆i.

— Vot potomu ih i razuvay̆ut vse komu ne leny.

On y̆ex̨e cto-to vorcal po etomu povodu, no y̆a uje napravilsa k hozıay̆inu postoy̆alovo dvora, predostaviv staromu pelikanu vyıskazyıvaty svoy̆i myısli Pugalu v namokxey̆ solomennoy̆ xlıape.



Narara, nesmotrıa na to cto eto ne samay̆a y̆ujnay̆a strana kontinenta, zimoy̆ otlicalasy kuda boley̆e mıagkim klimatom, cem tot je Lezerberg ili Vitilyska. Sneg v primorskih oblastıah padal obilyno, no morozyı slucalisy redko, a k konqu fevralıa dovolyno casto teplelo nastolyko, cto nacinal idti dojdy.

Konecno je holodnyıy̆ i nepriy̆atnyıy̆, no, y̆esli sravnivaty s ubiy̆stvennyım morozom, cto sey̆cas, po sluham, sobiray̆et x̨edruy̆u jatvu iz putnikov v Firvalydene, — zdesy, mojno skazaty, byıl ray̆ zemnoy̆. Vprocem, celovek nikogda ne byıvay̆et dovolen i castenyko penıay̆et na sudybu. K poludnıu y̆a voznenavidel dojdy, kotoryıy̆ xel ne perestavay̆a.

Stoy̆ilo mne podumaty o tom, cto lucxe uj xel byı sneg, kak kapli obernulisy krupnyımi belyımi hlopy̆ami i slucilasy ``cudesnay̆a" metely. Ona, tocno soly Y̆adovitovo morıa, ukryıla ryıje-krasnuy̆u zemlıu belyım naletom, kotoryıy̆ ne proderjalsa i casa — iz-za oblakov vyıglıanulo solnqe i rastopilo vsıu etu krasotu.

Y̆a byıstro ponıal, cto oxibsa v rascetah i k veceru v Kruso ne popadu. Okajisy zemlıa zamerzxey̆, eto byılo byı vpolne vozmojno, no doroga razmokla, i gnaty po ney̆ loxady ne imelo nikakovo smyısla.

Propovednik eto toje ponıal, no pomalkival, poglıadyıval na solnqe. I nakoneq, uje k veceru predlojil:

— Derevenyki po puti vstrecay̆utsa. Perenocuy̆em v odnoy̆ iz nih? Mestnyıy̆e dovolyno drujelıubnyı. Vedy uje ponıatno — tyı okajexysa v gorode ne ranyxe seredinyı zavtraxnevo dnıa.

— Predpocitay̆u postoy̆alyıy̆ dvor, a ne kresty̆anskiy̆ dom.

— Vse dvoryı zabityı palomnikami, piligrimami i sumasxedximi. Vcera myı y̆edva naxli mesto dlıa noclega. Cto tyı imey̆ex protiv kresty̆an?

— Y̆a verıu v dobrotu lıudey̆, Propovednik, no gorazdo menyxe, cem prejde. Za moy̆u jizny cetyırejdyı menıa pyıtalisy ubity vo vremıa takih vot noclegov. Odin raz, potomu cto y̆a straj, v drugoy̆ — potomu cto ponravilsa moy̆ kony, v tretiy̆ — iz-za prıajki na remne i dvuh serebrıanyıh monet v koxelyke.

— A v cetvertyıy̆? — utocnil on, kogda y̆a zamolcal. — Tyı skazal, cto cetyıre raza.

— Ne znay̆u pricinyı. Tot umnik umer prejde, cem uspel povedaty mne y̆ey̆e. V obx̨em, y̆a ne slixkom jajdu nastupity na te je grabli v pıatyıy̆ raz. Eto uje slixkom. Daje dlıa menıa.

— A y̆esli ne budet postoy̆alyıh dvorov?

— Cto-nibudy pridumay̆u. Les pod bokom.

Y̆a obognal neskolyko grupp strannikov — ustavxih, izmojdennyıh, no vdohnovenno xagay̆ux̨ih v Kruso, tocno okoldovannyıy̆e.

— Vera tvorit cudesa. — Propovednik s jadnyım lıubopyıtstvom rassmatrival ih liqa.

— Vera v slova malenykoy̆ devocki i sluhi, kotoryıy̆e ih preumnojay̆ut. Skolyko etih blajennyıh ostanutsa lejaty na obocine iz-za holoda, bolezney̆, pereutomleniy̆a i vstreci s durnyımi lıudymi? Po mne, eto bolyxe napominay̆et sumasxestviy̆e, a ne veru.

— Ne soglasen s toboy̆. — On ostorojno potrogal prolomlennyıy̆ visok, zatem glıanul na paleq. — Vera na to i vera. Y̆esli boy̆atsa za svoy̆u jizny, to konecno je nado sidety doma. No sleduy̆et cto-to sdelaty, ctobyı popasty v ray̆. Ne vsem otkryıvay̆utsa eti vrata i prox̨ay̆utsa grehi.

— To y̆esty, po logike, lucxe pogibnuty v puti i obresti vecnoy̆e blajenstvo na nebesah?

— A razve net?

Y̆a pokacal golovoy̆:

— Propovednik, y̆a kak nikto inoy̆ verıu v cudesa, ad, ray̆, demonov, angelov, duxi i voskrexeniy̆e Hristovo. Sıuda mojex dobavity potop, ishod iz hagjitskih zemely, znameniy̆a, ognennyıy̆e dojdi i cto tam y̆ex̨e napisano v bibliy̆i po drugim vajnyım povodam. No y̆a gotov pospority, kak speqialist po duxam, cto nelyzıa polucity klıuc ot ray̆a, sdohnuv v puti ot tifa, y̆esli tyı nasluxalsa basen, kotoryıy̆e ne imey̆ut nicevo obx̨evo s veroy̆.

— Lıubay̆a basnıa poy̆avlıay̆etsa po vole Y̆ego.

— Aga. Tak mojno skazaty obo vsem. Vot eta luja toje po vole y̆evo. I vot eta kanava zdesy ne slucay̆na. I von ta viseliqa na perekrestke poy̆avilasy isklıucitelyno po prikazu boga, a ne mestnovo zemlevladelyqa, kaznivxevo razboy̆nikov ili prosto kakih-to bedolag.

— Nax teologiceskiy̆ spor zahodit v tupik, — zametil on. — Potomu cto y̆a imey̆u v rukave odin i tot je kozyıry, ukladyıvay̆ux̨iy̆ lıuboy̆ tvoy̆ argument na obe lopatki. Y̆emu uje bez malovo poltoryı tyısıaci let, no on otlicno dey̆stvuy̆et. Hocex uslyıxaty eti volxebnyıy̆e slova?

Y̆a prix̨urilsa:

— Udivi menıa.

— Myı prosto ne v sostoy̆aniy̆i postic Y̆ego zamyıslov, — nevinno izrek on. — Ibo kto myı pered Nim? I vozmojno, eta kanava syıgray̆et roly v Y̆ego planah. Kak i luja. I viseliqa s y̆ey̆e gruzom. Tolyko myı ob etom nikogda ne uznay̆em.

— Da, eto oceny udobnoy̆e zaklinaniy̆e. I y̆evo mojno primenity k lıuboy̆ situaqiy̆i. K primeru, tvoy̆a smerty byıla y̆evo zamyıslom.

On hihiknul:

— Y̆a ni na minutu v etom ne somnevay̆usy.

— I poetomu poroy̆ nedelıami y̆a slyıxu ot tebıa potoki bogohulystv?

— Odno drugomu ne mexay̆et. Y̆esli moy̆a smerty nujna Y̆emu, to y̆a gotov vyıpolnity svoy̆e prednaznaceniy̆e.

Y̆a snıal s golovyı kapıuxon, i vlajnyıy̆ veter s morıa vzyeroxil moy̆i otrosxiy̆e volosyı:

— I tyı znay̆ex, kakovo ono?

— Myı prosto ne v sostoy̆aniy̆i postic Y̆ego zamyıslov, — terpelivo povtoril staryıy̆ pelikan. — Byıty mojet, On jelal, ctobyı y̆a skraxival tvoy̆e odinocestvo? A zatem otpravlıusy v ray̆.

— Tyı uje mojex tuda otpravitsa. Hoty sey̆cas, — napomnil y̆a y̆emu.

— Poka y̆a ne gotov. No vozvrax̨ay̆asy k naxey̆ besede o vere i veruy̆ux̨ih. Scitay̆u, cto nevajno, naskolyko pravdivyı sluhi i devocka, blagodarıa kotoroy̆ oni poy̆avilisy. Vajna lix vera. Daje y̆esli u ney̆e net pricinyı. Ibo ona — propusk v ray̆. Ne soglasen?

Vopros byıl obrax̨en k dolgovıazomu Pugalu, kotoroy̆e, tocno ay̆ist, razmerenno xagalo po druguy̆u storonu ot moy̆ey̆ loxadi. To lix uhmyılynulosy.

— Nu da, — provorcal Propovednik. — Kuda uj tebe o duhovnyıh vex̨ah rassujdaty.

— Vera ne y̆avlıay̆etsa propuskom. Tyı oxibay̆exysa. — Myı pocti dobralisy do viseliqi, i y̆a popravil palax, visevxiy̆ rıadom s sedelynyımi sumkami, tak kak v blijay̆xih pridorojnyıh kustah mne pocudilosy dvijeniy̆e. — Krome ney̆e doljnyı byıty i horoxiy̆e postupki. Otsutstviy̆e grehov. Slepay̆a vera ne pomogay̆et, drug Propovednik, a vredit. Eto vse ravno cto neupravlıay̆emay̆a kareta, nesux̨ay̆asıa pod gorku. Ugrobit i teh, kto sidit v ney̆, i teh, kto popadet pod kolesa.

Staryıy̆ pelikan skrivilsa:

— Y̆a ponimay̆u tvoy̆u analogiy̆u, Lıudvig. Daje priznay̆u, cto tyı prav. Myı, lıudi, iskajay̆em vse, do cevo dotıagivay̆emsıa. Ili vyıvoracivay̆em nay̆iznanku, cto odno i to je. No klıanusy krovy̆u Hristovoy̆, tak byıty ne doljno. Vera doljna spasaty, a ne ubivaty.

— I ne razobx̨aty, ne strax̨aty, ne sudity i ne kaznity. No otcevo-to imenno tak i proy̆ishodit. Odni jgut vedym, drugiy̆e — kaqerov iz Vitilyska, tretyi — teh, kto zabyıl pomolitsa pered obedom. Uveren, cto pomyıslyı Gospoda v etih slucay̆ah soverxenno ni pri cem. Eto uj myı sami, voplox̨eniy̆e ruk y̆evo, dodumalisy. No vsegda gotovyı spihnuty svoy̆i ne slixkom pravednyıy̆e postupki na cujuy̆u volıu, lixiv y̆ey̆e sebıa. Mol, ne y̆a srubil golovu tomu nehristıu-hagjitu, eto bog tak velel.

Myı vplotnuy̆u podyehali k viseliqe — perekladine mejdu dvuh stolbov. Na ney̆ boltalisy dva trupa. Sudıa po vnexnemu vidu, vstrecali putnikov oni uje oceny davno. Pokoy̆niki vyıglıadeli stoly jalko, cto ne zay̆interesovali daje Pugalo.

— Zabavno, — izrek Propovednik s takim vidom, slovno y̆evo zastavili proglotity tarelku jelci. — Myı jivem i myıslim, verim, jelay̆em, lıubim i nenavidim. Myı vse, sozdaniy̆a Bojyi s gorıacey̆ krovy̆u, v konqe puti prevrax̨ay̆emsıa vot v eto. V bezduxnyıy̆ kusok mıasa na radosty cervıam i voronam.

— Cto eto na tebıa naxlo?

On otvernulsa ot viselynikov:

— Umiraty ne straxno, Lıudvig. Prosto obidno. Nikogda ne uspevay̆ex sdelaty vse, cto hotel.

— Myı ne umiray̆em posle smerti, Propovednik. Tyı — tomu y̆avnoy̆e dokazatelystvo.

— Y̆a uznal ob etom, lix kogda menıa ubili. Do tovo momenta — veril i somnevalsa. Somnevalsa i veril. Pri vseh cudesah i dokazatelystvah ne vsegda mojno iskrenne byıty ubejdennyım do konqa, cto y̆esty jizny posle smerti.

— Y̆ex̨e odna celoveceskay̆a certa, — usmehnulsa y̆a. — Myı sklonnyı somnevatsa daje v ocevidnyıh faktah. Sploxnyıy̆e protivoreciy̆a.

On neojidanno ulyıbnulsa.

— Inogda tyı govorix zamecatelynyıy̆e vex̨i, moy̆ malycik. V takiy̆e minutyı y̆a uznay̆u cto-to novoy̆e o samom sebe, — proronil on i falyqetom zapel qerkovnyıy̆ gimn vo slavu blagodati.



Nocevaty pod otkryıtyım nebom ili v kakom-nibudy zabroxennom saray̆e ne prixlosy. Poctovay̆a stanqiy̆a s postoy̆alyım dvorom okazalasy kak nelyzıa kstati. I, nesmotrıa na to cto v krohotnom zale byılo narodu stolyko je, skolyko v bocke alybalandskih seledok, svobodnay̆a komnata naxlasy.

— Da neujeli? — izumilsa Propovednik i tknul suhim palyqem v cernıavuy̆u hozıay̆ku. — Sovetuy̆u sprosity u ney̆e, v cem zdesy podvoh. S tem kolicestvom jelay̆ux̨ih priobx̨itsa k svıatomu mestu komnatu nelyzıa nay̆ti daje za florin. A zdesy svobodnay̆a!

Y̆a sprosil. Jenx̨ina ne stala skryıvaty:

— Tri mesıaqa nazad zarezali tam odnovo putnika. Sam vinovat. Pustil cujakov, mnogo boltal, pil i soril denygami. Y̆a i opomnitsa ne uspela, kak y̆evo vyıpotroxili.

— S kakih eto por ubiy̆stvo pugay̆et gostey̆?

Ona nahmurilasy, zatem rexilasy:

— Horoxo. Ne budu y̆ulity, gospodin. Moy̆ syın videl vax kinjal, kogda zavodil loxady v stoy̆lo. Vyı straj, a znacit, ne boy̆itesy prizrakov. I ne potrebuy̆ete platu nazad.

— A vot s etovo momenta popodrobney̆e, — zay̆interesovanno poprosil y̆a.

Obyıcno prizraki i privideniy̆a ne boley̆e cem mif. Tak nazyıvay̆ut duxu, kotoruy̆u vnezapno vidıat vse, komu ona jelay̆et pokazatsa na glaza. O takom pixut v knigah, no v realynosti podobnoy̆e y̆avleniy̆e straji vstrecay̆ut reje govorıax̨evo kozla za obedennyım stolom Papyı.

— Otpeli i zakopali, vse kak polojeno. A on, svoloc, vse pokoy̆a ne day̆et. — Y̆ee liqo stalo zlyım. — Zanıal komnatu, pugay̆et lıudey̆. Te uje nacali boltaty, a mne, kak ponimay̆ete, ni k cemu razgovoryı. Sey̆cas narodu mnogo i mest net, a kak potok shlyınet, nikto ko mne ne idet, krome pridurkov, kotoryım ohota poglazety na mertveqa. Takih, kak vyı ponimay̆ete, gorazdo menyxe normalynyıh lıudey̆.

— Kto-nibudy posle y̆evo poy̆avleniy̆a zdesy umiral?

— Net.

— Bolel? Kalecilsa?

— Net, upasi Gospody. Nicevo takovo. I s dohodami poka vse horoxo. Da i ne zloy̆ on. Prosto pugaty lıubit. Slujanki uje tuda i ne zahodıat. Komnatu ne ubirali. Izbavyte menıa ot nevo, gospodin. A y̆a besplatno pux̨u. I y̆edu lucxuy̆u, i vino. I loxadke vaxey̆ pxeniqi otbornoy̆.

— Soblaznitelyno, — bez osobyıh emoqiy̆ proy̆iznes y̆a. — Nu pokazyıvay̆, gde u tebıa plohay̆a komnata.

Ona okazalasy na pervom etaje, v dalynem konqe doma, s y̆edinstvennyım oknom i vidom na skotnyıy̆ dvor.

— Nastoy̆ax̨iy̆ dvoreq. — Propovednik dal svoy̆u kriticeskuy̆u oqenku ubogomu interyeru i krovati s solomennyım matrasom.

— Nadey̆usy, prizrak smog napugaty ne tolyko lıudey̆, no i klopov. — Y̆a brosil sumku v temnyıy̆ ugol i, ne uderjavxisy, otkryıl malenykuy̆u fortocku. Zdesy davno sledovalo provetrity.

— Vrode byı obyıcnyıy̆e lıudi nas videty ne doljnyı. — Y̆ego pelikanye svıatey̆xestvo plıuhnulsa na moy̆u krovaty.

— Vsegda y̆esty isklıuceniy̆a iz pravil.

V dvery postucali.

— A vot i on. Kakoy̆ vejlivyıy̆, — hihiknul moy̆ sputnik.

Razumey̆etsa, eto byıla nikakay̆a ne duxa, a sama hozıay̆ka. Ne vhodıa, ona protıanula mne cistoy̆e postelynoy̆e belye, staray̆asy ne smotrety v komnatu:

— Ujinaty budete v zale ili sobraty vam zdesy, gospodin?

— V zale, — k y̆ey̆e y̆avnomu oblegceniy̆u otvetil y̆a.

Poka y̆a sidel v tolcey̆e, opustoxay̆a tarelku, v komnate poy̆avilsa gosty. No ne tot, kotorovo y̆a jdal. Eto byılo vsevo lix Pugalo. Ono besqeremonno izvleklo iz moy̆ey̆ obyemnoy̆ sumki glavnoy̆e y̆ey̆e soderjimoy̆e — tıajeluy̆u tolstuy̆u tetrady, perepletennuy̆u v xerxavuy̆u svinuy̆u koju.

Y̆a unes vse, cto naxel na stole pokoy̆novo burggrafa, no lix eta vex̨ opravdala moy̆i ojidaniy̆a. V moy̆i ruki popalo necto vrode dnevnika, buhgalterskoy̆ knigi i y̆ejednevnika za posledniy̆e polgoda — y̆evo milosty otlicalsa pedanticnosty̆u i doverıal bumage vse svoy̆i dela.

Oteq un Nomanna, k sojaleniy̆u, ne byıl nastolyko nay̆iven, ctobyı ne ispolyzovaty xifr. Posledniy̆ okazalsa slojen — izobreteniy̆e flotoliy̆skih bankirov. Procitaty y̆evo bez klıuca ne predstavlıalosy vozmojnyım. Poetomu y̆a sleduy̆ux̨im je utrom otpravil dnevnik cerez ``Fabyen Klemenz i syınovy̆a" Gertrude, znay̆a, cto s y̆ey̆e svıazıami i znakomstvami, v tom cisle i v Riapano, gde obojay̆ut ne tolyko sozdavaty, no i raskryıvaty cujiy̆e sekretyı, uznaty, cto napisano, polucitsa gorazdo byıstrey̆e.

Rovno cerez dve nedeli y̆a polucil tetrady obratno v drugom otdeleniy̆i ``Fabyen Klemenz", nahodıax̨emsıa za sto lig ot pervovo, i sredi straniq lejal klıuc s pravilynoy̆ kombinaqiy̆ey̆ i provox̨ennyıy̆ trafaret, v kotoryıy̆ trebovalosy podstavlıaty nujnyıy̆e bukvyı.

Y̆a nacal s samyıh poslednih zapisey̆ i ne oxibsa. Uje na tretyey̆ straniqe s konqa, mejdu otmetkami o vyıplate jalovany̆a slugam i sovex̨aniy̆i u burgomistra, naxlosy necto lıubopyıtnoy̆e:

``Interesnyıy̆ kinjal v kollekqiy̆u po proxloy̆ dogovorennosti. Polucen cerez „Fabyen Klemenz i syınovy̆a“. Otpravitely iz Kruso. Qerkovy Svıatovo Mihay̆ila. Avans v scet cernovo kamnıa. Rasplatitsa. Posyılku prosıat peredaty licno. Kuryer priy̆edet v nacale fevralıa".



Propovednik, uznav, cto y̆a sobiray̆usy v Kruso, daje rukami vsplesnul:

— Gospodi Iisuse, Lıudvig! A pocemu ne k hagjitam? Ili srazu k adskim vratam na vostocnoy̆ okray̆ine mira?! Do Nararyı puty neblizkiy̆, i tebe tam sovsem necevo delaty.

— Krome kak razobratsa s tem, cto slucilosy s Kristinoy̆, po sledu kotoroy̆ y̆a idu s samovo nacala oseni. Kto-to iz Kruso otpravil y̆ey̆e kinjal burggrafu. Tot, komu nujnyı byıli kamni serafima. I predpolagay̆u, dlıa tovo, ctobyı vyıkovaty temnoy̆e orujiy̆e.

— Da-da! Temnyıy̆ kuzneq jivet v qerkvi Svıatovo Mihay̆ila i tolyko i delay̆et, cto jdet tebıa. Ctobyı tyı priy̆ehal i zadal y̆emu svoy̆i voprosyı! Tovo, kto otpravil kinjal, uje mojet tam ne byıty!

— On tam, — s uverennosty̆u proy̆iznes y̆a. — V konqe fevralıa burggraf doljen byıl otpravity y̆emu kamni.

— Nu, mojet, prejde cem y̆ehaty sotni lig, tyı prosto zay̆dex v ``Fabyen Klemenz" i poy̆interesuy̆exysa, ot kovo byıla posyılka?

— S kakoy̆ stati im otvecaty? Oni ne raskryıvay̆ut postoronnim tay̆nyı svoy̆ih kliy̆entov.

— Tyı toje ih kliy̆ent. I s dovolyno vnuxitelynyım scetom.

— Eto nicevo ne menıay̆et. Oni ne stanut riskovaty reputaqiy̆ey̆ i otcityıvatsa o cujih tay̆nah.

On byıl nedovolen i ne jelal y̆ehaty na zapad. No, sobstvenno govorıa, kogda byılo inace? Propovednik, vsıu jizny provedxiy̆ v svoy̆ey̆ derevne, nesmotrıa na to cto motay̆etsa za mnoy̆ ne odin god, tak i ne privyık k castyım perey̆ezdam.

Odnako vernemsıa k nastoy̆ax̨emu. Tepery Pugalo rexilo zanıatsa cteniy̆em ili delalo vid, cto citay̆et dnevnik burggrafa. Ono nespexno perevoracivalo straniqi seryımi kogtistyımi palyqami, naklonivxisy k samoy̆ svece, kotoray̆a y̆edva ne podjigala y̆evo xlıapu, sdelannuy̆u iz plohoy̆ solomyı.

— Kak y̆a ponimay̆u, tebe ne smux̨ay̆et xifr.

Ono daje golovyı ne povernulo.

— Interesno, cto ono hocet tam nay̆ti. — Propovednik, podperev x̨eku, s nekotoroy̆ zavisty̆u sledil za oduxevlennyım.

Nelepoy̆e Pugalo s polıa delalo to, cevo ne mog selyskiy̆ svıax̨ennik — ono prekrasno umelo citaty.

Y̆a stal gotovitsa ko snu, kogda poy̆avilsa novyıy̆ gosty — blednyıy̆ celovek s izurodovannyım liqom i v zalitoy̆ krovy̆u odejde. Nad nim horoxenyko porabotali nojami i nanesli takoy̆e kolicestvo ran, cto vporu byılo lix pojalety y̆evo.

Vyıtarax̨iv glaza, on zaskrejetal zubami i medlenno dvinulsa ko mne.

— Ne nadoy̆elo? — s ucastiy̆em sprosil y̆a.

On ostanovilsa kak vkopannyıy̆, posmotrev nedovercivo, i ostorojno sprosil s silynyım litavskim akqentom:

— Cto, sovsem ne straxno?

— Uvyı, — s sojaleniy̆em razvel y̆a rukami.

— I vam ne straxno? — sprosil ney̆izvestnyıy̆ u Propovednika.

— Y̆a mertv, kak i tyı, poludurok, — provorcal tot. — Napugaty mertveqa mertveqom eto nado umudritsa. Klıanusy Devoy̆ Mariy̆ey̆, boley̆e glupoy̆ zatey̆i y̆a y̆ex̨e ne vidal.

— Navernoy̆e, stoy̆ilo podkrastsa szadi, — probormotal tot i doveritelyno skazal mne: — Obyıcno vse ubegali s voplem i y̆edva dvery ne snosili.

— Nekotoryıy̆e lıudi ne takiy̆e, kak vse. — Y̆a polojil na stol obnajennyıy̆ klinok.

— Bezdna! — proy̆iznes ubityıy̆ i rvanul proc, no ne tut-to byılo. Figura, kotoruy̆u y̆a kinul y̆emu pod nogi, byıla nicuty ne huje silka, kotoryıy̆e stavıat na krolika.

On dernulsa raz, drugoy̆, no lix zaputalsa y̆ex̨e silyney̆e.

— Ey̆, priy̆ately! Tyı ne imey̆ex nikakovo prava menıa trogaty! — Na y̆evo liqe byıl strah. — Y̆a ne temnyıy̆.

— Eto tyı tak scitay̆ex. — Y̆a vstal, vzıavxisy za kinjal. — Tyı ostalsa na meste svoy̆evo ubiy̆stva, i otcevo-to tebıa kto-to uvidel. Razumey̆etsa, on ispugalsa. I tebe eto ponravilosy.

— Vsevo lix malenykay̆a xalosty, — proskulil on.

— Cujoy̆ strah dobavil tebe sil. A oni dali vozmojnosty uvidety tebıa y̆ex̨e komu-to. I tyı snova napugal. I opıaty podpitalsa ujasom.

— No y̆a… — On zatknulsa, kogda y̆a podnıal ruku s klinkom, prizyıvay̆a y̆evo k molcaniy̆u.

— Y̆a rasskaju tebe o posledstviy̆ah. Pitaniy̆e strahom privedet k tomu, cto tvoy̆a svetlay̆a sux̨nosty stanet temnoy̆. Ne prıamo sey̆cas. Byıty mojet, cerez mesıaq, a mojet, i cerez god — smotrıa skolykih lıudey̆ tyı napugay̆ex i kak silyno im budet straxno. No povery mne, rano ili pozdno podobnoy̆e proy̆izoy̆det. Znay̆ex, cto togda slucitsa?

— Tyı pridex za mnoy̆? — xepotom sprosil tot.

— Y̆a uje prixel za toboy̆. I ne budu ojidaty toy̆ poryı, kogda tyı pererodixysa iz-za svoy̆ih glupyıh zabav i nacnex ubivaty lıudey̆. Poka pered toboy̆ otkryıtyı vrata ray̆a. No, kogda tyı naberexysa tymyı, otpravixysa ne naverh, a vniz. Zagremix v cistilix̨e. Grubo govorıa, sobstvennyımi rukami otpravix sebıa tuda, kuda nikto ne hocet. Ne slixkom prekrasnay̆a perspektiva, na moy̆ vzglıad. Y̆a day̆u tebe vyıbor. Uy̆dex sam ili mne vyıpolnity svoy̆u rabotu?

— Uy̆du sam, — byıstro otvetil on. — Nikakiy̆e xutki ne stoy̆at ada. Y̆a prosto snova hotel pocuvstvovaty jizny.

Y̆a razruxil figuru, uderjivay̆a nagotove znak. On vzdohnul, zakryıl glaza, a dalyxe slucilosy to, cto y̆a videl uje mnogo raz. Y̆ego siluet stal blednety, poka ne ostalosy y̆edva zametnovo kontura. Tot na mgnoveniy̆e zasiy̆al solnecnyım svetom, kotoryıy̆ ozaril vsıu komnatu, i vokrug vnovy nastupila polutyma, razgonıay̆emay̆a lix svecami na stole.

Cto primecatelyno, Pugalo daje golovyı ne povernulo, prodoljay̆a citaty dnevnik burggrafa.

— Nu, tepery hozıay̆ka postoy̆alovo dvora tocno skajet tebe spasibo. — Propovednik vyıglıadel zadumcivyım, y̆avno razmyıxlıay̆a o tom, cto kogda-nibudy necto podobnoy̆e predstoy̆it sdelaty i y̆emu. — Skaji, tyı byı i vpravdu zabral y̆evo kinjalom? On vedy vse-taki svetlyıy̆.

— Zabral byı. Potomu cto takoy̆ svetlyıy̆ byıstro stanovilsa temnyım, a eto otnositsa k prıamoy̆ ugroze lıudıam.

— Interesno, cto on vidit sey̆cas? Raspahnutyıy̆e vrata? Svıatovo Petra s klıucami? Ili arhangela Mihay̆ila?

— Boy̆usy, ne smogu udovletvority tvoy̆e lıubopyıtstvo. Pridetsa tebe proverity samomu. Davay̆ spaty. — I, povernuvxisy k Pugalu, dobavil: — Docitay̆ex, ne zabudy pogasity svecu.

I y̆a usnul pod tihiy̆ xelest perelistyıvay̆emyıh straniq.



— Vot sukin syın! — sgorıaca proy̆iznes y̆a.

Ot dnevnika burggrafa ostalasy lix odna oblojka. Straniqi byıli akkuratno vyırezanyı i raskley̆enyı po potolku. Cernila na nih namokli i raspolzlisy, tak cto procitaty bolyxe nicevo byılo nelyzıa.

— Y̆a… — probley̆al Propovednik, tak i ne zakonciv predlojeniy̆e.

Vse byılo ponıatno. Kogda oduxevlennyıy̆ eto prodelal, staryıy̆ pelikan gde-to brodil, poetomu ne smog razbudity menıa. Y̆a molca nacal odevatsa. Uje rassvelo, i pora byılo otpravlıatsa v dorogu.

— Tam soderjalosy cto-to qennoy̆e? — ostorojno poy̆interesovalsa Propovednik.

— Y̆ex̨e vcera y̆a byı skazal, cto net. Tepery uje ne uveren. — Y̆a zastegnul poy̆as s kinjalom.

— Mojet, eto odna iz y̆evo neponıatnyıh xutok? Mojet, ono prosto razvlekay̆etsa?

— Pojivem — uvidim.

— To y̆esty tyı nicevo ne budex delaty?

— V smyısle begaty po okrestnostıam i iskaty oduxevlennovo, kotoryıy̆ odnim x̨elckom palyqev mojet peremestitsa na tyısıacu lig, na pole, gde nahoditsa y̆evo obolocka? Pugalo vernetsa, ono vsegda vozvrax̨ay̆etsa. Moy̆ih planov eto nikak ne naruxalo.

— No tetrady…

— Y̆esli cestno, y̆a sobiralsa sjec y̆ey̆e y̆ex̨e neskolyko dney̆ nazad, no ruki nikak ne dohodili. Tak cto plevaty na tetrady. Kruso. Qerkovy Svıatovo Mihay̆ila. Vot moy̆a qely na segodnıa.

Kak tolyko y̆a okazalsa v zale, hozıay̆ka tut je kinulasy ko mne. V y̆ey̆e glazah citalsa vopros.

— On bolyxe ne pobespokoy̆it nikovo, — skazal y̆a, i ona rassyıpalasy v iskrennih blagodarnostıah.

Kogda y̆a vyıxel na uliqu, malycixka tut je podvel mne loxady.

Po sravneniy̆u s proxlyım dnem segodnıa byılo y̆asno i oceny teplo. Trakt konecno je okazalsa zabit telegami, vsadnikami i pexehodami. Vse xli v gorod, ctobyı poklonitsa novoy̆ svıatyıne i uvidety sled bosoy̆ stupni, kotoryıy̆ y̆akobyı angel ostavil pered domom devocki.

Kruso — sploxnyıy̆e stenyı i baxni iz jeltovo kamnıa. Gorod, ranyxe byıvxiy̆ stoliqey̆ korolevstva, narodom okazalsa zaprujen nicuty ne menyxe, cem doroga. U Ryıbnyıh vorot y̆a popal v nesusvetnuy̆u davku. Vokrug kricali molitvyı, peli, ponosili drug druga, vizjali svinyi i orali te, kto poterıal v tolcey̆e svoy̆i koxelyki. To i delo melykali belyıy̆e plax̨i palomnikov, qvetnyıy̆e lentocki na posohah. Kakay̆a-to gruppa kresty̆an nesla krest, obhodıa gorodskiy̆e stenyı po krugu. K nim kajduy̆u minutu prisoy̆edinıalisy novyıy̆e molıax̨iy̆esıa, raspevay̆a ``Velicit duxa moy̆a Gospoda".

Pod kopyıta moy̆ey̆ loxadi brosilsa nix̨iy̆, vopıa, cto grıadet koneq sveta i y̆a doljen pokay̆atsa i otdaty y̆emu vse denygi. Odnako, ponıav, cto y̆a ne otlicay̆usy osoboy̆ nabojnosty̆u, on tut je zabyıl obo mne i pristal k dvum dorodnyım kupqam, kotoryıy̆e byıli neskolyko perepuganyı tem bezumiy̆em, cto proy̆ishodilo vokrug.

U sleduy̆ux̨ih vorot byılo nicuty ne lucxe. Usilennyıy̆ otrıad straji sderjival tolpu. Lıudey̆ nabralosy stolyko, cto mnogiy̆e, poterıav nadejdu probratsa v gorod segodnıa, razbivali ogromnyıy̆ palatocnyıy̆ lagery na golom pole.

— Kuda prex? — po-nararski zaoral na menıa ustalyıy̆ strajnik v polosatom berete. — Vali nazad!

Y̆a pokazal y̆emu kinjal, i menıa, nesmotrıa na rugany oceredi, propustili.

— Qerkovy Mihay̆ila. Kak mne y̆ey̆e nay̆ti? — sprosil y̆a u soldata.

— Sprosi cevo polegce! — otmahnulsa tot. — Ih tut do certa, kak i bogomolyqev!

— Ne tyı odin ne lıubix palomnikov, — hihiknul Propovednik.

Prixlosy rasspraxivaty na uliqah. Kakoy̆-to pareny so znakom gilydiy̆i portnyıh na kamzole pocesal v zatyılke:

— Eto ta, kotoray̆a vozle kolodqa, cto ly?

— Znal byı, ne spraxival.

— Y̆episkopskay̆a, na Maloy̆ Zlotinke, po puti k vnutrenney̆ stene.

Y̆a poblagodaril y̆evo i napravilsa k qentru goroda. Kruso do etovo y̆a nikogda ne posex̨al, no rexil, cto zdesy, skorey̆e vsevo, budet tak je, kak i v drugih mestah pri prazdnestvah, y̆armarkah, svıatyıh palomnicestvah i svadybah knıazey̆ — vse dexevoy̆e jilye rashvatano, i lezty tuda ne imey̆et nikakovo smyısla. A vot v bogatyıh ray̆onah, gde poroy̆ mogut za noc sodraty i cetverty florina, y̆esli sovesty otsutstvuy̆et, krovaty dlıa putexestvennika vsegda nay̆detsa.

Moy̆ opyıt menıa ne obmanul. Postoy̆alyıy̆ dvor ``Pod koronoy̆ knıazıa", raspolojennyıy̆ naprotiv staryıh korolevskih konıuxen, qenami raspugal vseh jelay̆ux̨ih i prinimal lix lıudey̆, kotoryıy̆e ne oceny-to scitali denygi. Ostaviv loxady i sprosiv u hozıay̆ina dalyney̆xuy̆u dorogu, y̆a otpravilsa pexkom. Tak vyıhodilo gorazdo byıstrey̆e.

Qerkovy — seray̆a gromada, stisnutay̆a s dvuh storon jilyımi domami tak, cto predstavlıala s nimi y̆edinoy̆e qeloy̆e i otlicalasy ot nih lix xpilem, torcax̨im nad cerepicnyımi kryıxami. Na stupenykah sideli dvoy̆e cumazyıh malycixek let desıati, oni bez vsıakovo entuziazma prosili milostyınıu. Y̆a podergal dvery, no bezrezulytatno, hotıa slyıxal, cto vnutri igray̆et organ.

— Zakryıto, dıadecka, — skazal mne odin.

— No tam kto-to y̆esty.

— Cto s tovo? Svıax̨ennika-to net.

Y̆a dostal neskolyko medıakov, kinul im v xapku.

— Spraxivay̆te, — stepenno pozvolil vtoroy̆ i vyıter rukavom soplivyıy̆ nos.

— Gde on i pocemu zakryıta dvery?

— Vse svıax̨enniki tepery vozle casovni na toy̆ storone krutıatsa. Gde deviqa videla angela.

— Da ne mogla ona nicevo videty! — vozmutilsa y̆evo tovarix̨. — Ona ot rojdeniy̆a slepay̆a!

— Vot potomu i videla, cto ne videla! — zasporil tot. — Tak oteq Seliko govoril! A on-to pobole, cem tyı, znay̆et!

— A pocemu muzyıka igray̆et?

— Muzyıkant repetiruy̆et. On casto sıuda prihodit. No qerkovy otkroy̆etsa tolyko posle voskreseny̆a.

— A cto budet v voskresenye?

Malycixka mnogoznacitelyno posmotrel v xapku:

— Y̆esli uj vam leny u drugih uznavaty, gospodin, to vyı nam y̆ex̨e medıak na hlebuxek podkinyte.

Y̆a rassmey̆alsa y̆evo nahalystvu, kinul dva.

— Slujba torjestvennay̆a. Kardinal priy̆edet. Iz Riapano, govorıat. Ctobyı provesti messu dlıa uvajay̆emyıh jiteley̆ goroda. V cesty y̆avleniy̆a.

— Kak mne popasty v qerkovy?

Malycixki pereglıanulisy.

— Ne znay̆em, — otvetil tot, cto vyıglıadel postarxe.

— Vraty tyı ne umey̆ex, priy̆ately. — Mejdu ukazatelynyım i srednim palyqem y̆a derjal serebrıanuy̆u monetku.

Deti naklonilisy drug k drugu, poxuxukalisy.

— Ladno, dıadecka. Provedu.

Malycik zabral denejku, otdal y̆ey̆e svoy̆emu priy̆atelıu, kotoryıy̆ tut je sprıatal sokrovix̨e za pazuhu.

— Idemte, dıadecka.

On otvel menıa v pereulok, oglıadelsa i tocno kotenok y̆urknul v raspahnutoy̆e sluhovoy̆e okoxko, nahodıax̨ey̆esıa na urovne mostovoy̆. Nado priznatsa, tuda byı y̆a ne prolez pri vsem jelaniy̆i.

Liqo malycixki poy̆avilosy v okoxke:

— Idite k podvalu, von tomu. Y̆a x̨as dvery otkroy̆u.

Spusk v podval toje byıl na uliqe, zakryıtyıy̆ stalynyım x̨itom. Klaqnula zadvijka, y̆a podnıal nelegkiy̆ lıuk, spustilsa vniz. Malycixka provorno zax̨elknul zamok:

— Y̆esli dıadyka Mikely uznay̆et, cto y̆a snova zdesy lazay̆u, on uxi otorvet. Davay̆te byıstrey̆e, dıadecka.

Podvalynoy̆e pomex̨eniy̆e pod domom, s nizkim potolkom i zatıanutyımi pautinoy̆ uglami pohodilo na labirint. Lestniqa vyıvela nas v polutemnyıy̆ koridor. Zdesy silyno pahlo kvaxenoy̆ kapustoy̆ i koxkami. Gde-to za dvery̆u, nadryıvay̆asy, krical mladeneq. Xustryıy̆ malycixka bejal vpered, tak cto mne ostavalosy lix pospevaty za nim i ne vrezatsa golovoy̆ v vıazanki luka, svisay̆ux̨iy̆e s potolka.

Cernyıy̆ hod vyıvel nas v malenykiy̆ vnutrenniy̆ dvor doma — grıaznyıy̆, neuhojennyıy̆, s pokosivxey̆sıa golubıatney̆ vozle zabora.

— Cerez zabor vam, — skazal malycixka i, bolyxe nicevo ne oby̆asnıay̆a, skryılsa v zdaniy̆i.

Y̆a tak i postupil, blago perebratsa cerez pregradu byılo neslojno. Qerkovnyıy̆ dvor okazalsa y̆ex̨e menyxe — takoy̆ tesnyıy̆, cto napominal komnatu v kakoy̆-nibudy provinqialynoy̆ taverne. Organ prodoljal igraty, i daje tolstyıy̆e kamennyıy̆e stenyı ne mogli prigluxity y̆evo velicestvennyıy̆e zvuki.

Malenykay̆a kalitka byıla poluotkryıta, tak cto y̆a voxel. Krome zvuka organa y̆a slyıxal, kak nahodıax̨iy̆esıa vnizu podsobnyıy̆e rabociy̆e razduvay̆ut mehi muzyıkalynovo instrumenta. Uzkimi zakutkami vyıxel na balkon, otkuda otkryıvalsa vid na kolonnadu, pustyıy̆e skamyi i y̆arko-jeltyıy̆ uzor na polu ot vitrajey̆, v kotoryıy̆e svetilo solnqe. Spustilsa vniz po vitoy̆ lestniqe, rexiv ne mexaty nevidimomu organistu, i sel na pervuy̆u skamy̆u.

Zakryıl glaza, sluxay̆a muzyıku. Ona byıla grandioznoy̆, obyemnoy̆ i, kazalosy, pronzala tebıa naskvozy.

— Potrıasay̆ux̨e, — prozvucal u menıa nad uhom golos Propovednika. — V koy̆i-to veki tyı dovolen, nahodıasy v qerkvi.

— Cudesnay̆a muzyıka, — v otvet proy̆iznes y̆a. — Ne poboy̆usy etovo slova — bojestvennay̆a.

— I mnitsa mne, cto y̆a slyıxu y̆ey̆e v pervyıy̆ raz. — On byıl nemnogo rasterıan. — K kakoy̆ eto molitve?

— Ne imey̆u ponıatiy̆a.

— Togda cemu tyı ulyıbay̆exysa?

— Tomu, cto moy̆ dolgiy̆ puty okoncen.

On glıanul na menıa kak na sumasxedxevo. Hmyıknul i pristroy̆ilsa na lavke, ne jelay̆a bolyxe nicevo spraxivaty. Tak myı i sideli, poka zvuki ne stihli pod svodami.

Organist voxel v zal, i okazalosy, cto eto jenx̨ina. V rukah ona derjala stopku ispisannyıh not i na hodu cto-to cerkala v nih grifelem, ne zamecay̆a menıa. Tak cto y̆a otlicno smog y̆ey̆e rassmotrety. Oceny malenykay̆a, huday̆a i tonenykay̆a, kak devocka. Iz-pod barhatnovo bereta gilydiy̆i muziqirovaniy̆a vo vse storonyı torcali vihrastyıy̆e cernyıy̆e volosyı. Oni silyno otrosli i padali y̆ey̆ na pleci. Milovidnoy̆e liqo byılo sosredotoceno, lob nahmuren, krasivyıy̆e gubyı sjatyı, a v uglah nemnogo raskosyıh vostocnyıh glaz, harakternyıh dlıa teh, u kovo predki jili v Iliate, poy̆avilisy morx̨inki.

— Redko mojno vstretity v qerkvi jenx̨inu-muzyıkanta, — gromko skazal Propovednik.

Razumey̆etsa, skazal dlıa menıa, ne dumay̆a, cto kto-to drugoy̆ y̆evo uslyıxit.

No ona uslyıxala i, vzdrognuv, y̆edva ne uronila notyı, poy̆mav ih v posledniy̆ moment pokalecennoy̆ rukoy̆. Prix̨urivxisy, devuxka s podozreniy̆em glıanula na Propovednika, hotela cto-to skazaty i nakoneq uvidela menıa.

— Zdravstvuy̆, Kristina, — negromko proy̆iznes y̆a.

— Privet, Sineglazyıy̆, — otvetila ta, kovo y̆a tak dolgo iskal.

Voqarilosy molcaniy̆e. Porajennyıy̆ Propovednik tarax̨ilsa na nas, kak palomnik na snizoxedxevo na y̆evo molitvyı svıatovo.

— Tvoy̆ kony skucay̆et.

Y̆ee pleci rasslabilisy, i ona vzdohnula:

— Y̆a toje oceny skucay̆u po Vy̆unu. No sey̆cas y̆emu lucxe byıty s Miriam, cem so mnoy̆. Kak tyı menıa naxel?

— Cereda slucay̆nostey̆ i vezeniy̆e. Hocu vernuty tebe koy̆e-cto.

Y̆a protıanul svoy̆ey̆ byıvxey̆ naparniqe braslet iz dyımcatyıh rauhtopazov. Vot tepery straj dey̆stvitelyno byıla porajena. Y̆ee notyı — muzyıka, v kotoroy̆ devuxka duxi ne cay̆ala, — upali nam pod nogi. Y̆a videl, kak drojat y̆ey̆e palyqi, kogda ona zabirala svoy̆ braslet.

— Nam nado seryezno pogovority, Lıudvig. — Y̆ee golos sel i zvucal hriplo, no glaz ona ne opustila.

— Imenno eto y̆a i hotel predlojity.



Komnatyı, kotoryıy̆e ona snimala, nahodilisy nad bolyxoy̆ aptekoy̆, na vtorom etaje. Vhod byıl cerez torgovyıy̆ zal. Sedovlasyıy̆ i sedoborodyıy̆ aptekary, malenykiy̆ i nelepyıy̆, posmotrel na menıa poverh uvelicitelynyıh stekol, zakreplennyıh u nevo na nosu, no nicevo ne skazal, vernuvxisy k vesam, na kotoryıh otmerıal kakoy̆e-to koricnevoy̆e snadobye dlıa pokupatelıa.

Xurxa y̆ubkoy̆, Kristina brosila notyı na komod, dostala iz nevo butyılku vina, dva bokala:

— Tyı vse y̆ex̨e pyex krasnoy̆e?

— Vremıa ot vremeni.

— Otkroy̆. — Ona sela za stol, malenykimi palyqami zdorovoy̆ ruki perebiray̆a gladkiy̆e dyımcatyıy̆e kamni. — Znacit, tyı naxel y̆evo?

Imıa ne prozvucalo, no byılo i tak ponıatno, pro kovo ona spraxivay̆et. Pro Gansa.

— Da. — Y̆a vyıtax̨il probku iz butyılki, plesnul v bokalyı vina, sel naprotiv.

— I vyıjil. Tyı vsegda byıl vezucim, Lıudvig. Vezucim, kak cert. — Ona goryko usmehnulasy. — V otliciy̆e ot nevo.

— Vyı byıli vmeste?

Ona ne stala otriqaty:

— Kakoy̆e-to vremıa. — Pomolcala i dobavila: — Oceny kratkoy̆e vremıa. Tyı udivlen?

— Sey̆cas? Net. Vot kogda naxel tvoy̆ braslet u nevo — udivilsa. Vyı ne slixkom ladili posle tovo, kak tyı podderjala idey̆u Miriam, cto u kajdovo knıazıa doljen byıty personalynyıy̆ straj.

— Y̆a scitala, cto politiceski eto polezno dlıa Bratstva. — Byılo vidno, cto y̆ey̆ nepriy̆atnyı vospominaniy̆a. — Myı s Gansom rexili vse raznoglasiy̆a. Tebe on ne hotel govority.

— Vaxe pravo i vaxi dela, — pojal y̆a plecami. — Menıa bolyxe interesuy̆et, cto slucilosy v gorah.

Ona nervno krutanula stakan:

— Cert y̆evo znay̆et, Sineglazyıy̆. On y̆ex̨e v Ardenau vyıglıadel vstrevojennyım. Govoril, cto naxel necto interesnoy̆e. Zatem y̆evo otcitali starey̆xinyı na sovete, tyı vedy pomnix, kakovo slona oni sdelali iz toy̆ muhi?

Y̆a kivnul.

— V obx̨em, on uy̆ehal iz Alybalanda, a zatem, gde-to cerez mesıaq, myı vstretilisy v Liseqke. On skazal, cto y̆evo jdut dela na vostoke, zval s soboy̆, i y̆a poy̆ehala. V Bude myı natknulisy na temnuy̆u duxu, zasevxuy̆u v kolodqe. Gans toropilsa, govoril, cto y̆emu vo cto byı to ni stalo nado popasty v Dorc-gan-Toy̆n, poprosil menıa razobratsa s problemoy̆ i dojdatsa y̆evo. Obex̨al vernutsa cerez poltoryı nedeli.

— No tyı ne dojdalasy.

— Ne stala jdaty, — ulyıbnulasy ona, i y̆a vspomnil, kakoy̆ uprıamoy̆ poroy̆ stanovilasy Kristina. — Prikoncila tu tvary, vzıala deneg s burgomistra, ostavila Vy̆una v horoxey̆ konıuxne i napravilasy sledom za nim, v goryı. No ne uspela. Kalikveq na vorotah, na moy̆e scastye, okazalsa serdobolynyım celovekom. Skazal, cto y̆evo braty̆a i Orden Pravednosti ubili straja. Cto y̆a ne nay̆du telo i mne sleduy̆et uhodity kak mojno byıstrey̆e.

Y̆ee golos zadrojal, i ona po staroy̆ privyıcke prilojila pokalecennyıy̆ bezyımıannyıy̆ paleq klevoy̆ skule, prijala do boli, tak cto pobelela koja.

— Myı daje ne poprox̨alisy. I y̆a ne uvidela y̆evo mogilu.

— Tyı poverila monahu?

— O! On byıl oceny ubeditelen. Y̆a do sih por blagodarıu y̆evo za spaseniy̆e.

— On mertv, — jestko skazal y̆a. — Za to, cto predupredil tebıa, y̆evo raspıali v ledıanoy̆ pex̨ere.

Ona lix othlebnula vina:

— Pusty na nebe y̆evo duxe budet horoxo.

Kristina ne sprosila menıa, otkuda y̆a znay̆u, cto kalikveq mertv, a y̆a ne stal y̆ey̆ rasskazyıvaty, vo cto on prevratilsa posle smerti.

— Cto byılo dalyxe?

— Y̆a ne mogla mstity ublıudkam s krasnyımi verevkami na rıasah. No mne hvatilo sil na zakonnikov. — Ulyıbka u ney̆e byıla zloy̆ i oceny nepriy̆atnoy̆. Y̆a nevolyno podumal, cto Kristina cem-to napominay̆et mne Miriam v y̆ey̆e ne samyıy̆e lucxiy̆e dni.

— I tyı ubila vseh troy̆ih.

Ona potrıasenno morgnula:

— Tyı i eto znay̆ex.

— Slyıxal, hoty oni i pyıtalisy skryıty, cto v gorah, ne slixkom daleko ot monastyırıa, naxli dva tela.

— Verno. Tretyevo y̆a ranila iz arbaleta. Prijala y̆evo k kamnıam, no on pryıgnul v reku, i y̆evo unes potok. Nadey̆usy, on ne vyıplyıl.

— Vyıplyıl. Y̆emu hvatilo sil, ctobyı minovaty ux̨ely̆a i vyıy̆ti v dolinyı Brobergera, k objityım mestam.

I, vidıa vopros v y̆ey̆e glazah, poy̆asnil:

— Y̆a naxel y̆evo kosti vozle odnoy̆ derevuxki. Mestnyıy̆e socli, cto mertveq — straj. Sobstvenno govorıa, imenno poetomu y̆a okazalsa v Dorc-gan-Toy̆ne i tepery siju pered toboy̆.

— Straj? — nedoumenno naklonilasy ona ko mne. — Kakovo certa oni tak podumali?

— U nevo byıl kinjal Gansa.

— Proklıatye! — Ona zakryıla liqo rukami i prostonala: — Proklıatye!

Povisla tixina, y̆a slyıxal lix y̆ey̆e preryıvistoy̆e dyıhaniy̆e. Kogda ona ubrala ruki, y̆ey̆e glaza byıli soverxenno suhimi i zlyımi.

— Tyı sdal kinjal v Bratstvo?

— Konecno.

Ona vzdohnula.

— Horoxo. — I, slovno ubejday̆a sebıa, dobavila: — Da. Horoxo. Tak budet lucxe. Dalyxe y̆a znay̆u, cto slucilosy. Tyı ne sdalsa, kak byıvalo i prejde. I naxel y̆evo?

— V ledıanoy̆ pex̨ere. Gluboko pod monastyırem.

— Kak on umer?

— Srajalsa do poslednevo i zabral s soboy̆ neskolykih. Dumay̆u, cto usnul. Ot holoda i poteri krovi.

A cto y̆a y̆ex̨e mog y̆ey̆ skazaty? Cto y̆evo zakololi, slovno zverıa? Kak zakololi Hartviga.

— Tyı pohoronil y̆evo? — proxeptala moy̆a byıvxay̆a naparniqa.

— Net. No y̆a uveren, cto tepery telo Gansa nikto ne pobespokoy̆it.

Y̆ego ne kosnutsa ni cervi, ni trupoy̆edyı iz inyıh sux̨estv, ni zlo, ni svet. On navecno ostanetsa vo mrake, poka svod pex̨eryı ne obvalitsa i ne prevratitsa v savan dlıa moy̆evo starovo druga.

Odinokay̆a slezinka pokatilasy po y̆ey̆e x̨eke, i Kristina pospexno, tocno styıdıasy, vyıterla y̆ey̆e tyılynoy̆ storonoy̆ ladoni.

— Spasibo.

— Za cto?

— Za to, cto naxel y̆evo. Za to, cto rasskazal mne. Za to, cto y̆a tepery znay̆u.

— No pocemu tyı ne sdelala etovo? Stolyko let, Krista. Myı vse tak dolgo y̆evo iskali, ne sdavalisy, verili. A tyı vse znala. Znala s samovo nacala, no ni certa nicevo ne skazala! Nikomu iz nas!

Y̆a cuvstvoval, kak holodnyıy̆ gnev prosyıpay̆etsa u menıa v grudi. On jil tam y̆ex̨e s oseni, s teh por kak y̆a ponıal, cto devuxka kak-to svıazana s Gansom i y̆evo isceznoveniy̆em.

— Budy moy̆a volıa, nicevo ne govorila byı i dalyxe.

— Pocemu?

— A cto byılo byı?! Cto byı togda slucilosy, Lıudvig?! — kriknula ona mne v liqo, razom terıay̆a vse svoy̆e spokoy̆stviy̆e. — Skaji mne! Tyı byı prinıal eto?! Otoxel byı proc?! Skazal byı: nu cto podelaty, raz takova y̆evo sudyba?! Kto iz teh, kovo myı znay̆em, otstupil?! Kto?!

Tepery slezyı lilisy iz y̆ey̆e glaz nepreryıvno, i ona ne stesnıalasy ih.

— Y̆a sama otvecu: nikto! Tyı, Gertruda, Lyvenok, Xuko, Rozi ne ostalisy byı v storone, brosilisy byı spasaty to, cto uje nelyzıa spasti, ili tovo huje — mstity. Kto iz nas obladal osobyım razumom desıaty let nazad? Vyı pogibli byı, kak i on. A y̆esli byı vmexalisy ne myı, y̆ediniqi, a qeloy̆e Bratstvo? Tolyko predstavy, Lıudvig, samyıy̆ seryeznyıy̆ konflikt s Qerkovy̆u za vsıu naxu istoriy̆u. Nas byı smıali i unictojili, y̆esli byı myı tolyko posmeli podnıaty na nih ruku!

Ona byıla prava, no y̆a vse ravno scital, cto y̆ey̆e molcaniy̆e slixkom jestoko dlıa teh, kto do sih por jil nadejdoy̆.

— Dumay̆ obo mne cto hocex, no, zaklinay̆u, sohrani tay̆nu. Ne stoy̆it nikomu znaty, cto Gans naxel smerty v monastyıre kalikveqev. Eto slixkom opasnay̆a informaqiy̆a.

— Tyı byıla ne vprave rexaty za drugih, Kristina. Kakiy̆e byı blagiy̆e namereniy̆a toboy̆ ni dvigali.

— Y̆a ni o cem ne jaley̆u.

Y̆a vzıal sebıa v ruki, otkinuvxisy na stul:

— Pocemu y̆evo ubili? Cto on hotel ot monahov?

— Ne znay̆u.

Ona vyıderjala moy̆ vzglıad, no y̆a lix vzdohnul:

— Eto loj.

— Pusty tak, — legko soglasilasy ona. — No pravda o pricinah smerti Gansa tepery nicevo ne izmenit. Vse davno zakoncilosy, Lıudvig. Vse v proxlom. Ostavy y̆evo, inace ono prosnetsa i ubyet tebıa.

— Tyı vedy menıa znay̆ex. Y̆a vse ravno dokopay̆usy do istinyı, pusty dlıa etovo potrebuy̆etsa y̆ex̨e desıaty let.

— Ne s moy̆ey̆ pomox̨y̆u. Prosti, no y̆a ne jelay̆u braty na sebıa otvetstvennosty za tvoy̆u smerty.

Nastay̆ivaty ne imelo smyısla, i y̆a otstupil.

— Horoxo. Zabudem o pricinah, pobudivxih Gansa otpravitsa v monastyıry. Rasskaji o tom, cto byılo dalyxe. Posle tovo kak tyı razobralasy s zakonnikami.

Ona vstala, zakryıla okno, y̆ejasy ot holoda.

— Cto tebıa interesuy̆et?

— Cernyıy̆ kinjal.

Kristina hmyıknula:

— Y̆a nacinay̆u dumaty, cto tyı ne Lıudvig, a dy̆avol.

— Obley̆ menıa osvıax̨ennoy̆ vodoy̆, y̆esli tebıa cto-to smux̨ay̆et, — predlojil y̆a y̆ey̆.

— K sojaleniy̆u, net pod rukoy̆, — nevolyno ulyıbnulasy ona. — Tyı prav. Takoy̆ kinjal u menıa byıl. Cto tyı znay̆ex o klinke?

— Tyı vladela im kakoy̆e-to vremıa, zatem y̆evo ukrali, on oby̆avilsa v Xossiy̆i i pricinil nemalo nepriy̆atnostey̆, poka myı s Miriam ne razobralisy s y̆evo vladelyqem.

— Kinjal u ney̆e?

— Unictojen v prisutstviy̆i knıazey̆ Qerkvi.

Pro vtoroy̆ klinok, tot, cto prinadlejal imperatoru Konstantinu, dobyıtyıy̆ mnoy̆ i Ranse v tay̆nike prejnevo Bratstva, y̆a upominaty ne stal.

— Y̆ex̨e cto-nibudy?

— Sux̨iy̆e meloci, Krista. Kinjal, kotoryıy̆ tyı vyıpustila v mir, delay̆et duxi temnyımi.

Ona byıla nicuty ne udivlena. Ni kapli.

— Y̆a rada tvoy̆im znaniy̆am. Myı sekonomim kucu vremeni, Lıudvig. Mne ne potrebuy̆etsa rasskazyıvaty tebe vse s samovo nacala i ubejdaty, cto eto pravda.

— Vse daje lucxe, cem y̆a rasscityıval, — razdalsa cuty nasmexlivyıy̆ golos u menıa za spinoy̆. — Mojno srazu pristupaty k delu.

Y̆a obernulsa i uvidel v dverıah pervovo pomox̨nika nyıne mertvovo markgrafa Valentina.

Koldun Valyter, s kotoryım myı rasstalisy pri ne samyıh lucxih obstoy̆atelystvah, s ulyıbkoy̆ prislonilsa k kosıaku:

— Dobrovo tebe denecka, van Normay̆enn.

Rasstoy̆aniy̆e do nevo y̆a preodolel za odno mgnoveniy̆e. Stul uletel v protivopolojnuy̆u casty komnatyı, a y̆a okazalsa pered nenavistnyım koldunom. On, kajetsa, ne ojidal ot menıa takih skorostey̆. Y̆a uvidel, kak zastyıvay̆et ulyıbka na y̆evo liqe, i pervyım je udarom kulaka slomal y̆emu nos. Broskom povalil na pol i, ne dumay̆a, cto v lıuboy̆ moment on mojet primenity magiy̆u, nacal delaty to, o cem mectal bolyxe goda.

Kristina s voplem povisla u menıa na plecah:

— Lıudvig! Ostavy y̆evo! Prekrati! Nu je!

Certa s dva y̆a sobiralsa y̆ey̆e sluxaty. No menıa i vqepivxuy̆usıa Kristu otbrosilo v storonu. Potolok neskolyko raz krutanulsa pered glazami, i y̆a ox̨util silynuy̆u toxnotu. Dernulsa, pyıtay̆asy vstaty i vernutsa k koldunu. Na etot raz y̆a sobiralsa vospolyzovatsa ne kulakami, a kinjalom, no i tut moy̆a byıvxay̆a naparniqa ne razjala palyqev, povisnuv na mne, kak laska na ohotnicyem pse.

— Uspokoy̆sıa, cert tebıa poderi! Stop! Hvatit! On drug! On moy̆ drug!



Valyter to i delo trogal palyqami razbityıy̆e gubyı, na kotoryıh zapeklasy krovy. Nos u nevo raspuh, levyıy̆ glaz zaplyıl, no koldun ne sobiralsa jdaty polojennyıh dney̆ do svoy̆evo vyızdorovleniy̆a. Sidel sebe v uglu da xeptal nagovoryı.

— Tyı v norme? — Kristina protıanula y̆emu vlajnuy̆u trıapiqu, i etot certov ublıudok s blagodarnosty̆u y̆ey̆e prinıal.

— Byıvalo i huje, — prognusavil on. — K zavtraxnemu dnıu zajivet.

Y̆a hotel u nevo sprosity, cto je on ne zajivil sebe xram, kotoryıy̆ y̆a ostavil, kogda kinul arbalet y̆emu v liqo, no sderjalsa.

— Shodi k Filippu. On mojet pomoc.

Valyter lix skrivil gubyı i tut je ob etom pojalel, tak kak nacala socitsa krovy.

— Proklıatyıy̆ deny! — rugnulsa on. — Y̆a lucxe sam. Bez y̆evo adskih pritirok i boltuxek. Zay̆misy svoy̆im vspyılycivyım kollegoy̆.

— Y̆a byı tebe y̆ex̨e dobavil, y̆esli byı ne ona, — mracno zametil y̆a.

— Ohotno verıu. Y̆a byı s radosty̆u vskipıatil tvoy̆i mozgi, y̆esli byı ne ona, — ozlobilsa on.

— Zatknitesy oba i sidite tiho! — vspyılila Kristina. — Konfliktyı proxlovo ostanutsa v proxlom!

Y̆a ne sobiralsa zabyıvaty zastenki markgrafa Valentina, to, kak y̆a byıl kukloy̆ dlıa bity̆a, i to, kak etot sidıax̨iy̆ v pıati y̆ardah ot menıa hmyıry y̆edva ne ukral kinjal Natana.

— Tyı ranyxe takim ne byıl… — Kristina ustalo opustilasy na stul peredo mnoy̆, perekryıv puty k koldunu.

— U nas staryıy̆e scetyı.

— Znay̆u y̆a o vaxih scetah. On rasskazal.

— Togda ne ponimay̆u tvoy̆evo udivleniy̆a. Y̆esli byı zdesy byıla Gertruda, ona byı uje razmazala y̆evo po stenke.

— Vyıhodit, y̆a legko otdelalsa. — Valyter vnovy popyıtalsa ulyıbnutsa, no vspomnil o gubah, i ulyıbka prevratilasy v oskal.

Ona tıajelo vzdohnula:

— Ladno. O kinjale. Posle tovo kak y̆a y̆evo naxla, rexila, cto zakonniki pridumali cto-to svoy̆e dlıa sbora dux. No s duxami kinjal ne rabotal. Y̆a ne smogla ponıaty, dlıa cevo on nujen, vozila s soboy̆ pocti polgoda.

— Odin celovek skazal mne, cto, kogda im dolgo vladey̆ex, nacinay̆ut proy̆ishodity nepriy̆atnosti. U tebıa takoy̆e byılo?

— Na strajey̆ pravilo ne rasprostranıay̆etsa, — vlez v razgovor Valyter. — Klinok nikak ne vliy̆ay̆et na teh, u kovo uje y̆esty kinjalyı. V ostalynom — sux̨ay̆a pravda. Vex̨ dovolyno opasnay̆a.

Kristina razdrajenno dernula plecom i prodoljila:

— Y̆a sdala y̆evo na hraneniy̆e v ``Fabyen Klemenz i syınovy̆a" i, sobstvenno govorıa, zabyıla o nem na kakoy̆e-to kolicestvo let. Vspomnila, lix kogda uvidela opisaniy̆e cernovo kamnıa iz knigi, cto lejala na stole u Miriam. Redkiy̆ foliant, horoxiy̆e risunki. Hagjitskiy̆ y̆a znay̆u dovolyno poverhnostno, no procitannovo hvatilo, ctobyı ponıaty — glaz serafima dostatocno redkay̆a i qennay̆a vex̨iqa.

— I tyı zabrala orujiy̆e. Day̆ dogaday̆usy — eto slucilosy v Barburge. I v etot je deny polucila dva udara nojom.

Kristina pereglıanulasy s Valyterom, i tot proronil:

— Govoril y̆a tebe, on y̆ex̨e tot umnik.

— Vse verno. Kak y̆a ponimay̆u, tebe rasskazal ob etom tot, kto pohitil klinok iz moy̆ey̆ sumki.

— Nu tyı doljna byıty y̆emu blagodarna. On spas tvoy̆u jizny, oplatil lekarıa i komnatu. Kinjal ne prines y̆emu nikakovo scasty̆a, i on izbavilsa ot nevo. Otdal celoveku, kotorovo myı poy̆mali v Xossiy̆i. Kto te lıudi, cto napali na tebıa?

— Ne imey̆u ponıatiy̆a. Y̆a podozrevay̆u, cto oni nay̆emniki Ordena. On, — kivok v storonu kolduna, — scitay̆et, cto storonniki celoveka, sozdavxevo kinjal.

— Interesno, — s somneniy̆em protıanul y̆a.

— Cto ne tak? — Ona prekrasno cuvstvovala, kogda menıa smux̨ay̆ut faktyı.

— Na koy̆ cert eto Ordenu? Da i kak oni voobx̨e uznali? Tyı vedy ne begala po uliqam i ne razmahivala takim orujiy̆em napravo i nalevo. Troy̆e zakonnikov, kotoryıh tyı vstretila v gorah, mertvyı. Kalikveqi, y̆esli byı oni znali tvoy̆e imıa ili scitali, cto straj vyıjila, dostali byı tebıa iz-pod zemli i davno uje prikoncili. Dlıa nih tyı — vsevo lix bezyımıannay̆a jenx̨ina, kotoruy̆u v liqo videl tolyko pogibxiy̆ monah-privratnik. Myı vozvrax̨ay̆emsıa k samyım legkim iz moy̆ih voprosov: kak oni uznali tvoy̆e imıa, raz tyı nikomu y̆evo ne govorila? pocemu ponıali, cto kinjal u tebıa? otkuda dogadalisy, v kakom otdeleniy̆i ``Fabyen Klemenz i syınovy̆a" tyı y̆evo zaberex i v kakoy̆ deny, y̆esli napali srazu je posle etovo?

— Tvoy̆i predpolojeniy̆a? — Valyter byıl tak lıubezen, cto pozvolil mne vyıskazatsa.

— Kto napal — bez ponıatiy̆a. O tom, kak naxli, — Kristina ostavila sledyı. Zadela kolokolycik, kotoryıy̆ uslyıxali ne te uxi. No ona utverjday̆et, cto ni s kem ne govorila ni o sobyıtiy̆ah v gorah, ni o temnom kinjale.

— Eto tak, — podtverdila devuxka. — No y̆a zadavala voprosyı o glazah serafima. Spraxivala u kollekqionerov kamney̆ i u hagjitskih torgovqev.

— Vozmojno, kto-to iskal to je samoy̆e, cto i tyı, i zay̆interesovalsa celovekom, kotoryıy̆ proy̆avlıay̆et lıubopyıtstvo v stoly speqificeskoy̆ oblasti.

— No bolyxe nikto ne pyıtalsa napasty na tebıa posle tovo slucay̆a. — Valyter rabotal nad svoy̆im nosom, provodıa siy̆ay̆ux̨imi palyqami i postepenno snimay̆a otek.

— Kakoy̆ smyısl? Y̆a perestala byıty interesna. U menıa bolyxe ne byılo kinjala.

— No tyı vse ravno slixkom mnogo znala, — ulyıbnulsa y̆a. — Licno y̆a byı zaverxil delo, ctobyı celovek ne sozdaval problemyı.

— A tyı izmenilsa. — Kristina vnimatelyno posmotrela na menıa, zatem neohotno kivnula. — Y̆a byı postupila tocno tak je. Raz uj tyı menıa razyıskal, nesmotrıa na to cto y̆a skryıvay̆usy, to i ubiy̆qi mogli. Dva goda — bolyxoy̆ srok.

Valyter smotrel na menıa neotryıvno. Y̆a znal, cevo on boy̆itsa, i proy̆iznes to, cto uje davno sidelo u menıa v golove:

— Ostavity tebıa jivoy̆ mojno byılo lix po odnoy̆ pricine — eto komu-to vyıgodno. K primeru, tyı mojex privesti k klinku. Ili je y̆ex̨e kak-to pomoc. Vot, dopustim, tvoy̆ drug. On vpolne mog nanıaty lıudey̆, a zatem, kogda u nih nicevo ne vyıxlo, vteretsa k tebe v doveriy̆e i vsegda nahoditsa poblizosti.

Byıvxiy̆ sluga markgrafa Valentina rassmey̆alsa i podnıalsa so svoy̆evo mesta:

— Pojaluy̆, y̆a poy̆du shoju k Filippu. Inace y̆a vse-taki kovo-nibudy v samom dele uby̆u.

On vyıxel, a y̆a, dojdavxisy, kogda y̆evo xagi stihnut na lestniqe, vstal. Raspahnul dvery, proverıay̆a, dey̆stvitelyno li myı ostalisy odni.

Kristina sidela s neproniqay̆emyım liqom, no y̆a videl, kak v y̆ey̆e temnyıh glazah buxuy̆et burıa.

— Kak davno tyı y̆evo znay̆ex?

— S teh por, kak menıa y̆edva ne ubili. Kogda y̆a prixla v sebıa, on byıl rıadom.

Y̆a neveselo hohotnul:

— Oceny udobno. I vpisyıvay̆etsa v moy̆u teoriy̆u. Zabotlivo okazatsa podle posteli ranenoy̆ v tot moment, kogda nujno, raz uj ne udalosy polucity klinok.

Ona ne jelala verity:

— Eto vsevo lix teoriy̆a, Sineglazyıy̆. U tebıa net nikakih dokazatelystv, vprocem, kak i u menıa.

— Tyı ne slixkom dovercivay̆a natura, Krista. Otcevo je poverila prohodimqu?

Devuxka dopila vino, podumala:

— Krome tovo cto on neskolyko raz spasal moy̆u jizny, Valyter oceny ubeditelen. Y̆emu nujna pomox̨ straja. I y̆a verıu v y̆evo istoriy̆u. Myı stoy̆im na grani katastrofyı, Lıudvig. Do propasti, v kotoroy̆ buxuy̆et plamıa, vsevo odin xag. No nikto iz lıudey̆ daje ne podozrevay̆et ob etom.

V komnate byılo duxno, i y̆a rasstegnul vorot rubahi.

— Katastrofyı slucay̆utsa y̆ejegodno. Y̆esli ne epidemiy̆a cumyı, tak y̆ustirskiy̆ pot. Y̆esli ne ocerednay̆a y̆ereticeskay̆a sekta, risuy̆ux̨ay̆a na gravıurah Papu s kozlinyımi nogami, tak voy̆na. Celovek, sozday̆ux̨iy̆ kinjalyı, otravlıay̆ux̨iy̆e duxi, bez somneniy̆a, opasen. No ne slixkom li rano myı kricim ``apokalipsis!"?

— Etot nekto ruxit osnovyı, Lıudvig. On certovski talantlivyıy̆ master, no ispolyzuy̆et svoy̆ talant vo zlo. To, cto on delay̆et, nepravilyno. Valyter lovit kuzneqa uje ne pervyıy̆ god.

— Tvoy̆ koldun lovit ne tolyko y̆evo, no i strajey̆. On ubiy̆qa. Takih, kak myı s toboy̆.

— Y̆a znay̆u.

— No ostay̆exysa s nim?

Kristina uprıamo sjala gubyı.

— I dalyxe budu.

— Nesmotrıa na smerty teh, kto byıl tebe dorog?

Ona s sojaleniy̆em opustila golovu, no otvetila tverdo:

— Y̆esli kuzneq prodoljit, umret y̆ex̨e bolyxe takih, kak myı. Vse Bratstvo. A Valyter i y̆evo lıudi sey̆cas y̆edinstvennyıy̆e, kto mojet nay̆ti i ostanovity temnovo mastera. Vse oceny seryezno, Lıudvig. Valyter pokazyıval mne staryıy̆e manuskriptyı vremen Konstantina. Togda sux̨estvovalo lix dva takih klinka. Govorıat, ih dostavili s vostoka, s samoy̆ graniqi objityıh zemely. Znay̆ex, pocemu imperator sozdal strajey̆? Iz-za proklıatyıh kinjalov. On jelal jity vecno, a dlıa etovo y̆emu trebovalisy duxi.

Y̆a kivnul:

— Uje dumal ob etom. S pomox̨y̆u cernyıh klinkov on ubival lıudey̆, temnil ih duxi. A zatem zabiral svetlyım, dobavlıay̆a sebe jizny.

— Y̆a ne znay̆u, kogda eto nacalosy. Pocti vse svidetelystva tovo vremeni unictojenyı. No sobyıtiy̆a svıazyıvay̆u s tem momentom, kogda iz zemely hagjitov na nax materik priy̆ehala semy̆a svetlyıh kuzneqov. Kak govorıat legendyı, oni — potomki odnovo iz ucenikov Christa. Oni stali kovaty kinjalyı s sapfirami, no net ni odnovo podtverjdeniy̆a, cto temnyıy̆e klinki — ih ruk delo. Qerkovy vzıala masterov pod svoy̆u zax̨itu i na protıajeniy̆i mnogih pokoleniy̆ ih oberegala. Imperiy̆a Konstantina rosla, kak i y̆evo vlasty. A jizny dlilasy i dlilasy. No temnyıy̆e kinjalyı privlekali na materik temnyıy̆e duxi. Razumey̆etsa, te zarojdalisy i ranyxe. Grehi i prestupleniy̆a nikto ne otmenıal. No, po utverjdeniy̆am istorikov, ranyxe ih byılo gorazdo menyxe, cem posle tovo, kak imperator stal obmanyıvaty smerty. Poetomu y̆emu i potrebovalisy takiy̆e, kak myı, — ocix̨aty zemli ot y̆evo oxibok.

Nicevo udivitelynovo y̆a dlıa sebıa ne uznal:

— Konstantin davno prevratilsa v prah. Kinjalyı y̆emu ne pomogli. Ne pomogut i tomu, kto delay̆et ih sey̆cas. Cevo tyı boy̆ixysa? Naxestviy̆a zlyıh sux̨nostey̆? Myı spravimsıa. Volnyı temnyıh dux zahlestyıvali stranyı i ranyxe, no Bratstvo vsegda ih pobejdalo.

— Y̆a ne etovo opasay̆usy, Lıudvig. Menıa pugay̆et necto inoy̆e. Tyı znay̆ex, cto Konstantin otpravil vosemy ekspediqiy̆ na vostok, nadey̆asy razdobyıty y̆ex̨e podobnovo orujiy̆a?

— Zapaslivyıy̆ sukin syın, — nevolyno voshitilsa y̆a. — U nevo vedy ne polucilosy?

— Nikto ne vernulsa s kray̆a objityıh zemely. No Konstantin prodoljal iskaty i scital, cto y̆esli sobraty desıaty temnyıh klinkov, to oni stanut klıucom… — Ona sdelala pauzu, vnimatelyno nablıuday̆a za moy̆im liqom. — Klıucom dlıa tovo, ctobyı otkryıty adskiy̆e vrata.

Mne potrebovalosy neskolyko sekund, ctobyı perevarity y̆ey̆e slova i rassmey̆atsa:

— Potrıasay̆ux̨e! Eto Valyter tebe skazal? I tyı y̆emu verix?!

— Verıu, Lıudvig.

— Otkryıty dorogu v ad. Tak ne byıvay̆et.

— A byıvay̆et, cto kinjal prevrax̨ay̆et cistuy̆u duxu, na kotoroy̆ net grehov, v temnuy̆u sux̨nosty? — privela Kristina argument. — Skaji y̆a tebe takoy̆e god nazad, tyı byı mne poveril? Ili vot tak je smey̆alsa?

— Ne poveril byı, — prixlosy priznaty mne. — Znacit, desıaty cernyıh kinjalov otkroy̆ut vrata v ad? Skaji, pojaluy̆sta, kak Valyter otzyıvay̆etsa ob umstvennyıh sposobnostıah velikovo Konstantina? Na koy̆ cert tomu iskaty stoly slojnyıy̆ sposob samoubiy̆stva? Y̆esli gde-nibudy v Fringbou poy̆avitsa predstavitelystvo ada, to ploho budet ne v odnom gorode, a vo mnogih stranah. Legionyı demonov, sukkubov, certey̆, adskoy̆e plamıa, sera s nebes i prociy̆e vex̨i. Ne scitay̆a gibeli tyısıac lıudey̆. Eto nemnogo nerazumno. Daje dlıa Konstantina. Ne nahodix?

— Obıazatelyno iskaty logiku, Lıudvig?

— Y̆esli hocex ponıaty motivyı drugovo celoveka? Da. Obıazatelyno. Osobenno kogda ne verix na slovo koldunu, ohotivxemusıa za klinkami strajey̆.

— Ad toje mojet daty silu i vlasty. I Konstantin scital, cto, raz etovo ne day̆ut y̆emu nebesa, nesmotrıa na to cto on prinıal novuy̆u religiy̆u, otkazavxisy ot y̆azyıceskih bogov, sleduy̆et zaklıucity inuy̆u sdelku. Sozdaty prohod dlıa teh, komu popasty v nax mir ne tak-to prosto. On scital, cto priobretet gorazdo bolyxe, cem poterıay̆et.

Y̆a lix razvel rukami:

— I o takom otkroveniy̆i znay̆et tolyko nax obx̨iy̆ drug?

— Net. V Riapano eto toje izvestno. Poetomu dva klinka Konstantina unictojenyı uje oceny davno.

Konecno. Tolyko odin. Vtoroy̆ lejit v sumke naxey̆ obx̨ey̆ ucitelyniqi.

No skazal y̆a sovsem drugoy̆e:

— Polojim, vse tak, kak tyı govorix. Poraduy̆emsıa, cto u Konstantina nicevo ne vyıxlo. No tepery v mire poy̆avilsa y̆ex̨e odin bezumeq, kotoromu ne k cemu prilojity ruki, poetomu on sozday̆et orujiy̆e, boley̆e opasnoy̆e, cem hagjitskay̆a pescanay̆a kobra. No ne hranit y̆evo. I ne sobiray̆et, a vyıbrasyıvay̆et v narod. Inace byı myı s toboy̆ ne uvideli ni odnovo takovo klinka. Sledovatelyno, on y̆avno ne jelay̆et otkryıvaty nikakih mificeskih vrat. Tak?

— Net. Ne tak. Myı podozrevay̆em, cto tot, cto byıl u menıa, okazalsa vsevo lix probnyım ekzemplıarom. Y̆emu nado byılo uznaty, rabotay̆et li kinjal.

— I poetomu orujiy̆e kakim-to obrazom poy̆avilosy u predstavitelıa Ordena? — Y̆a byıl polon skeptiqizma.

— Pocemu byı i net? Oni speqialistyı v takih voprosah. Raz proverıay̆ut naxi klinki, vidıat istoriy̆u sobrannyıh dux, to, vozmojno, on nadey̆alsa, cto dadut oqenku i y̆evo rabote.

— Y̆esty dva ``no". Y̆a ne lıublıu Orden, no delo oni znay̆ut. Poy̆avisy sredi nih podobnyıy̆ celovek, oni y̆avno byı ne okazyıvali y̆emu uslugi, a potax̨ili k sebe v podvalyı. Ili je srazu prikoncili.

— A mojet, kuzneq zaklıucil sdelku tolyko s odnim iz nih. S samyım necistoplotnyım, — vesko vozrazila ona. — A tvoy̆e vtoroy̆e ``no"?

— Y̆ex̨e odin temnyıy̆ kinjal putexestvoval po miru bez svoy̆evo sozdatelıa. Klinok naxel odin inkvizitor v sumke gonqa, oderjimovo besom. Goneq umer, tak nicevo i ne uspev rasskazaty. A klinok kliriki unictojili.

— Tyı uveren v etoy̆ informaqiy̆i?

— Y̆a vmeste s Gertrudoy̆ videl oblomki orujiy̆a u kardinala di Travinno. V tvoy̆ey̆ teoriy̆i, cto kuzneq rexil proverity, rabotay̆et li y̆evo tvoreniy̆e, togda kak ostalynyıy̆e klinki on derjit pri sebe, y̆esty nekotoray̆a nesoglasovannosty. Proverka kinjala — zvucit kray̆ne natıanuto. On mog eto sdelaty i sam, raz umey̆et sozdavaty takiy̆e vex̨i. Y̆a dumay̆u, y̆edinstvennay̆a pricina, pocemu kuzneq mog otdaty temnyıy̆ klinok komu-to iz Ordena, — eto plata. Plata za pomox̨ i sotrudnicestvo.

No Kristina scitala inace:

— Otnıudy. Teoriy̆a lix ukrepilasy s tvoy̆im rasskazom. Podumay̆ sam. Pervoy̆e orujiy̆e ne doxlo do adresata. Y̆ego perehvatili i unictojili kliriki. Poetomu poy̆avilsa tot, vtoroy̆, v itoge popavxiy̆ mne v ruki.

Versiy̆a zvucala cuty boley̆e skladno, cem predyıdux̨ay̆a. No ne namnovo. Y̆a pomnil, kak v Riapano govorili o tom, cto oteq Mart naxel klinok tri goda nazad. A Kristina zapolucila svoy̆ na semy let ranyxe, sledovatelyno, y̆ey̆e klinok nikak ne mog byıty vtoryım.

— Valyter tak uveren v podobnom razvitiy̆i sobyıtiy̆? — Po moy̆emu tonu byılo ponıatno, naskolyko silyno y̆a ``qenıu" mneniy̆e kolduna.

Ona podnıala vverh ladoni:

— Sluxay̆. On neobyıcnyıy̆ celovek. I jestokiy̆. V drugoy̆e vremıa y̆a byı ubila y̆evo ne zadumyıvay̆asy. Za vse zlo, cto on pricinil Bratstvu. Da net! K certu Bratstvo! Za vse to, cto on sdelal tebe. No, kak y̆a govorila, situaqiy̆a oceny silyno izmenilasy. Y̆a mnogoy̆e uznala za paru poslednih let, i moy̆e otnoxeniy̆e k jizni perevernulosy. On — menyxey̆e zlo i mojet spasti vseh nas.

— Zlo ne mojet byıty malenykim ili bolyxim. Zlo ostay̆etsa zlom. Y̆a podhoju k slucivxemusıa bez emoqiy̆. Vo vsıakom slucay̆e, sey̆cas. On ne smog otsledity zakonnikov, no naxel tebıa. Ugaday̆, kuda spexil goneq, ubityıy̆ inkvizitorom? V zamok Latka, vladelyqem kotorovo byıl markgraf Valentin, a y̆emu slujil tvoy̆ razlıubeznyıy̆ koldun. Vse ukazyıvay̆et na to, cto on iskal temnyıy̆ kinjal.

— Y̆a toje y̆evo ix̨u, kak i drugiy̆e lıudi Valytera. Myı nadey̆emsıa, cto artefakt privedet nas k kuznequ. — Ona podalasy vpered, nakryıla moy̆u ruku svoy̆ey̆, vkradcivo skazav: — Sluxay̆. Skorey̆e vsevo, tyı prav. Za tem napadeniy̆em dey̆stvitelyno mog stoy̆aty on. Slixkom mnogo sovpadeniy̆. No eto nicevo ne menıay̆et, Lıudvig. Y̆a nujna y̆emu, a koldun nujen mne. U nas odna qely. I drug bez druga myı ne oboy̆demsıa. Y̆a mnogim pojertvovala radi tovo, ctobyı popyıtatsa nay̆ti temnovo mastera. Slixkom mnogim. I otstupaty sey̆cas… Povery, y̆a prosto ne mogu tak postupity.

— On opasen, Kristina.

— Y̆a eto znay̆u lucxe, cem tyı. U nevo tyısıaca i odin nedostatok, no daje takoy̆ celovek, kak on, mojet spasti nax mir.

Eto byılo tak smexno slyıxaty. Valyter — spasitely celovecestva. Po mne, tak eto mir sleduy̆et izbavlıaty ot nevo.

— Tyı iscezla i ne podavala o sebe vestey̆ pocti god. — Y̆a smenil temu. — Myı volnovalisy za tebıa.

Ona otvela glaza, skazav tihim golosom:

— Prosti. U menıa ne byılo vyıbora. Proxloy̆ vesnoy̆ myı s Valyterom vlipli v nepriy̆atnosti, kogda sbili so sleda kuzneqa klirikov. Uverena, v otliciy̆e ot nas oni ne hotıat y̆evo ubivaty. Poy̆mav kuzneqa, Riapano polucit v svoy̆i ruki ogromnuy̆u vlasty. Poetomu tyı ponimay̆ex, kak vajno nam nay̆ti y̆evo pervyımi?

— Znacit, vyı hotite ubity zagadocnovo mastera?

Ona gorıaco kivnula:

— Samyım byıstryım sposobom iz vseh vozmojnyıh. Ctobyı nikto ne uznal y̆evo sekretov. Oni doljnyı umerety vmeste s nim.

— A svıax̨enniki? Vdrug vyı oxibay̆etesy, i u nih takay̆a je qely, kak i u vas, — ubity y̆evo.

— Ne oxibay̆emsıa, — bezapellıaqionno zay̆avila ona. — Kak tolyko oni ponıali, cto myı toje ix̨em y̆evo, poslali za nami svoy̆ih ubiy̆q. Povery, Lıudvig, eto byılo straxno. Pıatero iz naxevo otrıada pogibli. Myı nasilu uxli i vot uje kotoryıy̆ mesıaq skryıvay̆emsıa. I y̆a ne mogu vernutsa v Ardenau. Daje pisymo napisaty komu-libo iz strajey̆ ne mogu. Ono podstavit pod udar lıubovo.

— Organist v qerkvi, — ulyıbnulsa y̆a. — Tyı ne slixkom-to horoxo skryıvay̆exysa.

— Lucxe prıatatsa ot sobaki v y̆ey̆e je budke. Tam ona budet iskaty v poslednıuy̆u oceredy. Nikto ne smotrit na muzyıkantov.

— Ubegaty vecno ne polucitsa.

— Y̆a i ne stanu. Nujno lix unictojity zlo. Vse ostalynoy̆e nevajno.

Y̆a videl, cto ona oderjima idey̆ey̆ nay̆ti celoveka, kuy̆ux̨evo temnyıy̆e kinjalyı, tocno tak je, kak Miriam vot uje vek ne day̆ut pokoy̆a poy̆iski kuzneqa, sozday̆ux̨evo klinki dlıa Bratstva. Y̆a ponimal, cto ne otgovorıu y̆ey̆e, cto spority bessmyıslenno i to, cto, priy̆ehav sıuda cerez neskolyko stran, y̆a uznal, cto ona jiva i o proy̆izoxedxem s ney̆ i Gansom, eto i y̆esty vse, cevo y̆a dostig.

— Viju, cto jelaniy̆e spasti mir veliko.

— Mir? — Ona izognula brovy. — Plevaty y̆a hotela na nevo. Y̆a spasay̆u ne mir, a Bratstvo. Zax̨ix̨ay̆u y̆evo v meru otpux̨ennyıh sil i umeniy̆a.

— Tolyko v Bratstve ob etom ne podozrevay̆ut.

— I horoxo. Menyxe problem i mne, i im.

— Myı sux̨estvuy̆em pocti poltoryı tyısıaci let. U nas slucalisy raznyıy̆e nepriy̆atnosti, no Bratstvo vsegda vyıjivalo i ostavalosy na nogah. Ostanetsa i vpredy. Tebe nezacem skladyıvaty golovu lix radi nepodtverjdennyıh slov kolduna. Poy̆ehali v Ardenau. Prıamo sey̆cas. Bratstvo dogovoritsa nascet tebıa s Riapano. Gertruda pomojet. A di Travinno tolyko poraduy̆etsa informaqiy̆i, kotoray̆a tebe izvestna. Myı zax̨itim tebıa.

Kristina grustno rassmey̆alasy:

— Mne otradno znaty, cto tyı do sih por pyıtay̆exysa spasti moy̆u golovu, Lıudvig. Y̆a byı oceny hotela, ctobyı vse byılo kak dvenadqaty let nazad, kogda myı plecom k plecu otrajali natisk temnyıh dux. No mne uje kajetsa, cto naxa y̆unosty ne boley̆e cem mif, kotoryıy̆ y̆a sama sebe pridumala. A sey̆cas vokrug menıa realynosty, i ona oceny straxna. Tyı prosto poka ne mojex oqenity tovo ujasa, kotoryıy̆ ispyıtyıvay̆u y̆a, ponıaty vsey̆ seryeznosti problemyı. I soverxay̆ex tu je samuy̆u oxibku, kak togda s tem kartografom.

V golove u menıa trevojno zvıaknulo.

— Tyı konecno je vse znay̆ex.

— Znay̆u. Vedy ispravlıaty tvoy̆i oxibki prixlosy mne.

Y̆a prix̨urilsa:

— Znacit, vot kto ubil y̆evo.

Ona daje ne pyıtalasy otriqaty:

— A tyı ostavil mne vyıbor, kogda proy̆avil jalosty? Tebe nado byılo privezti y̆evo v Bratstvo, a ne otpuskaty na vse cetyıre storonyı. Togda byı Miriam ne prosila menıa spasaty situaqiy̆u!

Tepery ponimay̆u, pocemu Gertruda nicevo mne ne rasskazala. Y̆a pokacal golovoy̆:

— Tyı govorila, cto y̆a izmenilsa. Tyı izmenilasy ne menyxe menıa. I y̆a sojaley̆u ob etom. Prejnıay̆a Kristina nikogda byı ne stala ubiy̆qey̆ na pobeguxkah u magistrov.

Y̆ee liqo iskazilosy ot obidyı, i ona proxipela:

— Tyı dey̆stvitelyno tak i ne ponıal, cto togda proy̆izoxlo?! Iz-za cevo oni stoy̆ali na uxah?

— Ponıal. Ispugalisy novovo messiy̆i i tovo, cto on mojet naucity drugih lıudey̆ snimaty grehi s lıudskih dux. V perspektive Bratstvo stalo byı nikomu ne nujno.

Ona dvajdyı bezzvucno hlopnula v ladoxi:

— Potrıasay̆ux̨e! Tyı uvidel myıx, no ne zametil koxku. Nikto iz teh, kto byıl v kurse situaqiy̆i, ne boy̆alsa dalekovo budux̨evo. Myı opasalisy nastoy̆ax̨evo. A ono takovo: kartograf po kakoy̆-to nasmexke sudybyı mog ocix̨aty duxi strajey̆. No on zabiral ne tolyko naxi grehi, no i nax dar. Myı stanovilisy obyıcnyımi lıudymi, takimi, kak tyısıaci drugih obyıvateley̆, Lıudvig. Lixivxisy dara, myı ne mogli delaty svoy̆u rabotu. A tepery tolyko predstavy, cto byı byılo, y̆esli byı y̆evo, k primeru, zahvatil Orden? I ispolyzoval protiv nas. A y̆esli byı kartograf naucil kovo-to i poy̆avilosy neskolyko takih lıudey̆? Desıatok? Sotnıa? Armiy̆a! Myı byıli byı unictojenyı, slovno gorod, v kotoryıy̆ popal celovek, zarajennyıy̆ y̆ustirskim potom.

— I mnogih li strajey̆ Hartvig lixil ih rabotyı?

— Slava bogu — ni odnovo.

— Togda prosti, no tvoy̆i slova ne boley̆e cem nelepay̆a fantaziy̆a.

— Odna takay̆a fantaziy̆a slujila poslednemu potomku imperatora Konstantina. I kogda Bratstvo ne vernulo y̆emu kinjal, na kotoryıy̆ pretendoval koroly Progansu, v delo vstupil takoy̆ je, kak tvoy̆ Hartvig. Cetvero magistrov i opomnitsa ne uspeli, kak lixilisy svoy̆evo dara. Togda y̆evo ubili vmeste s korolem, i zavertelasy vsıa eta kaxa. Tepery, pomnıa o proxlom, Bratstvo ne stalo jdaty nacala epidemiy̆i, a unictojilo bolynovo do y̆evo poy̆avleniy̆a v gorode. — Ona s vyızovom posmotrela na menıa. — Scitay̆ex, cto v Ardenau oxiblisy?

Y̆a vzdohnul, vstal iz-za stola, tak i ne pritronuvxisy k bokalu vina:

— Ne imey̆et smyısla terıaty vremıa na sporyı. Magistryı uvideli opasnosty. Realynuy̆u ili mnimuy̆u, ne mne sudity. No celovek, kotoryıy̆ mog sdelaty mir cuty lucxe, mertv. I mne jaly tovo, cto tepery nikogda ne slucitsa.

— Boy̆usy, y̆a ne smogu tebıa ponıaty, Lıudvig. Myı stali slixkom raznyımi, — s grusty̆u skazala ona. — Kogda tyı uy̆ezjay̆ex iz goroda?

— Y̆ex̨e ne rexil, — cestno otvetil y̆a.

— Ostanysa. Mne nujna tvoy̆a pomox̨, i tyı y̆edinstvennyıy̆, komu y̆a mogu doverıaty zdesy. Obex̨ay̆, cto primex vzvexennoy̆e rexeniy̆e.

Y̆a posmotrel v y̆ey̆e glaza i vopreki svoy̆emu jelaniy̆u otkazaty kivnul…

Propovednik sidel na pervom etaje, v apteke. On diplomaticno ne stal sluxaty nax razgovor s Kristinoy̆ i sey̆cas zanimalsa tem, cto s nenavisty̆u posyılal proklıatiy̆e za proklıatiy̆em na golovu Valytera, kotoryıy̆, ne zamecay̆a svetloy̆ duxi, negromko besedoval s sedoborodyım aptekarem.

— Van Normay̆enn. — Koldun vstal, zakryıvay̆a mne vyıhod. — Myı ploho nacali. Byıty mojet, sey̆cas samoy̆e vremıa vse ispravity?

— Ugolyev tebe nado v glaza napihaty, dy̆avolyskoy̆e otrodye! — besnovalsa Propovednik.

— Ne dumay̆u, — holodno proy̆iznes y̆a.

— Nu hotıa byı na vremıa. Ctobyı ne rasstray̆ivaty cudesnuy̆u Kristinu.

Y̆a xagnul k nemu navstrecu i skazal tak tiho, ctobyı slyıxal tolyko on:

— Y̆a ne verıu ni odnomu tvoy̆emu slovu. I tyı jiv tolyko potomu, cto ona menıa ostanovila. Poetomu s dorogi. Poka y̆a ne ubil tebıa za to, cto tyı delal so strajami.

On usmehnulsa, sdelal xag v storonu i, kogda y̆a uje vyıhodil, kriknul mne v spinu:

— Podumay̆ o tom, cto y̆a skazal! Adskiy̆e vrata! V odnom iz naxih gorodov. I kogda oni otkroy̆utsa, poblizosti ne okajetsa angelov, kotoryıy̆e steregut pokoy̆ celovecestva na vostoke. Myı budem predostavlenyı sami sebe!



Y̆a pisal byıstro, to i delo okunay̆a pero v cernilyniqu, i Propovednik, uznavxiy̆ osnovnoy̆e soderjaniy̆e naxevo s Kristinoy̆ razgovora, poy̆interesovalsa:

— Dlıa cevo vse eto?

— Konkretiziruy̆, — poprosil y̆a y̆evo.

— Pisymo. Zacem ono?

— Potomu cto sozdalasy opasnay̆a situaqiy̆a. I y̆esli so mnoy̆ cto-to slucitsa, hoty kto-to doljen byıty v kurse tovo, cto zdesy proy̆ishodit.

— Gertruda ne budet scastliva.

Y̆a podnıal na nevo vzglıad:

— Nadey̆usy, ona nicevo ne uznay̆et. Ne jelay̆u vputyıvaty y̆ey̆e v eto.

— Togda komu je tyı pixex?

— Miriam. Bratstvo doljno byıty gotovo k nepriy̆atnostıam, y̆esli kliriki rexat sprosity o Kristine i y̆ey̆e delah.

On pomolcal, sluxay̆a, kak skripit pero:

— Etot Valyter, on kak bexenay̆a sobaka. Y̆ego nado ubity.

— Priy̆atno znaty, cto myı shodimsıa vo mneniy̆ah i ne dumay̆em o bibley̆skih zapovedıah. — Y̆a dal cernilam vyısohnuty. — No beda v tom, cto u nevo y̆esty informaqiy̆a. O tom je temnom kuzneqe.

— Tyı ne dumal, cto on vret?

— Nascet adskih vrat? Vpolne vozmojno. No odno ne otmenıay̆et drugovo. Pohoje, on dey̆stvitelyno ix̨et temnovo mastera. I y̆esli ne dlıa ubiy̆stva, to dlıa svoy̆ih qeley̆. Ili cyih-to y̆ex̨e. Kristina uverena, cto oni pocti naxli kuzneqa. Razumno byılo byı nahoditsa rıadom s nimi.

— Tvoy̆a byıvxay̆a naparniqa — sumasxedxay̆a.

— Ona byı ne vvıazalasy v etu istoriy̆u, y̆esli byı ne verila v to, cto govorit.

Y̆a slojil bumagu, ubral y̆ey̆e v konvert.

— Kak Kristina lixilasy palyqev?

— Kogda nacala rabotaty odna, — neohotno otvetil y̆a.

— To y̆esty bez tebıa?

— Da.

On ponıal, cto y̆a ne jelay̆u prodoljaty etu temu:

— Y̆a shodil glıanul na mesto, gde y̆akobyı byıl angel. Tolpix̨a, kak pered ray̆skimi vratami. Soldatyı, zevaki, molıax̨iy̆esıa, svıax̨enniki. Vsıa eta lıudskay̆a massa kricit, gudit, oret, poy̆et i y̆edva li ne lay̆et. I vse radi odnovo otpecatka bosoy̆ nogi, ostavxegosıa na bruscatke.

— I cem je on neobyıcen?

— Tem, cto vplavlen v cernyıy̆ kameny, a sam oslepitelyno-bel. I govorıat, cto blagouhay̆et jasminom.

— Jasmin v konqe zimyı — eto pohoje na cudo. — Y̆a zapecatal konvert alyım surgucom.

— I za pravo prikosnutsa gubami k etomu cudu derutsa. A nekotoryıy̆e proday̆ut svoy̆e mesto v oceredi za desıaty dukatov.

— Byılo byı stranno, y̆esli byı zabyıli o najive, — melanholicno proy̆iznes y̆a.

Myı, lıudi, vsegda nay̆dem cto prodaty i cto kupity. Y̆edu, zemli, titulyı, zvaniy̆a, svıatyıy̆e mox̨i ili mesto poblije k ray̆skim vratam.

— Segodnıa y̆a vozblagodaril Gospoda, cto umer. Pravo, budy y̆a jiv, certa s dva smog byı dobratsa do relikviy̆i i uvidety svıatuy̆u Djuliy̆u.

Y̆a ubral pisymo za golenix̨e sapoga:

— A eto y̆ex̨e kto?

— Slepay̆a devocka, s kotoroy̆ govoril angel.

— Cto, uje byıla kanonizaqiy̆a? — s ironiy̆ey̆ proy̆iznes y̆a, i tak znay̆a otvet.

K liku svıatyıh pricislıali ne ranyxe cem cerez pıaty let posle smerti pretendenta, i ctobyı dostic stoly vyısokovo zvaniy̆a — slov o tom, cto govoril s angelom, nedostatocno.

— Konecno net. Prosto tak y̆ey̆e nazyıvay̆et narod.

— Narod… — Y̆a vyıxel na uliqu, takuy̆u je xumnuy̆u, kak i ranyxe. — V takih voprosah vajno to, cto govorit ne narod, a knıazy̆a qerkvi.

Tut on konecno je ne sporil. Vidno, kak i y̆a, vspomnil predyıdux̨evo knıazıa Lezerberga, prosivxevo, ctobyı v Riapano priznali y̆evo matuxku svıatoy̆. Takay̆a blaj stoy̆ila y̆emu pocti semysot tyısıac dukatov, a razrazivxiy̆sıa skandal privel k smerti desıatka podkuplennyıh kardinalov, reformaqiy̆i Qerkvi i poy̆avleniy̆u protestnyıh dvijeniy̆ v Vitilyska, trebovavxih lixity Riapano privilegiy̆, a vseh Pap priznaty ``ne namestnikami Boga na zemle, a vsevo lix prodajnyımi sukinyımi syınami i posledovatelıami dy̆avolyskih nauk, a takje jadnyımi kolcenogimi jabami".

S teh por Svıatoy̆ grad trijdyı dumay̆et, prejde cem kovo-libo pricislity hotıa byı k blajennyım, ne govorıa uje o svıatyıh. Oni trebuy̆ut dokazatelystv kak minimum dvuh soverxennyıh pri jizni cudes, pravednovo sux̨estvovaniy̆a i xesti monografiy̆ s razmyıxleniy̆ami o religiy̆i (y̆esli, konecno, pretendent umel citaty i pisaty).

Kontora ``Fabyen Klemenz i syınovy̆a" raspolagalasy nedaleko ot qerkvi, gde y̆a naxel Kristinu. Malenykiy̆ neprimetnyıy̆ klerk prinıal moy̆e pisymo, podslepovato x̨urıasy.

— Obyıcnay̆a otpravka?

— Otsrocennay̆a, — skazal y̆a. — Otoxlite y̆evo adresatu, y̆esli y̆a ne zaberu poslaniy̆e v teceniy̆e sleduy̆ux̨ih desıati dney̆.

— Eto budet cuty doroje. — On sdelal otmetku v tolstoy̆ knige. — Cto-nibudy y̆ex̨e?

— Net, blagodarıu.

Na uliqe menıa jdal Valyter.

— Dvadqaty pıaty klinkov strajey̆ trebuy̆etsa dlıa odnovo temnovo kinjala. — On skazal eto byıstro, prejde cem y̆a rexil, cto s nim sdelaty. — Dlıa desıati trebuy̆etsa dvesti pıatydesıat.

— Y̆a umey̆u scitaty. Y̆emu ne hvatit i vsey̆ jizni, ctobyı sobraty ih.

Koldun, tocno ptiqa, sklonil golovu:

— Jizny — vex̨ otnositelynay̆a, Lıudvig. K primeru, y̆esty te, kto jivet sebe posle smerti i v us ne duy̆et. A y̆esty takiy̆e, kak tyı. Kto mojet uvelicivaty svoy̆u jizny hoty do beskonecnosti. Ponimay̆ex, na cto y̆a namekay̆u?

— Cto temnyıy̆ kuzneq — straj.

— Y̆esty u menıa takay̆a teoriy̆a.

— No nikakih dokazatelystv.

— Nikakih, — priznal on. — Tot, kto sozday̆et temnoy̆e orujiy̆e, obladay̆et ogromnyım terpeniy̆em i beskonecnyım vremenem.

— Skaji mne, koldun. Skolyko kinjalov u nevo sey̆cas?

— Sprosi cto polegce. Semy. Byıty mojet, vosemy. I y̆ex̨e odin delay̆etsa. A znacit, vremeni u nas ne tak uj mnogo.

— A otkuda tebe izvestno ob etom?

— Y̆a umey̆u sluxaty. Y̆a znay̆u lıudey̆. I nelıudey̆. Y̆a ponimay̆u y̆evo lucxe, cem qerkovniki, no, k sojaleniy̆u, nedostatocno, ctobyı skazaty, kto on takoy̆. Etot celovek rabotay̆et uje davno, iz pokoleniy̆a v pokoleniy̆e sobiray̆a klinki strajey̆ i ostavay̆asy nezametnyım v naxem mire.

— On mojet byıty i ne odin. Naprimer, qelyıy̆ klan. Togda ne stoy̆it obrax̨aty vnimaniy̆a na skazku o bessmertiy̆i.

— Da nevajno, skolyko ih — odin, dvoy̆e ili sotnıa. Kogda y̆evo delo budet zakonceno, stanet slixkom pozdno. Byıty mojet, tyı proy̆avix blagorazumiy̆e i myı pogovorim? Prıamo sey̆cas?

— Govori, — skazal y̆a, prislonivxisy k stene doma.

— Horoxo, — legko soglasilsa on, byıstro oglıadevxisy i udostoverivxisy, cto nikto ne obrax̨ay̆et na nas vnimaniy̆a. — Y̆a uznal obo vsem etom davno, kogda y̆ex̨e byıl molodyım. Malenykiy̆ sluh, broxennay̆a fraza na odnom iz balov vedym. Y̆a zay̆interesovalsa, stal raskrucivaty nitocku. Nablıudal, rasspraxival. Daje v moy̆em soobx̨estve informaqiy̆i byılo malo, y̆a dovolystvovalsa lix sluhami i mifami, bolyxinstvo iz kotoryıh okazalisy vyıdumkoy̆. No y̆a ne sdavalsa, nahodil kollekqionerov staryıh knig, posex̨al castnyıy̆e biblioteki i daje y̆ezdil v Temnolesye.

— Ne ley̆ vodu, koldun. Cuty bolyxe konkretiki.

— Nakoneq y̆a uznal, cto y̆emu trebuy̆utsa kinjalyı strajey̆. Ne y̆unqov, a teh, kto uje ne pervyıy̆ god sobiray̆et duxi. I stal nablıudaty za vami. Pıaty let mne potrebovalosy, ctobyı ponıaty — vaxi umiray̆ut regulıarno, no v osnovnom molodnıak. Te, kto stanovıatsa masterami, pogibay̆ut dovolyno redko, a bessledno iscezay̆ut y̆ex̨e reje. I pocti vse ih kinjalyı popaday̆ut v Orden i unictojay̆utsa.

— Poka y̆a ne uznal nicevo novovo.

On hotel otvetity, no uvidel malıara, nesux̨evo vedro kraski, i ne otkryıval rta, poka celovek ne skryılsa za povorotom.

— Naprıagi mozg, straj. Y̆a govorıu o tom, cto y̆edinstvennyıy̆ sposob sobraty kinjalyı, ne vyırezay̆a vaxu bratiy̆u napravo i nalevo, eto zabiraty te klinki, cto Bratstvo otday̆et zakonnikam na unictojeniy̆e.

— Nevozmojno, — vozrazil y̆a. — Naxe orujiy̆e unictojay̆etsa pri svidetelıah.

On rassmey̆alsa, zapanibratski hlopnuv menıa po plecu:

— Let semy nazad y̆a byıl takim je nay̆ivnyım, kak tyı, van Normay̆enn. No zatem nacal dumaty. Kto vse eti svideteli? Straji na unictojeniy̆i byıvay̆ut kray̆ne redko — vas malo i del polno. Orden lomay̆et kinjal, kogda rıadom predstaviteli vlasti. A tepery podumay̆, mnogo li tolstyıy̆ burgomistr, zanoscivyıy̆ graf ili y̆edva umey̆ux̨iy̆ citaty prihodskoy̆ svıax̨ennik ponimay̆ut v kinjalah strajey̆?

S etimi slovami on dostal iz sumki klinok s sapfirom na rukoy̆ati i protıanul mne.

— Kopiy̆a, — posle beglovo osmotra skazal y̆a.

— Verno. No eto opredelit lix opyıtnyıy̆ glaz. Vseh ostalynyıh smutit zvezdcatyıy̆ sapfir, kotoryıy̆, priznay̆emsıa cestno, ne takay̆a uj i redkosty.

Y̆a poter x̨etinistyıy̆ podborodok:

— Tyı hocex ubedity menıa, cto Orden pomogay̆et temnomu koldunu?

— Orden ili kto-to iz sostoy̆ax̨ih v nem. Naprimer, lucxiy̆ drug markgrafa Valentina gospodin Aleksandr, horoxo tebe znakomyıy̆ po sobyıtiy̆am v Vione. Y̆esli zakonniki ne mogut vospolyzovatsa vaximi kinjalami sami, eto ne oznacay̆et, cto oni ne nay̆dut kuda ih pristroy̆ity.

— I myı vnovy utyıkay̆emsıa v stenu, koldun. Moy̆ vopros: ``Kakay̆a v etom vyıgoda?" — nikuda ne delsa. Y̆a ne jaluy̆u zakonnikov, no vse je ne poverıu v ih jelaniy̆e raspahnuty vrata ada i ustroy̆ity koneq dlıa vsevo sveta. S kakoy̆ stati vyırvavxiy̆esıa iz pekla certi ne nacnut im vredity?

Valyter po-priy̆atelyski pozdorovalsa s dvumıa prohodivximi mimo nas strajnikami i lix posle otvetil:

— Glıaju, tyı uje verix, cto vrata, kotoryıy̆e mogut otkryıty kinjalyı, ne prosto glupay̆a skazka.

— Ne verıu, — otrezal y̆a. — No eto vesomyıy̆ kontrargument v tvoy̆ey̆ nelepoy̆ istoriy̆i.

On ulyıbnulsa, no glaza y̆evo stali zlyımi i razdrajennyımi:

— Sporıu na dukat, cto zakonniki ne znay̆ut o tom, cto klinki mogut otkryıty vrata. Tot, kto splavlıay̆et kinjalyı kuznequ, delay̆et malyıy̆e pakosti, daje ne podozrevay̆a o bolyxoy̆. K primeru, on ne protiv, ctobyı u Bratstva byılo mnogo rabotyı. Kak togda, v Xossiy̆i. I nadey̆etsa, cto neskolyko temnyıh kinjalov diskreditiruy̆ut strajey̆, kotoryıy̆e prosto ne spravıatsa s valom temnyıh dux. Razve eto ne vyıgodno Ordenu? Panikuy̆ux̨ey̆e naseleniy̆e, nedovolynyıy̆e knıazy̆a, novyıy̆e volynosti, usileniy̆e, vlasty? Eto odna iz versiy̆. Drugay̆a — on banalyno zarabatyıvay̆et na etom. Lıudıam, znay̆ex li, nujno zoloto. A nekotoryım lıudıam y̆evo trebuy̆etsa kak mojno bolyxe. I nakoneq, trety̆a — Kristina scitay̆et, cto kuzneq otdal kinjal, ctobyı proverity y̆evo rabotu. Y̆a ne soglasen s etim. Tot, kto sozday̆et takoy̆e, znay̆et, cto vyıhodit iz-pod y̆evo molota. Po mne, eto byıla plata za klinki strajey̆, kotoryıy̆e y̆emu nujnyı.

— Y̆a znay̆u o dvuh temnyıh kinjalah. Odin zabrala u zakonnikov Kristina, drugoy̆ — qerkovy u kuryera, napravlıavxegosıa v Latku.

On vyıglıadel udivlennyım:

— Seryezno? V pervyıy̆ raz slyıxu. U tebıa tocnyıy̆e svedeniy̆a?

— Da.

— Kogda eto slucilosy?

— Ne mogu skazaty.

— Vozmojno, y̆ex̨e pri jizni gospodina Aleksandra, lıubivxevo gostity u markgrafa, — probormotal tot. — Gde kinjal tepery?

— Unictojen klirikami.

— Tepery mne ponıatno, kak oni vyıxli na kuzneqa. — On otvernulsa, sobiray̆asy uhodity. I brosil cerez pleco: — Kristina prosila peredaty, ctobyı tyı prixel cerez paru casov v apteku.

— Ey̆, koldun, — ostanovil y̆a y̆evo. — Ni odin celovek ne menıay̆etsa nastolyko byıstro. S cevo eto tyı stal takim lıubeznyım?

— Lıubeznyım? — On skrivil ugol rta. — Tyı menıa s kem-to sputal, straj. Y̆a govorıu s toboy̆ lix potomu, cto mne mojet ponadobitsa tvoy̆a pomox̨. Inace nikakih razgovorov byı ne polucilosy.

— Otvety mne vsevo lix na dva voprosa, a zatem mojex katitsa k certu.

— Vecerom.

— Net. Sey̆cas.

V y̆evo glazah byılo qeloy̆e more gneva, y̆a videl, kak on sjal kulaki, no tut je rasslabilsa i ney̆iskrenne ulyıbnulsa:

— Ladno. Prox̨e vse zakoncity sey̆cas. Valıay̆.

— Markgraf Valentin sobiral kinjalyı. Dlıa kovo?

— Aleksandr i y̆evo ney̆izvestnyıy̆e mne druzy̆a vnuxili markgrafu, cto tot obretet bessmertiy̆e. Na samom dele zakonniki prosto ispolyzovali y̆evo vozmojnosti dlıa sbora orujiy̆a Bratstva. Vtoroy̆ vopros?

— Kinjal, kotoryıy̆ tyı y̆edva ne ukral v Livette. Dlıa cevo nujen on?

— A… — protıanul koldun, i byılo vidno, cto y̆emu nepriy̆atno vspominaty o toy̆ neudace. — Dlıa obmena. Odin kollekqioner jelal sebe takuy̆u igruxku v obmen na bezdeluxku.

— Kakuy̆u bezdeluxku?

Kak nazlo, navstrecu xel y̆ex̨e odin patruly straji, i Valyter vospolyzovalsa etim, otodvinuv menıa plecom:

— Prihodi k Kristine. Uznay̆ex.

— To y̆esty tyı polucil y̆ey̆e? Naxel drugoy̆ klinok?

— Mne pora, van Normay̆enn.

Y̆a dal y̆emu proy̆ti, potomu cto i tak uje znal, cey̆ kinjal on ispolyzoval i na cto y̆evo hotel pomenıaty.



— Porajay̆usy tvoy̆emu terpeniy̆u. Drugoy̆, ne znay̆a tebıa, nazval byı eto slaboharakternosty̆u. Nu posle tovo, cto sdelal etot hmyıry. — Propovednik posmotrel na menıa po-starikovski hitro.

— No tyı menıa znay̆ex i… — Y̆a predostavil y̆emu zakoncity frazu.

— Tyı obuzdal emoqiy̆i i rexil ne pugaty kryısu, poka ona ne privedet tebıa k zernohranilix̨u.

— Skorey̆e uj zmey̆u, poka ta ne pokajet, gde y̆ey̆e kladka.

Propovednik pogrozil mne palyqem:

— Zmey̆a mojet i ukusity. A y̆ey̆e y̆ad opasen. Pomnıu, v moy̆ey̆ derevne odin pastuh…

— Menıa bolyxe interesuy̆et ta vajnay̆a novosty, kotoray̆a vsevo minutu nazad zanimala vse tvoy̆e voobrajeniy̆e.

— Tebıa ona udivit. Znay̆ex, kak zovut kardinala, kotoryıy̆ pribyıvay̆et v Kruso, ctobyı provesti torjestvennoy̆e bogoslujeniy̆e? Tvoy̆ staryıy̆ drug Urban.

Y̆a daje ostanovilsa:

— Nepriy̆atnoy̆e izvestiy̆e, y̆esli cestno. Kardinal Urban v gorode, a rıadom Valyter. Kak byı ne slucilosy vtorovo Viona.

— Tot celovek iz Ordena, Aleksandr, mertv.

— I vse ravno mne eto ne nravitsa.

Y̆a vyıxel na bolyxuy̆u krugluy̆u plox̨ady, na kotoroy̆ letom vo vremıa prazdnestva ustray̆ivali znamenityıy̆e pıatiminutnyıy̆e skacki. Sey̆cas zdesy goreli kostryı i byıli razbityı palatki. Palomniki, te scastlivciki, kovo pustili v gorod, jili prıamo na uliqe, ojiday̆a svoy̆ey̆ oceredi prikosnutsa gubami k svıatomu sledu.

Vozle odnoy̆ iz vıazanok hvorosta, vyıtıanuv nogi, sidelo Pugalo.

— Vot eto vstreca, — probubnil Propovednik. — Ne hocex procitaty y̆emu notaqiy̆u za to, cto ono razodralo dnevnik burgomistra? Inace v sleduy̆ux̨iy̆ raz ono ukradet u tebıa ispodney̆e. I spalit na ogne.

— A y̆esli ne procitaty y̆emu notaqiy̆u, y̆a sekonomlıu qeluy̆u minutu vremeni. Potomu cto itogovyıy̆ rezulytat budet odin i tot je — ono vse ravno propustit moy̆i slova mimo uxey̆. K tomu je mne sleduy̆et zaglıanuty v apteku.

— Tyı vse ravno nicevo ot nih ne dobyexysa.

— No hotıa byı uznay̆u, kakim obrazom oni hotıat poy̆maty kuzneqa.

— Takay̆a je bespoleznay̆a trata vremeni, kak ubejdaty Pugalo ostavatsa pay̆inykoy̆.

Ni tot, ni drugoy̆ ne zahoteli byıty moy̆imi soprovojday̆ux̨imi, tak cto y̆a ostavil ih na plox̨adi sluxaty lıudskuy̆u boltovnıu.

Apteka okazalasy zakryıta, stavni opux̨enyı, no v okne vtorovo etaja gorel svet. Y̆a postucal, i mne otkryıl sedoborodyıy̆ aptekary.

— A, gospodin straj. Myı uje dumali, vyı ne pridete. — On blagosklonno kivnul, vpuskay̆a menıa.

Starik vyıglıadel nervnyım i naprıajennyım. Za vsey̆ etoy̆ lıubeznosty̆u skryıvalsa kakoy̆-to suy̆etlivyıy̆ strah. Eto nicuty ne vnuxilo mne doveriy̆a.

— Lıudvig! Horoxo, cto tyı vernulsa! — Kristina stoy̆ala na lestniqe i ulyıbalasy, ne skryıvay̆a, cto rada menıa videty. — Idem, y̆a tebıa poznakomlıu s ostalynyımi.

V komnatah, kotoryıy̆e ona snimala, goreli sveci. Dva stola okazalisy sdvinutyı, i za nimi razmestilisy lıudi. Kogda y̆a voxel, na mne sosredotocilosy vnimaniy̆e vseh prisutstvuy̆ux̨ih.

— Pozvolyte poznakomity vas s gospodinom van Normay̆ennom, druzy̆a, — obratilasy Kristina k cetveryım neznakomqam. — On — straj, kak i y̆a. Odin iz lucxih v moy̆em pokoleniy̆i. A eto lıudi, kotoryıy̆e, kak i myı s Valyterom, jelay̆ut raz i navsegda pokoncity s temnyım kuzneqom.

Qelay̆a kompaniy̆a sumasxedxih mectateley̆, jelay̆ux̨ih spasti mir.

— Metr Filipp, — predstavila ona aptekarıa. — On zanimay̆etsa alhimiy̆ey̆ i byıl nastolyko lıubezen, cto okazal nam gostepriy̆imstvo.

Starik suy̆etlivo poklonilsa i, plıuhnuvxisy na stul, stal pomexivaty lojeckoy̆ v stakane s kakim-to varevom, to i delo gromko zvıakay̆a o tonkuy̆u steklıannuy̆u stenku.

— Adily aly Djuma — predstavitely Lavenduzzskovo soy̆uza v svoy̆ih zemlıah.

Tıurban na britoy̆ golove delal tox̨evo hagjita pohojim na strannyıy̆ lesnoy̆ grib. Glaza byıli podvedenyı surymoy̆.

— On okazal nam neoqenimuy̆u uslugu.

— Vyı priukraxivay̆ete moy̆i dostijeniy̆a, bay̆an[40] Kristina. — On ulyıbnulsa, i y̆a uvidel, cto dvuh qentralynyıh verhnih zubov u nevo net. — Y̆a vsevo lix skromnyıy̆ sluga pustyınnyıh mudreqov, i ih prikazyı priveli menıa sıuda.

— Cezare Motto. Kondotyer.

Vyısokiy̆ i plecistyıy̆ celovek s x̨etinistyım podborodkom i gustyımi, cuty ryıjevatyımi brovıami neohotno pripodnıal dva palyqa v privetstvennom jeste nay̆emnikov Kavarzere. Y̆a ne znal, cto on zdesy delay̆et, no soldat udaci kazalsa takim je lixnim, kak cert, zaglıanuvxiy̆ na voskresnuy̆u messu.

— I oteq Gotthod, kanonik sobora Svıatoy̆ Mariy̆i v Braselovette.

Borodatyıy̆ tolstıak v cernoy̆ rıase, krugloliqiy̆, s ospinami na x̨ekah i lbu, pripodnıalsa nad stulom:

— Master.

— S cevo mne nacaty rasskaz, Lıudvig? — Valyter sidel na podokonnike, na rukah u nevo dremala tox̨ay̆a pıatnistay̆a koxka. Liqo kolduna uje zajilo, slovno y̆a i ne kasalsa y̆evo svoy̆imi kulakami.

— Nacni s tovo, zacem y̆a zdesy. — Y̆a proy̆ignoriroval stul, vstav tak, ctobyı videty ih vseh. Razumey̆etsa, eto ne ostalosy nezamecennyım, no nikto, krome usmehnuvxegosıa kondotyera, ne podal vida.

— Delo ne v tom, cto tyı straj… — Koldun ne otryıval vzglıada ot koxki.

— Sam Gospody posyılay̆et vas nam, — vajno kivnul oteq Gotthod. — Ne inace eto y̆evo jelaniy̆e.

— Nam dey̆stvitelyno nujna tvoy̆a pomox̨, Lıudvig, — podhvatila Kristina. — Myı s Valyterom segodnıa pogovorili i ponıali, cto, y̆esli tyı budex s nami, vse proy̆det legko i ne budet nikakoy̆ krovi.

— Y̆a, pojaluy̆, nacnu s samovo nacala. — Koldun posmotrel na Kristinu, i ta obodrıay̆ux̨e kivnula. — Dnem tyı zadaval vopros, zacem mne byıl nujen kinjal tvoy̆evo druga…

— Natana, — podskazala straj.

— Tvoy̆evo druga Natana. Predyıstoriy̆a takova. Dostopoctimyıy̆ Adily aly Djuma, blagodarıa svoy̆im svıazıam v torgovle, mnogoy̆e slyıxit. Daje to, cto pyıtay̆utsa skryıty ot y̆evo uxey̆. Do nevo doxel sluh o tom, cto v Velikoy̆ pustyıne lovkiy̆e lıudi otyıskali dva cernyıh kamnıa i privezli na nax kontinent.

— Rec o glazah serafima?

— Verno, van Normay̆enn. Ih dostavili po osobomu zakazu. Etot kameny oceny redok i y̆avlıay̆etsa obıazatelynyım materialom dlıa izgotovleniy̆a temnovo klinka. I trebuy̆etsa kuznequ.

On sdelal znacitelynuy̆u pauzu, no, ne dojdavxisy nikakih kommentariy̆ev ot menıa, prodoljil:

— Nam prixlosy pobyıvaty v xestnadqati portah, prejde cem udalosy napasty na sled prodavqa. I y̆ex̨e neskolyko mesıaqev, ctobyı uznaty o pokupatele. On priobrel oba glaza serafima, tak kak sobiray̆et mineralyı — u nevo dovolyno obxirnay̆a kollekqiy̆a, kak y̆a slyıxal. Myı popyıtalisy vyıkupity hotıa byı odin kameny, no bogacu ne nujnyı denygi.

— Popyıtalisy ukrasty, — prodoljila Kristina. — No eto okazalosy ne tak-to prosto. Myı daje ne smogli uznaty, gde on ih hranit.

— Tvoy̆a reputaqiy̆a pod ugrozoy̆, koldun, — s usmexkoy̆ skazal y̆a Valyteru. — Neujeli tyı ne isproboval samyıy̆ vernyıy̆ sposob merzavqev — nasiliy̆e?

— Povery, mne oceny hotelosy. — On vernul usmexku. — No u nevo mnogo druzey̆. I eto izvestnyıy̆ celovek. Y̆ego isceznoveniy̆e, ne govorıa uje o smerti, vzbudorajilo byı vlasti. A y̆a v posledney̆e vremıa i tak privlek slixkom mnogo nezdorovovo vnimaniy̆a k poy̆iskam temnovo kuzneqa.

— Na samom dele eto y̆a otgovoril y̆evo ot pospexnyıh dey̆stviy̆. — Aptekary nervno sqepil palyqi. — Nasiliy̆e — ne vyıhod. Osobenno y̆esli ono mojet privesti k y̆ex̨e bolyxomu nasiliy̆u i provalu vajnoy̆ missiy̆i. Da i so smerty̆u kollekqionera myı byı ne naxli tay̆nik. Poetomu rexili dey̆stvovaty inace. Adily vyıstupil kak predstavitely drugovo lıubitelıa mineralov, predlojil, razumey̆etsa, denygi. Zatem obmen. Uznal, cevo hocet bur… tot celovek.

— I cto y̆emu nujno? — sprosil y̆a, hotıa znal otvet.

— Kinjal straja. Y̆ego interesoval zvezdcatyıy̆ sapfir v neobyıcnom ispolneniy̆i. Za takuy̆u relikviy̆u on gotov byıl ustupity odin iz dvuh svoy̆ih kamney̆.

— Tyı v kurse slucivxegosıa? — poy̆interesovalsa y̆a u Kristinyı.

— Net. V Livette menıa ne byılo. Y̆a byı ne dala y̆emu vzıaty kinjal Natana, tyı je znay̆ex.

Y̆esli cestno, y̆a uje ne znal, kto ona i na cto sposobna radi tovo, ctobyı poy̆maty temnovo mastera.

— I kak je vyı postupili, kogda u tvoy̆evo druga nicevo ne polucilosy i klinok vernulsa k zakonnomu vladelyqu?

— Y̆a izgotovil poddelku, — ojivilsa Filipp. — Otmennay̆a vex̨, nastoy̆ax̨iy̆ zvezdcatyıy̆ sapfir, i staly podhodıax̨ay̆a. Mojno obmanuty pocti vseh, no ne nastoy̆ax̨evo znatoka.

— On provozilsa do dekabrıa, myı poterıali pocti xesty mesıaqev. — Cezare prenebrejitelynyım x̨elckom otpravil cerez stol nevidimuy̆u sorinku. — A v itoge kollekqioner podnıal nas na smeh. Obmanuty y̆evo ne udalosy.

— I?.. — podstegnul y̆a ih.

Glubokay̆a tixina razlilasy po pomex̨eniy̆u, vse tepery smotreli na Kristinu, kak byı otstranıay̆asy ot tovo, cto slucilosy dalyxe.

— Y̆a otdala y̆emu svoy̆ kinjal! — nabrav vozduha v grudy, vyıpalila ona.

Mne prihodilosy igraty, i y̆a ne byıl uveren, cto akter iz menıa horoxiy̆.

— Cto?!

— Mne prixlosy, Lıudvig.

Y̆a s kamennyım liqom pomolcal, vidıa, cto ona to krasney̆et, to bledney̆et, i spokoy̆no proy̆iznes:

— Y̆a hocu y̆evo uvidety.

Liqo u Kristinyı stalo rasterıannyım:

— Tyı o kinjale? Y̆a je govorıu…

— K certu tvoy̆ kinjal, Kristina. Raz on tebe ne nujen i tyı rasstalasy s nim dobrovolyno, y̆a ne tot celovek, kotoryıy̆ budet ubejdaty tebıa v tvoy̆ey̆ gluposti! — rezko otvetil y̆a, i moy̆i slova byıli dlıa ney̆e kak pox̨ecina. — Pokajite mne kameny, radi kotorovo vyı ustroy̆ili vse eto.

— Em… — Filipp poter perenosiqu. — Ponimay̆ete, u nas y̆evo net. I kak byı… y̆a polagay̆u, cto uje i ne budet. Gospodin Cezare nedelıu nazad vernulsa s plohimi novostıami. Kollekqioner mertv, kamni tak i ne nay̆denyı. V delo vmexalasy inkviziqiy̆a, i myı ne mojem sey̆cas vernuty daje orujiy̆e Kristinyı. Ne znay̆em, gde ono.

— A y̆a govoril, cto klinok nado menıaty na kameny srazu. — Kondotyer qedil slova zlo. — Vyı je poxli na povodu u etoy̆ svoloci. Mirolıubiy̆e, ne nado nasiliy̆a, ne stoy̆it privlekaty k sebe dopolnitelynoy̆e vnimaniy̆e…

— Posle tovo kak myı pyıtalisy podsunuty y̆emu poddelku, on perestal nam doverıaty, — vinovato razvel rukami aptekary. — On potreboval pereslaty y̆emu kinjal cerez ``Fabyen Klemenz i syınovy̆a".

— No obmanul vas i ne peredal glaz serafima?

— Net, — gluho otvetila Kristina. — Valyter ne hotel polyzovatsa posrednikami. Myı rexili zabraty kameny licno, no ne uspeli.

— V itoge u vas net ni kur, ni lisyı, — otvetil y̆a staroy̆ pogovorkoy̆.

Na Kristinu byılo jalko smotrety, tepery ona vyıglıadela nastolyko podavlennoy̆, cto y̆a s trudom poborol v sebe jelaniy̆e otkryıty visevxuy̆u cerez pleco sumku i vernuty y̆ey̆e orujiy̆e. No y̆a sderjalsa. Ne sey̆cas. I ne pri etih lıudıah.

— Obrazno govorıa, vyı soverxenno pravyı, — podtverdil Filipp.

— Y̆esli vyı nadey̆etesy, cto tepery y̆a dam vam svoy̆ kinjal, ctobyı vyı y̆evo poterıali tak je bezdarno, kak i y̆ey̆e klinok, to obrax̨ay̆etesy ne po adresu.

— Gospody s vami! — vsplesnul rukami oteq Gotthod. — Nicevo podobnovo! Vam ne nado budet s nim rasstavatsa. Y̆esli cestno, to on nam sovsem ne nujen. Ne hotite vse je prisesty?

— Net. Tak cto je vam nujno?

— Odna vex̨, kotoray̆a prinadlejit tebe. — Koldun ostorojno opustil koxku na podokonnik, podoxel k Kristine, polojil ruku y̆ey̆ na pleco, i mne ne ponravilsa etot jest — sobstvennika, zay̆avlıay̆ux̨evo prava na svoy̆u vex̨. — Kolyqo, kotoroy̆e tebe podaril y̆episkop Urban, posle tovo kak tyı spas y̆emu jizny v Vione. Ono y̆ex̨e u tebıa?

Neojidannyıy̆ povorot. Priznatsa, y̆a ne byıl gotov k takomu voprosu.

— Ne privyık taskaty s soboy̆ goru bezdeluxek.

— No tyı i ne prodal y̆evo. — Kristina ne spraxivala, utverjdala. — Tyı slixkom umen, ctobyı razbrasyıvatsa podobnyımi podarkami i ostavlıaty ih v lombarde. Uverena, cto kak vsegda hranix na depozite v ``Fabyen Klemenz", ctobyı vzıaty v lıuboy̆ moment.

Ona slixkom horoxo znala moy̆i privyıcki, i sey̆cas menıa eto ne radovalo.

— Podrobney̆e, Krista. Y̆esli vam nujno kolyqo klirika, y̆a hocu znaty dlıa cevo. Vax kuzneq klıuy̆et na lıubuy̆u pobrıakuxku?

Valyter pokacal golovoy̆:

— Vse gorazdo slojney̆e i prox̨e. Tot slucay̆ v Vione imel nekotoryıy̆e predposyılki. Aleksandr i markgraf Valentin hoteli izbavitsa ot kardinala, tocney̆e togda y̆ex̨e y̆episkopa, po neskolykim pricinam. Razumey̆etsa, vse videli lix politiceskiy̆e — on mexal razvernutsa Ordenu v knıajestve i ne daval jizni markgrafu, oblicay̆a y̆evo prestupleniy̆a. No byılo y̆ex̨e odno ``no". Y̆a o nem uznal pozje, primerno za nedelıu do tovo, kak tyı prikoncil y̆evo milosty v Latke. On mne sam priznalsa, cto Aleksandr jajdal polucity cernyıy̆ kameny y̆episkopa. Mol, tot y̆emu nujen ne menyxe, cem kinjalyı strajey̆, i horoxo byı etu xtuku privezti v Latku, kak tolyko poy̆avitsa takay̆a vozmojnosty.

— Hm… — Y̆a glıadel na nevo ispodloby̆a. — U y̆evo vyısokopreosvıax̨enstva imey̆etsa glaz serafima?

— Imenno. Y̆a posluxal veter, i tot dones do menıa interesnyıy̆e sluhi. Celıady, kak tyı znay̆ex, redko hranit tay̆nyı. U kardinala y̆esty mineral, kotoryıy̆ on nosit na xey̆e, rıadom s raspıatiy̆em. Kameny plohoy̆ i zloy̆. Urban vrode kak dal obet mnogo let nazad, i eto y̆evo krest, kotoryıy̆ on tax̨it vo slavu Gospoda, istıazay̆a svoy̆u ploty takim obrazom.

— U glaza serafima y̆esty podobnyıy̆e svoy̆stva?

— Myı tolkom ne uverenyı, — otvetil aptekary. — Traktatyı govorıat raznoy̆e, v alhimiy̆i kameny scitay̆etsa temnyım i sposobnyım prinosity vred celoveku nestoy̆komu. U hagjitov na sey̆ scet voobx̨e mnojestvo legend.

— Oni raznıatsa, — ulyıbnulsa britogolovyıy̆ torgoveq. — Pustyınnyıy̆e starqi, da prodlıatsa ih goda vecno, nazyıvay̆ut y̆evo ubiy̆qey̆ sveta. I skazok o nem dey̆stvitelyno mnogo. Vse kak odna s plohim konqom. Ne skaju, skolyko v nih pravdyı, no moy̆ narod staray̆etsa ne derjaty takiy̆e vex̨i podolgu, osobenno blizko k domu.

— Predpocitay̆et sbagrivaty ih nam za kucu florinov, — poddel y̆evo Cezare, i hagjit ulyıbnulsa, no, sudıa po y̆evo liqu, isklıucitelyno iz vejlivosti.

— I raz ne polucilosy s kollekqionerom, vyı rexili pozay̆imstvovaty mineral u kardinala. Vmesto tovo ctobyı vzıaty lopatu i otpravitsa v bezvodnuy̆u pustyınıu. Po mne — posledniy̆ variant byıl byı gorazdo boley̆e razumen. V plane vyıjivaniy̆a.

— Y̆a ponimay̆u vaxu ironiy̆u, gospodin van Normay̆enn. Myı prıacemsıa ot Qerkvi, znay̆em, na cto ona sposobna. A tut sami lezem v volcy̆u pasty. No myı obıazanyı. Vo imıa lıudey̆ i vo blago vsevo mira, — pospexno utocnil aptekary.

— Razumey̆etsa, — ehom otkliknulsa y̆a. — A kolyqo vam trebuy̆etsa…

— Ctobyı podobratsa k Urbanu. Eto propusk, Lıudvig. Prosto otday̆ nam y̆evo. Tebe nezacem ucastvovaty v ostalynom.

Y̆a potıanulsa:

— Vyı, lıubeznyıy̆e gospoda, konecno, bolyxiy̆e fantazeryı, no, kak mne kajetsa, ne idiotyı. I ne stanete ubivaty kardinala. Klirikov takovo ranga ubivay̆ut lix drugiy̆e kliriki, no ne prostyıy̆e smertnyıy̆e vrode nas.

— Nikto ne govorit ob ubiy̆stve. Day̆ mne vozmojnosty podobratsa k nemu, vse ostalynoy̆e — delo tehniki. — Valyter nehoroxo ulyıbnulsa.

Y̆a znal, kak nekotoryıy̆e koldunyı umey̆ut usyıplıaty, rassey̆ivaty vnimaniy̆e ili pritormajivaty vremıa. Videl, cto prodelyıvay̆et Gertruda.

— Nu tak i sdelay̆ vse sam, — pojal y̆a plecami. — Dlıa etovo ne obıazatelyno kolyqo.

— U kardinala seryeznay̆a ohrana. Qerkovniki s magiy̆ey̆. So vsemi y̆a prosto ne spravlıusy. — On legko raspisalsa v svoy̆ey̆ bespomox̨nosti.

— A kogda u tebıa budet pobrıakuxka, oni cto? Rastvorıatsa v vozduhe, cto li?

— Qerkovniki budut meney̆e bditelynyı. Y̆a smogu podobratsa blizko, ogluxity ih. Bez propuska, izdali, eto nevozmojno.

— Kak tyı oby̆asnix, otkuda ono u tebıa?

— Ne budu oby̆asnıaty. Myı planiruy̆em vse provernuty vo vremıa torjestvennovo bogoslujeniy̆a. Y̆esli i budet proverka, to ne nastolyko seryeznay̆a. A potom, uveren, im stanet ne do nas.

Cezare zarjal, i Filipp, podderjivay̆a nay̆emnika, ulyıbnulsa.

— Y̆esty pricina dlıa smeha? — nahmurilsa y̆a.

— Nebolyxay̆a. — Aptekary ponizil golos. — Myı s Valyterom pridumali i osux̨estvili grandioznuy̆u aferu, sociniv, cto angel snizoxel v etot gorod.

— I proxlo, cert menıa deri! — hlopnul ladony̆u po stolu kondotyer. — Slucilosy cudo!

— Svıatotatqi, — so smireniy̆em pokacal golovoy̆ kanonik. — Nadey̆usy, Gospody poy̆met, cto ne radi zla myı eto sdelali, i prostit nas.

— To y̆esty lıudi, cto sidıat na uliqah i pyıtay̆utsa popasty v gorod, proxli sotni lig radi nesux̨estvuy̆ux̨evo cuda? Da, smexno. Kak vyı eto ustroy̆ili?

Valyter skromno razvel rukami:

— Nemnovo ney̆tralynoy̆ magiy̆i, kotoruy̆u ne srazu opredelıat kliriki, nemnogo alhimiceskih smesey̆ Filippa, odin slepoy̆ rebenok i umeniy̆e rasprostranıaty sluhi. Vot reqept bojestvennovo cuda v naxi dni.

— Vyı zatey̆ali eto, ctobyı vyımanity Urbana, tak kak Kruso pod y̆evo pokrovitelystvom, i on ne mog proy̆ignorirovaty takoy̆e sobyıtiy̆e i ne priy̆ehaty sıuda.

— Tyı pravilyno ponimay̆ex.

Da cto uj tut ponimaty? I idiotu y̆asno.

— Znacit, vyı vse splanirovali davno. Y̆ex̨e do tovo, kak ponıali, cto s kollekqionera kameny ne polucity.

— Rezervnyıy̆ variant. Odno rovnyım scetom ne mexalo drugomu, i, kak vidix, y̆a okazalsa prav. Y̆esli byı tyı ne poy̆avilsa, myı spravilisy byı bez tebıa. Prosto vse stalo byı gorazdo slojney̆e. Kogda nacnetsa bogoslujeniy̆e, lucxiy̆e mesta budut otdanyı pocetnyım jitelıam goroda i blagorodnyım, ostalynyım pridetsa dovolystvovatsa liqezreniy̆em cujih spin — kardinalyskay̆a ohrana ne imey̆et privyıcki puskaty kovo ni popadıa. Persteny — moy̆ propusk na samyıy̆ verh. Celoveku s koldovskim darom tıajelo prohodity cerez stroy̆ klirikov, v otliciy̆e ot obyıcnoy̆ straji. A tvoy̆e kolyqo mne v etom pomojet. I kogda y̆a zaberu kameny, to razvey̆u volxebstvo — sled angela v kamne isceznet, i svıatoxam, pravo, y̆ex̨e dolgo budet cem zanıatsa. Poka oni hvatıatsa propaji, myı budem uje daleko.

S tolku on menıa ne sbil. Y̆a videl, on nedogovarivay̆et, i znal, cto imenno. Propaja ``cuda", byıty mojet, na kakoy̆e-to vremıa otodvinet obnarujeniy̆e vnezapnovo isceznoveniy̆a kamnıa kardinala, no i tolyko. Vse ravno nacnut iskaty i ryıty. Ibo svıax̨enniki ne nastolyko kretinyı, ctobyı ne pocuvstvovaty ostatki cujoy̆ magiy̆i. A znacit, imey̆etsa lix odin sposob zamesti sledyı — ubity vseh pricastnyıh.

Teh, kto propustit y̆evo k kardinalu. Teh, kto uvidit y̆evo. Nu i menıa zaodno.

Y̆a znal, cto Valyter opasen, no ne dumal, cto nastolyko. Y̆ego hladnokroviy̆e, besstyıdstvo i umeniy̆e manipulirovaty lıudymi byıli potrıasay̆ux̨imi. Imenno sey̆cas y̆a okoncatelyno utverdilsa v myısli, cto kolduna sleduy̆et ubity srazu, kak tolyko predstavitsa takay̆a vozmojnosty, nevajno, kakiy̆e qeli on presleduy̆et. Etot celovek proy̆det po golovam i unictojit vseh.

On vedet svoy̆u igru, no vmexivay̆et v ney̆e Bratstvo. Y̆esli y̆a pomogu y̆emu, y̆esli u nevo vse polucitsa, kliriki budut zlyı. Da net. Cto tam! Oni budut v y̆arosti. I poy̆dut po lojnomu sledu, kotoryıy̆ privedet ih v Ardenau.

No skazal y̆a sovsem inoy̆e:

— Y̆a pomogu vam.

Y̆a uvidel, kak oni ojivilisy. Vse, krome Kristinyı, v glazah kotoroy̆ citalosy somneniy̆e.

— Na neskolykih usloviy̆ah.

— Nazovi ih, — predlojila moy̆a byıvxay̆a naparniqa.

— Y̆esli on hocet polucity kolyqo, to pusty vernet to, cto snıal s moy̆evo palyqa v Latke.

Valyter pokacal golovoy̆:

— Izvini, van Normay̆enn, no eto nevozmojno. Y̆a srazu je y̆evo unictojil — rabota vedymyı, i ono moglo navesti y̆ey̆e na nas. Viju, cto sey̆cas u tebıa takoy̆e je. Byıty mojet, y̆a kak-to mogu kompensirovaty tvoy̆u poterıu?

Da. Raspahnuty okno i siganuty golovoy̆ vniz.

— Zabudem. — Skrepıa serdqe y̆a otkazalsa ot svoy̆ey̆ mectyı. — Daley̆e. Y̆a idu s toboy̆.

— Zacem tebe riskovaty?

Ctobyı tyı ne ubil menıa srazu posle tovo, kak vozymex jelay̆emoy̆e.

— Potomu cto, y̆esli kto-to iz vas predstavitsa mnoy̆, y̆a ne sobiray̆usy potom rashlebyıvaty y̆evo oxibki.

— Cto-nibudy y̆ex̨e?

— Nikakoy̆ krovi.

— Nu, razumey̆etsa. — On skazal eto tak cestno, cto daje y̆a byı poveril y̆emu, y̆esli byı ne znal kolduna slixkom horoxo i ne pobyıval v zastenkah zamka Latka.

— Togda net problem. Cas pozdniy̆, gospoda. S vaxevo pozvoleniy̆a, y̆a otpravlıusy domoy̆.

— Tyı legko soglasilsa, van Normay̆enn. Y̆a dumal, pridetsa ubejdaty tebıa do utra.

Y̆a zametil legkuy̆u usmexku kondotyera. Interesno, kak oni planirovali menıa ubejdaty?

— Tyı mne vse y̆ex̨e ne nravixysa, koldun. Kak i vse, cto proy̆ishodit. No y̆a eto delay̆u tolyko radi ney̆e. Zapomni eto.

— Vesomyıy̆ argument, — kivnul on. — I y̆a v nevo verıu. Uvidimsıa zavtra, van Normay̆enn.

— Y̆a provoju, — vyızvalasy Kristina.

Myı vmeste spustilisy na pervyıy̆ etaj i, ne sgovarivay̆asy, vyıxli na uliqu.

— Y̆ex̨e ne pozdno uy̆ehaty. Prıamo sey̆cas, — vnovy predlojil y̆a.

Ona izbegala smotrety na menıa i otriqatelyno pokacala golovoy̆:

— Y̆a poterıala svoy̆ kinjal. I cto samoy̆e ujasnoy̆e — nicuty ne jaley̆u. Y̆a bolyxe ne straj, Lıudvig. Mne nekuda vozvrax̨atsa.

— Propaja orujiy̆a ne lixay̆et tebıa dara. Daje bez nevo tyı — straj.

Ona neojidanno prislonilasy lbom k moy̆ey̆ grudi:

— Y̆a ustala, Sineglazyıy̆. Ustala byıty strajem, ustala opravdyıvaty ojidaniy̆a Miriam, ustala spasaty Bratstvo, otcityıvatsa pered magistrami, vrajdovaty s zakonnikami i vyıcix̨aty vse te goryı deryma, cto ostavlıay̆ut lıudi posle svoy̆ey̆ smerti, ne jelay̆a otpravlıatsa v cistilix̨e. Vse moy̆e sux̨estvovaniy̆e, skolyko y̆a sebıa pomnıu, posvıax̨eno imenno etomu. Znay̆ex, cevo y̆a hocu? Pokoy̆a. Sbejaty daleko-daleko, tuda, gde net temnyıh dux i lıudey̆ s proklıatiy̆em dara, magiy̆i, i ostavity vse za spinoy̆, jity svoy̆ey̆ malenykoy̆ jizny̆u, pisaty muzyıku i rastity detey̆. No samoy̆e straxnoy̆e v etom to, cto moy̆i jelaniy̆a nicevo ne znacat. V menıa vbili to je samoy̆e, cto i v tebıa, — spasaty lıudey̆ i zax̨ix̨aty Bratstvo. Temnyıy̆ kuzneq — samoy̆e opasnoy̆e, s cem myı stalkivalisy. Y̆a obıazana razobratsa s nim.

Byılo obidno, cto myı ponimay̆em zax̨itu Bratstva soverxenno po-raznomu. Ona namerena podstavity y̆evo, ctobyı potom pyıtatsa spasti to, cto uje budet unictojeno. Y̆a gotov predaty y̆ey̆e nadejdyı, ctobyı xkola v Ardenau sux̨estvovala i dalyxe.

— Cto je. Eto tvoy̆ vyıbor, — s sojaleniy̆em skazal y̆a y̆ey̆.

— Postoy̆! — Ona shvatila menıa za ruku. — Tebe ne obıazatelyno ucastvovaty. Pravda. Prosto day̆ nam kolyqo i uhodi. Myı cto-nibudy pridumay̆em.

— Y̆a doljen cto-to y̆ex̨e znaty?

Ona sdelala xag nazad:

— Mogu govority tolyko za sebıa. Ne ponimay̆u, kak pri ogromnom steceniy̆i naroda i klirikah mojno ukrasty vex̨ u kardinala… I boy̆usy za tvoy̆u jizny. Vozmojno, y̆a oxibay̆usy, no skazaty ob etom — pravilyno, i tyı doljen byıty v kurse moy̆ih razmyıxleniy̆.

— Do zavtra, Krista. Budy ostorojna s nimi.

Ona ulyıbnulasy:

— Eto im sleduy̆et byıty ostorojnyımi.

I vernulasy obratno v apteku.

Vstrecaty kardinala Urbana y̆a planiroval ne vyıbiray̆asy iz doma. Okna moy̆ih komnat vyıhodili na qentralynuy̆u gorodskuy̆u uliqu, po kotoryım doljna proy̆ehaty torjestvennay̆a proqessiy̆a, tak cto y̆a byıl obespecen neplohim zritelyskim mestom.

Pugalu toje byılo interesno poglıadety. Ono torcalo zdesy s samovo utra, vprocem, ne bez dela. Vcera vecerom oduxevlennyıy̆ sper v kakoy̆-to lavke derevıannuy̆u marionetku, za cto y̆emu prixlosy vyısluxaty notaqiy̆u ot Propovednika, tak kak ``kakoy̆-to rebenok tepery lixilsa radosti". Po mne, kakoy̆-to rebenok izbejal nocnyıh koxmarov, kukla vyıglıadela cudovix̨no — grotesknay̆a kopiy̆a celoveka s cuty vyıtıanutoy̆ golovoy̆, xirocennyımi rukami i bockoobraznoy̆ grudy̆u. Raskraxena ona okazalasy nicuty ne meney̆e bezdarno, cem vyırezana: glaza raznoy̆ velicinyı, gubki bantikom, soloma vmesto volos.

Pugalo proy̆avilo tvorceskuy̆u jilku, cuty podpraviv vse, cto nujno, serpom. V itoge igruxka obrela harakternuy̆u i znakomuy̆u zlovex̨uy̆u uhmyılocku.

— Prelestno, — oqenil Propovednik. — Tepery mojex privıazaty k ney̆ verevocki i korcity kuklovoda iz Litaviy̆i.

Pugalo dey̆stvitelyno privıazalo verevocku, no lix odnu, i za xey̆u. Zatem izvleklo iz karmana mundira klocok atlasnoy̆ tkani, igolki, nitki i za polcasa svarganilo odejonku. Ono razve cto ne nasvistyıvalo ot udovolystviy̆a, naslajday̆asy rukodeliy̆em. Kogda vse byılo gotovo i kukla povisla pod potolkom, Propovednik ostorojno izrek:

— Svıatoy̆ Vitt i vse y̆evo plıaski! Mne odnomu kajetsa, cto eta xtuka pohoja na kardinala? Ono tepery vyısunet y̆ey̆e v okno i budet mahaty pri vsey̆ cestnoy̆ kompaniy̆i?

Na stoly slojnyıy̆ vopros u menıa otveta ne naxlosy. I Propovednik vnes ocerednoy̆e predlojeniy̆e:

— K cemu vsıa eta suy̆eta, Lıudvig? V tvoy̆ey̆ sumke qelyıh dva proklıatyıh kamnıa. Na koy̆ cert nagrevaty kardinala na y̆ex̨e odin?

— Nu, ``nagrety" y̆evo vyısokopreosvıax̨enstvo y̆ex̨e nado sumety. Vprocem, y̆a ne nedooqenivay̆u silyı Valytera. On mojet provernuty necto podobnoy̆e.

Propovednik skorcil minu, stav pohojim na zamorskuy̆u obezy̆anku:

— Tyı sam sebe protivorecix, Lıudvig. To ``ne sumey̆et", to ``mojet". Pocemu byı tebe vse-taki ne otdaty im to, cto u tebıa y̆esty, i ne vputyıvatsa v seryeznyıy̆e nepriy̆atnosti? V Riapano jivut opasnyıy̆e lıudi. Stoy̆it li perebegaty im dorogu?

— A tyı zadumyıvalsa, cto slucitsa, kogda zagovorx̨iki polucat kamni?

On razvel rukami:

— Y̆a ne znay̆u.

— V tom-to i beda. Byıty mojet, oni uberut menıa kak lixnevo svidetelıa, byıty mojet, Kristinu. Y̆a horoxo uspel uznaty kolduna. Otdavaty v y̆evo ruki to, cto on hocet, — opasno. Potıanu vremıa.

Staryıy̆ pelikan rassey̆anno vyıter krovy so x̨eki, otcevo ta ne stala boley̆e cistoy̆:

— Eto neskolyko ne po zakonam Bojyim, i tebe, navernoy̆e, stranno slyıxaty podobnoy̆e ot menıa, no, y̆esli on tak opasen, pocemu byı prosto vse ne zaverxity? Vzıaty pistolet i raznesti y̆emu golovu? Hotıa byı za to, cto iz-za nevo tyı provel v podzemnoy̆ kamere paru mesıaqev i y̆edva ne otdal Bogu duxu.

— Voobx̨e-to otdal, i, y̆esli byı ne Sofiy̆a, y̆a byı s toboy̆ sey̆cas ne razgovarival, — napomnil y̆a y̆emu. — Y̆a dumal nad etim, no menıa ostanavlivay̆et to, cto u nevo y̆esty qennoy̆e znaniy̆e o temnom kuzneqe. Stoly vajnyıy̆e svedeniy̆a mogut byıty poleznyı dlıa Bratstva, a pulıa, drug Propovednik, raz i navsegda postavit tocku. Ot trupa oceny tıajelo cto-nibudy uznaty.

— Da jivyım-to on toje tebe mnogo ne rasskajet.

— Posmotrim. Y̆a dumay̆u o tom, cto byı slucilosy, y̆esli byı Valyter polucil glaz serafima? Otdal mineral kuznequ? Vyımanil y̆evo i ubil, kak on govorit? Ili je poznakomilsa i popyıtalsa ispolyzovaty mastera v svoy̆ih qelıah? No samyıy̆ glavnyıy̆ vopros, kotoryıy̆ ne day̆et mne pokoy̆a, Propovednik, kak on svıajetsa s etim neulovimyım ney̆izvestnyım celovekom?

Tot aj podskocil:

— Tyı namekay̆ex, cto on znay̆et, kuda sleduy̆et otpravity vestocku, ctobyı kuzneq naznacil vstrecu!

— Imenno.

— I hocex vyısledity y̆evo svıaznovo.

— ``Svıaznovo"? Propovednik, gde tyı uslyıxal eto slovo?

— V bordelıah, — nebrejno mahnul on. — Tam xpionov ne menyxe, cem xlıuh. Vo vsıakom slucay̆e, v nekotoryıh. Poroy̆ takiy̆e razgovoryı vedutsa, cto y̆a jaley̆u o svoy̆ey̆ smerti i o tom, cto nekomu prodaty cujiy̆e tay̆nyı. Da i grehovno eto. Tak nascet istoriy̆i s Urbanom. Kak y̆a ponimay̆u, u tebıa y̆esty plan?

— Y̆ego nametki. — Y̆a posmotrel na kuklu, medlenno krutıax̨uy̆usıa pod potolkom.

— Nu y̆asno. Iz tebıa slova ne vyıjmex. Tyı huje xpiona. Oni hotıa byı boltay̆ut.

Y̆a rassmey̆alsa.

— A Kristina? S ney̆-to kak? Y̆a cto-to ne viju, ctobyı tyı spexil vernuty y̆ey̆ propaju.

— Ona rasstalasy s kinjalom po dobroy̆ vole. Dumay̆u, cto projivet bez nevo y̆ex̨e kakoy̆e-to vremıa. Do teh por poka vse ne koncitsa. Inace, boy̆usy, vnovy pojertvuy̆et im radi neponıatnyıh vyısxih qeley̆, i rıadom uje ne okajetsa menıa dlıa tovo, ctobyı vernuty y̆evo.

— Razumno, hotıa i neskolyko jestoko… O! Kajetsa, nacinay̆etsa!

Y̆a raspahnul okno, tak kak toje uslyıxal trubyı gornistov.

Obe storonyı uliqi byıli zaprujenyı narodom.

— Kak budto dıadıuxka tvoy̆ey̆ vedymyı priy̆ehal, a ne kardinal, — uslyıxal y̆a vozle uha polnyıy̆ skeptiqizma golos Propovednika i otvetil:

— Slava ob Urbane bejit vperedi y̆evo. Mnogiy̆e scitay̆ut y̆evo y̆edva li ne svıatyım, oplotom veryı i budux̨im Papoy̆. Horoxay̆a reputaqiy̆a, pravilynyıy̆e postupki, vsenarodnay̆a lıubovy. On lucxiy̆ pravednik iz teh, cto y̆esty v Qerkvi na dannyıy̆ moment. Nikakih skandalov, vzıatok, podkupov, ubiy̆stv i procevo. Lix vera, a realynay̆a vera, kak tyı pomnix iz naxey̆ proxloy̆ besedyı na etu temu, zarajay̆et lıudey̆.

— Daje takih, kak tyı? — poddel on menıa.

— Lıubyıh.

Proqessiy̆a vyıglıadela vnuxitelyno i seryezno. Vperedi marxirovala rota kantonskih nay̆emnikov, nahodıax̨ay̆asıa na slujbe goroda. V paradnyıh zolotistyıh kirasah, xlemah s plıumajami, s serebrıanyımi alebardami na cernyıh drevkah. Oni vyıglıadeli y̆arko, slovno rojdestvenskay̆a igruxka, i ih prazdnicnay̆a odejda silyno otlicalasy ot toy̆ koji, xersti i stali, v kotoryıh oni predpocitali ne radovaty tolpu, a ubivaty y̆ey̆e.

Za rotoy̆ nay̆emnikov sledovala kavalykada pocetnyıh jiteley̆ goroda i blagorodnyıh gospod. V dorogih odejdah iz barhata, mehovyıh xubah, oni pyıtalisy perex̨egolıaty drug druga. Vyısxey̆e duhovenstvo vyıglıadelo cuty skromney̆e, no nenamnovo. Mestnyıy̆e kliriki dlıa vstreci dorogovo gostıa dostali svoy̆i lucxiy̆e narıadyı. Kardinal Urban na fone vstrecay̆ux̨ey̆ delegaqiy̆i vyıglıadel nastoy̆ax̨im skromnıagoy̆ — xirokopolay̆a alay̆a xlıapa, krasnay̆a sutana iz horoxey̆ xersti, na plecah pelerina iz meha belovo krolika.

Y̆a vpervyıy̆e videl celoveka, kotorovo kogda-to spas. Y̆emu okazalosy okolo semidesıati, no dlıa svoy̆evo vozrasta on otlicno derjalsa v sedle. Osanka, posadka golovyı i to, kak uverenno i spokoy̆no on pravil jerebqom, govorili o tom, cto sil etomu celoveku ne zanimaty, nesmotrıa na vnexnıuy̆u hudobu, beskrovnyıy̆e gubyı i zapavxiy̆e glaza. Navernoy̆e, v molodosti u nevo byılo priy̆atnoy̆e liqo, no sey̆cas sozdavalosy vpecatleniy̆e, cto y̆a smotrıu na staruy̆u hix̨nuy̆u ptiqu.

Kardinal to i delo podnimal ruku, blagoslovlıay̆a privetstvuy̆ux̨ih y̆evo jiteley̆ goroda.

Y̆ego vyısokopreosvıax̨enstvo soprovojdala vosymerka gvardey̆qev-alybalandqev v paradnyıh mundirah i beretah. Vse kak odin svetlovolosyıy̆e i vyısocennyıy̆e. Takje v svite prisutstvovalo neskolyko klirikov v skromnyıh odejdah, sredi kotoryıh y̆a zametil odnovo kalikveqa. Za svıax̨ennoslujitelıami y̆ehali slugi. Dva desıatka celıadi, na kotoryıh nikto i ne smotrel. A zrıa.

Y̆a srazu uvidel tovo, kovo jdal. Celovek v belo-koricnevom plax̨e palomnika. Smugloliqiy̆, temnoglazyıy̆, v usah i nepokryıtyıh volosah bolyxoy̆e kolicestvo sedinyı, a na pravoy̆ skule vyıdelıalsa zametnyıy̆ izdaleka krestoobraznyıy̆ xram. On s interesom glazel po storonam i toje pocti srazu uvidel menıa, napolovinu vyısunuvxegosıa iz okna. V y̆evo temnyıh glazah na mgnoveniy̆e prostupila zolotistay̆a jeltizna, i v sleduy̆ux̨uy̆u sekundu on smotrel v druguy̆u storonu, a y̆ex̨e cerez polminutyı uje skryılsa za povorotom so vsey̆ kavalykadoy̆.



Valyter prixel v naznacennoy̆e vremıa, vyınyırnuv iz uzkovo, propahxevo kryısami pereulka. On priodelsa, tocno zajitocnyıy̆ gorojanin, i y̆evo volosyı i usyı zametno pobeleli.

— Gde kolyqo? — sprosil u menıa koldun vmesto privetstviy̆a.

— Gde Kristina?

— O ney̆ ne bespokoy̆sıa. U ney̆e s ostalynyımi svoy̆a zadaca. Ot tebıa mnogoy̆e ne trebuy̆etsa, van Normay̆enn. Prosto day̆ mne podobratsa k kardinalu.

— Ostalynyıy̆e budut tam je?

— Dumay̆u, tebe plevaty na ostalynyıh. Kristinyı na prazdnike net. Ona jdet s loxadymi v uslovlennom meste. Y̆esli vse proy̆det horoxo, uy̆edem iz goroda kak mojno byıstrey̆e i kak mojno dalyxe. Tebe y̆a sovetuy̆u sdelaty to je samoy̆e, raz uj tyı idex so mnoy̆ do konqa.

— Ne do konqa, — vozrazil y̆a. — Lix do tovo mesta, kogda moy̆a pomox̨ bolyxe ne budet nujna Kristine.

On rassmey̆alsa:

— Y̆a vsegda znal, cto straji drug za druga gotovyı risknuty golovoy̆.

— I polyzuy̆exysa etim.

Koldun otvesil legkiy̆ poklon:

— Glupo otriqaty. Idem. U nas menyxe polucasa.

Uliqi byıli zaprujenyı narodom, i Propovednik s utra poxutil, cto daje mertvyıy̆e, y̆esli byı oni mogli, spolzlisy byı sıuda s gorodskih pogostov. Y̆a ne stal rasstray̆ivaty starovo pelikana i govority y̆emu, cto y̆avleniy̆e angela — eto ne boley̆e cem falyxivka, sozdannay̆a kuckoy̆ moxennikov.

Myı opazdyıvali, prodiray̆asy cerez lıudskoy̆e stolpotvoreniy̆e. V konqe konqov Valyteru eto nadoy̆elo, i on vse je ispolyzoval kakoy̆-to fokus. Gorojane, stoy̆avxiy̆e pered nami, nevolyno stali delaty xag v storonu, tolkay̆a drugih i nastupay̆a im na nogi, a myı vklinivalisy v otkryıvay̆ux̨iy̆esıa i tut je zahlopyıvay̆ux̨iy̆esıa brexi. Vprocem, vskore etot effekt propal, koldun, opasay̆asy privlec vnimaniy̆e, ostorojnical, i nam snova prixlosy rabotaty loktıami, kak samyım obyıcnyım lıudıam.

Vyıhod na Maluy̆u Karetnuy̆u okazalsa perekryıt qepy̆u straji — ona uderjivala svobodnuy̆u proy̆ezjuy̆u casty dlıa bogatyıh priglaxennyıh, spexivxih k bogoslujeniy̆u.

— Proklıatye! — zlo rugnulsa Valyter. — Pridetsa iskaty drugoy̆ puty.

— Postoy̆, — skazal y̆a, dostav svoy̆ ``propusk", i pokazal kinjal odnomu iz soldat. — Myı iz Bratstva.

Tot daje spority ne stal i, ne trebuy̆a u kolduna pokazaty orujiy̆e, otkryıl nam puty na uliqu.

— I bez vsıakovo volxebstva, — probormotal sebe pod nos moy̆ nedrug.

Tepery k gorodskoy̆ svıatyıne prodvigatsa stalo gorazdo legce — na prıamoy̆ doroge, svobodnoy̆ ot lıudey̆, myı lix dvajdyı postoronilisy, propuskay̆a zapozdavxih blagorodnyıh gospod.

Daleko-daleko gulko udarili casyı na novoy̆ ratuxe, i ih boy̆ podhvatili kolokola. Torjestvennay̆a messa nacalasy. Ohrana, puskavxay̆a na osnovnoy̆e dey̆stvo, okazalasy kuda boley̆e vnuxitelynoy̆ — kantonskiy̆e nay̆emniki, s nimi troy̆e alybalandqev i para klirikov v prostyıh seryıh rıasah. Vse te, kto ne smog popasty na plox̨ady, zanıali sosedniy̆e uliqi, kryıxi okrestnyıh domov i daje derevy̆a.

— Tolyko po priglaxeniy̆am, — skazal mne cernoborodyıy̆, pohojiy̆ na medvedıa nay̆emnik.

— Myı straji, — otvetil y̆a.

— Viju, cto straji. No prikaz puskaty tolyko po gramotam, na kotoryıh pecaty burgomistra, — nicuty ne smutilsa tot. — Y̆esty gramota?

— Y̆esty koy̆e-cto polucxe, — otvetil y̆a i dostal persteny, podarennyıy̆ mne v Vione.

— Kupity, cto li, hocex? — opexil tot.

— Pogodi, dobryıy̆ celovek. — Klirik, prisluxivavxiy̆sıa k naxemu razgovoru, protıanul ruku. — Day̆ posmotrety.

Y̆a polojil bezdeluxku y̆emu na ladony.

— Y̆a licnyıy̆ duhovnik y̆evo vyısokopreosvıax̨enstva. I pomnıu eto kolyqo. Ono dey̆stvitelyno kogda-to prinadlejalo y̆emu. Tebıa priglasil kardinal?

— Net, — ne stal y̆a lgaty, cuvstvuy̆a, kak naprıagsıa Valyter. — No kogda-to y̆a okazal uslugu y̆evo vyısokopreosvıax̨enstvu i uveren, cto on pomnit menıa. Y̆a i moy̆ drug hotim prisutstvovaty na stoly vajnom bogoslujeniy̆i.

Klirik kivnul lyısoy̆ golovoy̆:

— Cto je, y̆a ponimay̆u vaxe jelaniy̆e. Kto y̆a takoy̆, ctobyı mexaty priobx̨itsa k cudu i Gospodu.

— No priglaxeniy̆e… — popyıtalsa zaspority podoxedxiy̆ kapitan nay̆emnikov.

— Etot celovek sdelal horoxey̆e delo dlıa Qerkvi i licno y̆evo vyısokopreosvıax̨enstva. Vyı jelay̆ete potom oby̆asnıaty kardinalu, pocemu on ne byıl dopux̨en na bogoslujeniy̆e?

— Konecno net!

— Togda propustite ih.

Kantoneq neohotno mahnul svoy̆im lıudıam, i te podnıali alebardyı, otkryıvay̆a dorogu.

— Vse kuda legce, cem y̆a ojidal, — usmehnulsa koldun, kogda ohrana ostalasy daleko pozadi.

Propovednik, vse eto vremıa tocno teny sledovavxiy̆ za mnoy̆, nakoneq-to dal volıu svoy̆im emoqiy̆ami:

— Tyı mnogovo ne ojiday̆ex, certov ublıudok!

Po scasty̆u, krome menıa, y̆evo nikto ne slyıxal.

— Cto tepery? — sprosil y̆a.

— Tyı svoy̆e delo sdelal, van Normay̆enn. Mojex naslajdatsa predstavleniy̆em. A mne nado podobratsa k kardinalu kak mojno blije.

No y̆a ne dal y̆emu uy̆ti, polojiv ruku na pleco.

— Ne tak byıstro. Tyı prixel vmeste so mnoy̆ i uy̆dex so mnoy̆.

— Kak znay̆ex. Tolyko ne mexay̆, — legko soglasilsa on.

— Eto byılo byı prox̨e osux̨estvity, y̆esli byı tyı rasskazal, cto sobiray̆exysa sdelaty.

— Povery, nikto nicevo ne poy̆met. Myı uy̆dem prejde, cem oni zametıat, cto cto-to slucilosy.

Y̆ego slova ne vnuxali doveriy̆a, no y̆a nadey̆alsa na kozyıri v rukave, hotıa obyıgraty opyıtnovo kolduna ne tak prosto.

Usilennyıy̆ svıatoy̆ magiy̆ey̆ golos kardinala raznosilsa nad plox̨ady̆u. On cital na staroqerkovnom, y̆azyıke vremen Konstantina. Lıudi, raspolojivxiy̆esıa na plox̨adi plotnoy̆ tolpoy̆, molilisy, xevelıa gubami i v obx̨em-to ne slixkom horoxo vidıa, cto proy̆ishodit za spinami vperedi stoy̆ax̨ih.

Y̆a skorey̆e pocuvstvoval, cem uvidel, kak k nam prisoy̆edinilsa tretiy̆ celovek v kaftane slugi burgomistra i vyısokoy̆ xapke. Sudıa po vsemu, Cezare proniknuty sıuda okazalosy gorazdo legce, cem koldunu.

— U menıa vse gotovo, — xepnul on Valyteru. — Gotthod na meste i jdet tvoy̆ey̆ komandyı.

— Pozabotsa o svoy̆ey̆ zadace, — otvetil tot.

Y̆a uje, kajetsa, znal, v cem zaklıucay̆etsa rabota nay̆emnika. Vozle ``svıatovo" mesta postavili raspıatiy̆e, rıadom s nim nahodilosy nebolyxoy̆e vozvyıxeniy̆e, s kotorovo vyıstupal kardinal. Vnizu stoy̆ali predstaviteli duhovenstva i gorodskih vlastey̆. Tolpa ostavila svobodnyım lix nebolyxoy̆ pıatacok plox̨adi, to samoy̆e mesto, gde nahodilsa otpecatok ``angela".

Kondotyer vnezapno vyıbrosil ruku, i y̆a, ojidavxiy̆ cevo-to podobnovo, blokiroval y̆ey̆e predplecyem, ne dav stiletu udarity menıa v gorlo.

V sleduy̆ux̨ey̆e mgnoveniy̆e v qentre Kruso razverzlasy bezdna i voqarilsa ognennyıy̆ ad.



…Krıucy̆a, svisavxiy̆e s potolka, byıli v bezobraznom sostoy̆aniy̆i. Rjavyıy̆e, s ostatkami temnoy̆ ploti na granıah, oni smerdeli zastareloy̆ krovy̆u i gnilyım mıasom. Tocno tak je pahla i rexetka, na kotoroy̆ zdesy lıubili podjarivaty teh, kto ne raskay̆ivay̆etsa v svoy̆ih oxibkah.

Nesmotrıa na to cto v bolyxoy̆ jarovne besnovalosy plamıa, v podvalah qentralynoy̆ tıurymyı Kruso okazalosy jutko holodno.

Master doprosa, nemolodoy̆ jilistyıy̆ subyekt s y̆arkimi lucistyımi golubyımi glazami, snıal kojanyıy̆ fartuk, percatki i peredal stalynoy̆ prut odnomu iz pomox̨nikov.

— S etim vse.

``Etot" lejal rastıanutyım na xirokoy̆ mıasniqkoy̆ stolexniqe i bolyxe pohodil ne na celoveka, a na kusok otbivnoy̆. Y̆a ne prisutstvoval na pyıtke, tak cto s trudom uznal otqa Gotthoda, kanonika sobora Svıatoy̆ Mariy̆i v Braselovette. Po perelomannyım konecnostıam i krovavyım puzyırıam, kotoryıy̆e naduvalisy i lopalisy u nevo na gubah, byılo ponıatno, cto on ne protıanet i polucasa. Vot-vot ispustit duh. Propovednik, uvidev takoy̆e zrelix̨e, razvernulsa na kablukah i, nicevo ne skazav, vyıxel von. On predpocital ne smotrety na to, cto y̆emu byılo nepriy̆atno.

Pugala y̆a nigde ne videl s momenta sobyıtiy̆ na plox̨adi. Vpolne vozmojno, cto y̆evo stoy̆it iskaty vozle trupovozok, kotoryıy̆e sey̆cas v bolyxom kolicestve sobiray̆utsa na qentralynoy̆ uliqe.

Roman, moy̆ staryıy̆ znakomyıy̆, s kotoryım y̆a perejil napadeniy̆e rugaru v Hrustalynyıh gorah, privalivxisy k stene, smotrel na umiray̆ux̨evo s polnyım ravnoduxiy̆em. Kak na tot samyıy̆ kusok otbivnoy̆, o kotorom y̆a tolyko cto upomıanul.

— Zakancivay̆ s nim, master.

Palac vzıal so stolika xirokiy̆ mıasniqkiy̆ noj i odnim dvijeniy̆em prekratil stradaniy̆a umiray̆ux̨evo.

— Kardinalyskay̆a milosty, — oby̆asnil mne qigan, tocno opravdyıvay̆asy.

— Tyı privel menıa uvidety y̆ey̆e?

— Y̆ego vyısokopreosvıax̨enstvo platit svoy̆i dolgi. Hotıa byı casty ih.

— Stranno. — Y̆a posmotrel, kak dva pomox̨nika palaca razrezay̆ut verevki, stıagivay̆ux̨iy̆e ruki i nogi trupa, a zatem sbrasyıvay̆ut okrovavlennoy̆e telo v telejku. — Somnevay̆usy, cto mne nujna byıla smerty klirika.

— Tyı ne ponıal. Tyı mog okazatsa na etom okrovavlennom stole. Nekotoryıy̆e iz okrujeniy̆a Urbana scitay̆ut, cto imenno tam ono i doljno byıty. I oni byıli oceny ubeditelynyı.

— I y̆a scastlivo izbejal etoy̆ ucasti, potomu cto… — Y̆a predostavil y̆emu zakoncity moy̆u frazu.

— Potomu cto myı znakomyı, van Normay̆enn. Potomu cto tyı snova spas jizny kardinalu, i on v dolgu pered toboy̆. Potomu cto te, kto hotel rastıanuty tebıa na dyıbe, sey̆cas uje napravlıay̆utsa v Ny̆ugort. Im trebuy̆etsa pokay̆aniy̆e za ih gluposty. A lucxe vsevo y̆evo dobitsa, izucay̆a vereskovyıy̆e pustoxi.

Telo uvezli, ostalsa lix gustoy̆, lipkiy̆ zapah krovi, napolnivxiy̆ holodnoy̆e pomex̨eniy̆e.

— Nu, cto je. Peredavay̆ kardinalu moy̆i blagodarnosti. Hotıa y̆a i ne rad, cto celovek umer.

— Vzdumal sebıa vinity?

— Net. Y̆a postupil pravilyno. A oni znali, cem riskuy̆ut.

— Vmesto pıati soten trupov v Kruso segodnıa byılo byı tri-cetyıre tyısıaci, y̆esli byı nikto iz nas ne okazalsa gotov k podobnomu povorotu sobyıtiy̆.

Y̆a s sodroganiy̆em vspomnil vzryıv, y̆arkuy̆u vspyıxku i potoki jivovo, pohojevo na zmey̆, zolotovo ognıa, rinuvxegosıa vo vse storonyı. Za sekundu on projeg v rıadah molivxihsıa brex, jgucim molotom udaril po klirikam i otprıanul ot siy̆ay̆ux̨ey̆ svetom pregradyı. Kupol stremitelyno rvanul vverh i v storonyı, nakryıv snacala plox̨ady, zatem okrestnyıy̆e doma, a potom i uliqi, poglox̨ay̆a v sebıa strannoy̆e, oslepitelyno-zolotoy̆e plamıa, ne davay̆a y̆emu rasprostranitsa i navredity y̆ex̨e komu-nibudy. No vse ravno etovo okazalosy nedostatocno, ctobyı spasti vseh.

Panikuy̆ux̨ay̆a tolpa, begux̨ay̆a proc, voy̆ux̨ay̆a ot ujasa iz-za tovo, cto koneq sveta, obex̨annyıy̆ angelom, uje nastupil, razlucila menıa s Cezare v tot ``cudesnyıy̆" dlıa nas oboy̆ih moment, kogda myı sobiralisy ubity drug druga.

Master doprosa vyısluxal xepot pomox̨nika i soobx̨il:

— Vtoroy̆ gotov k razgovoru. Smotrety budete?

— Vtoroy̆? — udivilsa y̆a. — Vyı smogli vzıaty dvoy̆ih?

— K sojaleniy̆u, tolyko dvoy̆ih. Ih, v otliciy̆e ot ostalynyıh, poy̆maty okazalosy ne tak uj i slojno. Y̆a pristavil k nim lıudey̆ srazu posle naxevo s toboy̆ razgovora.

Myı proxli v sosedniy̆ zal, zerkalynuy̆u kopiy̆u predyıdux̨evo, i y̆a uvidel starovo aptekarıa. On sidel za stolom, y̆evo ruki i nogi byıli zafiksirovanyı xirokimi kojanyımi remnıami, a golova zasunuta v mehanizm kray̆ne nepriglıadnovo vida. Tiski plotno ohvatyıvali nijnıuy̆u celıusty i temıa.

— Horoxo, — pohvalil Roman palaca i sklonilsa nad Filippom. — Udobno li vam, metr?

Tot ne smog nicevo otvetity, lix promyıcal cto-to neclenorazdelynoy̆e i umolıay̆ux̨ey̆e.

— Dumay̆u, cto ne slixkom. Navernoy̆e, stranno spraxivaty takoy̆e u celoveka, nahodıax̨egosıa v cerepodrobilke. Davay̆te y̆a nemnogo rasskaju vam, cto eto takoy̆e. Na samom dele, metr, vse predelyno prosto. Palac krutit vint, i tiski nacinay̆ut sdavlivaty vaxu golovu. Sperva lomay̆utsa zubyı, a byıty mojet, nijnıay̆a celıusty. Y̆esli cestno, y̆a nikogda ne vnikal v detali. Zatem lopnet cerep, no, poveryte, posle etovo vyı projivete dostatocno dolgo, ctobyı pojalety o glupostıah, kotoryıy̆e ucinili.

Filipp zaplakal, zamyıcal aktivney̆e.

— Moy̆ vam sovet, aptekary. Y̆esli hotite izbejaty lixney̆… golovnoy̆ boli i zaslujity prox̨eniy̆e kardinala, na kotorovo vyı tak bezdarno pokuxalisy, pokay̆tesy, soznay̆tesy i nacinay̆te sotrudnicaty. — Roman pohlopal uznika po plecu. — Y̆a vernusy cerez polcasa i s radosty̆u uslyıxu pravilynyıy̆ otvet. Idem, Lıudvig.

Myı pokinuli pyıtocnuy̆u, podnıavxisy na dva etaja vverh, peresekli pustoy̆ tıuremnyıy̆ dvor, minovali ohranu i vyıxli na uliqu. Eto byıla okray̆ina goroda, v dvuh xagah ot vnexney̆ stenyı.

— Pogovorim. — On zabralsa na goru kirpica, svalennovo rabocimi, sobiravximisıa remontirovaty ukrepleniy̆e.

— Zdesy? — udivilsa y̆a, no prisoy̆edinilsa k nemu.

— Vidno vsıu okrugu. Uj lucxe, cem kabinet nacalynika tıurymyı, gde kajdyıy̆ durak mojet podsluxaty.

— U vas y̆esty ofiqialynay̆a versiy̆a proy̆izoxedxevo v gorode?

— A kogda y̆ey̆e ne byılo? — On usmehnulsa v usyı. — Lıudi uje rabotay̆ut y̆azyıkami, sluhi odin huje drugovo letıat po dorogam da mnojatsa v traktirah. Za mesıaq uznay̆ut vse i vezde. Eto slucilosy oceny ne vovremıa. Hotıa kogda podobnyıy̆e sobyıtiy̆a voobx̨e mogut byıty k mestu? Nam pridetsa prilojity massu usiliy̆, ctobyı sgladity posledstviy̆a proy̆isxedxevo.

— Nam?

— Y̆a sluju kardinalu, a on — Qerkvi. Tak cto v dannom kontekste daje koldunu i rugaru mojno govority ``nam". Naskolyko y̆a ponıal, vseh sobak povesıat na zlobnovo dy̆avolopoklonnika, kotoryıy̆ hotel isportity radosty veruy̆ux̨im, oskvernity svıatyınıu, popraty zakonyı Bojyi i procay̆a, procay̆a, procay̆a.

Y̆a ojidal cevo-to podobnovo:

— V obx̨em, skazka o cudovix̨e, rexivxem ustroy̆ity bessmyıslennoy̆e ubiy̆stvo.

— Ili kakoy̆-nibudy jertvennyıy̆ ritual. Lıudıam soverxenno nezacem znaty nastoy̆ax̨ih pricin.

— A myı ih znay̆em? Eti samyıy̆e pricinyı?

Roman ne otvetil, i eto govorilo o tom, cto on, kak i y̆a, terıay̆etsa v dogadkah.

— Scitay̆ex, cto koldun urovnıa Valytera sposoben ustroy̆ity ognennyıy̆ vulkan v qentre goroda? — zadal y̆a y̆ex̨e odin vopros.

On posmotrel na menıa dolgim, tıajelyım vzglıadom:

— I tyı dumay̆ex tocno tak je. Zapomni eto, y̆esli dorojix svoy̆ey̆ xkuroy̆. — I, cuty smıagcivxisy, proy̆iznes: — Y̆esty vex̨i, o kotoryıh ne stoy̆it boltaty, Lıudvig. Naprimer, y̆a staray̆usy pomalkivaty, cto polnay̆a luna ne slixkom horoxo vliy̆ay̆et na menıa v posledney̆e vremıa.

Myı oba usmehnulisy tolyko nam ponıatnoy̆ xutke.

— Tak cto vinovat Valyter. Poka ne budet dokazano obratnoy̆e. A ono, kak tyı ponimay̆ex, dokazano, skorey̆e vsevo, ne budet. Vo vsıakom slucay̆e, tolpe.

— No tyı ne verix, cto koldun markgrafa Valentina imey̆et stolyko sil.

— Razve eto tak vajno? — On ustalo poter veki.

— Dlıa menıa — vajno.

Qyıgan sdalsa:

— Ne verıu. Y̆a znay̆u o temnom iskusstve ne ponaslyıxke. Na moy̆ vzglıad, podobnoy̆e ne mojet provernuty nikto iz teh, s kem y̆a znakom. Zdesy nujna mox̨ velefa ili kakovo-nibudy legendarnovo carodey̆a iz Temnolesy̆a. Ili oceny seryeznovo demona. Takovo, kto odnim x̨elckom palyqev sjigay̆et pıaty soten dux i y̆edva ne prolamyıvay̆et x̨it desıati gotovyıh k otrajeniy̆u ataki klirikov. Koldunyı na takoy̆e ne sposobnyı. Inace mirom pravili byı oni, a ne knıazy̆a i qerkovniki.

— Y̆esli on byıl nastolyko silen, cto ne boy̆alsa klirikov, to pocemu otstupil? Pocemu ne ubil vseh, kto nahodilsa na plox̨adi?

— Y̆a ne znay̆u.

Y̆a zadumalsa.

— No y̆esli eto byıl ne Valyter, a kto-to inoy̆, to slucivxey̆esıa na plox̨adi — sovpadeniy̆e? Komanda zagovorx̨ikov, pyıtavxihsıa ubity kardinala, i ney̆izvestnyıy̆, rexivxiy̆ ustroy̆ity lokalynyıy̆ apokalipsis?

— Inovo oby̆asneniy̆a u menıa net. I ne tolyko u menıa. Inkviziqiy̆a cexet v zatyılke, razvodit rukami i lihoradocno listay̆et grimuaryı. Tolyko dumay̆etsa mne, cto bez tolku eto vse. Nikovo myı ne nay̆dem. No im nujen kozel otpux̨eniy̆a. Lucxe vsevo tot hagjit, o kotorom tyı govoril. Otlicnay̆a kandidatura dlıa kostra. Obyıvateli ne jaluy̆ut cujezemqev i inoverqev, s radosty̆u sojgut zlobnoy̆e cudovix̨e, voshitıatsa tem, cto vozmezdiy̆e Qerkvi nastiglo prestupnika, i uspokoy̆atsa. Nu a y̆esli ne poy̆may̆em etovo hagjita, nay̆dem drugovo. Ili je pridetsa ispolyzovaty bednıagu-aptekarıa. Morx̨ixysa? Ne stoy̆it. Eto zvucit jestoko, no inace prosto nelyzıa. Simvolyı veryı i silyı doljnyı ostavatsa nezyıblemyı. Inace nacnetsa haos.

— Kogda vyı ponıali, cto sled angela — falyxivka?

Roman slojil uzlovatyıy̆e palyqi v kulak, poter im zatyılok:

— Slixkom pozdno dlıa tovo, ctobyı vse ostanovity. Sluhi poleteli, a palomniki i mestnyıy̆e kliriki-tupiqi uje skolacivali krest da jgli sveci. Pervyıy̆ je inkvizitor s magiy̆ey̆ vse raskusil.

— Valytera nelyzıa nazvaty nay̆ivnyım. On ne veril, cto y̆evo obman proderjitsa dolgo, znacit, znal, cto vyı ne ostanovite predstavleniy̆e.

— U nas ne byılo vyıbora. Cto myı doljnyı byıli delaty? Oby̆avity vo vseuslyıxaniy̆e, cto kucka moxennikov rexila naduty veruy̆ux̨ih? Cto nikakovo cuda net, a poslednevo angela videli poltoryı tyısıaci let nazad? Zacem rubity suk, na kotorom sidix, Lıudvig? Lıudi hotıat verity v cudesa, i kto myı takiy̆e, ctobyı razruxaty ih illıuziy̆i?

Y̆a lix neveselo rassmey̆alsa:

— Mne povezlo, Roman. Moy̆a rabota gorazdo prox̨e.

— A moy̆a ne jdet. — On vstal i protıanul ruku. — Tyı spas mne jizny v teh proklıatyıh gorah, i y̆a ne lıublıu byıty doljen. Y̆a sobiray̆u sluhi i informaqiy̆u. Staray̆a ferma v cetverti ligi za gorodom, y̆esli y̆ehaty cerez Koxacyi vorota. Y̆a smogu priderjivaty etu novosty paru-troy̆ku casov, ne bolyxe. Uvezi jenx̨inu kak mojno dalyxe, y̆esli ne hocex, ctobyı ona sgnila v podvale.

— Oni budut preduprejdenyı. Vse.

— I cto s tovo? Myı ih poy̆may̆em. No y̆esli straja ne budet sredi nih, y̆ey̆e nikto ne stanet iskaty. Obex̨ay̆u.

— Spasibo, Roman. Myı v rascete.

— Boy̆usy, cto net. Eto byıla usluga za uslugu.



Ferma vyıglıadela zabroxennoy̆, no v y̆ey̆e okoxkah gorel prigluxennyıy̆ svet.

Nebo byıstro temnelo.

Y̆a napravilsa k vorotam, dumay̆a, cto snova dey̆stvuy̆u naobum, pryıgay̆u v omut golovoy̆ i Valyteru v obx̨em-to nicevo ne stoy̆it sdelaty to, cto ne polucilosy u nay̆emnika.

Y̆a zametil, kak drognula zanaveska — nablıudavxiy̆ za dorogoy̆ uvidel menıa. Tem lucxe.

Vorota byıli ne zapertyı. Y̆a voxel vo dvor — grıaznyıy̆, s dvumıa ogromnyımi lujami i telegoy̆, perevernutoy̆ nabok. Dvery v dom otkryılasy, i na poroge poy̆avilsa Cezare. Posmotrel na menıa tıajelyım vzglıadom, ne ubiray̆a ruki s visıax̨ey̆ na poy̆ase dagi:

— Tyı odin?

— Kak vidix.

On postoronilsa i, kogda y̆a voxel, ostalsa na uliqe, rexiv proverity, ne prixel li za mnoy̆ kto-to y̆ex̨e.

Vnutri pahlo staryım zaplesnevevxim domom, pol byıl zemlıanoy̆. Bolyxuy̆u casty y̆edinstvennoy̆ komnatyı zanimal ostyıvxiy̆ ocag. Skudnay̆a kresty̆anskay̆a mebely, a takje prıalka okazalisy sdvinutyı k dalyney̆ stene. V uglu, zavernuvxisy v tonkoy̆e odey̆alo, spal hagjit. Koldun, do etovo cto-to pisavxiy̆, tepery smotrel na menıa.

Kristina poryıvisto brosilasy ko mne, krepko obnıala:

— Y̆a dumala, tyı pogib! Valyter skazal, cto tam, gde tyı stoy̆al, nikto ne vyıjil!

Byıvxiy̆ sluga markgrafa Valentina vstretil moy̆ krasnorecivyıy̆ vzglıad s ponimay̆ux̨ey̆ ulyıbkoy̆:

— Kak vidno, y̆a oxibalsa. Etot Cezare vecno vse putay̆et.

— I nedodelyıvay̆et. Oni pyıtalisy izbavitsa ot menıa.

— Cto?! — Ona gnevno nahmurila brovi, rezko povernuvxisy k Valyteru. — Tyı je obex̨al!

— Ey̆! Ey̆! Uspokoy̆sıa.

— Uspokoy̆itsa?! — razozlilasy Kristina. — Tyı sukin syın, Valyter! Ne tolyko zavalil delo, no i hotel ubity tovo, kto soglasilsa pomoc nam!

— Eto byıla iniqiativa Cezare. Y̆a ne daval y̆emu takih rasporıajeniy̆. Klıanusy!

Ona xagnula k nemu s y̆avnyım namereniy̆em udarity, no y̆a vzıal y̆ey̆e za predplecye:

— Myı uhodim. Prıamo sey̆cas.

— Neujeli? — vkradcivo skazal on, polojil pero na stol i vstal.

Y̆a vosprinıal eto kak ugrozu i opustil ruku na kinjal.

— Dumay̆ex, amulet tvoy̆ey̆ vedymyı spaset tebıa ot magiy̆i?

— Proverim?

— Prekratite! Oba! — kriknula Kristina, razbudiv Adilıa. — I tak vse ploho. Stolyko mesıaqev rabotyı nasmarku! Myı ne znay̆em, gde oteq Gotthod i Filipp, oni do sih por ne prixli. Xans vyıy̆ti na kuzneqa poterıan! A vyı dumay̆ete tolyko o tom, kak pustity drug drugu krovy!

Valyter mirolıubivo podnıal ruki vverh i vnovy ugnezdilsa na stule:

— Prejde cem tyı primex rexeniy̆e, uznay̆ u van Normay̆enna, kak tot naxel nas. On, navernoy̆e, velikiy̆ volxebnik, raz okazalsa zdesy. Vedy tyı y̆emu ne govorila o naxem ubejix̨e?

— Ne govorila. — Ona voprositelyno posmotrela na menıa. — Tak kak, Lıudvig?

Etot ublıudok cudesno perevel razgovor na druguy̆u temu.

— Nevajno. Y̆a zdesy. A u vas malo vremeni. Nado vyıvezti tebıa otsıuda, Kristina. Ostanexysa s nim — propadex.

— Pff! — Koldun provorno nacal sobiraty vex̨i, a britogolovyıy̆ hagjit zastegnul na sıurtuke poy̆as s krivoy̆ sabley̆.

Y̆a pocuvstvoval, kak zatyılok ukololo cto-to holodnoy̆e.

— Cezare! — kriknula Kristina. — Stoy̆!

— Y̆a ne sluxay̆u tvoy̆ih prikazov, jenx̨ina. Valyter? — Kondotyer podobralsa ko mne soverxenno nezametno.

— Ostavy y̆evo, — poprosil koldun, okoncatelyno sobrav sumku. — Nam on ne nujen. Zacem rasstray̆ivaty Kristinu? Pora uy̆ezjaty. I nacinaty vse snacala.

— Ne budet nikakovo nacala. Tyı uje proy̆igral.

— Tyı vse-taki prosto tupoy̆ gromila, alybalandeq. U menıa ne byılo vyıbora. Jizny qelovo mira postavlena na kartu. Y̆a poxel byı daje v pasty drakona, y̆esli byı eto priblizilo menıa k temnomu kuznequ. Kak tyı ne poy̆mex takuy̆u prostuy̆u vex̨: vse, cto y̆a govoril tebe o temnyıh kinjalah, — pravda. Y̆a nay̆du novyıy̆ glaz serafima. I poprobuy̆u snova. A tyı mojex bejaty v Ardenau, zaryıty golovu v pesok i dumaty, cto y̆a lgu, raz tebe tak legce.

— Tak i postuplıu. No snacala zaberu y̆ey̆e.

— Y̆a ne poy̆edu s toboy̆, Lıudvig, — tiho skazala Kristina.

— Hocex ostatsa s nimi? Znay̆ex, cto slucitsa, kogda tebıa poy̆may̆ut? Straj budet obvinen v zagovore protiv Riapano. Eto udarit po Bratstvu. Po kajdomu iz nas, gde byı myı ni nahodilisy. Uedem, poka ne pozdno. Ostavy ih. Tyı doljna jity, a ne umerety na dyıbe. Eto posledniy̆ xans.

Kristina vzıala iz ruk Valytera svoy̆u kurtku, nadela, drojax̨imi palyqami, nemnogo nelovko, zastegnula pugoviqi:

— Tebe y̆a tocno nicevo ne doljna, van Normay̆enn. Vse, cto byılo mejdu nami v dalekom proxlom, tepery ne imey̆et znaceniy̆a. U menıa svoy̆ puty, a u tebıa svoy̆. Uhodi, Lıudvig. Prıamo sey̆cas. I bolyxe ne ix̨i menıa.

Y̆a posmotrel y̆ey̆ v glaza, ponıal, cto vse bespolezno, cto y̆a ne smogu pereubedity y̆ey̆e, cto Kristinu, celoveka, s kotoryım y̆a kogda-to ucilsa, ros, jil i srajalsa plecom k plecu, uje ne vernex. I vyıxel iz doma, plotno prikryıv za soboy̆ dvery…

Y̆a xel po temnomu pustomu traktu k Kruso, a na duxe u menıa skrebli koxki.

Cto je. Y̆a hotıa byı popyıtalsa. Ona prinıala rexeniy̆e. Vyıbrala svoy̆u sudybu, svoy̆u jizny, svoy̆u qely. I bessmyıslenno lovity rukami uskolyzay̆ux̨uy̆u teny. Tratity vremıa i silyı. Y̆a uznal otvetyı na voprosyı, kotoryıy̆e menıa volnovali, i tepery sleduy̆et dvigatsa dalyxe. Idti vpered i ne oglıadyıvatsa.

Vperedi pokazalisy dva znakomyıh silueta. Odin vyısocennyıy̆ i dolgovıazyıy̆, drugoy̆ nevyısokiy̆ i suhonykiy̆.

— Ne vyıxlo? — negromko sprosil Propovednik. — Pocemu?

— Y̆a ne mogu spasti y̆ey̆e ot samoy̆ sebıa, drujix̨e.

— Tıajelo vvesti lıudey̆ v Qarstviy̆e Nebesnoy̆e, y̆esli oni ne jelay̆ut spaseniy̆a, — probormotal on i skazal kuda gromce: — No tepery Qerkovy nay̆det y̆ey̆e i nakajet kak zagovorx̨iqu.

— Ili ne nay̆det, y̆esli udaca budet na y̆ey̆e storone.

Y̆a sunul ruki v karmanyı, xagay̆a dalyxe, i oni pristroy̆ilisy rıadom.

— I cto tepery? — ne vyıderjal Propovednik.

— Zay̆musy delami. V mire polno temnyıh dux. Syezju v Ardenau. Y̆a ne byıl na rodine neskolyko let. Vstrecusy s Gertrudoy̆. Rasskaju obo vsem, cto zdesy proy̆izoxlo, starey̆xinam. Bratstvo doljno byıty gotovo k nepriy̆atnostıam.

Myı so staryım pelikanom sdelali y̆ex̨e neskolyko xagov, prejde cem ponıali, cto Pugalo nas ne soprovojday̆et. Ono stoy̆alo na doroge, vyıtıanuvxisy v strunku, tocno teryer, pocuy̆avxiy̆ lisu, i smotrelo tuda, otkuda y̆a prixel.

— Ey̆, Solomennay̆a golova! — okliknul y̆evo Propovednik. — Zabyılo, kuda nado idti? E-ey̆! Myı zdesy. Iisuse Hriste, tyı ne tolyko onemelo, no i oglohlo?!

— Pogodi, — nahmurilsa y̆a i podoxel k Pugalu.

Ono melko drojalo, i v uzkih glazah to zagoralisy, to gasli dva malenykih ugolyka.

— Cto tam? Cto tyı vidix?

Ono polojilo mne na pleco tıajeluy̆u, kostlıavuy̆u ruku i razvernulo, predlagay̆a smotrety ne na nevo, a na mracnuy̆u dorogu, barhatnoy̆e zvezdnoy̆e nebo i temnyıy̆e siluetyı derevyev, vyıstupay̆ux̨iy̆e na etom fone. Vokrug byıla poslednıay̆a noc zimyı, strannay̆a toy̆ zlovex̨ey̆ tixinoy̆, kotoray̆a zastigay̆et odinokovo putnika na pustyınnom trakte. Y̆a, kajetsa, ne dyıxal, vmeste s Pugalom smotrıa vo mrak. I tot otvetil mne.

Zolotistoy̆ iskroy̆. Zolotoy̆ vspyıxkoy̆. Zolotyım svetom.

Ogony qveta jidkovo zolota podnıalsa vyıxe drevesnyıh kron i tut je opal, ostaviv v nebe zolotoy̆e zarevo.

— O, Gospodi! — ahnul Propovednik.

No y̆a uje ne sluxal y̆evo. Bejal obratno.



Zolotyıy̆e kostryı, takiy̆e teplyıy̆e, prekrasnyıy̆e, pohojiy̆e ne na obyıcnyıy̆ ogony, a na rasplavlennyıy̆ dragoqennyıy̆ metall, goreli povsıudu. V lesu, na ogromnom pustom pole i tam, gde y̆ex̨e sovsem nedavno stoy̆ala staray̆a ferma.

Ih byılo neskolyko desıatkov, haoticnyıh, razbrosannyıh po okruge, soverxenno neveroy̆atnyıh. Volxebnyıh. I smertelyno opasnyıh.

Eto byılo to je plamıa, cto svirepstvovalo na plox̨adi v Kruso, puskay̆ i meney̆e y̆arostnoy̆e. Ono gorelo, popiray̆a vse zakonyı mirozdaniy̆a, samo po sebe, ne nujday̆asy v toplive i ne zavisıa ot kaprizov vetra.

Pugalo ne stalo podhodity k blijay̆xemu kostru, a ostanovilosy kak vkopannoy̆e i, kazalosy, nıuhalo vozduh, pahnux̨iy̆ tıajeloy̆ gary̆u i, cuty ulovimo, perejarennyım mıasom. Zatem ono opustilo golovu, ssutulilosy i s nekotoryım razocarovaniy̆em selo na zemlıu. Na y̆evo vzglıad, tut uje ne byılo nicevo interesnovo.

Propovednik ne poxel so mnoy̆ po inoy̆ pricine — on boy̆alsa, hotıa ni odin ogony ne mog pricinity duxe vreda.

— Mojet, ne stoy̆it tebe tuda lezty?! — kriknul on mne v spinu.

No y̆a ne mog postupity inace.

Pervoy̆e telo, obuglennoy̆e do golovexek, vse y̆ex̨e dyımıax̨ey̆esıa, y̆a naxel rıadom s mertvyımi loxadymi. Lix po krivoy̆ polose metalla, v kotoroy̆ trudno byılo opoznaty sablıu, y̆a ponıal, cto eto hagjit.

V dom y̆a zay̆ti ne smog, tot vse y̆ex̨e polyıhal, poetomu napravilsa ot kostra k kostru, po vyıjjennoy̆ zemle.

I y̆edva ne spotknulsa o trup Cezare. On lejal na jivote, i v y̆evo spine byıla projjena skvoznay̆a dyıra velicinoy̆ s dva moy̆ih kulaka. Glaza okazalisy raspahnutyı, na liqe zastyıli udivleniy̆e i obida.

Y̆a vse dalyxe othodil ot fermyı, prodoljay̆a iskaty, i v glazah postepenno nacinalo dvoy̆itsa ot zolotyıh ogney̆. Ih byılo kuda bolyxe, cem mne pokazalosy vnacale.

Y̆a byı proxel mimo, y̆esli byı ona menıa ne okliknula. Y̆ee liqo pocernelo ot kopoti, pravay̆a ruka napominala obgorevxuy̆u vetku, a na to, cto byılo nije grudi, nelyzıa smotrety bez slez — odin sploxnoy̆ ojog.

Ona popyıtalasy ulyıbnutsa, pokazaty, cto vse horoxo, no polucilosy eto nevajno. Kristina splıunula temno-koricnevuy̆u slıunu, y̆a ox̨util prıanyıy̆, y̆edkiy̆ zapah i ponıal, cto ona tolyko cto syela koreny zolotovo lyva, silynyıy̆ hagjitskiy̆ narkotik, izbavlıay̆ux̨iy̆ ot lıuboy̆ boli.

— Ne povezlo, — tolyko i skazala ona. — Myı iskali y̆evo, a on naxel nas.

Byılo ponıatno, o kom ona govorit.

— Tyı videla temnovo kuzneqa?

— Izdali. — Straj uronila golovu na zemlıu. — Y̆a ne smogu tebe pomoc. Poobex̨ay̆ mne sdelaty koy̆e-cto.

— Obex̨ay̆u.

— Otpravlıay̆sıa v Klagenfurt. Tam jivet doc Valytera. Uliqa Stenyı. U ney̆e dar. Y̆a poklıalasy y̆emu, cto Bratstvo y̆ey̆e primet. Ne perebivay̆. Sluxay̆.

Ona vzıala iz raspotroxennoy̆ sumki y̆ex̨e odin koreny, otpravila y̆evo sebe za x̨eku:

— Pod polom v y̆evo dome y̆esty kniga. Sojgi y̆ey̆e. Eto vajno. Sdelay̆ex?

— Da.

— Vozymi sebe Vy̆una. Bolyxe y̆a nikomu y̆evo ne doverıu.

— Horoxo.

Y̆a videl, kak mutney̆ut y̆ey̆e glaza ot narkotika, i predstavlıal, kakuy̆u boly ona doljna ispyıtyıvaty sey̆cas.

— Tretye. Ne ix̨i temnovo kuzneqa. Inace on pridet i za toboy̆. Kak prixel za nami. Poklıanisy!

— Klıanusy. — Na etot raz y̆a lgal.

Ona ustalo zakryıla glaza i skazala cuty zapletay̆ux̨imsıa y̆azyıkom:

— I skaji Miriam: mne ujasno jaly, cto y̆a y̆ey̆e podvela.

— Eto ne tak. No y̆a peredam.

Y̆a dostal iz sumki kinjal, vlojil v y̆ey̆e ruku:

— Prosti, cto ne smog sdelaty etovo ranyxe.

Po y̆ey̆e x̨eke sbejala odinokay̆a slezinka:

— Y̆a ne ponimay̆u…

— Uje nevajno, Kristina. Glavnoy̆e, cto tvoy̆ klinok tepery s toboy̆. Tyı ne poterıala y̆evo.

Ona ulyıbnulasy prizrakom svoy̆ey̆ proxloy̆ ulyıbki, xepnula:

— Spasibo. V tom monastyıre, gde pogib Gans… Tam kuzneq, cto kuy̆et nam kinjalyı. Etu tay̆nu on uznal, i poetomu y̆evo ubili. Ne govori Miriam, horoxo? Y̆ey̆ ne stoy̆it znaty. — Kristina prervalasy, provalivay̆asy v zabyıtye, no s usiliy̆em zakoncila: — Inace budet beda. Dlıa vseh nas. Y̆a posplıu nemnogo. Razbudix menıa k utru?

— Konecno. Ni o cem ne volnuy̆sıa, — skazal y̆a, no ne byıl uveren, cto ona menıa y̆ex̨e slyıxit.

Y̆a sidel rıadom s ney̆, ox̨ux̨ay̆a tocno takuy̆u je zluy̆u bespomox̨nosty, kak kogda umirala Hanna. Y̆a nicevo ne mog dlıa ney̆e sdelaty.

Tolyko byıty rıadom…

\end{document}
