\documentclass[10pt]{book}
\usepackage{fontspec}
\setmainfont{Linux Libertine O}
\begin{document}

\newcommand{\e}{ë}
%\newcommand{\e}{e}
%\newcommand{\e}{é}
%\newcommand{\e}{ó}

\renewcommand{\i}{ı}
%\renewcommand{\i}{i}

\newcommand{\yi}{yı}
%\newcommand{\yi}{yi}
%\newcommand{\yi}{ǝ}

\newcommand{\ia}{ıa}
%\newcommand{\ia}{ia}
%\newcommand{\ia}{ía}
%\newcommand{\ia}{á}

\newcommand{\iu}{ıu}
%\newcommand{\iu}{iu}
%\newcommand{\iu}{ıo}
%\newcommand{\iu}{io}
%\newcommand{\iu}{ío}
%\newcommand{\iu}{íu}
%\newcommand{\iu}{ú}

\newcommand{\y}{y̆}
%\newcommand{\y}{y}

\newcommand{\Y}{Y̆}
%\newcommand{\Y}{Y}


Pugalo sidelo na kr{\yi}xe fligel{\ia}, nabl{\iu}da{\y}a rassvet. Nebo, kak eto castenyko b{\yi}va{\y}et v fevrale nedaleko ot mor{\ia}, neskolyko minut napominalo qvetom morsku{\y}u rakovinu, takim nejno-rozov{\yi}m ono b{\yi}lo, a zatem srazu potusknelo, nalilosy svinqom, v glubine kotorovo, kazalosy, raspleskali lucxi{\y}e vostocn{\yi}{\y}e cernila. Pocti srazu nacal nakrap{\yi}vaty dojdy, i oduxevl{\e}nnovo s kr{\yi}xi kak vetrom sdulo.

Dojdy Pugalo l{\iu}bilo daje menyxe, cem otsutstvi{\y}e razvleceni{\y}. A poslednih ne slucalosy uje dovolyno davno.

{\Y}a zanimalsa tem, cto zakancival ranni{\y} zavtrak — var{\e}n{\yi}{\y}e {\y}a{\y}qa, jarena{\y}a sardina i cesnocn{\yi}{\y} sup, kotor{\yi}{\y} okazalsa por{\ia}dkom peresolen.

Vs{\e} telo cesalosy, noc{\y}u {\y}a otrazil cet{\yi}re ataki klopov, no propustil organizovann{\yi}{\y} udar s levovo flanga i tepery m{\yi}slenno proklinal etot posto{\y}al{\yi}{\y} dvor i to, cto nasto{\y}ka dl{\ia} otpugivani{\y}a nasekom{\yi}h zakoncilasy tak ne vovrem{\ia}.

Vladeleq zavedeni{\y}a, vid{\ia} mo{\y}u hmuru{\y}u roju, ne rexalsa prosity rascot i toptalsa vozle kladovki, pogl{\ia}d{\yi}va{\y}a to na men{\ia}, to na oblaconn{\yi}h v belo{\y}e.[39]

Stranniki v odejdah, kotor{\yi}{\y}e za vrem{\ia} putexestvi{\y}a davno uje priobreli ser{\yi}{\y} qvet, s prost{\yi}mi posohami, {\y}eli postnu{\y}u grecku. Ustavxi{\y}e ot beskonecno{\y} dorogi bogomolyqi xli ot m{\yi}sa Del Sur, samo{\y} {\y}ujno{\y} tocki Narar{\yi}, v Kruso.

Ne perv{\yi}{\y}e piligrim{\yi} na mo{\y}om puti. I vse kak odin tverd{\ia}t, cto devocka, jivux̨a{\y}a v Kruso, uzrela cudo. Mol, priletel k ne{\y} kr{\yi}lat{\yi}{\y} vestnik, i kr{\yi}l{\y}a {\y}evo b{\yi}li podobn{\yi} d{\yi}mu. Soobx̨il on, razume{\y}etsa, cto Straxn{\yi}{\y} sud ne za gorami i nado poka{\y}atsa, prejde cem vostrubit rog i podnimetsa iz zemli prah.

— T{\yi} verix v eto? — sprosil u men{\ia} Propovednik, kogda m{\yi} tolyko usl{\yi}xali novosty.

— Cto angel sletel s nebes? Posledni{\y} angel, kotorovo, kak govor{\ia}t, videli, po{\y}avl{\ia}lsa, kogda rasp{\ia}li Christa, i izvestil namestnika, deda imperatora Konstantina, o tom, cto gr{\ia}dut bolyxi{\y}e peremen{\yi}. No v to{\y} istori{\y}i hot{\ia} b{\yi} b{\yi}l ser{\y}ozn{\yi}{\y} povod.

— Koneq sveta, po-tvo{\y}emu, ne povod?

— {\Y}a nemnogo ustal ot konqov sveta, Propovednik. Kajd{\yi}{\y} god vs{\ia}ki{\y}, kto scita{\y}et, cto videl angela ili sl{\yi}xal boga, za{\y}avl{\ia}{\y}et o tom, cto mir na kra{\y}u gibeli, cto vot-vot slucitsa treti{\y} potop, cetv{\e}rta{\y}a velika{\y}a epidemi{\y}a {\y}ustirskovo pota i v kajdom gorode na meste domov grexnikov v{\yi}rastut ogned{\yi}xax̨i{\y}e gor{\yi}, kotor{\yi}{\y}e budut plevatsa sero{\y} i jabami.

— To {\y}esty t{\yi} ne jd{\e}x Apokalipsisa?

— Ne somneva{\y}usy, cto rano ili pozdno m{\yi} dostanem nebesa i te provedut pokazatelynu{\y}u cistku parxiv{\yi}h oveq, no uveren, eto slucitsa ne pri mo{\y}e{\y} jizni.

— {\Y}esli cestno, {\y}a toje ne ver{\iu} v etu istori{\y}u. Na ko{\y} cort, prosti Gospodi, angelu priletaty k kako{\y}-to des{\ia}tiletne{\y} devconke, kogda v {\y}evo raspor{\ia}jeni{\y}i kuda bole{\y}e interesn{\yi}{\y}e predstaviteli celovecestva?

Odin iz piligrimov, uje davno pogl{\ia}d{\yi}va{\y}ux̨i{\y} na men{\ia}, otodvinul pustu{\y}u misku, nespexno v{\yi}ter gub{\yi} rukavom i podoxol k mo{\y}emu stolu:

— Bog v pomox̨. Napravl{\ia}{\y}etesy v Kruso?

{\Y}a podumal, sto{\y}it li delitsa svo{\y}imi planami s perv{\yi}m vstrecn{\yi}m, rexil, cto huje ne budet, i prosto kivnul.

— Nas vosemnadqaty. M{\yi} mirn{\yi}{\y}e l{\iu}di, a dorogi vdoly poberej{\y}a b{\yi}va{\y}ut opasn{\yi}. Zaplatim za zax̨itu.

Vot tolyko etovo mne ne hvatalo. Plestisy dva s polovino{\y} dn{\ia} vmeste s raspeva{\y}ux̨imi sv{\ia}t{\yi}{\y}e gimn{\yi} bogomolyqami, kogda na loxadi mojno okazatsa v gorode uje k veceru. U men{\ia} prosto net lixnevo vremeni.

— T{\yi} oxibsa, dobr{\yi}{\y} celovek, — l{\iu}bezno otvetil {\y}a {\y}emu. — {\Y}a ne na{\y}omnik i ne vo{\y}in.

Ctob{\yi} ne b{\yi}lo bolyxe voprosov, pokazal ruko{\y}atku kinjala:

— U men{\ia} svo{\y}i qeli. {\Y}esli hocex zax̨it{\yi}, shodi na kupeceski{\y} post. Oni ob{\yi}cno proda{\y}ut uslugi ohrannikov.

On, kajetsa, udovletvorilsa mo{\y}im otvetom i vernulsa k sputnikam. Propovednik pokrutil palyqem u viska:

— Otdaty denygi odnomu, ctob{\yi} on ohran{\ia}l vosemnadqaty. Voistinu Bojyi l{\iu}di. Takovo idiotizma {\y}a ne vstrecal s teh por, kak rexil ostanovity na{\y}omnikov na kr{\yi}lyqe mo{\y}e{\y} qerkvi. Lucxe b{\yi} sideli doma, cem xl{\ia}tsa po dorogam.

— Kak t{\yi} surov s utra. — {\Y}a s usmexko{\y} otlojil lojku, okonciv zavtrak. Sledovalo rasplatitsa i otpravl{\ia}tsa v dorogu. Pod merzkim dojd{\e}m.

— {\Y}a istinu govor{\iu}, Ludwig. A uj {\y}esli doma ne siditsa i v zadniqe sverbit, naucisy strel{\ia}ty iz arbaleta. Vosemnadqaty celovek s arbaletami. Oni l{\iu}b{\yi}h razbo{\y}nikov udela{\y}ut.

— Otpravl{\ia}{\y}ux̨imsa k sv{\ia}t{\yi}n{\ia}m ne pristalo nosity pri sebe cto-to t{\ia}jele{\y}e bibli{\y}i.

— Vot potomu ih i razuva{\y}ut vse komu ne leny.

On {\y}ex̨o cto-to vorcal po etomu povodu, no {\y}a uje napravilsa k hoz{\ia}{\y}inu posto{\y}alovo dvora, predostaviv staromu pelikanu v{\yi}skaz{\yi}vaty svo{\y}i m{\yi}sli Pugalu v namokxe{\y} solomenno{\y} xl{\ia}pe.



Narara, nesmotr{\ia} na to cto eto ne sama{\y}a {\y}ujna{\y}a strana kontinenta, zimo{\y} otlicalasy kuda bole{\y}e m{\ia}gkim klimatom, cem tot je Leserberg ili Vitilyska. Sneg v primorskih oblast{\ia}h padal obilyno, no moroz{\yi} slucalisy redko, a k konqu fevral{\ia} dovolyno casto teplelo nastolyko, cto nacinal idti dojdy.

Konecno je holodn{\yi}{\y} i nepri{\y}atn{\yi}{\y}, no, {\y}esli sravnivaty s ubi{\y}stvenn{\yi}m morozom, cto se{\y}cas, po sluham, sobira{\y}et x̨edru{\y}u jatvu iz putnikov v Firvaldene, — zdesy, mojno skazaty, b{\yi}l ra{\y} zemno{\y}. Vprocem, celovek nikogda ne b{\yi}va{\y}et dovolen i castenyko pen{\ia}{\y}et na sudybu. K poludn{\iu} {\y}a voznenavidel dojdy, kotor{\yi}{\y} xol ne perestava{\y}a.

Sto{\y}ilo mne podumaty o tom, cto lucxe uj xol b{\yi} sneg, kak kapli obernulisy krupn{\yi}mi bel{\yi}mi hlop{\y}ami i slucilasy ``cudesna{\y}a" metely. Ona, tocno soly {\Y}adovitovo mor{\ia}, ukr{\yi}la r{\yi}je-krasnu{\y}u zeml{\iu} bel{\yi}m nal{\e}tom, kotor{\yi}{\y} ne proderjalsa i casa — iz-za oblakov v{\yi}gl{\ia}nulo solnqe i rastopilo vs{\iu} etu krasotu.

{\Y}a b{\yi}stro pon{\ia}l, cto oxibsa v rascetah i k veceru v Kruso ne popadu. Okajisy zeml{\ia} zamerzxe{\y}, eto b{\yi}lo b{\yi} vpolne vozmojno, no doroga razmokla, i gnaty po ne{\y} loxady ne imelo nikakovo sm{\yi}sla.

Propovednik eto toje pon{\ia}l, no pomalkival, pogl{\ia}d{\yi}val na solnqe. I nakoneq, uje k veceru predlojil:

— Derevenyki po puti vstreca{\y}utsa. Perenocu{\y}em v odno{\y} iz nih? Mestn{\yi}{\y}e dovolyno drujel{\iu}bn{\yi}. Vedy uje pon{\ia}tno — t{\yi} okajexsa v gorode ne ranyxe seredin{\yi} zavtraxnevo dn{\ia}.

— Predpocita{\y}u posto{\y}al{\yi}{\y} dvor, a ne krest{\y}anski{\y} dom.

— Vse dvor{\yi} zabit{\yi} palomnikami, piligrimami i sumasxedximi. Vcera m{\yi} {\y}edva naxli mesto dl{\ia} noclega. Cto t{\yi} ime{\y}ex protiv krest{\y}an?

— {\Y}a ver{\iu} v dobrotu l{\iu}de{\y}, Propovednik, no gorazdo menyxe, cem prejde. Za mo{\y}u jizny cet{\yi}rejd{\yi} men{\ia} p{\yi}talisy ubity vo vrem{\ia} takih vot noclegov. Odin raz, potomu cto {\y}a straj, v drugo{\y} — potomu cto ponravilsa mo{\y} kony, v treti{\y} — iz-za pr{\ia}jki na remne i dvuh serebr{\ia}n{\yi}h monet v koxelyke.

— A v cetv{\e}rt{\yi}{\y}? — utocnil on, kogda {\y}a zamolcal. — T{\yi} skazal, cto cet{\yi}re raza.

— Ne zna{\y}u pricin{\yi}. Tot umnik umer prejde, cem uspel povedaty mne {\y}e{\y}o. V obx̨em, {\y}a ne slixkom jajdu nastupity na te je grabli v p{\ia}t{\yi}{\y} raz. Eto uje slixkom. Daje dl{\ia} men{\ia}.

— A {\y}esli ne budet posto{\y}al{\yi}h dvorov?

— Cto-nibudy priduma{\y}u. Les pod bokom.

{\Y}a obognal neskolyko grupp strannikov — ustavxih, izmojd{\e}nn{\yi}h, no vdohnovenno xaga{\y}ux̨ih v Kruso, tocno okoldovann{\yi}{\y}e.

— Vera tvorit cudesa. — Propovednik s jadn{\yi}m l{\iu}bop{\yi}tstvom rassmatrival ih liqa.

— Vera v slova malenyko{\y} devocki i sluhi, kotor{\yi}{\y}e ih preumnoja{\y}ut. Skolyko etih blajenn{\yi}h ostanutsa lejaty na obocine iz-za holoda, bolezne{\y}, pereutomleni{\y}a i vstreci s durn{\yi}mi l{\iu}dymi? Po mne, eto bolyxe napomina{\y}et sumasxestvi{\y}e, a ne veru.

— Ne soglasen s tobo{\y}. — On ostorojno potrogal prolomlenn{\yi}{\y} visok, zatem gl{\ia}nul na paleq. — Vera na to i vera. {\Y}esli bo{\y}atsa za svo{\y}u jizny, to konecno je nado sidety doma. No sledu{\y}et cto-to sdelaty, ctob{\yi} popasty v ra{\y}. Ne vsem otkr{\yi}va{\y}utsa eti vrata i prox̨a{\y}utsa grehi.

— To {\y}esty, po logike, lucxe pogibnuty v puti i obresti vecno{\y}e blajenstvo na nebesah?

— A razve net?

{\Y}a pokacal golovo{\y}:

— Propovednik, {\y}a kak nikto ino{\y} ver{\iu} v cudesa, ad, ra{\y}, demonov, angelov, duxi i voskrexeni{\y}e Christovo. S{\iu}da mojex dobavity potop, ishod iz hagjitskih zemely, znameni{\y}a, ognenn{\yi}{\y}e dojdi i cto tam {\y}ex̨o napisano v bibli{\y}i po drugim vajn{\yi}m povodam. No {\y}a gotov pospority, kak specialist po duxam, cto nelyz{\ia} polucity kl{\iu}c ot ra{\y}a, sdohnuv v puti ot tifa, {\y}esli t{\yi} nasluxalsa basen, kotor{\yi}{\y}e ne ime{\y}ut nicevo obx̨evo s vero{\y}.

— L{\iu}ba{\y}a basn{\ia} po{\y}avl{\ia}{\y}etsa po vole {\Y}evo.

— Aga. Tak mojno skazaty obo vs{\e}m. Vot eta luja toje po vole {\y}evo. I vot eta kanava zdesy ne sluca{\y}na. I von ta viseliqa na perekr{\e}stke po{\y}avilasy iskl{\iu}citelyno po prikazu boga, a ne mestnovo zemlevladelyqa, kaznivxevo razbo{\y}nikov ili prosto kakih-to bedolag.

— Nax teologiceski{\y} spor zahodit v tupik, — zametil on. — Potomu cto {\y}a ime{\y}u v rukave odin i tot je koz{\yi}ry, uklad{\yi}va{\y}ux̨i{\y} l{\iu}bo{\y} tvo{\y} argument na obe lopatki. {\Y}emu uje bez malovo poltor{\yi} t{\yi}s{\ia}ci let, no on otlicno de{\y}stvu{\y}et. Hocex usl{\yi}xaty eti volxebn{\yi}{\y}e slova?

{\Y}a prix̨urilsa:

— Udivi men{\ia}.

— M{\yi} prosto ne v sosto{\y}ani{\y}i postic {\Y}evo zam{\yi}slov, — nevinno izr{\e}k on. — Ibo kto m{\yi} pered Nim? I vozmojno, eta kanava s{\yi}gra{\y}et roly v {\Y}evo planah. Kak i luja. I viseliqa s {\y}e{\y}o gruzom. Tolyko m{\yi} ob etom nikogda ne uzna{\y}em.

— Da, eto oceny udobno{\y}e zaklinani{\y}e. I {\y}evo mojno primenity k l{\iu}bo{\y} situaqi{\y}i. K primeru, tvo{\y}a smerty b{\yi}la {\y}evo zam{\yi}slom.

On hihiknul:

— {\Y}a ni na minutu v etom ne somneva{\y}usy.

— I poetomu poro{\y} nedel{\ia}mi {\y}a sl{\yi}xu ot teb{\ia} potoki bogohulystv?

— Odno drugomu ne mexa{\y}et. {\Y}esli mo{\y}a smerty nujna {\Y}emu, to {\y}a gotov v{\yi}polnity svo{\y}e prednaznaceni{\y}e.

{\Y}a sn{\ia}l s golov{\yi} capuchon, i vlajn{\yi}{\y} veter s mor{\ia} vzyeroxil mo{\y}i otrosxi{\y}e volos{\yi}:

— I t{\yi} zna{\y}ex, kakovo ono?

— M{\yi} prosto ne v sosto{\y}ani{\y}i postic {\Y}evo zam{\yi}slov, — terpelivo povtoril star{\yi}{\y} pelikan. — B{\yi}ty mojet, On jelal, ctob{\yi} {\y}a skraxival tvo{\y}e odinocestvo? A zatem otpravl{\iu}sy v ra{\y}.

— T{\yi} uje mojex tuda otpravitsa. Hoty se{\y}cas, — napomnil {\y}a {\y}emu.

— Poka {\y}a ne gotov. No vozvrax̨a{\y}asy k naxe{\y} besede o vere i veru{\y}ux̨ih. Scita{\y}u, cto nevajno, naskolyko pravdiv{\yi} sluhi i devocka, blagodar{\ia} kotoro{\y} oni po{\y}avilisy. Vajna lix vera. Daje {\y}esli u ne{\y}o net pricin{\yi}. Ibo ona — propusk v ra{\y}. Ne soglasen?

Vopros b{\yi}l obrax̨on k dolgov{\ia}zomu Pugalu, kotoro{\y}e, tocno aist, razmerenno xagalo po drugu{\y}u storonu ot mo{\y}e{\y} loxadi. To lix uhm{\yi}lynulosy.

— Nu da, — provorcal Propovednik. — Kuda uj tebe o duhovn{\yi}h vex̨ah rassujdaty.

— Vera ne {\y}avl{\ia}{\y}etsa propuskom. T{\yi} oxiba{\y}exsa. — M{\yi} pocti dobralisy do viseliqi, i {\y}a popravil palax, visevxi{\y} r{\ia}dom s sedelyn{\yi}mi sumkami, tak kak v blija{\y}xih pridorojn{\yi}h kustah mne pocudilosy dvijeni{\y}e. — Krome ne{\y}o doljn{\yi} b{\yi}ty i horoxi{\y}e postupki. Otsutstvi{\y}e grehov. Slepa{\y}a vera ne pomoga{\y}et, drug Propovednik, a vredit. Eto vs{\e} ravno cto neupravl{\ia}{\y}ema{\y}a kareta, nesux̨a{\y}asa pod gorku. Ugrobit i teh, kto sidit v ne{\y}, i teh, kto popad{\e}t pod kolesa.

Star{\yi}{\y} pelikan skrivilsa:

— {\Y}a ponima{\y}u tvo{\y}u analogi{\y}u, Ludwig. Daje prizna{\y}u, cto t{\yi} prav. M{\yi}, l{\iu}di, iskaja{\y}em vs{\e}, do cevo dot{\ia}giva{\y}emsa. Ili v{\yi}voraciva{\y}em naiznanku, cto odno i to je. No kl{\ia}nusy krov{\y}u Christovo{\y}, tak b{\yi}ty ne doljno. Vera doljna spasaty, a ne ubivaty.

— I ne razobx̨aty, ne strax̨aty, ne sudity i ne kaznity. No otcevo-to imenno tak i proishodit. Odni jgut vedym, drugi{\y}e — katzerov iz Vitilyska, tretyi — teh, kto zab{\yi}l pomolitsa pered obedom. Uveren, cto pom{\yi}sl{\yi} Gospoda v etih sluca{\y}ah soverxenno ni pri cem. Eto uj m{\yi} sami, voplox̨eni{\y}e ruk {\y}evo, dodumalisy. No vsegda gotov{\yi} spihnuty svo{\y}i ne slixkom pravedn{\yi}{\y}e postupki na cuju{\y}u vol{\iu}, lixiv {\y}e{\y}o seb{\ia}. Mol, ne {\y}a srubil golovu tomu nechrist{\i}u-hagjitu, eto bog tak velel.

M{\yi} vplotnu{\y}u podyehali k viseliqe — perekladine mejdu dvuh stolbov. Na ne{\y} boltalisy dva trupa. Sud{\ia} po vnexnemu vidu, vstrecali putnikov oni uje oceny davno. Poko{\y}niki v{\yi}gl{\ia}deli stoly jalko, cto ne zainteresovali daje Pugalo.

— Zabavno, — izr{\e}k Propovednik s takim vidom, slovno {\y}evo zastavili proglotity tarelku jelci. — M{\yi} jiv{\e}m i m{\yi}slim, verim, jela{\y}em, l{\iu}bim i nenavidim. M{\yi} vse, sozdani{\y}a Bojyi s gor{\ia}ce{\y} krov{\y}u, v konqe puti prevrax̨a{\y}emsa vot v eto. V bezduxn{\yi}{\y} kusok m{\ia}sa na radosty cerv{\ia}m i voronam.

— Cto eto na teb{\ia} naxlo?

On otvernulsa ot viselynikov:

— Umiraty ne straxno, Ludwig. Prosto obidno. Nikogda ne uspeva{\y}ex sdelaty vse, cto hotel.

— M{\yi} ne umira{\y}em posle smerti, Propovednik. T{\yi} — tomu {\y}avno{\y}e dokazatelystvo.

— {\Y}a uznal ob etom, lix kogda men{\ia} ubili. Do tovo momenta — veril i somnevalsa. Somnevalsa i veril. Pri vseh cudesah i dokazatelystvah ne vsegda mojno iskrenne b{\yi}ty ubejdenn{\yi}m do konqa, cto {\y}esty jizny posle smerti.

— {\Y}ex̨o odna celoveceska{\y}a certa, — usmehnulsa {\y}a. — M{\yi} sklonn{\yi} somnevatsa daje v ocevidn{\yi}h faktah. Sploxn{\yi}{\y}e protivoreci{\y}a.

On neojidanno ul{\yi}bnulsa.

— Inogda t{\yi} govorix zamecatelyn{\yi}{\y}e vex̨i, mo{\y} malycik. V taki{\y}e minut{\yi} {\y}a uzna{\y}u cto-to novo{\y}e o samom sebe, — proronil on i falyqetom zapel qerkovn{\yi}{\y} gimn vo slavu blagodati.



Nocevaty pod otkr{\yi}t{\yi}m nebom ili v kakom-nibudy zabroxennom sara{\y}e ne prixlosy. Poctova{\y}a stanqi{\y}a s posto{\y}al{\yi}m dvorom okazalasy kak nelyz{\ia} kstati. I, nesmotr{\ia} na to cto v krohotnom zale b{\yi}lo narodu stolyko je, skolyko v bocke alybalandskih seledok, svobodna{\y}a komnata naxlasy.

— Da neujeli? — izumilsa Propovednik i tknul suhim palyqem v cern{\ia}vu{\y}u hoz{\ia}{\y}ku. — Sovetu{\y}u sprosity u ne{\y}e, v cem zdesy podvoh. S tem kolicestvom jela{\y}ux̨ih priobx̨itsa k sv{\ia}tomu mestu komnatu nelyz{\ia} na{\y}ti daje za florin. A zdesy svobodna{\y}a!

{\Y}a sprosil. Jenx̨ina ne stala skr{\yi}vaty:

— Tri mes{\ia}qa nazad zarezali tam odnovo putnika. Sam vinovat. Pustil cujakov, mnogo boltal, pil i soril denygami. {\Y}a i opomnitsa ne uspela, kak {\y}evo v{\yi}potroxili.

— S kakih eto por ubi{\y}stvo puga{\y}et goste{\y}?

Ona nahmurilasy, zatem rexilasy:

— Horoxo. Ne budu {\y}ulity, gospodin. Mo{\y} s{\yi}n videl vax kinjal, kogda zavodil loxady v sto{\y}lo. V{\yi} straj, a znacit, ne bo{\y}itesy prizrakov. I ne potrebu{\y}ete platu nazad.

— A vot s etovo momenta popodrobne{\y}e, — za{\y}interesovanno poprosil {\y}a.

Ob{\yi}cno prizraki i privideni{\y}a ne bole{\y}e cem mif. Tak naz{\yi}va{\y}ut duxu, kotoru{\y}u vnezapno vid{\ia}t vse, komu ona jela{\y}et pokazatsa na glaza. O takom pixut v knigah, no v realynosti podobno{\y}e {\y}avleni{\y}e straji vstreca{\y}ut reje govor{\ia}x̨evo kozla za obedenn{\yi}m stolom Pap{\yi}.

— Otpeli i zakopali, vs{\e} kak polojeno. A on, svoloc, vs{\e} poko{\y}a ne da{\y}et. — {\y}e{\y}o liqo stalo zl{\yi}m. — Zan{\ia}l komnatu, puga{\y}et l{\iu}de{\y}. Te uje nacali boltaty, a mne, kak ponima{\y}ete, ni k cemu razgovor{\yi}. Se{\y}cas narodu mnogo i mest net, a kak potok shl{\yi}net, nikto ko mne ne idet, krome pridurkov, kotor{\yi}m ohota poglazety na mertveqa. Takih, kak v{\yi} ponima{\y}ete, gorazdo menyxe normalyn{\yi}h l{\iu}de{\y}.

— Kto-nibudy posle {\y}evo po{\y}avleni{\y}a zdesy umiral?

— Net.

— Bolel? Kalecilsa?

— Net, upasi Gospody. Nicevo takovo. I s dohodami poka vs{\e} horoxo. Da i ne zlo{\y} on. Prosto pugaty l{\iu}bit. Slujanki uje tuda i ne zahod{\ia}t. Komnatu ne ubirali. Izbavyte men{\ia} ot nevo, gospodin. A {\y}a besplatno pux̨u. I {\y}edu lucxu{\y}u, i vino. I loxadke vaxe{\y} pxeniqi otborno{\y}.

— Soblaznitelyno, — bez osob{\yi}h emoqi{\y} pro{\y}iznes {\y}a. — Nu pokaz{\yi}va{\y}, gde u teb{\ia} ploha{\y}a komnata.

Ona okazalasy na pervom etaje, v dalynem konqe doma, s {\y}edinstvenn{\yi}m oknom i vidom na skotn{\yi}{\y} dvor.

— Nasto{\y}ax̨i{\y} dvoreq. — Propovednik dal svo{\y}u kriticesku{\y}u oqenku ubogomu interyeru i krovati s solomenn{\yi}m matrasom.

— Nade{\y}usy, prizrak smog napugaty ne tolyko l{\iu}de{\y}, no i klopov. — {\Y}a brosil sumku v temn{\yi}{\y} ugol i, ne uderjavxisy, otkr{\yi}l malenyku{\y}u fortocku. Zdesy davno sledovalo provetrity.

— Vrode b{\yi} ob{\yi}cn{\yi}{\y}e l{\iu}di nas videty ne doljn{\yi}. — {\Y}evo pelikanye sv{\ia}te{\y}xestvo pl{\iu}hnulsa na mo{\y}u krovaty.

— Vsegda {\y}esty iskl{\iu}ceni{\y}a iz pravil.

V dvery postucali.

— A vot i on. Kako{\y} vejliv{\yi}{\y}, — hihiknul mo{\y} sputnik.

Razume{\y}etsa, eto b{\yi}la nikaka{\y}a ne duxa, a sama hoz{\ia}{\y}ka. Ne vhod{\ia}, ona prot{\ia}nula mne cisto{\y}e postelyno{\y}e belye, stara{\y}asy ne smotrety v komnatu:

— Ujinaty budete v zale ili sobraty vam zdesy, gospodin?

— V zale, — k {\y}e{\y}o {\y}avnomu oblegceni{\y}u otvetil {\y}a.

Poka {\y}a sidel v tolce{\y}e, opustoxa{\y}a tarelku, v komnate po{\y}avilsa gosty. No ne tot, kotorovo {\y}a jdal. Eto b{\yi}lo vsevo lix Pugalo. Ono besqeremonno izvleklo iz mo{\y}e{\y} obyemno{\y} sumki glavno{\y}e {\y}e{\y}o soderjimo{\y}e — t{\ia}jelu{\y}u tolstu{\y}u tetrady, perepletennu{\y}u v xerxavu{\y}u svinu{\y}u koju.

{\Y}a unes vse, cto naxel na stole poko{\y}novo burggrafa, no lix eta vex̨ opravdala mo{\y}i ojidani{\y}a. V mo{\y}i ruki popalo necto vrode dnevnika, buhgaltersko{\y} knigi i {\y}ejednevnika za posledni{\y}e polgoda — {\y}evo milosty otlicalsa pedanticnost{\y}u i dover{\ia}l bumage vse svo{\y}i dela.

Oteq un Nomanna, k sojaleni{\y}u, ne b{\yi}l nastolyko na{\y}iven, ctob{\yi} ne ispolyzovaty xifr. Posledni{\y} okazalsa slojen — izobreteni{\y}e flotoli{\y}skih bankirov. Procitaty {\y}evo bez kl{\iu}ca ne predstavl{\ia}losy vozmojn{\yi}m. Poetomu {\y}a sledu{\y}ux̨im je utrom otpravil dnevnik cerez ``Fabyen Klemenz i s{\yi}nov{\y}a" Gertrude, zna{\y}a, cto s {\y}e{\y}o sv{\ia}z{\ia}mi i znakomstvami, v tom cisle i v Riapano, gde oboja{\y}ut ne tolyko sozdavaty, no i raskr{\yi}vaty cuji{\y}e sekret{\yi}, uznaty, cto napisano, polucitsa gorazdo b{\yi}stre{\y}e.

Rovno cerez dve nedeli {\y}a polucil tetrady obratno v drugom otdeleni{\y}i ``Fabyen Klemenz", nahod{\ia}x̨emsa za sto lig ot pervovo, i sredi straniq lejal kl{\iu}c s pravilyno{\y} kombinaqi{\y}e{\y} i provox̨enn{\yi}{\y} trafaret, v kotor{\yi}{\y} trebovalosy podstavl{\ia}ty nujn{\yi}{\y}e bukv{\yi}.

{\Y}a nacal s sam{\yi}h poslednih zapise{\y} i ne oxibsa. Uje na tretye{\y} straniqe s konqa, mejdu otmetkami o v{\yi}plate jalovan{\y}a slugam i sovex̨ani{\y}i u burgomistra, naxlosy necto l{\iu}bop{\yi}tno{\y}e:

``Interesn{\yi}{\y} kinjal v kollekqi{\y}u po proxlo{\y} dogovorennosti. Polucen cerez „Fabyen Klemenz i s{\yi}nov{\y}a“. Otpravitely iz Kruso. Qerkovy Sv{\ia}tovo Miha{\y}ila. Avans v scet cernovo kamn{\ia}. Rasplatitsa. Pos{\yi}lku pros{\ia}t peredaty licno. Kuryer pri{\y}edet v nacale fevral{\ia}".



Propovednik, uznav, cto {\y}a sobira{\y}usy v Kruso, daje rukami vsplesnul:

— Gospodi Iisuse, Ludwig! A pocemu ne k hagjitam? Ili srazu k adskim vratam na vostocno{\y} okra{\y}ine mira?! Do Narar{\yi} puty neblizki{\y}, i tebe tam sovsem necevo delaty.

— Krome kak razobratsa s tem, cto slucilosy s Kristino{\y}, po sledu kotoro{\y} {\y}a idu s samovo nacala oseni. Kto-to iz Kruso otpravil {\y}e{\y}o kinjal burggrafu. Tot, komu nujn{\yi} b{\yi}li kamni serafima. I predpolaga{\y}u, dl{\ia} tovo, ctob{\yi} v{\yi}kovaty temno{\y}e oruji{\y}e.

— Da-da! Temn{\yi}{\y} kuzneq jivet v qerkvi Sv{\ia}tovo Miha{\y}ila i tolyko i dela{\y}et, cto jdet teb{\ia}. Ctob{\yi} t{\yi} pri{\y}ehal i zadal {\y}emu svo{\y}i vopros{\yi}! Tovo, kto otpravil kinjal, uje mojet tam ne b{\yi}ty!

— On tam, — s uverennost{\y}u pro{\y}iznes {\y}a. — V konqe fevral{\ia} burggraf doljen b{\yi}l otpravity {\y}emu kamni.

— Nu, mojet, prejde cem {\y}ehaty sotni lig, t{\yi} prosto za{\y}dex v ``Fabyen Klemenz" i po{\y}interesu{\y}exsa, ot kovo b{\yi}la pos{\yi}lka?

— S kako{\y} stati im otvecaty? Oni ne raskr{\yi}va{\y}ut postoronnim ta{\y}n{\yi} svo{\y}ih kli{\y}entov.

— T{\yi} toje ih kli{\y}ent. I s dovolyno vnuxitelyn{\yi}m scetom.

— Eto nicevo ne men{\ia}{\y}et. Oni ne stanut riskovaty reputaqi{\y}e{\y} i otcit{\yi}vatsa o cujih ta{\y}nah.

On b{\yi}l nedovolen i ne jelal {\y}ehaty na zapad. No, sobstvenno govor{\ia}, kogda b{\yi}lo inace? Propovednik, vs{\iu} jizny provedxi{\y} v svo{\y}e{\y} derevne, nesmotr{\ia} na to cto mota{\y}etsa za mno{\y} ne odin god, tak i ne priv{\yi}k k cast{\yi}m pere{\y}ezdam.

Odnako vernemsa k nasto{\y}ax̨emu. Tepery Pugalo rexilo zan{\ia}tsa cteni{\y}em ili delalo vid, cto cita{\y}et dnevnik burggrafa. Ono nespexno perevoracivalo straniqi ser{\yi}mi kogtist{\yi}mi palyqami, naklonivxisy k samo{\y} svece, kotora{\y}a {\y}edva ne podjigala {\y}evo xl{\ia}pu, sdelannu{\y}u iz ploho{\y} solom{\yi}.

— Kak {\y}a ponima{\y}u, tebe ne smux̨a{\y}et xifr.

Ono daje golov{\yi} ne povernulo.

— Interesno, cto ono hocet tam na{\y}ti. — Propovednik, podperev x̨eku, s nekotoro{\y} zavist{\y}u sledil za oduxevlenn{\yi}m.

Nelepo{\y}e Pugalo s pol{\ia} delalo to, cevo ne mog selyski{\y} sv{\ia}x̨ennik — ono prekrasno umelo citaty.

{\Y}a stal gotovitsa ko snu, kogda po{\y}avilsa nov{\yi}{\y} gosty — bledn{\yi}{\y} celovek s izurodovann{\yi}m liqom i v zalito{\y} krov{\y}u odejde. Nad nim horoxenyko porabotali nojami i nanesli tako{\y}e kolicestvo ran, cto vporu b{\yi}lo lix pojalety {\y}evo.

V{\yi}tarax̨iv glaza, on zaskrejetal zubami i medlenno dvinulsa ko mne.

— Ne nado{\y}elo? — s ucasti{\y}em sprosil {\y}a.

On ostanovilsa kak vkopann{\yi}{\y}, posmotrev nedovercivo, i ostorojno sprosil s silyn{\yi}m litavskim akqentom:

— Cto, sovsem ne straxno?

— Uv{\yi}, — s sojaleni{\y}em razvel {\y}a rukami.

— I vam ne straxno? — sprosil ne{\y}izvestn{\yi}{\y} u Propovednika.

— {\Y}a mertv, kak i t{\yi}, poludurok, — provorcal tot. — Napugaty mertveqa mertveqom eto nado umudritsa. Kl{\ia}nusy Devo{\y} Mari{\y}e{\y}, bole{\y}e glupo{\y} zate{\y}i {\y}a {\y}ex̨o ne vidal.

— Naverno{\y}e, sto{\y}ilo podkrastsa szadi, — probormotal tot i doveritelyno skazal mne: — Ob{\yi}cno vse ubegali s voplem i {\y}edva dvery ne snosili.

— Nekotor{\yi}{\y}e l{\iu}di ne taki{\y}e, kak vse. — {\Y}a polojil na stol obnajenn{\yi}{\y} klinok.

— Bezdna! — pro{\y}iznes ubit{\yi}{\y} i rvanul proc, no ne tut-to b{\yi}lo. Figura, kotoru{\y}u {\y}a kinul {\y}emu pod nogi, b{\yi}la nicuty ne huje silka, kotor{\yi}{\y}e stav{\ia}t na krolika.

On dernulsa raz, drugo{\y}, no lix zaputalsa {\y}ex̨o silyne{\y}e.

— E{\y}, pri{\y}ately! T{\yi} ne ime{\y}ex nikakovo prava men{\ia} trogaty! — Na {\y}evo liqe b{\yi}l strah. — {\Y}a ne temn{\yi}{\y}.

— Eto t{\yi} tak scita{\y}ex. — {\Y}a vstal, vz{\ia}vxisy za kinjal. — T{\yi} ostalsa na meste svo{\y}evo ubi{\y}stva, i otcevo-to teb{\ia} kto-to uvidel. Razume{\y}etsa, on ispugalsa. I tebe eto ponravilosy.

— Vsevo lix malenyka{\y}a xalosty, — proskulil on.

— Cujo{\y} strah dobavil tebe sil. A oni dali vozmojnosty uvidety teb{\ia} {\y}ex̨o komu-to. I t{\yi} snova napugal. I op{\ia}ty podpitalsa ujasom.

— No {\y}a… — On zatknulsa, kogda {\y}a podn{\ia}l ruku s klinkom, priz{\yi}va{\y}a {\y}evo k molcani{\y}u.

— {\Y}a rasskaju tebe o posledstvi{\y}ah. Pitani{\y}e strahom privedet k tomu, cto tvo{\y}a svetla{\y}a sux̨nosty stanet temno{\y}. Ne pr{\ia}mo se{\y}cas. B{\yi}ty mojet, cerez mes{\ia}q, a mojet, i cerez god — smotr{\ia} skolykih l{\iu}de{\y} t{\yi} napuga{\y}ex i kak silyno im budet straxno. No povery mne, rano ili pozdno podobno{\y}e pro{\y}izo{\y}det. Zna{\y}ex, cto togda slucitsa?

— T{\yi} pridex za mno{\y}? — xepotom sprosil tot.

— {\Y}a uje prixel za tobo{\y}. I ne budu ojidaty to{\y} por{\yi}, kogda t{\yi} pererodixsa iz-za svo{\y}ih glup{\yi}h zabav i nacnex ubivaty l{\iu}de{\y}. Poka pered tobo{\y} otkr{\yi}t{\yi} vrata ra{\y}a. No, kogda t{\yi} naberexsa tym{\yi}, otpravixsa ne naverh, a vniz. Zagremix v cistilix̨e. Grubo govor{\ia}, sobstvenn{\yi}mi rukami otpravix seb{\ia} tuda, kuda nikto ne hocet. Ne slixkom prekrasna{\y}a perspektiva, na mo{\y} vzgl{\ia}d. {\Y}a da{\y}u tebe v{\yi}bor. U{\y}dex sam ili mne v{\yi}polnity svo{\y}u rabotu?

— U{\y}du sam, — b{\yi}stro otvetil on. — Nikaki{\y}e xutki ne sto{\y}at ada. {\Y}a prosto snova hotel pocuvstvovaty jizny.

{\Y}a razruxil figuru, uderjiva{\y}a nagotove znak. On vzdohnul, zakr{\yi}l glaza, a dalyxe slucilosy to, cto {\y}a videl uje mnogo raz. {\Y}evo siluet stal blednety, poka ne ostalosy {\y}edva zametnovo kontura. Tot na mgnoveni{\y}e zasi{\y}al solnecn{\yi}m svetom, kotor{\yi}{\y} ozaril vs{\iu} komnatu, i vokrug vnovy nastupila polutyma, razgon{\ia}{\y}ema{\y}a lix svecami na stole.

Cto primecatelyno, Pugalo daje golov{\yi} ne povernulo, prodolja{\y}a citaty dnevnik burggrafa.

— Nu, tepery hoz{\ia}{\y}ka posto{\y}alovo dvora tocno skajet tebe spasibo. — Propovednik v{\yi}gl{\ia}del zadumciv{\yi}m, {\y}avno razm{\yi}xl{\ia}{\y}a o tom, cto kogda-nibudy necto podobno{\y}e predsto{\y}it sdelaty i {\y}emu. — Skaji, t{\yi} b{\yi} i vpravdu zabral {\y}evo kinjalom? On vedy vse-taki svetl{\yi}{\y}.

— Zabral b{\yi}. Potomu cto tako{\y} svetl{\yi}{\y} b{\yi}stro stanovilsa temn{\yi}m, a eto otnositsa k pr{\ia}mo{\y} ugroze l{\iu}d{\ia}m.

— Interesno, cto on vidit se{\y}cas? Raspahnut{\yi}{\y}e vrata? Sv{\ia}tovo Petra s kl{\iu}cami? Ili arhangela Miha{\y}ila?

— Bo{\y}usy, ne smogu udovletvority tvo{\y}e l{\iu}bop{\yi}tstvo. Pridetsa tebe proverity samomu. Dava{\y} spaty. — I, povernuvxisy k Pugalu, dobavil: — Docita{\y}ex, ne zabudy pogasity svecu.

I {\y}a usnul pod tihi{\y} xelest perelist{\yi}va{\y}em{\yi}h straniq.



— Vot sukin s{\yi}n! — sgor{\ia}ca pro{\y}iznes {\y}a.

Ot dnevnika burggrafa ostalasy lix odna oblojka. Straniqi b{\yi}li akkuratno v{\yi}rezan{\yi} i raskle{\y}en{\yi} po potolku. Cernila na nih namokli i raspolzlisy, tak cto procitaty bolyxe nicevo b{\yi}lo nelyz{\ia}.

— {\Y}a… — proble{\y}al Propovednik, tak i ne zakonciv predlojeni{\y}e.

Vse b{\yi}lo pon{\ia}tno. Kogda oduxevlenn{\yi}{\y} eto prodelal, star{\yi}{\y} pelikan gde-to brodil, poetomu ne smog razbudity men{\ia}. {\Y}a molca nacal odevatsa. Uje rassvelo, i pora b{\yi}lo otpravl{\ia}tsa v dorogu.

— Tam soderjalosy cto-to qenno{\y}e? — ostorojno po{\y}interesovalsa Propovednik.

— {\y}ex̨o vcera {\y}a b{\yi} skazal, cto net. Tepery uje ne uveren. — {\Y}a zastegnul po{\y}as s kinjalom.

— Mojet, eto odna iz {\y}evo nepon{\ia}tn{\yi}h xutok? Mojet, ono prosto razvleka{\y}etsa?

— Pojivem — uvidim.

— To {\y}esty t{\yi} nicevo ne budex delaty?

— V sm{\yi}sle begaty po okrestnost{\ia}m i iskaty oduxevlennovo, kotor{\yi}{\y} odnim x̨elckom palyqev mojet peremestitsa na t{\yi}s{\ia}cu lig, na pole, gde nahoditsa {\y}evo obolocka? Pugalo vernetsa, ono vsegda vozvrax̨a{\y}etsa. Mo{\y}ih planov eto nikak ne naruxalo.

— No tetrady…

— {\Y}esli cestno, {\y}a sobiralsa sjec {\y}e{\y}o {\y}ex̨o neskolyko dne{\y} nazad, no ruki nikak ne dohodili. Tak cto plevaty na tetrady. Kruso. Qerkovy Sv{\ia}tovo Miha{\y}ila. Vot mo{\y}a qely na segodn{\ia}.

Kak tolyko {\y}a okazalsa v zale, hoz{\ia}{\y}ka tut je kinulasy ko mne. V {\y}e{\y}o glazah citalsa vopros.

— On bolyxe ne pobespoko{\y}it nikovo, — skazal {\y}a, i ona rass{\yi}palasy v iskrennih blagodarnost{\ia}h.

Kogda {\y}a v{\yi}xel na uliqu, malycixka tut je podvel mne loxady.

Po sravneni{\y}u s proxl{\yi}m dnem segodn{\ia} b{\yi}lo {\y}asno i oceny teplo. Trakt konecno je okazalsa zabit telegami, vsadnikami i pexehodami. Vse xli v gorod, ctob{\yi} poklonitsa novo{\y} sv{\ia}t{\yi}ne i uvidety sled boso{\y} stupni, kotor{\yi}{\y} {\y}akob{\yi} angel ostavil pered domom devocki.

Kruso — sploxn{\yi}{\y}e sten{\yi} i baxni iz jeltovo kamn{\ia}. Gorod, ranyxe b{\yi}vxi{\y} stoliqe{\y} korolevstva, narodom okazalsa zaprujen nicuty ne menyxe, cem doroga. U R{\yi}bn{\yi}h vorot {\y}a popal v nesusvetnu{\y}u davku. Vokrug kricali molitv{\yi}, peli, ponosili drug druga, vizjali svinyi i orali te, kto poter{\ia}l v tolce{\y}e svo{\y}i koxelyki. To i delo melykali bel{\yi}{\y}e plax̨i palomnikov, qvetn{\yi}{\y}e lentocki na posohah. Kaka{\y}a-to gruppa krest{\y}an nesla krest, obhod{\ia} gorodski{\y}e sten{\yi} po krugu. K nim kajdu{\y}u minutu priso{\y}edin{\ia}lisy nov{\yi}{\y}e mol{\ia}x̨i{\y}es{\ia}, raspeva{\y}a ``Velicit duxa mo{\y}a Gospoda".

Pod kop{\yi}ta mo{\y}e{\y} loxadi brosilsa nix̨i{\y}, vop{\ia}, cto gr{\ia}det koneq sveta i {\y}a doljen poka{\y}atsa i otdaty {\y}emu vse denygi. Odnako, pon{\ia}v, cto {\y}a ne otlica{\y}usy osobo{\y} nabojnost{\y}u, on tut je zab{\yi}l obo mne i pristal k dvum dorodn{\yi}m kupqam, kotor{\yi}{\y}e b{\yi}li neskolyko perepugan{\yi} tem bezumi{\y}em, cto pro{\y}ishodilo vokrug.

U sledu{\y}ux̨ih vorot b{\yi}lo nicuty ne lucxe. Usilenn{\yi}{\y} otr{\ia}d straji sderjival tolpu. L{\iu}de{\y} nabralosy stolyko, cto mnogi{\y}e, poter{\ia}v nadejdu probratsa v gorod segodn{\ia}, razbivali ogromn{\yi}{\y} palatocn{\yi}{\y} lagery na golom pole.

— Kuda prex? — po-nararski zaoral na men{\ia} ustal{\yi}{\y} strajnik v polosatom berete. — Vali nazad!

{\Y}a pokazal {\y}emu kinjal, i men{\ia}, nesmotr{\ia} na rugany oceredi, propustili.

— Qerkovy Miha{\y}ila. Kak mne {\y}e{\y}o na{\y}ti? — sprosil {\y}a u soldata.

— Sprosi cevo polegce! — otmahnulsa tot. — Ih tut do certa, kak i bogomolyqev!

— Ne t{\yi} odin ne l{\iu}bix palomnikov, — hihiknul Propovednik.

Prixlosy rasspraxivaty na uliqah. Kako{\y}-to pareny so znakom gilydi{\y}i portn{\yi}h na kamzole pocesal v zat{\yi}lke:

— Eto ta, kotora{\y}a vozle kolodqa, cto ly?

— Znal b{\yi}, ne spraxival.

— {\Y}episkopska{\y}a, na Malo{\y} Zlotinke, po puti k vnutrenne{\y} stene.

{\Y}a poblagodaril {\y}evo i napravilsa k qentru goroda. Kruso do etovo {\y}a nikogda ne posex̨al, no rexil, cto zdesy, skore{\y}e vsevo, budet tak je, kak i v drugih mestah pri prazdnestvah, {\y}armarkah, sv{\ia}t{\yi}h palomnicestvah i svadybah kn{\ia}ze{\y} — vs{\e} dexevo{\y}e jilye rashvatano, i lezty tuda ne ime{\y}et nikakovo sm{\yi}sla. A vot v bogat{\yi}h ra{\y}onah, gde poro{\y} mogut za noc sodraty i cetverty florina, {\y}esli sovesty otsutstvu{\y}et, krovaty dl{\ia} putexestvennika vsegda na{\y}detsa.

Mo{\y} op{\yi}t men{\ia} ne obmanul. Posto{\y}al{\yi}{\y} dvor ``Pod korono{\y} kn{\ia}z{\ia}", raspolojenn{\yi}{\y} naprotiv star{\yi}h korolevskih kon{\iu}xen, qenami raspugal vseh jela{\y}ux̨ih i prinimal lix l{\iu}de{\y}, kotor{\yi}{\y}e ne oceny-to scitali denygi. Ostaviv loxady i sprosiv u hoz{\ia}{\y}ina dalyne{\y}xu{\y}u dorogu, {\y}a otpravilsa pexkom. Tak v{\yi}hodilo gorazdo b{\yi}stre{\y}e.

Qerkovy — sera{\y}a gromada, stisnuta{\y}a s dvuh storon jil{\yi}mi domami tak, cto predstavl{\ia}la s nimi {\y}edino{\y}e qelo{\y}e i otlicalasy ot nih lix xpilem, torcax̨im nad cerepicn{\yi}mi kr{\yi}xami. Na stupenykah sideli dvo{\y}e cumaz{\yi}h malycixek let des{\ia}ti, oni bez vs{\ia}kovo entuziazma prosili milost{\yi}n{\iu}. {\Y}a podergal dvery, no bezrezulytatno, hot{\ia} sl{\yi}xal, cto vnutri igra{\y}et organ.

— Zakr{\yi}to, d{\ia}decka, — skazal mne odin.

— No tam kto-to {\y}esty.

— Cto s tovo? Sv{\ia}x̨ennika-to net.

{\Y}a dostal neskolyko med{\ia}kov, kinul im v xapku.

— Spraxiva{\y}te, — stepenno pozvolil vtoro{\y} i v{\yi}ter rukavom sopliv{\yi}{\y} nos.

— Gde on i pocemu zakr{\yi}ta dvery?

— Vse sv{\ia}x̨enniki tepery vozle casovni na to{\y} storone krut{\ia}tsa. Gde deviqa videla angela.

— Da ne mogla ona nicevo videty! — vozmutilsa {\y}evo tovarix̨. — Ona ot rojdeni{\y}a slepa{\y}a!

— Vot potomu i videla, cto ne videla! — zasporil tot. — Tak oteq Seliko govoril! A on-to pobole, cem t{\yi}, zna{\y}et!

— A pocemu muz{\yi}ka igra{\y}et?

— Muz{\yi}kant repetiru{\y}et. On casto s{\iu}da prihodit. No qerkovy otkro{\y}etsa tolyko posle voskresen{\y}a.

— A cto budet v voskresenye?

Malycixka mnogoznacitelyno posmotrel v xapku:

— {\Y}esli uj vam leny u drugih uznavaty, gospodin, to v{\yi} nam {\y}ex̨o med{\ia}k na hlebuxek podkinyte.

{\Y}a rassme{\y}alsa {\y}evo nahalystvu, kinul dva.

— Slujba torjestvenna{\y}a. Kardinal pri{\y}edet. Iz Riapano, govor{\ia}t. Ctob{\yi} provesti messu dl{\ia} uvaja{\y}em{\yi}h jitele{\y} goroda. V cesty {\y}avleni{\y}a.

— Kak mne popasty v qerkovy?

Malycixki peregl{\ia}nulisy.

— Ne zna{\y}em, — otvetil tot, cto v{\yi}gl{\ia}del postarxe.

— Vraty t{\yi} ne ume{\y}ex, pri{\y}ately. — Mejdu ukazatelyn{\yi}m i srednim palyqem {\y}a derjal serebr{\ia}nu{\y}u monetku.

Deti naklonilisy drug k drugu, poxuxukalisy.

— Ladno, d{\ia}decka. Provedu.

Malycik zabral denejku, otdal {\y}e{\y}o svo{\y}emu pri{\y}atel{\iu}, kotor{\yi}{\y} tut je spr{\ia}tal sokrovix̨e za pazuhu.

— Idemte, d{\ia}decka.

On otvel men{\ia} v pereulok, ogl{\ia}delsa i tocno kotenok {\y}urknul v raspahnuto{\y}e sluhovo{\y}e okoxko, nahod{\ia}x̨e{\y}es{\ia} na urovne mostovo{\y}. Nado priznatsa, tuda b{\yi} {\y}a ne prolez pri vsem jelani{\y}i.

Liqo malycixki po{\y}avilosy v okoxke:

— Idite k podvalu, von tomu. {\Y}a x̨as dvery otkro{\y}u.

Spusk v podval toje b{\yi}l na uliqe, zakr{\yi}t{\yi}{\y} stalyn{\yi}m x̨itom. Klaqnula zadvijka, {\y}a podn{\ia}l nelegki{\y} l{\iu}k, spustilsa vniz. Malycixka provorno zax̨elknul zamok:

— {\Y}esli d{\ia}dyka Mikely uzna{\y}et, cto {\y}a snova zdesy laza{\y}u, on uxi otorvet. Dava{\y}te b{\yi}stre{\y}e, d{\ia}decka.

Podvalyno{\y}e pomex̨eni{\y}e pod domom, s nizkim potolkom i zat{\ia}nut{\yi}mi pautino{\y} uglami pohodilo na labirint. Lestniqa v{\yi}vela nas v polutemn{\yi}{\y} koridor. Zdesy silyno pahlo kvaxeno{\y} kapusto{\y} i koxkami. Gde-to za dver{\y}u, nadr{\yi}va{\y}asy, krical mladeneq. Xustr{\yi}{\y} malycixka bejal vpered, tak cto mne ostavalosy lix pospevaty za nim i ne vrezatsa golovo{\y} v v{\ia}zanki luka, svisa{\y}ux̨i{\y}e s potolka.

Cern{\yi}{\y} hod v{\yi}vel nas v malenyki{\y} vnutrenni{\y} dvor doma — gr{\ia}zn{\yi}{\y}, neuhojenn{\yi}{\y}, s pokosivxe{\y}s{\ia} golub{\ia}tne{\y} vozle zabora.

— Cerez zabor vam, — skazal malycixka i, bolyxe nicevo ne ob{\y}asn{\ia}{\y}a, skr{\yi}lsa v zdani{\y}i.

{\Y}a tak i postupil, blago perebratsa cerez pregradu b{\yi}lo neslojno. Qerkovn{\yi}{\y} dvor okazalsa {\y}ex̨o menyxe — tako{\y} tesn{\yi}{\y}, cto napominal komnatu v kako{\y}-nibudy provinqialyno{\y} taverne. Organ prodoljal igraty, i daje tolst{\yi}{\y}e kamenn{\yi}{\y}e sten{\yi} ne mogli prigluxity {\y}evo velicestvenn{\yi}{\y}e zvuki.

Malenyka{\y}a kalitka b{\yi}la poluotkr{\yi}ta, tak cto {\y}a voxel. Krome zvuka organa {\y}a sl{\yi}xal, kak nahod{\ia}x̨i{\y}es{\ia} vnizu podsobn{\yi}{\y}e raboci{\y}e razduva{\y}ut mehi muz{\yi}kalynovo instrumenta. Uzkimi zakutkami v{\yi}xel na balkon, otkuda otkr{\yi}valsa vid na kolonnadu, pust{\yi}{\y}e skamyi i {\y}arko-jelt{\yi}{\y} uzor na polu ot vitraje{\y}, v kotor{\yi}{\y}e svetilo solnqe. Spustilsa vniz po vito{\y} lestniqe, rexiv ne mexaty nevidimomu organistu, i sel na pervu{\y}u skam{\y}u.

Zakr{\yi}l glaza, sluxa{\y}a muz{\yi}ku. Ona b{\yi}la grandiozno{\y}, obyemno{\y} i, kazalosy, pronzala teb{\ia} naskvozy.

— Potr{\ia}sa{\y}ux̨e, — prozvucal u men{\ia} nad uhom golos Propovednika. — V ko{\y}i-to veki t{\yi} dovolen, nahod{\ia}sy v qerkvi.

— Cudesna{\y}a muz{\yi}ka, — v otvet pro{\y}iznes {\y}a. — Ne pobo{\y}usy etovo slova — bojestvenna{\y}a.

— I mnitsa mne, cto {\y}a sl{\yi}xu {\y}e{\y}o v perv{\yi}{\y} raz. — On b{\yi}l nemnogo raster{\ia}n. — K kako{\y} eto molitve?

— Ne ime{\y}u pon{\ia}ti{\y}a.

— Togda cemu t{\yi} ul{\yi}ba{\y}exsa?

— Tomu, cto mo{\y} dolgi{\y} puty okoncen.

On gl{\ia}nul na men{\ia} kak na sumasxedxevo. Hm{\yi}knul i pristro{\y}ilsa na lavke, ne jela{\y}a bolyxe nicevo spraxivaty. Tak m{\yi} i sideli, poka zvuki ne stihli pod svodami.

Organist voxel v zal, i okazalosy, cto eto jenx̨ina. V rukah ona derjala stopku ispisann{\yi}h not i na hodu cto-to cerkala v nih grifelem, ne zameca{\y}a men{\ia}. Tak cto {\y}a otlicno smog {\y}e{\y}o rassmotrety. Oceny malenyka{\y}a, huda{\y}a i tonenyka{\y}a, kak devocka. Iz-pod barhatnovo bereta gilydi{\y}i muziqirovani{\y}a vo vse storon{\yi} torcali vihrast{\yi}{\y}e cern{\yi}{\y}e volos{\yi}. Oni silyno otrosli i padali {\y}e{\y} na pleci. Milovidno{\y}e liqo b{\yi}lo sosredotoceno, lob nahmuren, krasiv{\yi}{\y}e gub{\yi} sjat{\yi}, a v uglah nemnogo raskos{\yi}h vostocn{\yi}h glaz, haraktern{\yi}h dl{\ia} teh, u kovo predki jili v Iliate, po{\y}avilisy morx̨inki.

— Redko mojno vstretity v qerkvi jenx̨inu-muz{\yi}kanta, — gromko skazal Propovednik.

Razume{\y}etsa, skazal dl{\ia} men{\ia}, ne duma{\y}a, cto kto-to drugo{\y} {\y}evo usl{\yi}xit.

No ona usl{\yi}xala i, vzdrognuv, {\y}edva ne uronila not{\yi}, po{\y}mav ih v posledni{\y} moment pokalecenno{\y} ruko{\y}. Prix̨urivxisy, devuxka s podozreni{\y}em gl{\ia}nula na Propovednika, hotela cto-to skazaty i nakoneq uvidela men{\ia}.

— Zdravstvu{\y}, Kristina, — negromko pro{\y}iznes {\y}a.

— Privet, Sineglaz{\yi}{\y}, — otvetila ta, kovo {\y}a tak dolgo iskal.

Voqarilosy molcani{\y}e. Porajenn{\yi}{\y} Propovednik tarax̨ilsa na nas, kak palomnik na snizoxedxevo na {\y}evo molitv{\yi} sv{\ia}tovo.

— Tvo{\y} kony skuca{\y}et.

{\Y}ee pleci rasslabilisy, i ona vzdohnula:

— {\Y}a toje oceny skuca{\y}u po V{\y}unu. No se{\y}cas {\y}emu lucxe b{\yi}ty s Miriam, cem so mno{\y}. Kak t{\yi} men{\ia} naxel?

— Cereda sluca{\y}noste{\y} i vezeni{\y}e. Hocu vernuty tebe ko{\y}e-cto.

{\Y}a prot{\ia}nul svo{\y}e{\y} b{\yi}vxe{\y} naparniqe braslet iz d{\yi}mcat{\yi}h rauhtopazov. Vot tepery straj de{\y}stvitelyno b{\yi}la porajena. {\y}e{\y}o not{\yi} — muz{\yi}ka, v kotoro{\y} devuxka duxi ne ca{\y}ala, — upali nam pod nogi. {\Y}a videl, kak drojat {\y}e{\y}o palyqi, kogda ona zabirala svo{\y} braslet.

— Nam nado seryezno pogovority, Ludwig. — {\y}e{\y}o golos sel i zvucal hriplo, no glaz ona ne opustila.

— Imenno eto {\y}a i hotel predlojity.



Komnat{\yi}, kotor{\yi}{\y}e ona snimala, nahodilisy nad bolyxo{\y} apteko{\y}, na vtorom etaje. Vhod b{\yi}l cerez torgov{\yi}{\y} zal. Sedovlas{\yi}{\y} i sedoborod{\yi}{\y} aptekary, malenyki{\y} i nelep{\yi}{\y}, posmotrel na men{\ia} poverh uvelicitelyn{\yi}h stekol, zakreplenn{\yi}h u nevo na nosu, no nicevo ne skazal, vernuvxisy k vesam, na kotor{\yi}h otmer{\ia}l kako{\y}e-to koricnevo{\y}e snadobye dl{\ia} pokupatel{\ia}.

Xurxa {\y}ubko{\y}, Kristina brosila not{\yi} na komod, dostala iz nevo but{\yi}lku vina, dva bokala:

— T{\yi} vs{\e} {\y}ex̨o pyex krasno{\y}e?

— Vrem{\ia} ot vremeni.

— Otkro{\y}. — Ona sela za stol, malenykimi palyqami zdorovo{\y} ruki perebira{\y}a gladki{\y}e d{\yi}mcat{\yi}{\y}e kamni. — Znacit, t{\yi} naxel {\y}evo?

Im{\ia} ne prozvucalo, no b{\yi}lo i tak pon{\ia}tno, pro kovo ona spraxiva{\y}et. Pro Gansa.

— Da. — {\Y}a v{\yi}tax̨il probku iz but{\yi}lki, plesnul v bokal{\yi} vina, sel naprotiv.

— I v{\yi}jil. T{\yi} vsegda b{\yi}l vezucim, Ludwig. Vezucim, kak cert. — Ona goryko usmehnulasy. — V otlici{\y}e ot nevo.

— V{\yi} b{\yi}li vmeste?

Ona ne stala otriqaty:

— Kako{\y}e-to vrem{\ia}. — Pomolcala i dobavila: — Oceny kratko{\y}e vrem{\ia}. T{\yi} udivlen?

— Se{\y}cas? Net. Vot kogda naxel tvo{\y} braslet u nevo — udivilsa. V{\yi} ne slixkom ladili posle tovo, kak t{\yi} podderjala ide{\y}u Miriam, cto u kajdovo kn{\ia}z{\ia} doljen b{\yi}ty personalyn{\yi}{\y} straj.

— {\Y}a scitala, cto politiceski eto polezno dl{\ia} Bratstva. — B{\yi}lo vidno, cto {\y}e{\y} nepri{\y}atn{\yi} vospominani{\y}a. — M{\yi} s Gansom rexili vse raznoglasi{\y}a. Tebe on ne hotel govority.

— Vaxe pravo i vaxi dela, — pojal {\y}a plecami. — Men{\ia} bolyxe interesu{\y}et, cto slucilosy v gorah.

Ona nervno krutanula stakan:

— Cert {\y}evo zna{\y}et, Sineglaz{\yi}{\y}. On {\y}ex̨o v Ardenau v{\yi}gl{\ia}del vstrevojenn{\yi}m. Govoril, cto naxel necto interesno{\y}e. Zatem {\y}evo otcitali stare{\y}xin{\yi} na sovete, t{\yi} vedy pomnix, kakovo slona oni sdelali iz to{\y} muhi?

{\Y}a kivnul.

— V obx̨em, on u{\y}ehal iz Alybalanda, a zatem, gde-to cerez mes{\ia}q, m{\yi} vstretilisy v Liseqke. On skazal, cto {\y}evo jdut dela na vostoke, zval s sobo{\y}, i {\y}a po{\y}ehala. V Bude m{\yi} natknulisy na temnu{\y}u duxu, zasevxu{\y}u v kolodqe. Gans toropilsa, govoril, cto {\y}emu vo cto b{\yi} to ni stalo nado popasty v Dorc-gan-To{\y}n, poprosil men{\ia} razobratsa s problemo{\y} i dojdatsa {\y}evo. Obex̨al vernutsa cerez poltor{\yi} nedeli.

— No t{\yi} ne dojdalasy.

— Ne stala jdaty, — ul{\yi}bnulasy ona, i {\y}a vspomnil, kako{\y} upr{\ia}mo{\y} poro{\y} stanovilasy Kristina. — Prikoncila tu tvary, vz{\ia}la deneg s burgomistra, ostavila V{\y}una v horoxe{\y} kon{\iu}xne i napravilasy sledom za nim, v gor{\yi}. No ne uspela. Kalikveq na vorotah, na mo{\y}e scastye, okazalsa serdobolyn{\yi}m celovekom. Skazal, cto {\y}evo brat{\y}a i Orden Pravednosti ubili straja. Cto {\y}a ne na{\y}du telo i mne sledu{\y}et uhodity kak mojno b{\yi}stre{\y}e.

{\Y}ee golos zadrojal, i ona po staro{\y} priv{\yi}cke prilojila pokalecenn{\yi}{\y} bez{\yi}m{\ia}nn{\yi}{\y} paleq klevo{\y} skule, prijala do boli, tak cto pobelela koja.

— M{\yi} daje ne poprox̨alisy. I {\y}a ne uvidela {\y}evo mogilu.

— T{\yi} poverila monahu?

— O! On b{\yi}l oceny ubeditelen. {\Y}a do sih por blagodar{\iu} {\y}evo za spaseni{\y}e.

— On mertv, — jestko skazal {\y}a. — Za to, cto predupredil teb{\ia}, {\y}evo rasp{\ia}li v led{\ia}no{\y} pex̨ere.

Ona lix othlebnula vina:

— Pusty na nebe {\y}evo duxe budet horoxo.

Kristina ne sprosila men{\ia}, otkuda {\y}a zna{\y}u, cto kalikveq mertv, a {\y}a ne stal {\y}e{\y} rasskaz{\yi}vaty, vo cto on prevratilsa posle smerti.

— Cto b{\yi}lo dalyxe?

— {\Y}a ne mogla mstity ubl{\iu}dkam s krasn{\yi}mi verevkami na r{\ia}sah. No mne hvatilo sil na zakonnikov. — Ul{\yi}bka u ne{\y}o b{\yi}la zlo{\y} i oceny nepri{\y}atno{\y}. {\Y}a nevolyno podumal, cto Kristina cem-to napomina{\y}et mne Miriam v {\y}e{\y}o ne sam{\yi}{\y}e lucxi{\y}e dni.

— I t{\yi} ubila vseh tro{\y}ih.

Ona potr{\ia}senno morgnula:

— T{\yi} i eto zna{\y}ex.

— Sl{\yi}xal, hoty oni i p{\yi}talisy skr{\yi}ty, cto v gorah, ne slixkom daleko ot monast{\yi}r{\ia}, naxli dva tela.

— Verno. Tretyevo {\y}a ranila iz arbaleta. Prijala {\y}evo k kamn{\ia}m, no on pr{\yi}gnul v reku, i {\y}evo unes potok. Nade{\y}usy, on ne v{\yi}pl{\yi}l.

— V{\yi}pl{\yi}l. {\Y}emu hvatilo sil, ctob{\yi} minovaty ux̨el{\y}a i v{\yi}{\y}ti v dolin{\yi} Brobergera, k objit{\yi}m mestam.

I, vid{\ia} vopros v {\y}e{\y}o glazah, po{\y}asnil:

— {\Y}a naxel {\y}evo kosti vozle odno{\y} derevuxki. Mestn{\yi}{\y}e socli, cto mertveq — straj. Sobstvenno govor{\ia}, imenno poetomu {\y}a okazalsa v Dorc-gan-To{\y}ne i tepery siju pered tobo{\y}.

— Straj? — nedoumenno naklonilasy ona ko mne. — Kakovo certa oni tak podumali?

— U nevo b{\yi}l kinjal Gansa.

— Prokl{\ia}tye! — Ona zakr{\yi}la liqo rukami i prostonala: — Prokl{\ia}tye!

Povisla tixina, {\y}a sl{\yi}xal lix {\y}e{\y}o prer{\yi}visto{\y}e d{\yi}hani{\y}e. Kogda ona ubrala ruki, {\y}e{\y}o glaza b{\yi}li soverxenno suhimi i zl{\yi}mi.

— T{\yi} sdal kinjal v Bratstvo?

— Konecno.

Ona vzdohnula.

— Horoxo. — I, slovno ubejda{\y}a seb{\ia}, dobavila: — Da. Horoxo. Tak budet lucxe. Dalyxe {\y}a zna{\y}u, cto slucilosy. T{\yi} ne sdalsa, kak b{\yi}valo i prejde. I naxel {\y}evo?

— V led{\ia}no{\y} pex̨ere. Gluboko pod monast{\yi}rem.

— Kak on umer?

— Srajalsa do poslednevo i zabral s sobo{\y} neskolykih. Duma{\y}u, cto usnul. Ot holoda i poteri krovi.

A cto {\y}a {\y}ex̨o mog {\y}e{\y} skazaty? Cto {\y}evo zakololi, slovno zver{\ia}? Kak zakololi Hartviga.

— T{\yi} pohoronil {\y}evo? — proxeptala mo{\y}a b{\yi}vxa{\y}a naparniqa.

— Net. No {\y}a uveren, cto tepery telo Gansa nikto ne pobespoko{\y}it.

{\Y}evo ne kosnutsa ni cervi, ni trupo{\y}ed{\yi} iz in{\yi}h sux̨estv, ni zlo, ni svet. On navecno ostanetsa vo mrake, poka svod pex̨er{\yi} ne obvalitsa i ne prevratitsa v savan dl{\ia} mo{\y}evo starovo druga.

Odinoka{\y}a slezinka pokatilasy po {\y}e{\y}o x̨eke, i Kristina pospexno, tocno st{\yi}d{\ia}sy, v{\yi}terla {\y}e{\y}o t{\yi}lyno{\y} storono{\y} ladoni.

— Spasibo.

— Za cto?

— Za to, cto naxel {\y}evo. Za to, cto rasskazal mne. Za to, cto {\y}a tepery zna{\y}u.

— No pocemu t{\yi} ne sdelala etovo? Stolyko let, Krista. M{\yi} vse tak dolgo {\y}evo iskali, ne sdavalisy, verili. A t{\yi} vs{\e} znala. Znala s samovo nacala, no ni certa nicevo ne skazala! Nikomu iz nas!

{\Y}a cuvstvoval, kak holodn{\yi}{\y} gnev pros{\yi}pa{\y}etsa u men{\ia} v grudi. On jil tam {\y}ex̨o s oseni, s teh por kak {\y}a pon{\ia}l, cto devuxka kak-to sv{\ia}zana s Gansom i {\y}evo isceznoveni{\y}em.

— Budy mo{\y}a vol{\ia}, nicevo ne govorila b{\yi} i dalyxe.

— Pocemu?

— A cto b{\yi}lo b{\yi}?! Cto b{\yi} togda slucilosy, Ludwig?! — kriknula ona mne v liqo, razom ter{\ia}{\y}a vs{\e} svo{\y}e spoko{\y}stvi{\y}e. — Skaji mne! T{\yi} b{\yi} prin{\ia}l eto?! Otoxel b{\yi} proc?! Skazal b{\yi}: nu cto podelaty, raz takova {\y}evo sudyba?! Kto iz teh, kovo m{\yi} zna{\y}em, otstupil?! Kto?!

Tepery slez{\yi} lilisy iz {\y}e{\y}o glaz neprer{\yi}vno, i ona ne stesn{\ia}lasy ih.

— {\Y}a sama otvecu: nikto! T{\yi}, Gertruda, Lyvenok, Xuko, Rozi ne ostalisy b{\yi} v storone, brosilisy b{\yi} spasaty to, cto uje nelyz{\ia} spasti, ili tovo huje — mstity. Kto iz nas obladal osob{\yi}m razumom des{\ia}ty let nazad? V{\yi} pogibli b{\yi}, kak i on. A {\y}esli b{\yi} vmexalisy ne m{\yi}, {\y}ediniqi, a qelo{\y}e Bratstvo? Tolyko predstavy, Ludwig, sam{\yi}{\y} seryezn{\yi}{\y} konflikt s Qerkov{\y}u za vs{\iu} naxu istori{\y}u. Nas b{\yi} sm{\ia}li i unictojili, {\y}esli b{\yi} m{\yi} tolyko posmeli podn{\ia}ty na nih ruku!

Ona b{\yi}la prava, no {\y}a vs{\e} ravno scital, cto {\y}e{\y}o molcani{\y}e slixkom jestoko dl{\ia} teh, kto do sih por jil nadejdo{\y}.

— Duma{\y} obo mne cto hocex, no, zaklina{\y}u, sohrani ta{\y}nu. Ne sto{\y}it nikomu znaty, cto Gans naxel smerty v monast{\yi}re kalikveqev. Eto slixkom opasna{\y}a informaqi{\y}a.

— T{\yi} b{\yi}la ne vprave rexaty za drugih, Kristina. Kaki{\y}e b{\yi} blagi{\y}e namereni{\y}a tobo{\y} ni dvigali.

— {\Y}a ni o cem ne jale{\y}u.

{\Y}a vz{\ia}l seb{\ia} v ruki, otkinuvxisy na stul:

— Pocemu {\y}evo ubili? Cto on hotel ot monahov?

— Ne zna{\y}u.

Ona v{\yi}derjala mo{\y} vzgl{\ia}d, no {\y}a lix vzdohnul:

— Eto loj.

— Pusty tak, — legko soglasilasy ona. — No pravda o pricinah smerti Gansa tepery nicevo ne izmenit. Vse davno zakoncilosy, Ludwig. Vse v proxlom. Ostavy {\y}evo, inace ono prosnetsa i ubyet teb{\ia}.

— T{\yi} vedy men{\ia} zna{\y}ex. {\Y}a vs{\e} ravno dokopa{\y}usy do istin{\yi}, pusty dl{\ia} etovo potrebu{\y}etsa {\y}ex̨o des{\ia}ty let.

— Ne s mo{\y}e{\y} pomox̨{\y}u. Prosti, no {\y}a ne jela{\y}u braty na seb{\ia} otvetstvennosty za tvo{\y}u smerty.

Nasta{\y}ivaty ne imelo sm{\yi}sla, i {\y}a otstupil.

— Horoxo. Zabudem o pricinah, pobudivxih Gansa otpravitsa v monast{\yi}ry. Rasskaji o tom, cto b{\yi}lo dalyxe. Posle tovo kak t{\yi} razobralasy s zakonnikami.

Ona vstala, zakr{\yi}la okno, {\y}ejasy ot holoda.

— Cto teb{\ia} interesu{\y}et?

— Cern{\yi}{\y} kinjal.

Kristina hm{\yi}knula:

— {\Y}a nacina{\y}u dumaty, cto t{\yi} ne Ludwig, a d{\y}avol.

— Oble{\y} men{\ia} osv{\ia}x̨enno{\y} vodo{\y}, {\y}esli teb{\ia} cto-to smux̨a{\y}et, — predlojil {\y}a {\y}e{\y}.

— K sojaleni{\y}u, net pod ruko{\y}, — nevolyno ul{\yi}bnulasy ona. — T{\yi} prav. Tako{\y} kinjal u men{\ia} b{\yi}l. Cto t{\yi} zna{\y}ex o klinke?

— T{\yi} vladela im kako{\y}e-to vrem{\ia}, zatem {\y}evo ukrali, on ob{\y}avilsa v Xossi{\y}i i pricinil nemalo nepri{\y}atnoste{\y}, poka m{\yi} s Miriam ne razobralisy s {\y}evo vladelyqem.

— Kinjal u ne{\y}e?

— Unictojen v prisutstvi{\y}i kn{\ia}ze{\y} Qerkvi.

Pro vtoro{\y} klinok, tot, cto prinadlejal imperatoru Konstantinu, dob{\yi}t{\yi}{\y} mno{\y} i Ranse v ta{\y}nike prejnevo Bratstva, {\y}a upominaty ne stal.

— {\y}ex̨o cto-nibudy?

— Sux̨i{\y}e meloci, Krista. Kinjal, kotor{\yi}{\y} t{\yi} v{\yi}pustila v mir, dela{\y}et duxi temn{\yi}mi.

Ona b{\yi}la nicuty ne udivlena. Ni kapli.

— {\Y}a rada tvo{\y}im znani{\y}am. M{\yi} sekonomim kucu vremeni, Ludwig. Mne ne potrebu{\y}etsa rasskaz{\yi}vaty tebe vs{\e} s samovo nacala i ubejdaty, cto eto pravda.

— Vse daje lucxe, cem {\y}a rasscit{\yi}val, — razdalsa cuty nasmexliv{\yi}{\y} golos u men{\ia} za spino{\y}. — Mojno srazu pristupaty k delu.

{\Y}a obernulsa i uvidel v dver{\ia}h pervovo pomox̨nika n{\yi}ne mertvovo markgrafa Valentina.

Koldun Valyter, s kotor{\yi}m m{\yi} rasstalisy pri ne sam{\yi}h lucxih obsto{\y}atelystvah, s ul{\yi}bko{\y} prislonilsa k kos{\ia}ku:

— Dobrovo tebe denecka, van Norma{\y}enn.

Rassto{\y}ani{\y}e do nevo {\y}a preodolel za odno mgnoveni{\y}e. Stul uletel v protivopolojnu{\y}u casty komnat{\yi}, a {\y}a okazalsa pered nenavistn{\yi}m koldunom. On, kajetsa, ne ojidal ot men{\ia} takih skoroste{\y}. {\Y}a uvidel, kak zast{\yi}va{\y}et ul{\yi}bka na {\y}evo liqe, i perv{\yi}m je udarom kulaka slomal {\y}emu nos. Broskom povalil na pol i, ne duma{\y}a, cto v l{\iu}bo{\y} moment on mojet primenity magi{\y}u, nacal delaty to, o cem mectal bolyxe goda.

Kristina s voplem povisla u men{\ia} na plecah:

— Ludwig! Ostavy {\y}evo! Prekrati! Nu je!

Certa s dva {\y}a sobiralsa {\y}e{\y}o sluxaty. No men{\ia} i vqepivxu{\y}us{\ia} Kristu otbrosilo v storonu. Potolok neskolyko raz krutanulsa pered glazami, i {\y}a ox̨util silynu{\y}u toxnotu. Dernulsa, p{\yi}ta{\y}asy vstaty i vernutsa k koldunu. Na etot raz {\y}a sobiralsa vospolyzovatsa ne kulakami, a kinjalom, no i tut mo{\y}a b{\yi}vxa{\y}a naparniqa ne razjala palyqev, povisnuv na mne, kak laska na ohotnicyem pse.

— Uspoko{\y}s{\ia}, cert teb{\ia} poderi! Stop! Hvatit! On drug! On mo{\y} drug!



Valyter to i delo trogal palyqami razbit{\yi}{\y}e gub{\yi}, na kotor{\yi}h zapeklasy krovy. Nos u nevo raspuh, lev{\yi}{\y} glaz zapl{\yi}l, no koldun ne sobiralsa jdaty polojenn{\yi}h dne{\y} do svo{\y}evo v{\yi}zdorovleni{\y}a. Sidel sebe v uglu da xeptal nagovor{\yi}.

— T{\yi} v norme? — Kristina prot{\ia}nula {\y}emu vlajnu{\y}u tr{\ia}piqu, i etot certov ubl{\iu}dok s blagodarnost{\y}u {\y}e{\y}o prin{\ia}l.

— B{\yi}valo i huje, — prognusavil on. — K zavtraxnemu dn{\iu} zajivet.

{\Y}a hotel u nevo sprosity, cto je on ne zajivil sebe xram, kotor{\yi}{\y} {\y}a ostavil, kogda kinul arbalet {\y}emu v liqo, no sderjalsa.

— Shodi k Filippu. On mojet pomoc.

Valyter lix skrivil gub{\yi} i tut je ob etom pojalel, tak kak nacala socitsa krovy.

— Prokl{\ia}t{\yi}{\y} deny! — rugnulsa on. — {\Y}a lucxe sam. Bez {\y}evo adskih pritirok i boltuxek. Za{\y}misy svo{\y}im vsp{\yi}lyciv{\yi}m kollego{\y}.

— {\Y}a b{\yi} tebe {\y}ex̨o dobavil, {\y}esli b{\yi} ne ona, — mracno zametil {\y}a.

— Ohotno ver{\iu}. {\Y}a b{\yi} s radost{\y}u vskip{\ia}til tvo{\y}i mozgi, {\y}esli b{\yi} ne ona, — ozlobilsa on.

— Zatknitesy oba i sidite tiho! — vsp{\yi}lila Kristina. — Konflikt{\yi} proxlovo ostanutsa v proxlom!

{\Y}a ne sobiralsa zab{\yi}vaty zastenki markgrafa Valentina, to, kak {\y}a b{\yi}l kuklo{\y} dl{\ia} bit{\y}a, i to, kak etot sid{\ia}x̨i{\y} v p{\ia}ti {\y}ardah ot men{\ia} hm{\yi}ry {\y}edva ne ukral kinjal Natana.

— T{\yi} ranyxe takim ne b{\yi}l… — Kristina ustalo opustilasy na stul peredo mno{\y}, perekr{\yi}v puty k koldunu.

— U nas star{\yi}{\y}e scet{\yi}.

— Zna{\y}u {\y}a o vaxih scetah. On rasskazal.

— Togda ne ponima{\y}u tvo{\y}evo udivleni{\y}a. {\Y}esli b{\yi} zdesy b{\yi}la Gertruda, ona b{\yi} uje razmazala {\y}evo po stenke.

— V{\yi}hodit, {\y}a legko otdelalsa. — Valyter vnovy pop{\yi}talsa ul{\yi}bnutsa, no vspomnil o gubah, i ul{\yi}bka prevratilasy v oskal.

Ona t{\ia}jelo vzdohnula:

— Ladno. O kinjale. Posle tovo kak {\y}a {\y}evo naxla, rexila, cto zakonniki pridumali cto-to svo{\y}e dl{\ia} sbora dux. No s duxami kinjal ne rabotal. {\Y}a ne smogla pon{\ia}ty, dl{\ia} cevo on nujen, vozila s sobo{\y} pocti polgoda.

— Odin celovek skazal mne, cto, kogda im dolgo vlade{\y}ex, nacina{\y}ut pro{\y}ishodity nepri{\y}atnosti. U teb{\ia} tako{\y}e b{\yi}lo?

— Na straje{\y} pravilo ne rasprostran{\ia}{\y}etsa, — vlez v razgovor Valyter. — Klinok nikak ne vli{\y}a{\y}et na teh, u kovo uje {\y}esty kinjal{\yi}. V ostalynom — sux̨a{\y}a pravda. Vex̨ dovolyno opasna{\y}a.

Kristina razdrajenno dernula plecom i prodoljila:

— {\Y}a sdala {\y}evo na hraneni{\y}e v ``Fabyen Klemenz i s{\yi}nov{\y}a" i, sobstvenno govor{\ia}, zab{\yi}la o nem na kako{\y}e-to kolicestvo let. Vspomnila, lix kogda uvidela opisani{\y}e cernovo kamn{\ia} iz knigi, cto lejala na stole u Miriam. Redki{\y} foliant, horoxi{\y}e risunki. Hagjitski{\y} {\y}a zna{\y}u dovolyno poverhnostno, no procitannovo hvatilo, ctob{\yi} pon{\ia}ty — glaz serafima dostatocno redka{\y}a i qenna{\y}a vex̨iqa.

— I t{\yi} zabrala oruji{\y}e. Da{\y} dogada{\y}usy — eto slucilosy v Barburge. I v etot je deny polucila dva udara nojom.

Kristina peregl{\ia}nulasy s Valyterom, i tot proronil:

— Govoril {\y}a tebe, on {\y}ex̨o tot umnik.

— Vse verno. Kak {\y}a ponima{\y}u, tebe rasskazal ob etom tot, kto pohitil klinok iz mo{\y}e{\y} sumki.

— Nu t{\yi} doljna b{\yi}ty {\y}emu blagodarna. On spas tvo{\y}u jizny, oplatil lekar{\ia} i komnatu. Kinjal ne prines {\y}emu nikakovo scast{\y}a, i on izbavilsa ot nevo. Otdal celoveku, kotorovo m{\yi} po{\y}mali v Xossi{\y}i. Kto te l{\iu}di, cto napali na teb{\ia}?

— Ne ime{\y}u pon{\ia}ti{\y}a. {\Y}a podozreva{\y}u, cto oni na{\y}emniki Ordena. On, — kivok v storonu kolduna, — scita{\y}et, cto storonniki celoveka, sozdavxevo kinjal.

— Interesno, — s somneni{\y}em prot{\ia}nul {\y}a.

— Cto ne tak? — Ona prekrasno cuvstvovala, kogda men{\ia} smux̨a{\y}ut fakt{\yi}.

— Na ko{\y} cert eto Ordenu? Da i kak oni voobx̨e uznali? T{\yi} vedy ne begala po uliqam i ne razmahivala takim oruji{\y}em napravo i nalevo. Tro{\y}e zakonnikov, kotor{\yi}h t{\yi} vstretila v gorah, mertv{\yi}. Kalikveqi, {\y}esli b{\yi} oni znali tvo{\y}e im{\ia} ili scitali, cto straj v{\yi}jila, dostali b{\yi} teb{\ia} iz-pod zemli i davno uje prikoncili. Dl{\ia} nih t{\yi} — vsevo lix bez{\yi}m{\ia}nna{\y}a jenx̨ina, kotoru{\y}u v liqo videl tolyko pogibxi{\y} monah-privratnik. M{\yi} vozvrax̨a{\y}emsa k sam{\yi}m legkim iz mo{\y}ih voprosov: kak oni uznali tvo{\y}e im{\ia}, raz t{\yi} nikomu {\y}evo ne govorila? pocemu pon{\ia}li, cto kinjal u teb{\ia}? otkuda dogadalisy, v kakom otdeleni{\y}i ``Fabyen Klemenz i s{\yi}nov{\y}a" t{\yi} {\y}evo zaberex i v kako{\y} deny, {\y}esli napali srazu je posle etovo?

— Tvo{\y}i predpolojeni{\y}a? — Valyter b{\yi}l tak l{\iu}bezen, cto pozvolil mne v{\yi}skazatsa.

— Kto napal — bez pon{\ia}ti{\y}a. O tom, kak naxli, — Kristina ostavila sled{\yi}. Zadela kolokolycik, kotor{\yi}{\y} usl{\yi}xali ne te uxi. No ona utverjda{\y}et, cto ni s kem ne govorila ni o sob{\yi}ti{\y}ah v gorah, ni o temnom kinjale.

— Eto tak, — podtverdila devuxka. — No {\y}a zadavala vopros{\yi} o glazah serafima. Spraxivala u kollekqionerov kamne{\y} i u hagjitskih torgovqev.

— Vozmojno, kto-to iskal to je samo{\y}e, cto i t{\yi}, i za{\y}interesovalsa celovekom, kotor{\yi}{\y} pro{\y}avl{\ia}{\y}et l{\iu}bop{\yi}tstvo v stoly speqificesko{\y} oblasti.

— No bolyxe nikto ne p{\yi}talsa napasty na teb{\ia} posle tovo sluca{\y}a. — Valyter rabotal nad svo{\y}im nosom, provod{\ia} si{\y}a{\y}ux̨imi palyqami i postepenno snima{\y}a otek.

— Kako{\y} sm{\yi}sl? {\Y}a perestala b{\yi}ty interesna. U men{\ia} bolyxe ne b{\yi}lo kinjala.

— No t{\yi} vs{\e} ravno slixkom mnogo znala, — ul{\yi}bnulsa {\y}a. — Licno {\y}a b{\yi} zaverxil delo, ctob{\yi} celovek ne sozdaval problem{\yi}.

— A t{\yi} izmenilsa. — Kristina vnimatelyno posmotrela na men{\ia}, zatem neohotno kivnula. — {\Y}a b{\yi} postupila tocno tak je. Raz uj t{\yi} men{\ia} raz{\yi}skal, nesmotr{\ia} na to cto {\y}a skr{\yi}va{\y}usy, to i ubi{\y}qi mogli. Dva goda — bolyxo{\y} srok.

Valyter smotrel na men{\ia} neotr{\yi}vno. {\Y}a znal, cevo on bo{\y}itsa, i pro{\y}iznes to, cto uje davno sidelo u men{\ia} v golove:

— Ostavity teb{\ia} jivo{\y} mojno b{\yi}lo lix po odno{\y} pricine — eto komu-to v{\yi}godno. K primeru, t{\yi} mojex privesti k klinku. Ili je {\y}ex̨o kak-to pomoc. Vot, dopustim, tvo{\y} drug. On vpolne mog nan{\ia}ty l{\iu}de{\y}, a zatem, kogda u nih nicevo ne v{\yi}xlo, vteretsa k tebe v doveri{\y}e i vsegda nahoditsa poblizosti.

B{\yi}vxi{\y} sluga markgrafa Valentina rassme{\y}alsa i podn{\ia}lsa so svo{\y}evo mesta:

— Pojalu{\y}, {\y}a po{\y}du shoju k Filippu. Inace {\y}a vse-taki kovo-nibudy v samom dele ub{\y}u.

On v{\yi}xel, a {\y}a, dojdavxisy, kogda {\y}evo xagi stihnut na lestniqe, vstal. Raspahnul dvery, prover{\ia}{\y}a, de{\y}stvitelyno li m{\yi} ostalisy odni.

Kristina sidela s neproniqa{\y}em{\yi}m liqom, no {\y}a videl, kak v {\y}e{\y}o temn{\yi}h glazah buxu{\y}et bur{\ia}.

— Kak davno t{\yi} {\y}evo zna{\y}ex?

— S teh por, kak men{\ia} {\y}edva ne ubili. Kogda {\y}a prixla v seb{\ia}, on b{\yi}l r{\ia}dom.

{\Y}a neveselo hohotnul:

— Oceny udobno. I vpis{\yi}va{\y}etsa v mo{\y}u teori{\y}u. Zabotlivo okazatsa podle posteli raneno{\y} v tot moment, kogda nujno, raz uj ne udalosy polucity klinok.

Ona ne jelala verity:

— Eto vsevo lix teori{\y}a, Sineglaz{\yi}{\y}. U teb{\ia} net nikakih dokazatelystv, vprocem, kak i u men{\ia}.

— T{\yi} ne slixkom doverciva{\y}a natura, Krista. Otcevo je poverila prohodimqu?

Devuxka dopila vino, podumala:

— Krome tovo cto on neskolyko raz spasal mo{\y}u jizny, Valyter oceny ubeditelen. {\Y}emu nujna pomox̨ straja. I {\y}a ver{\iu} v {\y}evo istori{\y}u. M{\yi} sto{\y}im na grani katastrof{\yi}, Ludwig. Do propasti, v kotoro{\y} buxu{\y}et plam{\ia}, vsevo odin xag. No nikto iz l{\iu}de{\y} daje ne podozreva{\y}et ob etom.

V komnate b{\yi}lo duxno, i {\y}a rasstegnul vorot rubahi.

— Katastrof{\yi} sluca{\y}utsa {\y}ejegodno. {\Y}esli ne epidemi{\y}a cum{\yi}, tak {\y}ustirski{\y} pot. {\Y}esli ne oceredna{\y}a {\y}ereticeska{\y}a sekta, risu{\y}ux̨a{\y}a na grav{\iu}rah Papu s kozlin{\yi}mi nogami, tak vo{\y}na. Celovek, sozda{\y}ux̨i{\y} kinjal{\yi}, otravl{\ia}{\y}ux̨i{\y}e duxi, bez somneni{\y}a, opasen. No ne slixkom li rano m{\yi} kricim ``apokalipsis!"?

— Etot nekto ruxit osnov{\yi}, Ludwig. On certovski talantliv{\yi}{\y} master, no ispolyzu{\y}et svo{\y} talant vo zlo. To, cto on dela{\y}et, nepravilyno. Valyter lovit kuzneqa uje ne perv{\yi}{\y} god.

— Tvo{\y} koldun lovit ne tolyko {\y}evo, no i straje{\y}. On ubi{\y}qa. Takih, kak m{\yi} s tobo{\y}.

— {\Y}a zna{\y}u.

— No osta{\y}exsa s nim?

Kristina upr{\ia}mo sjala gub{\yi}.

— I dalyxe budu.

— Nesmotr{\ia} na smerty teh, kto b{\yi}l tebe dorog?

Ona s sojaleni{\y}em opustila golovu, no otvetila tverdo:

— {\Y}esli kuzneq prodoljit, umret {\y}ex̨o bolyxe takih, kak m{\yi}. Vse Bratstvo. A Valyter i {\y}evo l{\iu}di se{\y}cas {\y}edinstvenn{\yi}{\y}e, kto mojet na{\y}ti i ostanovity temnovo mastera. Vse oceny seryezno, Ludwig. Valyter pokaz{\yi}val mne star{\yi}{\y}e manuskript{\yi} vremen Konstantina. Togda sux̨estvovalo lix dva takih klinka. Govor{\ia}t, ih dostavili s vostoka, s samo{\y} graniqi objit{\yi}h zemely. Zna{\y}ex, pocemu imperator sozdal straje{\y}? Iz-za prokl{\ia}t{\yi}h kinjalov. On jelal jity vecno, a dl{\ia} etovo {\y}emu trebovalisy duxi.

{\Y}a kivnul:

— Uje dumal ob etom. S pomox̨{\y}u cern{\yi}h klinkov on ubival l{\iu}de{\y}, temnil ih duxi. A zatem zabiral svetl{\yi}m, dobavl{\ia}{\y}a sebe jizny.

— {\Y}a ne zna{\y}u, kogda eto nacalosy. Pocti vse svidetelystva tovo vremeni unictojen{\yi}. No sob{\yi}ti{\y}a sv{\ia}z{\yi}va{\y}u s tem momentom, kogda iz zemely hagjitov na nax materik pri{\y}ehala sem{\y}a svetl{\yi}h kuzneqov. Kak govor{\ia}t legend{\yi}, oni — potomki odnovo iz ucenikov Christa. Oni stali kovaty kinjal{\yi} s sapfirami, no net ni odnovo podtverjdeni{\y}a, cto temn{\yi}{\y}e klinki — ih ruk delo. Qerkovy vz{\ia}la masterov pod svo{\y}u zax̨itu i na prot{\ia}jeni{\y}i mnogih pokoleni{\y} ih oberegala. Imperi{\y}a Konstantina rosla, kak i {\y}evo vlasty. A jizny dlilasy i dlilasy. No temn{\yi}{\y}e kinjal{\yi} privlekali na materik temn{\yi}{\y}e duxi. Razume{\y}etsa, te zarojdalisy i ranyxe. Grehi i prestupleni{\y}a nikto ne otmen{\ia}l. No, po utverjdeni{\y}am istorikov, ranyxe ih b{\yi}lo gorazdo menyxe, cem posle tovo, kak imperator stal obman{\yi}vaty smerty. Poetomu {\y}emu i potrebovalisy taki{\y}e, kak m{\yi}, — ocix̨aty zemli ot {\y}evo oxibok.

Nicevo udivitelynovo {\y}a dl{\ia} seb{\ia} ne uznal:

— Konstantin davno prevratilsa v prah. Kinjal{\yi} {\y}emu ne pomogli. Ne pomogut i tomu, kto dela{\y}et ih se{\y}cas. Cevo t{\yi} bo{\y}ixsa? Naxestvi{\y}a zl{\yi}h sux̨noste{\y}? M{\yi} spravimsa. Voln{\yi} temn{\yi}h dux zahlest{\yi}vali stran{\yi} i ranyxe, no Bratstvo vsegda ih pobejdalo.

— {\Y}a ne etovo opasa{\y}usy, Ludwig. Men{\ia} puga{\y}et necto ino{\y}e. T{\yi} zna{\y}ex, cto Konstantin otpravil vosemy ekspediqi{\y} na vostok, nade{\y}asy razdob{\yi}ty {\y}ex̨o podobnovo oruji{\y}a?

— Zapasliv{\yi}{\y} sukin s{\yi}n, — nevolyno voshitilsa {\y}a. — U nevo vedy ne polucilosy?

— Nikto ne vernulsa s kra{\y}a objit{\yi}h zemely. No Konstantin prodoljal iskaty i scital, cto {\y}esli sobraty des{\ia}ty temn{\yi}h klinkov, to oni stanut kl{\iu}com… — Ona sdelala pauzu, vnimatelyno nabl{\iu}da{\y}a za mo{\y}im liqom. — Kl{\iu}com dl{\ia} tovo, ctob{\yi} otkr{\yi}ty adski{\y}e vrata.

Mne potrebovalosy neskolyko sekund, ctob{\yi} perevarity {\y}e{\y}o slova i rassme{\y}atsa:

— Potr{\ia}sa{\y}ux̨e! Eto Valyter tebe skazal? I t{\yi} {\y}emu verix?!

— Ver{\iu}, Ludwig.

— Otkr{\yi}ty dorogu v ad. Tak ne b{\yi}va{\y}et.

— A b{\yi}va{\y}et, cto kinjal prevrax̨a{\y}et cistu{\y}u duxu, na kotoro{\y} net grehov, v temnu{\y}u sux̨nosty? — privela Kristina argument. — Skaji {\y}a tebe tako{\y}e god nazad, t{\yi} b{\yi} mne poveril? Ili vot tak je sme{\y}alsa?

— Ne poveril b{\yi}, — prixlosy priznaty mne. — Znacit, des{\ia}ty cern{\yi}h kinjalov otkro{\y}ut vrata v ad? Skaji, pojalu{\y}sta, kak Valyter otz{\yi}va{\y}etsa ob umstvenn{\yi}h sposobnost{\ia}h velikovo Konstantina? Na ko{\y} cert tomu iskaty stoly slojn{\yi}{\y} sposob samoubi{\y}stva? {\Y}esli gde-nibudy v Fringbou po{\y}avitsa predstavitelystvo ada, to ploho budet ne v odnom gorode, a vo mnogih stranah. Legion{\yi} demonov, sukkubov, certe{\y}, adsko{\y}e plam{\ia}, sera s nebes i proci{\y}e vex̨i. Ne scita{\y}a gibeli t{\yi}s{\ia}c l{\iu}de{\y}. Eto nemnogo nerazumno. Daje dl{\ia} Konstantina. Ne nahodix?

— Ob{\ia}zatelyno iskaty logiku, Ludwig?

— {\Y}esli hocex pon{\ia}ty motiv{\yi} drugovo celoveka? Da. Ob{\ia}zatelyno. Osobenno kogda ne verix na slovo koldunu, ohotivxemus{\ia} za klinkami straje{\y}.

— Ad toje mojet daty silu i vlasty. I Konstantin scital, cto, raz etovo ne da{\y}ut {\y}emu nebesa, nesmotr{\ia} na to cto on prin{\ia}l novu{\y}u religi{\y}u, otkazavxisy ot {\y}az{\yi}ceskih bogov, sledu{\y}et zakl{\iu}city inu{\y}u sdelku. Sozdaty prohod dl{\ia} teh, komu popasty v nax mir ne tak-to prosto. On scital, cto priobretet gorazdo bolyxe, cem poter{\ia}{\y}et.

{\Y}a lix razvel rukami:

— I o takom otkroveni{\y}i zna{\y}et tolyko nax obx̨i{\y} drug?

— Net. V Riapano eto toje izvestno. Poetomu dva klinka Konstantina unictojen{\yi} uje oceny davno.

Konecno. Tolyko odin. Vtoro{\y} lejit v sumke naxe{\y} obx̨e{\y} ucitelyniqi.

No skazal {\y}a sovsem drugo{\y}e:

— Polojim, vs{\e} tak, kak t{\yi} govorix. Poradu{\y}emsa, cto u Konstantina nicevo ne v{\yi}xlo. No tepery v mire po{\y}avilsa {\y}ex̨o odin bezumeq, kotoromu ne k cemu prilojity ruki, poetomu on sozda{\y}et oruji{\y}e, bole{\y}e opasno{\y}e, cem hagjitska{\y}a pescana{\y}a kobra. No ne hranit {\y}evo. I ne sobira{\y}et, a v{\yi}bras{\yi}va{\y}et v narod. Inace b{\yi} m{\yi} s tobo{\y} ne uvideli ni odnovo takovo klinka. Sledovatelyno, on {\y}avno ne jela{\y}et otkr{\yi}vaty nikakih mificeskih vrat. Tak?

— Net. Ne tak. M{\yi} podozreva{\y}em, cto tot, cto b{\yi}l u men{\ia}, okazalsa vsevo lix probn{\yi}m ekzempl{\ia}rom. {\Y}emu nado b{\yi}lo uznaty, rabota{\y}et li kinjal.

— I poetomu oruji{\y}e kakim-to obrazom po{\y}avilosy u predstavitel{\ia} Ordena? — {\Y}a b{\yi}l polon skeptiqizma.

— Pocemu b{\yi} i net? Oni speqialist{\yi} v takih voprosah. Raz prover{\ia}{\y}ut naxi klinki, vid{\ia}t istori{\y}u sobrann{\yi}h dux, to, vozmojno, on nade{\y}alsa, cto dadut oqenku i {\y}evo rabote.

— {\Y}esty dva ``no". {\Y}a ne l{\iu}bl{\iu} Orden, no delo oni zna{\y}ut. Po{\y}avisy sredi nih podobn{\yi}{\y} celovek, oni {\y}avno b{\yi} ne okaz{\yi}vali {\y}emu uslugi, a potax̨ili k sebe v podval{\yi}. Ili je srazu prikoncili.

— A mojet, kuzneq zakl{\iu}cil sdelku tolyko s odnim iz nih. S sam{\yi}m necistoplotn{\yi}m, — vesko vozrazila ona. — A tvo{\y}e vtoro{\y}e ``no"?

— {\y}ex̨o odin temn{\yi}{\y} kinjal putexestvoval po miru bez svo{\y}evo sozdatel{\ia}. Klinok naxel odin inkvizitor v sumke gonqa, oderjimovo besom. Goneq umer, tak nicevo i ne uspev rasskazaty. A klinok kliriki unictojili.

— T{\yi} uveren v eto{\y} informaqi{\y}i?

— {\Y}a vmeste s Gertrudo{\y} videl oblomki oruji{\y}a u kardinala di Travinno. V tvo{\y}e{\y} teori{\y}i, cto kuzneq rexil proverity, rabota{\y}et li {\y}evo tvoreni{\y}e, togda kak ostalyn{\yi}{\y}e klinki on derjit pri sebe, {\y}esty nekotora{\y}a nesoglasovannosty. Proverka kinjala — zvucit kra{\y}ne nat{\ia}nuto. On mog eto sdelaty i sam, raz ume{\y}et sozdavaty taki{\y}e vex̨i. {\Y}a duma{\y}u, {\y}edinstvenna{\y}a pricina, pocemu kuzneq mog otdaty temn{\yi}{\y} klinok komu-to iz Ordena, — eto plata. Plata za pomox̨ i sotrudnicestvo.

No Kristina scitala inace:

— Otn{\iu}dy. Teori{\y}a lix ukrepilasy s tvo{\y}im rasskazom. Poduma{\y} sam. Pervo{\y}e oruji{\y}e ne doxlo do adresata. {\Y}evo perehvatili i unictojili kliriki. Poetomu po{\y}avilsa tot, vtoro{\y}, v itoge popavxi{\y} mne v ruki.

Versi{\y}a zvucala cuty bole{\y}e skladno, cem pred{\yi}dux̨a{\y}a. No ne namnovo. {\Y}a pomnil, kak v Riapano govorili o tom, cto oteq Mart naxel klinok tri goda nazad. A Kristina zapolucila svo{\y} na semy let ranyxe, sledovatelyno, {\y}e{\y}o klinok nikak ne mog b{\yi}ty vtor{\yi}m.

— Valyter tak uveren v podobnom razviti{\y}i sob{\yi}ti{\y}? — Po mo{\y}emu tonu b{\yi}lo pon{\ia}tno, naskolyko silyno {\y}a ``qen{\iu}" mneni{\y}e kolduna.

Ona podn{\ia}la vverh ladoni:

— Sluxa{\y}. On neob{\yi}cn{\yi}{\y} celovek. I jestoki{\y}. V drugo{\y}e vrem{\ia} {\y}a b{\yi} ubila {\y}evo ne zadum{\yi}va{\y}asy. Za vs{\e} zlo, cto on pricinil Bratstvu. Da net! K certu Bratstvo! Za vs{\e} to, cto on sdelal tebe. No, kak {\y}a govorila, situaqi{\y}a oceny silyno izmenilasy. {\Y}a mnogo{\y}e uznala za paru poslednih let, i mo{\y}e otnoxeni{\y}e k jizni perevernulosy. On — menyxe{\y}e zlo i mojet spasti vseh nas.

— Zlo ne mojet b{\yi}ty malenykim ili bolyxim. Zlo osta{\y}etsa zlom. {\Y}a podhoju k slucivxemus{\ia} bez emoqi{\y}. Vo vs{\ia}kom sluca{\y}e, se{\y}cas. On ne smog otsledity zakonnikov, no naxel teb{\ia}. Ugada{\y}, kuda spexil goneq, ubit{\yi}{\y} inkvizitorom? V zamok Latka, vladelyqem kotorovo b{\yi}l markgraf Valentin, a {\y}emu slujil tvo{\y} razl{\iu}bezn{\yi}{\y} koldun. Vse ukaz{\yi}va{\y}et na to, cto on iskal temn{\yi}{\y} kinjal.

— {\Y}a toje {\y}evo ix̨u, kak i drugi{\y}e l{\iu}di Valytera. M{\yi} nade{\y}emsa, cto artefakt privedet nas k kuznequ. — Ona podalasy vpered, nakr{\yi}la mo{\y}u ruku svo{\y}e{\y}, vkradcivo skazav: — Sluxa{\y}. Skore{\y}e vsevo, t{\yi} prav. Za tem napadeni{\y}em de{\y}stvitelyno mog sto{\y}aty on. Slixkom mnogo sovpadeni{\y}. No eto nicevo ne men{\ia}{\y}et, Ludwig. {\Y}a nujna {\y}emu, a koldun nujen mne. U nas odna qely. I drug bez druga m{\yi} ne obo{\y}demsa. {\Y}a mnogim pojertvovala radi tovo, ctob{\yi} pop{\yi}tatsa na{\y}ti temnovo mastera. Slixkom mnogim. I otstupaty se{\y}cas… Povery, {\y}a prosto ne mogu tak postupity.

— On opasen, Kristina.

— {\Y}a eto zna{\y}u lucxe, cem t{\yi}. U nevo t{\yi}s{\ia}ca i odin nedostatok, no daje tako{\y} celovek, kak on, mojet spasti nax mir.

Eto b{\yi}lo tak smexno sl{\yi}xaty. Valyter — spasitely celovecestva. Po mne, tak eto mir sledu{\y}et izbavl{\ia}ty ot nevo.

— T{\yi} iscezla i ne podavala o sebe veste{\y} pocti god. — {\Y}a smenil temu. — M{\yi} volnovalisy za teb{\ia}.

Ona otvela glaza, skazav tihim golosom:

— Prosti. U men{\ia} ne b{\yi}lo v{\yi}bora. Proxlo{\y} vesno{\y} m{\yi} s Valyterom vlipli v nepri{\y}atnosti, kogda sbili so sleda kuzneqa klirikov. Uverena, v otlici{\y}e ot nas oni ne hot{\ia}t {\y}evo ubivaty. Po{\y}mav kuzneqa, Riapano polucit v svo{\y}i ruki ogromnu{\y}u vlasty. Poetomu t{\yi} ponima{\y}ex, kak vajno nam na{\y}ti {\y}evo perv{\yi}mi?

— Znacit, v{\yi} hotite ubity zagadocnovo mastera?

Ona gor{\ia}co kivnula:

— Sam{\yi}m b{\yi}str{\yi}m sposobom iz vseh vozmojn{\yi}h. Ctob{\yi} nikto ne uznal {\y}evo sekretov. Oni doljn{\yi} umerety vmeste s nim.

— A sv{\ia}x̨enniki? Vdrug v{\yi} oxiba{\y}etesy, i u nih taka{\y}a je qely, kak i u vas, — ubity {\y}evo.

— Ne oxiba{\y}emsa, — bezapell{\ia}qionno za{\y}avila ona. — Kak tolyko oni pon{\ia}li, cto m{\yi} toje ix̨em {\y}evo, poslali za nami svo{\y}ih ubi{\y}q. Povery, Ludwig, eto b{\yi}lo straxno. P{\ia}tero iz naxevo otr{\ia}da pogibli. M{\yi} nasilu uxli i vot uje kotor{\yi}{\y} mes{\ia}q skr{\yi}va{\y}emsa. I {\y}a ne mogu vernutsa v Ardenau. Daje pisymo napisaty komu-libo iz straje{\y} ne mogu. Ono podstavit pod udar l{\iu}bovo.

— Organist v qerkvi, — ul{\yi}bnulsa {\y}a. — T{\yi} ne slixkom-to horoxo skr{\yi}va{\y}exsa.

— Lucxe pr{\ia}tatsa ot sobaki v {\y}e{\y}o je budke. Tam ona budet iskaty v posledn{\iu}{\y}u oceredy. Nikto ne smotrit na muz{\yi}kantov.

— Ubegaty vecno ne polucitsa.

— {\Y}a i ne stanu. Nujno lix unictojity zlo. Vse ostalyno{\y}e nevajno.

{\Y}a videl, cto ona oderjima ide{\y}e{\y} na{\y}ti celoveka, ku{\y}ux̨evo temn{\yi}{\y}e kinjal{\yi}, tocno tak je, kak Miriam vot uje vek ne da{\y}ut poko{\y}a po{\y}iski kuzneqa, sozda{\y}ux̨evo klinki dl{\ia} Bratstva. {\Y}a ponimal, cto ne otgovor{\iu} {\y}e{\y}e, cto spority bessm{\yi}slenno i to, cto, pri{\y}ehav s{\iu}da cerez neskolyko stran, {\y}a uznal, cto ona jiva i o pro{\y}izoxedxem s ne{\y} i Gansom, eto i {\y}esty vse, cevo {\y}a dostig.

— Viju, cto jelani{\y}e spasti mir veliko.

— Mir? — Ona izognula brovy. — Plevaty {\y}a hotela na nevo. {\Y}a spasa{\y}u ne mir, a Bratstvo. Zax̨ix̨a{\y}u {\y}evo v meru otpux̨enn{\yi}h sil i umeni{\y}a.

— Tolyko v Bratstve ob etom ne podozreva{\y}ut.

— I horoxo. Menyxe problem i mne, i im.

— M{\yi} sux̨estvu{\y}em pocti poltor{\yi} t{\yi}s{\ia}ci let. U nas slucalisy razn{\yi}{\y}e nepri{\y}atnosti, no Bratstvo vsegda v{\yi}jivalo i ostavalosy na nogah. Ostanetsa i vpredy. Tebe nezacem sklad{\yi}vaty golovu lix radi nepodtverjdenn{\yi}h slov kolduna. Po{\y}ehali v Ardenau. Pr{\ia}mo se{\y}cas. Bratstvo dogovoritsa nascet teb{\ia} s Riapano. Gertruda pomojet. A di Travinno tolyko poradu{\y}etsa informaqi{\y}i, kotora{\y}a tebe izvestna. M{\yi} zax̨itim teb{\ia}.

Kristina grustno rassme{\y}alasy:

— Mne otradno znaty, cto t{\yi} do sih por p{\yi}ta{\y}exsa spasti mo{\y}u golovu, Ludwig. {\Y}a b{\yi} oceny hotela, ctob{\yi} vs{\e} b{\yi}lo kak dvenadqaty let nazad, kogda m{\yi} plecom k plecu otrajali natisk temn{\yi}h dux. No mne uje kajetsa, cto naxa {\y}unosty ne bole{\y}e cem mif, kotor{\yi}{\y} {\y}a sama sebe pridumala. A se{\y}cas vokrug men{\ia} realynosty, i ona oceny straxna. T{\yi} prosto poka ne mojex oqenity tovo ujasa, kotor{\yi}{\y} isp{\yi}t{\yi}va{\y}u {\y}a, pon{\ia}ty vse{\y} seryeznosti problem{\yi}. I soverxa{\y}ex tu je samu{\y}u oxibku, kak togda s tem kartografom.

V golove u men{\ia} trevojno zv{\ia}knulo.

— T{\yi} konecno je vs{\e} zna{\y}ex.

— Zna{\y}u. Vedy ispravl{\ia}ty tvo{\y}i oxibki prixlosy mne.

{\Y}a prix̨urilsa:

— Znacit, vot kto ubil {\y}evo.

Ona daje ne p{\yi}talasy otriqaty:

— A t{\yi} ostavil mne v{\yi}bor, kogda pro{\y}avil jalosty? Tebe nado b{\yi}lo privezti {\y}evo v Bratstvo, a ne otpuskaty na vse cet{\yi}re storon{\yi}. Togda b{\yi} Miriam ne prosila men{\ia} spasaty situaqi{\y}u!

Tepery ponima{\y}u, pocemu Gertruda nicevo mne ne rasskazala. {\Y}a pokacal golovo{\y}:

— T{\yi} govorila, cto {\y}a izmenilsa. T{\yi} izmenilasy ne menyxe men{\ia}. I {\y}a sojale{\y}u ob etom. Prejn{\ia}{\y}a Kristina nikogda b{\yi} ne stala ubi{\y}qe{\y} na pobeguxkah u magistrov.

{\Y}ee liqo iskazilosy ot obid{\yi}, i ona proxipela:

— T{\yi} de{\y}stvitelyno tak i ne pon{\ia}l, cto togda pro{\y}izoxlo?! Iz-za cevo oni sto{\y}ali na uxah?

— Pon{\ia}l. Ispugalisy novovo messi{\y}i i tovo, cto on mojet naucity drugih l{\iu}de{\y} snimaty grehi s l{\iu}dskih dux. V perspektive Bratstvo stalo b{\yi} nikomu ne nujno.

Ona dvajd{\yi} bezzvucno hlopnula v ladoxi:

— Potr{\ia}sa{\y}ux̨e! T{\yi} uvidel m{\yi}x, no ne zametil koxku. Nikto iz teh, kto b{\yi}l v kurse situaqi{\y}i, ne bo{\y}alsa dalekovo budux̨evo. M{\yi} opasalisy nasto{\y}ax̨evo. A ono takovo: kartograf po kako{\y}-to nasmexke sudyb{\yi} mog ocix̨aty duxi straje{\y}. No on zabiral ne tolyko naxi grehi, no i nax dar. M{\yi} stanovilisy ob{\yi}cn{\yi}mi l{\iu}dymi, takimi, kak t{\yi}s{\ia}ci drugih ob{\yi}vatele{\y}, Ludwig. Lixivxisy dara, m{\yi} ne mogli delaty svo{\y}u rabotu. A tepery tolyko predstavy, cto b{\yi} b{\yi}lo, {\y}esli b{\yi} {\y}evo, k primeru, zahvatil Orden? I ispolyzoval protiv nas. A {\y}esli b{\yi} kartograf naucil kovo-to i po{\y}avilosy neskolyko takih l{\iu}de{\y}? Des{\ia}tok? Sotn{\ia}? Armi{\y}a! M{\yi} b{\yi}li b{\yi} unictojen{\yi}, slovno gorod, v kotor{\yi}{\y} popal celovek, zarajenn{\yi}{\y} {\y}ustirskim potom.

— I mnogih li straje{\y} Hartvig lixil ih rabot{\yi}?

— Slava bogu — ni odnovo.

— Togda prosti, no tvo{\y}i slova ne bole{\y}e cem nelepa{\y}a fantazi{\y}a.

— Odna taka{\y}a fantazi{\y}a slujila poslednemu potomku imperatora Konstantina. I kogda Bratstvo ne vernulo {\y}emu kinjal, na kotor{\yi}{\y} pretendoval koroly Progansu, v delo vstupil tako{\y} je, kak tvo{\y} Hartvig. Cetvero magistrov i opomnitsa ne uspeli, kak lixilisy svo{\y}evo dara. Togda {\y}evo ubili vmeste s korolem, i zavertelasy vs{\ia} eta kaxa. Tepery, pomn{\ia} o proxlom, Bratstvo ne stalo jdaty nacala epidemi{\y}i, a unictojilo bolynovo do {\y}evo po{\y}avleni{\y}a v gorode. — Ona s v{\yi}zovom posmotrela na men{\ia}. — Scita{\y}ex, cto v Ardenau oxiblisy?

{\Y}a vzdohnul, vstal iz-za stola, tak i ne pritronuvxisy k bokalu vina:

— Ne ime{\y}et sm{\yi}sla ter{\ia}ty vrem{\ia} na spor{\yi}. Magistr{\yi} uvideli opasnosty. Realynu{\y}u ili mnimu{\y}u, ne mne sudity. No celovek, kotor{\yi}{\y} mog sdelaty mir cuty lucxe, mertv. I mne jaly tovo, cto tepery nikogda ne slucitsa.

— Bo{\y}usy, {\y}a ne smogu teb{\ia} pon{\ia}ty, Ludwig. M{\yi} stali slixkom razn{\yi}mi, — s grust{\y}u skazala ona. — Kogda t{\yi} u{\y}ezja{\y}ex iz goroda?

— {\y}ex̨o ne rexil, — cestno otvetil {\y}a.

— Ostanysa. Mne nujna tvo{\y}a pomox̨, i t{\yi} {\y}edinstvenn{\yi}{\y}, komu {\y}a mogu dover{\ia}ty zdesy. Obex̨a{\y}, cto primex vzvexenno{\y}e rexeni{\y}e.

{\Y}a posmotrel v {\y}e{\y}o glaza i vopreki svo{\y}emu jelani{\y}u otkazaty kivnul…

Propovednik sidel na pervom etaje, v apteke. On diplomaticno ne stal sluxaty nax razgovor s Kristino{\y} i se{\y}cas zanimalsa tem, cto s nenavist{\y}u pos{\yi}lal prokl{\ia}ti{\y}e za prokl{\ia}ti{\y}em na golovu Valytera, kotor{\yi}{\y}, ne zameca{\y}a svetlo{\y} duxi, negromko besedoval s sedoborod{\yi}m aptekarem.

— Van Norma{\y}enn. — Koldun vstal, zakr{\yi}va{\y}a mne v{\yi}hod. — M{\yi} ploho nacali. B{\yi}ty mojet, se{\y}cas samo{\y}e vrem{\ia} vs{\e} ispravity?

— Ugolyev tebe nado v glaza napihaty, d{\y}avolysko{\y}e otrodye! — besnovalsa Propovednik.

— Ne duma{\y}u, — holodno pro{\y}iznes {\y}a.

— Nu hot{\ia} b{\yi} na vrem{\ia}. Ctob{\yi} ne rasstra{\y}ivaty cudesnu{\y}u Kristinu.

{\Y}a xagnul k nemu navstrecu i skazal tak tiho, ctob{\yi} sl{\yi}xal tolyko on:

— {\Y}a ne ver{\iu} ni odnomu tvo{\y}emu slovu. I t{\yi} jiv tolyko potomu, cto ona men{\ia} ostanovila. Poetomu s dorogi. Poka {\y}a ne ubil teb{\ia} za to, cto t{\yi} delal so strajami.

On usmehnulsa, sdelal xag v storonu i, kogda {\y}a uje v{\yi}hodil, kriknul mne v spinu:

— Poduma{\y} o tom, cto {\y}a skazal! Adski{\y}e vrata! V odnom iz naxih gorodov. I kogda oni otkro{\y}utsa, poblizosti ne okajetsa angelov, kotor{\yi}{\y}e steregut poko{\y} celovecestva na vostoke. M{\yi} budem predostavlen{\yi} sami sebe!



{\Y}a pisal b{\yi}stro, to i delo okuna{\y}a pero v cernilyniqu, i Propovednik, uznavxi{\y} osnovno{\y}e soderjani{\y}e naxevo s Kristino{\y} razgovora, po{\y}interesovalsa:

— Dl{\ia} cevo vs{\e} eto?

— Konkretiziru{\y}, — poprosil {\y}a {\y}evo.

— Pisymo. Zacem ono?

— Potomu cto sozdalasy opasna{\y}a situaqi{\y}a. I {\y}esli so mno{\y} cto-to slucitsa, hoty kto-to doljen b{\yi}ty v kurse tovo, cto zdesy proishodit.

— Gertruda ne budet scastliva.

{\Y}a podn{\ia}l na nevo vzgl{\ia}d:

— Nade{\y}usy, ona nicevo ne uzna{\y}et. Ne jela{\y}u vput{\yi}vaty {\y}e{\y}o v eto.

— Togda komu je t{\yi} pixex?

— Miriam. Bratstvo doljno b{\yi}ty gotovo k nepri{\y}atnost{\ia}m, {\y}esli kliriki rexat sprosity o Kristine i {\y}e{\y}o delah.

On pomolcal, sluxa{\y}a, kak skripit pero:

— Etot Valyter, on kak bexena{\y}a sobaka. {\Y}evo nado ubity.

— Pri{\y}atno znaty, cto m{\yi} shodimsa vo mneni{\y}ah i ne duma{\y}em o bible{\y}skih zapoved{\ia}h. — {\Y}a dal cernilam v{\yi}sohnuty. — No beda v tom, cto u nevo {\y}esty informaqi{\y}a. O tom je temnom kuzneqe.

— T{\yi} ne dumal, cto on vret?

— Nascet adskih vrat? Vpolne vozmojno. No odno ne otmen{\ia}{\y}et drugovo. Pohoje, on de{\y}stvitelyno ix̨et temnovo mastera. I {\y}esli ne dl{\ia} ubi{\y}stva, to dl{\ia} svo{\y}ih qele{\y}. Ili cyih-to {\y}ex̨e. Kristina uverena, cto oni pocti naxli kuzneqa. Razumno b{\yi}lo b{\yi} nahoditsa r{\ia}dom s nimi.

— Tvo{\y}a b{\yi}vxa{\y}a naparniqa — sumasxedxa{\y}a.

— Ona b{\yi} ne vv{\ia}zalasy v etu istori{\y}u, {\y}esli b{\yi} ne verila v to, cto govorit.

{\Y}a slojil bumagu, ubral {\y}e{\y}o v konvert.

— Kak Kristina lixilasy palyqev?

— Kogda nacala rabotaty odna, — neohotno otvetil {\y}a.

— To {\y}esty bez teb{\ia}?

— Da.

On pon{\ia}l, cto {\y}a ne jela{\y}u prodoljaty etu temu:

— {\Y}a shodil gl{\ia}nul na mesto, gde {\y}akob{\yi} b{\yi}l angel. Tolpix̨a, kak pered ra{\y}skimi vratami. Soldat{\yi}, zevaki, mol{\ia}x̨i{\y}es{\ia}, sv{\ia}x̨enniki. Vs{\ia} eta l{\iu}dska{\y}a massa kricit, gudit, oret, po{\y}et i {\y}edva li ne la{\y}et. I vs{\e} radi odnovo otpecatka boso{\y} nogi, ostavxegos{\ia} na bruscatke.

— I cem je on neob{\yi}cen?

— Tem, cto vplavlen v cern{\yi}{\y} kameny, a sam oslepitelyno-bel. I govor{\ia}t, cto blagouha{\y}et jasminom.

— Jasmin v konqe zim{\yi} — eto pohoje na cudo. — {\Y}a zapecatal konvert al{\yi}m surgucom.

— I za pravo prikosnutsa gubami k etomu cudu derutsa. A nekotor{\yi}{\y}e proda{\y}ut svo{\y}e mesto v oceredi za des{\ia}ty dukatov.

— B{\yi}lo b{\yi} stranno, {\y}esli b{\yi} zab{\yi}li o najive, — melanholicno pro{\y}iznes {\y}a.

M{\yi}, l{\iu}di, vsegda na{\y}dem cto prodaty i cto kupity. {\Y}edu, zemli, titul{\yi}, zvani{\y}a, sv{\ia}t{\yi}{\y}e mox̨i ili mesto poblije k ra{\y}skim vratam.

— Segodn{\ia} {\y}a vozblagodaril Gospoda, cto umer. Pravo, budy {\y}a jiv, certa s dva smog b{\yi} dobratsa do relikvi{\y}i i uvidety sv{\ia}tu{\y}u Djuli{\y}u.

{\Y}a ubral pisymo za golenix̨e sapoga:

— A eto {\y}ex̨o kto?

— Slepa{\y}a devocka, s kotoro{\y} govoril angel.

— Cto, uje b{\yi}la kanonizaqi{\y}a? — s ironi{\y}e{\y} pro{\y}iznes {\y}a, i tak zna{\y}a otvet.

K liku sv{\ia}t{\yi}h pricisl{\ia}li ne ranyxe cem cerez p{\ia}ty let posle smerti pretendenta, i ctob{\yi} dostic stoly v{\yi}sokovo zvani{\y}a — slov o tom, cto govoril s angelom, nedostatocno.

— Konecno net. Prosto tak {\y}e{\y}o naz{\yi}va{\y}et narod.

— Narod… — {\Y}a v{\yi}xel na uliqu, taku{\y}u je xumnu{\y}u, kak i ranyxe. — V takih voprosah vajno to, cto govorit ne narod, a kn{\ia}z{\y}a qerkvi.

Tut on konecno je ne sporil. Vidno, kak i {\y}a, vspomnil pred{\yi}dux̨evo kn{\ia}z{\ia} Lezerberga, prosivxevo, ctob{\yi} v Riapano priznali {\y}evo matuxku sv{\ia}to{\y}. Taka{\y}a blaj sto{\y}ila {\y}emu pocti semysot t{\yi}s{\ia}c dukatov, a razrazivxi{\y}s{\ia} skandal privel k smerti des{\ia}tka podkuplenn{\yi}h kardinalov, reformaqi{\y}i Qerkvi i po{\y}avleni{\y}u protestn{\yi}h dvijeni{\y} v Vitilyska, trebovavxih lixity Riapano privilegi{\y}, a vseh Pap priznaty ``ne namestnikami Boga na zemle, a vsevo lix prodajn{\yi}mi sukin{\yi}mi s{\yi}nami i posledovatel{\ia}mi d{\y}avolyskih nauk, a takje jadn{\yi}mi kolcenogimi jabami".

S teh por Sv{\ia}to{\y} grad trijd{\yi} duma{\y}et, prejde cem kovo-libo pricislity hot{\ia} b{\yi} k blajenn{\yi}m, ne govor{\ia} uje o sv{\ia}t{\yi}h. Oni trebu{\y}ut dokazatelystv kak minimum dvuh soverxenn{\yi}h pri jizni cudes, pravednovo sux̨estvovani{\y}a i xesti monografi{\y} s razm{\yi}xleni{\y}ami o religi{\y}i ({\y}esli, konecno, pretendent umel citaty i pisaty).

Kontora ``Fabyen Klemenz i s{\yi}nov{\y}a" raspolagalasy nedaleko ot qerkvi, gde {\y}a naxel Kristinu. Malenyki{\y} neprimetn{\yi}{\y} klerk prin{\ia}l mo{\y}e pisymo, podslepovato x̨ur{\ia}sy.

— Ob{\yi}cna{\y}a otpravka?

— Otsrocenna{\y}a, — skazal {\y}a. — Otoxlite {\y}evo adresatu, {\y}esli {\y}a ne zaberu poslani{\y}e v teceni{\y}e sledu{\y}ux̨ih des{\ia}ti dne{\y}.

— Eto budet cuty doroje. — On sdelal otmetku v tolsto{\y} knige. — Cto-nibudy {\y}ex̨e?

— Net, blagodar{\iu}.

Na uliqe men{\ia} jdal Valyter.

— Dvadqaty p{\ia}ty klinkov straje{\y} trebu{\y}etsa dl{\ia} odnovo temnovo kinjala. — On skazal eto b{\yi}stro, prejde cem {\y}a rexil, cto s nim sdelaty. — Dl{\ia} des{\ia}ti trebu{\y}etsa dvesti p{\ia}tydes{\ia}t.

— {\Y}a ume{\y}u scitaty. {\Y}emu ne hvatit i vse{\y} jizni, ctob{\yi} sobraty ih.

Koldun, tocno ptiqa, sklonil golovu:

— Jizny — vex̨ otnositelyna{\y}a, Ludwig. K primeru, {\y}esty te, kto jivet sebe posle smerti i v us ne du{\y}et. A {\y}esty taki{\y}e, kak t{\yi}. Kto mojet uvelicivaty svo{\y}u jizny hoty do beskonecnosti. Ponima{\y}ex, na cto {\y}a nameka{\y}u?

— Cto temn{\yi}{\y} kuzneq — straj.

— {\Y}esty u men{\ia} taka{\y}a teori{\y}a.

— No nikakih dokazatelystv.

— Nikakih, — priznal on. — Tot, kto sozda{\y}et temno{\y}e oruji{\y}e, oblada{\y}et ogromn{\yi}m terpeni{\y}em i beskonecn{\yi}m vremenem.

— Skaji mne, koldun. Skolyko kinjalov u nevo se{\y}cas?

— Sprosi cto polegce. Semy. B{\yi}ty mojet, vosemy. I {\y}ex̨o odin dela{\y}etsa. A znacit, vremeni u nas ne tak uj mnogo.

— A otkuda tebe izvestno ob etom?

— {\Y}a ume{\y}u sluxaty. {\Y}a zna{\y}u l{\iu}de{\y}. I nel{\iu}de{\y}. {\Y}a ponima{\y}u {\y}evo lucxe, cem qerkovniki, no, k sojaleni{\y}u, nedostatocno, ctob{\yi} skazaty, kto on tako{\y}. Etot celovek rabota{\y}et uje davno, iz pokoleni{\y}a v pokoleni{\y}e sobira{\y}a klinki straje{\y} i ostava{\y}asy nezametn{\yi}m v naxem mire.

— On mojet b{\yi}ty i ne odin. Naprimer, qel{\yi}{\y} klan. Togda ne sto{\y}it obrax̨aty vnimani{\y}a na skazku o bessmerti{\y}i.

— Da nevajno, skolyko ih — odin, dvo{\y}e ili sotn{\ia}. Kogda {\y}evo delo budet zakonceno, stanet slixkom pozdno. B{\yi}ty mojet, t{\yi} pro{\y}avix blagorazumi{\y}e i m{\yi} pogovorim? Pr{\ia}mo se{\y}cas?

— Govori, — skazal {\y}a, prislonivxisy k stene doma.

— Horoxo, — legko soglasilsa on, b{\yi}stro ogl{\ia}devxisy i udostoverivxisy, cto nikto ne obrax̨a{\y}et na nas vnimani{\y}a. — {\Y}a uznal obo vsem etom davno, kogda {\y}ex̨o b{\yi}l molod{\yi}m. Malenyki{\y} sluh, broxenna{\y}a fraza na odnom iz balov vedym. {\Y}a za{\y}interesovalsa, stal raskrucivaty nitocku. Nabl{\iu}dal, rasspraxival. Daje v mo{\y}em soobx̨estve informaqi{\y}i b{\yi}lo malo, {\y}a dovolystvovalsa lix sluhami i mifami, bolyxinstvo iz kotor{\yi}h okazalisy v{\yi}dumko{\y}. No {\y}a ne sdavalsa, nahodil kollekqionerov star{\yi}h knig, posex̨al castn{\yi}{\y}e biblioteki i daje {\y}ezdil v Temnolesye.

— Ne le{\y} vodu, koldun. Cuty bolyxe konkretiki.

— Nakoneq {\y}a uznal, cto {\y}emu trebu{\y}utsa kinjal{\yi} straje{\y}. Ne {\y}unqov, a teh, kto uje ne perv{\yi}{\y} god sobira{\y}et duxi. I stal nabl{\iu}daty za vami. P{\ia}ty let mne potrebovalosy, ctob{\yi} pon{\ia}ty — vaxi umira{\y}ut regul{\ia}rno, no v osnovnom molodn{\ia}k. Te, kto stanov{\ia}tsa masterami, pogiba{\y}ut dovolyno redko, a bessledno isceza{\y}ut {\y}ex̨o reje. I pocti vse ih kinjal{\yi} popada{\y}ut v Orden i unictoja{\y}utsa.

— Poka {\y}a ne uznal nicevo novovo.

On hotel otvetity, no uvidel mal{\ia}ra, nesux̨evo vedro kraski, i ne otkr{\yi}val rta, poka celovek ne skr{\yi}lsa za povorotom.

— Napr{\ia}gi mozg, straj. {\Y}a govor{\iu} o tom, cto {\y}edinstvenn{\yi}{\y} sposob sobraty kinjal{\yi}, ne v{\yi}reza{\y}a vaxu brati{\y}u napravo i nalevo, eto zabiraty te klinki, cto Bratstvo otda{\y}et zakonnikam na unictojeni{\y}e.

— Nevozmojno, — vozrazil {\y}a. — Naxe oruji{\y}e unictoja{\y}etsa pri svidetel{\ia}h.

On rassme{\y}alsa, zapanibratski hlopnuv men{\ia} po plecu:

— Let semy nazad {\y}a b{\yi}l takim je na{\y}ivn{\yi}m, kak t{\yi}, van Norma{\y}enn. No zatem nacal dumaty. Kto vse eti svideteli? Straji na unictojeni{\y}i b{\yi}va{\y}ut kra{\y}ne redko — vas malo i del polno. Orden loma{\y}et kinjal, kogda r{\ia}dom predstaviteli vlasti. A tepery poduma{\y}, mnogo li tolst{\yi}{\y} burgomistr, zanosciv{\yi}{\y} graf ili {\y}edva ume{\y}ux̨i{\y} citaty prihodsko{\y} sv{\ia}x̨ennik ponima{\y}ut v kinjalah straje{\y}?

S etimi slovami on dostal iz sumki klinok s sapfirom na ruko{\y}ati i prot{\ia}nul mne.

— Kopi{\y}a, — posle beglovo osmotra skazal {\y}a.

— Verno. No eto opredelit lix op{\yi}tn{\yi}{\y} glaz. Vseh ostalyn{\yi}h smutit zvezdcat{\yi}{\y} sapfir, kotor{\yi}{\y}, prizna{\y}emsa cestno, ne taka{\y}a uj i redkosty.

{\Y}a poter x̨etinist{\yi}{\y} podborodok:

— T{\yi} hocex ubedity men{\ia}, cto Orden pomoga{\y}et temnomu koldunu?

— Orden ili kto-to iz sosto{\y}ax̨ih v nem. Naprimer, lucxi{\y} drug markgrafa Valentina gospodin Aleksandr, horoxo tebe znakom{\yi}{\y} po sob{\yi}ti{\y}am v Vione. {\Y}esli zakonniki ne mogut vospolyzovatsa vaximi kinjalami sami, eto ne oznaca{\y}et, cto oni ne na{\y}dut kuda ih pristro{\y}ity.

— I m{\yi} vnovy ut{\yi}ka{\y}emsa v stenu, koldun. Mo{\y} vopros: ``Kaka{\y}a v etom v{\yi}goda?" — nikuda ne delsa. {\Y}a ne jalu{\y}u zakonnikov, no vs{\e} je ne pover{\iu} v ih jelani{\y}e raspahnuty vrata ada i ustro{\y}ity koneq dl{\ia} vsevo sveta. S kako{\y} stati v{\yi}rvavxi{\y}es{\ia} iz pekla certi ne nacnut im vredity?

Valyter po-pri{\y}atelyski pozdorovalsa s dvum{\ia} prohodivximi mimo nas strajnikami i lix posle otvetil:

— Gl{\ia}ju, t{\yi} uje verix, cto vrata, kotor{\yi}{\y}e mogut otkr{\yi}ty kinjal{\yi}, ne prosto glupa{\y}a skazka.

— Ne ver{\iu}, — otrezal {\y}a. — No eto vesom{\yi}{\y} kontrargument v tvo{\y}e{\y} nelepo{\y} istori{\y}i.

On ul{\yi}bnulsa, no glaza {\y}evo stali zl{\yi}mi i razdrajenn{\yi}mi:

— Spor{\iu} na dukat, cto zakonniki ne zna{\y}ut o tom, cto klinki mogut otkr{\yi}ty vrata. Tot, kto splavl{\ia}{\y}et kinjal{\yi} kuznequ, dela{\y}et mal{\yi}{\y}e pakosti, daje ne podozreva{\y}a o bolyxo{\y}. K primeru, on ne protiv, ctob{\yi} u Bratstva b{\yi}lo mnogo rabot{\yi}. Kak togda, v Xossi{\y}i. I nade{\y}etsa, cto neskolyko temn{\yi}h kinjalov diskreditiru{\y}ut straje{\y}, kotor{\yi}{\y}e prosto ne sprav{\ia}tsa s valom temn{\yi}h dux. Razve eto ne v{\yi}godno Ordenu? Paniku{\y}ux̨e{\y}e naseleni{\y}e, nedovolyn{\yi}{\y}e kn{\ia}z{\y}a, nov{\yi}{\y}e volynosti, usileni{\y}e, vlasty? Eto odna iz versi{\y}. Druga{\y}a — on banalyno zarabat{\yi}va{\y}et na etom. L{\iu}d{\ia}m, zna{\y}ex li, nujno zoloto. A nekotor{\yi}m l{\iu}d{\ia}m {\y}evo trebu{\y}etsa kak mojno bolyxe. I nakoneq, tret{\y}a — Kristina scita{\y}et, cto kuzneq otdal kinjal, ctob{\yi} proverity {\y}evo rabotu. {\Y}a ne soglasen s etim. Tot, kto sozda{\y}et tako{\y}e, zna{\y}et, cto v{\yi}hodit iz-pod {\y}evo molota. Po mne, eto b{\yi}la plata za klinki straje{\y}, kotor{\yi}{\y}e {\y}emu nujn{\yi}.

— {\Y}a zna{\y}u o dvuh temn{\yi}h kinjalah. Odin zabrala u zakonnikov Kristina, drugo{\y} — qerkovy u kuryera, napravl{\ia}vxegos{\ia} v Latku.

On v{\yi}gl{\ia}del udivlenn{\yi}m:

— Seryezno? V perv{\yi}{\y} raz sl{\yi}xu. U teb{\ia} tocn{\yi}{\y}e svedeni{\y}a?

— Da.

— Kogda eto slucilosy?

— Ne mogu skazaty.

— Vozmojno, {\y}ex̨o pri jizni gospodina Aleksandra, l{\iu}bivxevo gostity u markgrafa, — probormotal tot. — Gde kinjal tepery?

— Unictojen klirikami.

— Tepery mne pon{\ia}tno, kak oni v{\yi}xli na kuzneqa. — On otvernulsa, sobira{\y}asy uhodity. I brosil cerez pleco: — Kristina prosila peredaty, ctob{\yi} t{\yi} prixel cerez paru casov v apteku.

— E{\y}, koldun, — ostanovil {\y}a {\y}evo. — Ni odin celovek ne men{\ia}{\y}etsa nastolyko b{\yi}stro. S cevo eto t{\yi} stal takim l{\iu}bezn{\yi}m?

— L{\iu}bezn{\yi}m? — On skrivil ugol rta. — T{\yi} men{\ia} s kem-to sputal, straj. {\Y}a govor{\iu} s tobo{\y} lix potomu, cto mne mojet ponadobitsa tvo{\y}a pomox̨. Inace nikakih razgovorov b{\yi} ne polucilosy.

— Otvety mne vsevo lix na dva voprosa, a zatem mojex katitsa k certu.

— Vecerom.

— Net. Se{\y}cas.

V {\y}evo glazah b{\yi}lo qelo{\y}e more gneva, {\y}a videl, kak on sjal kulaki, no tut je rasslabilsa i ne{\y}iskrenne ul{\yi}bnulsa:

— Ladno. Prox̨e vs{\e} zakoncity se{\y}cas. Val{\ia}{\y}.

— Markgraf Valentin sobiral kinjal{\yi}. Dl{\ia} kovo?

— Aleksandr i {\y}evo ne{\y}izvestn{\yi}{\y}e mne druz{\y}a vnuxili markgrafu, cto tot obretet bessmerti{\y}e. Na samom dele zakonniki prosto ispolyzovali {\y}evo vozmojnosti dl{\ia} sbora oruji{\y}a Bratstva. Vtoro{\y} vopros?

— Kinjal, kotor{\yi}{\y} t{\yi} {\y}edva ne ukral v Livette. Dl{\ia} cevo nujen on?

— A… — prot{\ia}nul koldun, i b{\yi}lo vidno, cto {\y}emu nepri{\y}atno vspominaty o to{\y} neudace. — Dl{\ia} obmena. Odin kollekqioner jelal sebe taku{\y}u igruxku v obmen na bezdeluxku.

— Kaku{\y}u bezdeluxku?

Kak nazlo, navstrecu xel {\y}ex̨o odin patruly straji, i Valyter vospolyzovalsa etim, otodvinuv men{\ia} plecom:

— Prihodi k Kristine. Uzna{\y}ex.

— To {\y}esty t{\yi} polucil {\y}e{\y}e? Naxel drugo{\y} klinok?

— Mne pora, van Norma{\y}enn.

{\Y}a dal {\y}emu pro{\y}ti, potomu cto i tak uje znal, ce{\y} kinjal on ispolyzoval i na cto {\y}evo hotel pomen{\ia}ty.



— Poraja{\y}usy tvo{\y}emu terpeni{\y}u. Drugo{\y}, ne zna{\y}a teb{\ia}, nazval b{\yi} eto slaboharakternost{\y}u. Nu posle tovo, cto sdelal etot hm{\yi}ry. — Propovednik posmotrel na men{\ia} po-starikovski hitro.

— No t{\yi} men{\ia} zna{\y}ex i… — {\Y}a predostavil {\y}emu zakoncity frazu.

— T{\yi} obuzdal emoqi{\y}i i rexil ne pugaty kr{\yi}su, poka ona ne privedet teb{\ia} k zernohranilix̨u.

— Skore{\y}e uj zme{\y}u, poka ta ne pokajet, gde {\y}e{\y}o kladka.

Propovednik pogrozil mne palyqem:

— Zme{\y}a mojet i ukusity. A {\y}e{\y}o {\y}ad opasen. Pomn{\iu}, v mo{\y}e{\y} derevne odin pastuh…

— Men{\ia} bolyxe interesu{\y}et ta vajna{\y}a novosty, kotora{\y}a vsevo minutu nazad zanimala vs{\e} tvo{\y}e voobrajeni{\y}e.

— Teb{\ia} ona udivit. Zna{\y}ex, kak zovut kardinala, kotor{\yi}{\y} prib{\yi}va{\y}et v Kruso, ctob{\yi} provesti torjestvenno{\y}e bogoslujeni{\y}e? Tvo{\y} star{\yi}{\y} drug Urban.

{\Y}a daje ostanovilsa:

— Nepri{\y}atno{\y}e izvesti{\y}e, {\y}esli cestno. Kardinal Urban v gorode, a r{\ia}dom Valyter. Kak b{\yi} ne slucilosy vtorovo Viona.

— Tot celovek iz Ordena, Aleksandr, mertv.

— I vs{\e} ravno mne eto ne nravitsa.

{\Y}a v{\yi}xel na bolyxu{\y}u kruglu{\y}u plox̨ady, na kotoro{\y} letom vo vrem{\ia} prazdnestva ustra{\y}ivali znamenit{\yi}{\y}e p{\ia}timinutn{\yi}{\y}e skacki. Se{\y}cas zdesy goreli kostr{\yi} i b{\yi}li razbit{\yi} palatki. Palomniki, te scastlivciki, kovo pustili v gorod, jili pr{\ia}mo na uliqe, ojida{\y}a svo{\y}e{\y} oceredi prikosnutsa gubami k sv{\ia}tomu sledu.

Vozle odno{\y} iz v{\ia}zanok hvorosta, v{\yi}t{\ia}nuv nogi, sidelo Pugalo.

— Vot eto vstreca, — probubnil Propovednik. — Ne hocex procitaty {\y}emu notaqi{\y}u za to, cto ono razodralo dnevnik burgomistra? Inace v sledu{\y}ux̨i{\y} raz ono ukradet u teb{\ia} ispodne{\y}e. I spalit na ogne.

— A {\y}esli ne procitaty {\y}emu notaqi{\y}u, {\y}a sekonoml{\iu} qelu{\y}u minutu vremeni. Potomu cto itogov{\yi}{\y} rezulytat budet odin i tot je — ono vs{\e} ravno propustit mo{\y}i slova mimo uxe{\y}. K tomu je mne sledu{\y}et zagl{\ia}nuty v apteku.

— T{\yi} vs{\e} ravno nicevo ot nih ne dobyexsa.

— No hot{\ia} b{\yi} uzna{\y}u, kakim obrazom oni hot{\ia}t po{\y}maty kuzneqa.

— Taka{\y}a je bespolezna{\y}a trata vremeni, kak ubejdaty Pugalo ostavatsa pa{\y}inyko{\y}.

Ni tot, ni drugo{\y} ne zahoteli b{\yi}ty mo{\y}imi soprovojda{\y}ux̨imi, tak cto {\y}a ostavil ih na plox̨adi sluxaty l{\iu}dsku{\y}u boltovn{\iu}.

Apteka okazalasy zakr{\yi}ta, stavni opux̨en{\yi}, no v okne vtorovo etaja gorel svet. {\Y}a postucal, i mne otkr{\yi}l sedoborod{\yi}{\y} aptekary.

— A, gospodin straj. M{\yi} uje dumali, v{\yi} ne pridete. — On blagosklonno kivnul, vpuska{\y}a men{\ia}.

Starik v{\yi}gl{\ia}del nervn{\yi}m i napr{\ia}jenn{\yi}m. Za vse{\y} eto{\y} l{\iu}beznost{\y}u skr{\yi}valsa kako{\y}-to su{\y}etliv{\yi}{\y} strah. Eto nicuty ne vnuxilo mne doveri{\y}a.

— Ludwig! Horoxo, cto t{\yi} vernulsa! — Kristina sto{\y}ala na lestniqe i ul{\yi}balasy, ne skr{\yi}va{\y}a, cto rada men{\ia} videty. — Idem, {\y}a teb{\ia} poznakoml{\iu} s ostalyn{\yi}mi.

V komnatah, kotor{\yi}{\y}e ona snimala, goreli sveci. Dva stola okazalisy sdvinut{\yi}, i za nimi razmestilisy l{\iu}di. Kogda {\y}a voxel, na mne sosredotocilosy vnimani{\y}e vseh prisutstvu{\y}ux̨ih.

— Pozvolyte poznakomity vas s gospodinom van Norma{\y}ennom, druz{\y}a, — obratilasy Kristina k cetver{\yi}m neznakomqam. — On — straj, kak i {\y}a. Odin iz lucxih v mo{\y}em pokoleni{\y}i. A eto l{\iu}di, kotor{\yi}{\y}e, kak i m{\yi} s Valyterom, jela{\y}ut raz i navsegda pokoncity s temn{\yi}m kuzneqom.

Qela{\y}a kompani{\y}a sumasxedxih mectatele{\y}, jela{\y}ux̨ih spasti mir.

— Metr Filipp, — predstavila ona aptekar{\ia}. — On zanima{\y}etsa alhimi{\y}e{\y} i b{\yi}l nastolyko l{\iu}bezen, cto okazal nam gostepri{\y}imstvo.

Starik su{\y}etlivo poklonilsa i, pl{\iu}hnuvxisy na stul, stal pomexivaty lojecko{\y} v stakane s kakim-to varevom, to i delo gromko zv{\ia}ka{\y}a o tonku{\y}u stekl{\ia}nnu{\y}u stenku.

— Adily aly Djuma — predstavitely Lavenduzzskovo so{\y}uza v svo{\y}ih zeml{\ia}h.

T{\iu}rban na brito{\y} golove delal tox̨evo hagjita pohojim na strann{\yi}{\y} lesno{\y} grib. Glaza b{\yi}li podveden{\yi} surymo{\y}.

— On okazal nam neoqenimu{\y}u uslugu.

— V{\yi} priukraxiva{\y}ete mo{\y}i dostijeni{\y}a, ba{\y}an[40] Kristina. — On ul{\yi}bnulsa, i {\y}a uvidel, cto dvuh qentralyn{\yi}h verhnih zubov u nevo net. — {\Y}a vsevo lix skromn{\yi}{\y} sluga pust{\yi}nn{\yi}h mudreqov, i ih prikaz{\yi} priveli men{\ia} s{\iu}da.

— Cezare Motto. Kondotyer.

V{\yi}soki{\y} i plecist{\yi}{\y} celovek s x̨etinist{\yi}m podborodkom i gust{\yi}mi, cuty r{\yi}jevat{\yi}mi brov{\ia}mi neohotno pripodn{\ia}l dva palyqa v privetstvennom jeste na{\y}emnikov Kavarzere. {\Y}a ne znal, cto on zdesy dela{\y}et, no soldat udaci kazalsa takim je lixnim, kak cert, zagl{\ia}nuvxi{\y} na voskresnu{\y}u messu.

— I oteq Gotthod, kanonik sobora Sv{\ia}to{\y} Mari{\y}i v Braselovette.

Borodat{\yi}{\y} tolst{\ia}k v cerno{\y} r{\ia}se, krugloliqi{\y}, s ospinami na x̨ekah i lbu, pripodn{\ia}lsa nad stulom:

— Master.

— S cevo mne nacaty rasskaz, Ludwig? — Valyter sidel na podokonnike, na rukah u nevo dremala tox̨a{\y}a p{\ia}tnista{\y}a koxka. Liqo kolduna uje zajilo, slovno {\y}a i ne kasalsa {\y}evo svo{\y}imi kulakami.

— Nacni s tovo, zacem {\y}a zdesy. — {\Y}a pro{\y}ignoriroval stul, vstav tak, ctob{\yi} videty ih vseh. Razume{\y}etsa, eto ne ostalosy nezamecenn{\yi}m, no nikto, krome usmehnuvxegos{\ia} kondotyera, ne podal vida.

— Delo ne v tom, cto t{\yi} straj… — Koldun ne otr{\yi}val vzgl{\ia}da ot koxki.

— Sam Gospody pos{\yi}la{\y}et vas nam, — vajno kivnul oteq Gotthod. — Ne inace eto {\y}evo jelani{\y}e.

— Nam de{\y}stvitelyno nujna tvo{\y}a pomox̨, Ludwig, — podhvatila Kristina. — M{\yi} s Valyterom segodn{\ia} pogovorili i pon{\ia}li, cto, {\y}esli t{\yi} budex s nami, vs{\e} pro{\y}det legko i ne budet nikako{\y} krovi.

— {\Y}a, pojalu{\y}, nacnu s samovo nacala. — Koldun posmotrel na Kristinu, i ta obodr{\ia}{\y}ux̨e kivnula. — Dnem t{\yi} zadaval vopros, zacem mne b{\yi}l nujen kinjal tvo{\y}evo druga…

— Natana, — podskazala straj.

— Tvo{\y}evo druga Natana. Pred{\yi}stori{\y}a takova. Dostopoctim{\yi}{\y} Adily aly Djuma, blagodar{\ia} svo{\y}im sv{\ia}z{\ia}m v torgovle, mnogo{\y}e sl{\yi}xit. Daje to, cto p{\yi}ta{\y}utsa skr{\yi}ty ot {\y}evo uxe{\y}. Do nevo doxel sluh o tom, cto v Veliko{\y} pust{\yi}ne lovki{\y}e l{\iu}di ot{\yi}skali dva cern{\yi}h kamn{\ia} i privezli na nax kontinent.

— Rec o glazah serafima?

— Verno, van Norma{\y}enn. Ih dostavili po osobomu zakazu. Etot kameny oceny redok i {\y}avl{\ia}{\y}etsa ob{\ia}zatelyn{\yi}m materialom dl{\ia} izgotovleni{\y}a temnovo klinka. I trebu{\y}etsa kuznequ.

On sdelal znacitelynu{\y}u pauzu, no, ne dojdavxisy nikakih kommentari{\y}ev ot men{\ia}, prodoljil:

— Nam prixlosy pob{\yi}vaty v xestnadqati portah, prejde cem udalosy napasty na sled prodavqa. I {\y}ex̨o neskolyko mes{\ia}qev, ctob{\yi} uznaty o pokupatele. On priobrel oba glaza serafima, tak kak sobira{\y}et mineral{\yi} — u nevo dovolyno obxirna{\y}a kollekqi{\y}a, kak {\y}a sl{\yi}xal. M{\yi} pop{\yi}talisy v{\yi}kupity hot{\ia} b{\yi} odin kameny, no bogacu ne nujn{\yi} denygi.

— Pop{\yi}talisy ukrasty, — prodoljila Kristina. — No eto okazalosy ne tak-to prosto. M{\yi} daje ne smogli uznaty, gde on ih hranit.

— Tvo{\y}a reputaqi{\y}a pod ugrozo{\y}, koldun, — s usmexko{\y} skazal {\y}a Valyteru. — Neujeli t{\yi} ne isproboval sam{\yi}{\y} vern{\yi}{\y} sposob merzavqev — nasili{\y}e?

— Povery, mne oceny hotelosy. — On vernul usmexku. — No u nevo mnogo druze{\y}. I eto izvestn{\yi}{\y} celovek. {\Y}evo isceznoveni{\y}e, ne govor{\ia} uje o smerti, vzbudorajilo b{\yi} vlasti. A {\y}a v posledne{\y}e vrem{\ia} i tak privlek slixkom mnogo nezdorovovo vnimani{\y}a k po{\y}iskam temnovo kuzneqa.

— Na samom dele eto {\y}a otgovoril {\y}evo ot pospexn{\yi}h de{\y}stvi{\y}. — Aptekary nervno sqepil palyqi. — Nasili{\y}e — ne v{\yi}hod. Osobenno {\y}esli ono mojet privesti k {\y}ex̨o bolyxomu nasili{\y}u i provalu vajno{\y} missi{\y}i. Da i so smert{\y}u kollekqionera m{\yi} b{\yi} ne naxli ta{\y}nik. Poetomu rexili de{\y}stvovaty inace. Adily v{\yi}stupil kak predstavitely drugovo l{\iu}bitel{\ia} mineralov, predlojil, razume{\y}etsa, denygi. Zatem obmen. Uznal, cevo hocet bur… tot celovek.

— I cto {\y}emu nujno? — sprosil {\y}a, hot{\ia} znal otvet.

— Kinjal straja. {\Y}evo interesoval zvezdcat{\yi}{\y} sapfir v neob{\yi}cnom ispolneni{\y}i. Za taku{\y}u relikvi{\y}u on gotov b{\yi}l ustupity odin iz dvuh svo{\y}ih kamne{\y}.

— T{\yi} v kurse slucivxegos{\ia}? — po{\y}interesovalsa {\y}a u Kristin{\yi}.

— Net. V Livette men{\ia} ne b{\yi}lo. {\Y}a b{\yi} ne dala {\y}emu vz{\ia}ty kinjal Natana, t{\yi} je zna{\y}ex.

{\Y}esli cestno, {\y}a uje ne znal, kto ona i na cto sposobna radi tovo, ctob{\yi} po{\y}maty temnovo mastera.

— I kak je v{\yi} postupili, kogda u tvo{\y}evo druga nicevo ne polucilosy i klinok vernulsa k zakonnomu vladelyqu?

— {\Y}a izgotovil poddelku, — ojivilsa Filipp. — Otmenna{\y}a vex̨, nasto{\y}ax̨i{\y} zvezdcat{\yi}{\y} sapfir, i staly podhod{\ia}x̨a{\y}a. Mojno obmanuty pocti vseh, no ne nasto{\y}ax̨evo znatoka.

— On provozilsa do dekabr{\ia}, m{\yi} poter{\ia}li pocti xesty mes{\ia}qev. — Cezare prenebrejitelyn{\yi}m x̨elckom otpravil cerez stol nevidimu{\y}u sorinku. — A v itoge kollekqioner podn{\ia}l nas na smeh. Obmanuty {\y}evo ne udalosy.

— I?.. — podstegnul {\y}a ih.

Gluboka{\y}a tixina razlilasy po pomex̨eni{\y}u, vse tepery smotreli na Kristinu, kak b{\yi} otstran{\ia}{\y}asy ot tovo, cto slucilosy dalyxe.

— {\Y}a otdala {\y}emu svo{\y} kinjal! — nabrav vozduha v grudy, v{\yi}palila ona.

Mne prihodilosy igraty, i {\y}a ne b{\yi}l uveren, cto akter iz men{\ia} horoxi{\y}.

— Cto?!

— Mne prixlosy, Ludwig.

{\Y}a s kamenn{\yi}m liqom pomolcal, vid{\ia}, cto ona to krasne{\y}et, to bledne{\y}et, i spoko{\y}no pro{\y}iznes:

— {\Y}a hocu {\y}evo uvidety.

Liqo u Kristin{\yi} stalo raster{\ia}nn{\yi}m:

— T{\yi} o kinjale? {\Y}a je govor{\iu}…

— K certu tvo{\y} kinjal, Kristina. Raz on tebe ne nujen i t{\yi} rasstalasy s nim dobrovolyno, {\y}a ne tot celovek, kotor{\yi}{\y} budet ubejdaty teb{\ia} v tvo{\y}e{\y} gluposti! — rezko otvetil {\y}a, i mo{\y}i slova b{\yi}li dl{\ia} ne{\y}o kak pox̨ecina. — Pokajite mne kameny, radi kotorovo v{\yi} ustro{\y}ili vs{\e} eto.

— Em… — Filipp poter perenosiqu. — Ponima{\y}ete, u nas {\y}evo net. I kak b{\yi}… {\y}a polaga{\y}u, cto uje i ne budet. Gospodin Cezare nedel{\iu} nazad vernulsa s plohimi novost{\ia}mi. Kollekqioner mertv, kamni tak i ne na{\y}den{\yi}. V delo vmexalasy inkviziqi{\y}a, i m{\yi} ne mojem se{\y}cas vernuty daje oruji{\y}e Kristin{\yi}. Ne zna{\y}em, gde ono.

— A {\y}a govoril, cto klinok nado men{\ia}ty na kameny srazu. — Kondotyer qedil slova zlo. — V{\yi} je poxli na povodu u eto{\y} svoloci. Mirol{\iu}bi{\y}e, ne nado nasili{\y}a, ne sto{\y}it privlekaty k sebe dopolnitelyno{\y}e vnimani{\y}e…

— Posle tovo kak m{\yi} p{\yi}talisy podsunuty {\y}emu poddelku, on perestal nam dover{\ia}ty, — vinovato razvel rukami aptekary. — On potreboval pereslaty {\y}emu kinjal cerez ``Fabyen Klemenz i s{\yi}nov{\y}a".

— No obmanul vas i ne peredal glaz serafima?

— Net, — gluho otvetila Kristina. — Valyter ne hotel polyzovatsa posrednikami. M{\yi} rexili zabraty kameny licno, no ne uspeli.

— V itoge u vas net ni kur, ni lis{\yi}, — otvetil {\y}a staro{\y} pogovorko{\y}.

Na Kristinu b{\yi}lo jalko smotrety, tepery ona v{\yi}gl{\ia}dela nastolyko podavlenno{\y}, cto {\y}a s trudom poborol v sebe jelani{\y}e otkr{\yi}ty visevxu{\y}u cerez pleco sumku i vernuty {\y}e{\y}o oruji{\y}e. No {\y}a sderjalsa. Ne se{\y}cas. I ne pri etih l{\iu}d{\ia}h.

— Obrazno govor{\ia}, v{\yi} soverxenno prav{\yi}, — podtverdil Filipp.

— {\Y}esli v{\yi} nade{\y}etesy, cto tepery {\y}a dam vam svo{\y} kinjal, ctob{\yi} v{\yi} {\y}evo poter{\ia}li tak je bezdarno, kak i {\y}e{\y}o klinok, to obrax̨a{\y}etesy ne po adresu.

— Gospody s vami! — vsplesnul rukami oteq Gotthod. — Nicevo podobnovo! Vam ne nado budet s nim rasstavatsa. {\Y}esli cestno, to on nam sovsem ne nujen. Ne hotite vs{\e} je prisesty?

— Net. Tak cto je vam nujno?

— Odna vex̨, kotora{\y}a prinadlejit tebe. — Koldun ostorojno opustil koxku na podokonnik, podoxel k Kristine, polojil ruku {\y}e{\y} na pleco, i mne ne ponravilsa etot jest — sobstvennika, za{\y}avl{\ia}{\y}ux̨evo prava na svo{\y}u vex̨. — Kolyqo, kotoro{\y}e tebe podaril {\y}episkop Urban, posle tovo kak t{\yi} spas {\y}emu jizny v Vione. Ono {\y}ex̨o u teb{\ia}?

Neojidann{\yi}{\y} povorot. Priznatsa, {\y}a ne b{\yi}l gotov k takomu voprosu.

— Ne priv{\yi}k taskaty s sobo{\y} goru bezdeluxek.

— No t{\yi} i ne prodal {\y}evo. — Kristina ne spraxivala, utverjdala. — T{\yi} slixkom umen, ctob{\yi} razbras{\yi}vatsa podobn{\yi}mi podarkami i ostavl{\ia}ty ih v lombarde. Uverena, cto kak vsegda hranix na depozite v ``Fabyen Klemenz", ctob{\yi} vz{\ia}ty v l{\iu}bo{\y} moment.

Ona slixkom horoxo znala mo{\y}i priv{\yi}cki, i se{\y}cas men{\ia} eto ne radovalo.

— Podrobne{\y}e, Krista. {\Y}esli vam nujno kolyqo klirika, {\y}a hocu znaty dl{\ia} cevo. Vax kuzneq kl{\iu}{\y}et na l{\iu}bu{\y}u pobr{\ia}kuxku?

Valyter pokacal golovo{\y}:

— Vse gorazdo slojne{\y}e i prox̨e. Tot sluca{\y} v Vione imel nekotor{\yi}{\y}e predpos{\yi}lki. Aleksandr i markgraf Valentin hoteli izbavitsa ot kardinala, tocne{\y}e togda {\y}ex̨o {\y}episkopa, po neskolykim pricinam. Razume{\y}etsa, vse videli lix politiceski{\y}e — on mexal razvernutsa Ordenu v kn{\ia}jestve i ne daval jizni markgrafu, oblica{\y}a {\y}evo prestupleni{\y}a. No b{\yi}lo {\y}ex̨o odno ``no". {\Y}a o nem uznal pozje, primerno za nedel{\iu} do tovo, kak t{\yi} prikoncil {\y}evo milosty v Latke. On mne sam priznalsa, cto Aleksandr jajdal polucity cern{\yi}{\y} kameny {\y}episkopa. Mol, tot {\y}emu nujen ne menyxe, cem kinjal{\yi} straje{\y}, i horoxo b{\yi} etu xtuku privezti v Latku, kak tolyko po{\y}avitsa taka{\y}a vozmojnosty.

— Hm… — {\Y}a gl{\ia}del na nevo ispodlob{\y}a. — U {\y}evo v{\yi}sokopreosv{\ia}x̨enstva ime{\y}etsa glaz serafima?

— Imenno. {\Y}a posluxal veter, i tot dones do men{\ia} interesn{\yi}{\y}e sluhi. Cel{\ia}dy, kak t{\yi} zna{\y}ex, redko hranit ta{\y}n{\yi}. U kardinala {\y}esty mineral, kotor{\yi}{\y} on nosit na xe{\y}e, r{\ia}dom s rasp{\ia}ti{\y}em. Kameny ploho{\y} i zlo{\y}. Urban vrode kak dal obet mnogo let nazad, i eto {\y}evo krest, kotor{\yi}{\y} on tax̨it vo slavu Gospoda, ist{\ia}za{\y}a svo{\y}u ploty takim obrazom.

— U glaza serafima {\y}esty podobn{\yi}{\y}e svo{\y}stva?

— M{\yi} tolkom ne uveren{\yi}, — otvetil aptekary. — Traktat{\yi} govor{\ia}t razno{\y}e, v alhimi{\y}i kameny scita{\y}etsa temn{\yi}m i sposobn{\yi}m prinosity vred celoveku nesto{\y}komu. U hagjitov na se{\y} scet voobx̨e mnojestvo legend.

— Oni razn{\ia}tsa, — ul{\yi}bnulsa britogolov{\yi}{\y} torgoveq. — Pust{\yi}nn{\yi}{\y}e starqi, da prodl{\ia}tsa ih goda vecno, naz{\yi}va{\y}ut {\y}evo ubi{\y}qe{\y} sveta. I skazok o nem de{\y}stvitelyno mnogo. Vse kak odna s plohim konqom. Ne skaju, skolyko v nih pravd{\yi}, no mo{\y} narod stara{\y}etsa ne derjaty taki{\y}e vex̨i podolgu, osobenno blizko k domu.

— Predpocita{\y}et sbagrivaty ih nam za kucu florinov, — poddel {\y}evo Cezare, i hagjit ul{\yi}bnulsa, no, sud{\ia} po {\y}evo liqu, iskl{\iu}citelyno iz vejlivosti.

— I raz ne polucilosy s kollekqionerom, v{\yi} rexili poza{\y}imstvovaty mineral u kardinala. Vmesto tovo ctob{\yi} vz{\ia}ty lopatu i otpravitsa v bezvodnu{\y}u pust{\yi}n{\iu}. Po mne — posledni{\y} variant b{\yi}l b{\yi} gorazdo bole{\y}e razumen. V plane v{\yi}jivani{\y}a.

— {\Y}a ponima{\y}u vaxu ironi{\y}u, gospodin van Norma{\y}enn. M{\yi} pr{\ia}cemsa ot Qerkvi, zna{\y}em, na cto ona sposobna. A tut sami lezem v volc{\y}u pasty. No m{\yi} ob{\ia}zan{\yi}. Vo im{\ia} l{\iu}de{\y} i vo blago vsevo mira, — pospexno utocnil aptekary.

— Razume{\y}etsa, — ehom otkliknulsa {\y}a. — A kolyqo vam trebu{\y}etsa…

— Ctob{\yi} podobratsa k Urbanu. Eto propusk, Ludwig. Prosto otda{\y} nam {\y}evo. Tebe nezacem ucastvovaty v ostalynom.

{\Y}a pot{\ia}nulsa:

— V{\yi}, l{\iu}bezn{\yi}{\y}e gospoda, konecno, bolyxi{\y}e fantazer{\yi}, no, kak mne kajetsa, ne idiot{\yi}. I ne stanete ubivaty kardinala. Klirikov takovo ranga ubiva{\y}ut lix drugi{\y}e kliriki, no ne prost{\yi}{\y}e smertn{\yi}{\y}e vrode nas.

— Nikto ne govorit ob ubi{\y}stve. Da{\y} mne vozmojnosty podobratsa k nemu, vs{\e} ostalyno{\y}e — delo tehniki. — Valyter nehoroxo ul{\yi}bnulsa.

{\Y}a znal, kak nekotor{\yi}{\y}e koldun{\yi} ume{\y}ut us{\yi}pl{\ia}ty, rasse{\y}ivaty vnimani{\y}e ili pritormajivaty vrem{\ia}. Videl, cto prodel{\yi}va{\y}et Gertruda.

— Nu tak i sdela{\y} vs{\e} sam, — pojal {\y}a plecami. — Dl{\ia} etovo ne ob{\ia}zatelyno kolyqo.

— U kardinala seryezna{\y}a ohrana. Qerkovniki s magi{\y}e{\y}. So vsemi {\y}a prosto ne spravl{\iu}sy. — On legko raspisalsa v svo{\y}e{\y} bespomox̨nosti.

— A kogda u teb{\ia} budet pobr{\ia}kuxka, oni cto? Rastvor{\ia}tsa v vozduhe, cto li?

— Qerkovniki budut mene{\y}e bditelyn{\yi}. {\Y}a smogu podobratsa blizko, ogluxity ih. Bez propuska, izdali, eto nevozmojno.

— Kak t{\yi} ob{\y}asnix, otkuda ono u teb{\ia}?

— Ne budu ob{\y}asn{\ia}ty. M{\yi} planiru{\y}em vs{\e} provernuty vo vrem{\ia} torjestvennovo bogoslujeni{\y}a. {\Y}esli i budet proverka, to ne nastolyko seryezna{\y}a. A potom, uveren, im stanet ne do nas.

Cezare zarjal, i Filipp, podderjiva{\y}a na{\y}emnika, ul{\yi}bnulsa.

— {\Y}esty pricina dl{\ia} smeha? — nahmurilsa {\y}a.

— Nebolyxa{\y}a. — Aptekary ponizil golos. — M{\yi} s Valyterom pridumali i osux̨estvili grandioznu{\y}u aferu, sociniv, cto angel snizoxel v etot gorod.

— I proxlo, cert men{\ia} deri! — hlopnul ladon{\y}u po stolu kondotyer. — Slucilosy cudo!

— Sv{\ia}totatqi, — so smireni{\y}em pokacal golovo{\y} kanonik. — Nade{\y}usy, Gospody po{\y}met, cto ne radi zla m{\yi} eto sdelali, i prostit nas.

— To {\y}esty l{\iu}di, cto sid{\ia}t na uliqah i p{\yi}ta{\y}utsa popasty v gorod, proxli sotni lig radi nesux̨estvu{\y}ux̨evo cuda? Da, smexno. Kak v{\yi} eto ustro{\y}ili?

Valyter skromno razvel rukami:

— Nemnovo ne{\y}tralyno{\y} magi{\y}i, kotoru{\y}u ne srazu opredel{\ia}t kliriki, nemnogo alhimiceskih smese{\y} Filippa, odin slepo{\y} rebenok i umeni{\y}e rasprostran{\ia}ty sluhi. Vot reqept bojestvennovo cuda v naxi dni.

— V{\yi} zate{\y}ali eto, ctob{\yi} v{\yi}manity Urbana, tak kak Kruso pod {\y}evo pokrovitelystvom, i on ne mog pro{\y}ignorirovaty tako{\y}e sob{\yi}ti{\y}e i ne pri{\y}ehaty s{\iu}da.

— T{\yi} pravilyno ponima{\y}ex.

Da cto uj tut ponimaty? I idiotu {\y}asno.

— Znacit, v{\yi} vs{\e} splanirovali davno. {\y}ex̨o do tovo, kak pon{\ia}li, cto s kollekqionera kameny ne polucity.

— Rezervn{\yi}{\y} variant. Odno rovn{\yi}m scetom ne mexalo drugomu, i, kak vidix, {\y}a okazalsa prav. {\Y}esli b{\yi} t{\yi} ne po{\y}avilsa, m{\yi} spravilisy b{\yi} bez teb{\ia}. Prosto vs{\e} stalo b{\yi} gorazdo slojne{\y}e. Kogda nacnetsa bogoslujeni{\y}e, lucxi{\y}e mesta budut otdan{\yi} pocetn{\yi}m jitel{\ia}m goroda i blagorodn{\yi}m, ostalyn{\yi}m pridetsa dovolystvovatsa liqezreni{\y}em cujih spin — kardinalyska{\y}a ohrana ne ime{\y}et priv{\yi}cki puskaty kovo ni popad{\ia}. Persteny — mo{\y} propusk na sam{\yi}{\y} verh. Celoveku s koldovskim darom t{\ia}jelo prohodity cerez stro{\y} klirikov, v otlici{\y}e ot ob{\yi}cno{\y} straji. A tvo{\y}e kolyqo mne v etom pomojet. I kogda {\y}a zaberu kameny, to razve{\y}u volxebstvo — sled angela v kamne isceznet, i sv{\ia}toxam, pravo, {\y}ex̨o dolgo budet cem zan{\ia}tsa. Poka oni hvat{\ia}tsa propaji, m{\yi} budem uje daleko.

S tolku on men{\ia} ne sbil. {\Y}a videl, on nedogovariva{\y}et, i znal, cto imenno. Propaja ``cuda", b{\yi}ty mojet, na kako{\y}e-to vrem{\ia} otodvinet obnarujeni{\y}e vnezapnovo isceznoveni{\y}a kamn{\ia} kardinala, no i tolyko. Vse ravno nacnut iskaty i r{\yi}ty. Ibo sv{\ia}x̨enniki ne nastolyko kretin{\yi}, ctob{\yi} ne pocuvstvovaty ostatki cujo{\y} magi{\y}i. A znacit, ime{\y}etsa lix odin sposob zamesti sled{\yi} — ubity vseh pricastn{\yi}h.

Teh, kto propustit {\y}evo k kardinalu. Teh, kto uvidit {\y}evo. Nu i men{\ia} zaodno.

{\Y}a znal, cto Valyter opasen, no ne dumal, cto nastolyko. {\Y}evo hladnokrovi{\y}e, besst{\yi}dstvo i umeni{\y}e manipulirovaty l{\iu}dymi b{\yi}li potr{\ia}sa{\y}ux̨imi. Imenno se{\y}cas {\y}a okoncatelyno utverdilsa v m{\yi}sli, cto kolduna sledu{\y}et ubity srazu, kak tolyko predstavitsa taka{\y}a vozmojnosty, nevajno, kaki{\y}e qeli on presledu{\y}et. Etot celovek pro{\y}det po golovam i unictojit vseh.

On vedet svo{\y}u igru, no vmexiva{\y}et v ne{\y}o Bratstvo. {\Y}esli {\y}a pomogu {\y}emu, {\y}esli u nevo vs{\e} polucitsa, kliriki budut zl{\yi}. Da net. Cto tam! Oni budut v {\y}arosti. I po{\y}dut po lojnomu sledu, kotor{\yi}{\y} privedet ih v Ardenau.

No skazal {\y}a sovsem ino{\y}e:

— {\Y}a pomogu vam.

{\Y}a uvidel, kak oni ojivilisy. Vse, krome Kristin{\yi}, v glazah kotoro{\y} citalosy somneni{\y}e.

— Na neskolykih uslovi{\y}ah.

— Nazovi ih, — predlojila mo{\y}a b{\yi}vxa{\y}a naparniqa.

— {\Y}esli on hocet polucity kolyqo, to pusty vernet to, cto sn{\ia}l s mo{\y}evo palyqa v Latke.

Valyter pokacal golovo{\y}:

— Izvini, van Norma{\y}enn, no eto nevozmojno. {\Y}a srazu je {\y}evo unictojil — rabota vedym{\yi}, i ono moglo navesti {\y}e{\y}o na nas. Viju, cto se{\y}cas u teb{\ia} tako{\y}e je. B{\yi}ty mojet, {\y}a kak-to mogu kompensirovaty tvo{\y}u poter{\iu}?

Da. Raspahnuty okno i siganuty golovo{\y} vniz.

— Zabudem. — Skrep{\ia} serdqe {\y}a otkazalsa ot svo{\y}e{\y} mect{\yi}. — Dale{\y}e. {\Y}a idu s tobo{\y}.

— Zacem tebe riskovaty?

Ctob{\yi} t{\yi} ne ubil men{\ia} srazu posle tovo, kak vozymex jela{\y}emo{\y}e.

— Potomu cto, {\y}esli kto-to iz vas predstavitsa mno{\y}, {\y}a ne sobira{\y}usy potom rashleb{\yi}vaty {\y}evo oxibki.

— Cto-nibudy {\y}ex̨e?

— Nikako{\y} krovi.

— Nu, razume{\y}etsa. — On skazal eto tak cestno, cto daje {\y}a b{\yi} poveril {\y}emu, {\y}esli b{\yi} ne znal kolduna slixkom horoxo i ne pob{\yi}val v zastenkah zamka Latka.

— Togda net problem. Cas pozdni{\y}, gospoda. S vaxevo pozvoleni{\y}a, {\y}a otpravl{\iu}sy domo{\y}.

— T{\yi} legko soglasilsa, van Norma{\y}enn. {\Y}a dumal, pridetsa ubejdaty teb{\ia} do utra.

{\Y}a zametil legku{\y}u usmexku kondotyera. Interesno, kak oni planirovali men{\ia} ubejdaty?

— T{\yi} mne vs{\e} {\y}ex̨o ne nravixsa, koldun. Kak i vse, cto proishodit. No {\y}a eto dela{\y}u tolyko radi ne{\y}e. Zapomni eto.

— Vesom{\yi}{\y} argument, — kivnul on. — I {\y}a v nevo ver{\iu}. Uvidimsa zavtra, van Norma{\y}enn.

— {\Y}a provoju, — v{\yi}zvalasy Kristina.

M{\yi} vmeste spustilisy na perv{\yi}{\y} etaj i, ne sgovariva{\y}asy, v{\yi}xli na uliqu.

— {\y}ex̨o ne pozdno u{\y}ehaty. Pr{\ia}mo se{\y}cas, — vnovy predlojil {\y}a.

Ona izbegala smotrety na men{\ia} i otriqatelyno pokacala golovo{\y}:

— {\Y}a poter{\ia}la svo{\y} kinjal. I cto samo{\y}e ujasno{\y}e — nicuty ne jale{\y}u. {\Y}a bolyxe ne straj, Ludwig. Mne nekuda vozvrax̨atsa.

— Propaja oruji{\y}a ne lixa{\y}et teb{\ia} dara. Daje bez nevo t{\yi} — straj.

Ona neojidanno prislonilasy lbom k mo{\y}e{\y} grudi:

— {\Y}a ustala, Sineglaz{\yi}{\y}. Ustala b{\yi}ty strajem, ustala opravd{\yi}vaty ojidani{\y}a Miriam, ustala spasaty Bratstvo, otcit{\yi}vatsa pered magistrami, vrajdovaty s zakonnikami i v{\yi}cix̨aty vse te gor{\yi} deryma, cto ostavl{\ia}{\y}ut l{\iu}di posle svo{\y}e{\y} smerti, ne jela{\y}a otpravl{\ia}tsa v cistilix̨e. Vse mo{\y}e sux̨estvovani{\y}e, skolyko {\y}a seb{\ia} pomn{\iu}, posv{\ia}x̨eno imenno etomu. Zna{\y}ex, cevo {\y}a hocu? Poko{\y}a. Sbejaty daleko-daleko, tuda, gde net temn{\yi}h dux i l{\iu}de{\y} s prokl{\ia}ti{\y}em dara, magi{\y}i, i ostavity vs{\e} za spino{\y}, jity svo{\y}e{\y} malenyko{\y} jizn{\y}u, pisaty muz{\yi}ku i rastity dete{\y}. No samo{\y}e straxno{\y}e v etom to, cto mo{\y}i jelani{\y}a nicevo ne znacat. V men{\ia} vbili to je samo{\y}e, cto i v teb{\ia}, — spasaty l{\iu}de{\y} i zax̨ix̨aty Bratstvo. Temn{\yi}{\y} kuzneq — samo{\y}e opasno{\y}e, s cem m{\yi} stalkivalisy. {\Y}a ob{\ia}zana razobratsa s nim.

B{\yi}lo obidno, cto m{\yi} ponima{\y}em zax̨itu Bratstva soverxenno po-raznomu. Ona namerena podstavity {\y}evo, ctob{\yi} potom p{\yi}tatsa spasti to, cto uje budet unictojeno. {\Y}a gotov predaty {\y}e{\y}o nadejd{\yi}, ctob{\yi} xkola v Ardenau sux̨estvovala i dalyxe.

— Cto je. Eto tvo{\y} v{\yi}bor, — s sojaleni{\y}em skazal {\y}a {\y}e{\y}.

— Posto{\y}! — Ona shvatila men{\ia} za ruku. — Tebe ne ob{\ia}zatelyno ucastvovaty. Pravda. Prosto da{\y} nam kolyqo i uhodi. M{\yi} cto-nibudy priduma{\y}em.

— {\Y}a doljen cto-to {\y}ex̨o znaty?

Ona sdelala xag nazad:

— Mogu govority tolyko za seb{\ia}. Ne ponima{\y}u, kak pri ogromnom steceni{\y}i naroda i klirikah mojno ukrasty vex̨ u kardinala… I bo{\y}usy za tvo{\y}u jizny. Vozmojno, {\y}a oxiba{\y}usy, no skazaty ob etom — pravilyno, i t{\yi} doljen b{\yi}ty v kurse mo{\y}ih razm{\yi}xleni{\y}.

— Do zavtra, Krista. Budy ostorojna s nimi.

Ona ul{\yi}bnulasy:

— Eto im sledu{\y}et b{\yi}ty ostorojn{\yi}mi.

I vernulasy obratno v apteku.

Vstrecaty kardinala Urbana {\y}a planiroval ne v{\yi}bira{\y}asy iz doma. Okna mo{\y}ih komnat v{\yi}hodili na qentralynu{\y}u gorodsku{\y}u uliqu, po kotor{\yi}m doljna pro{\y}ehaty torjestvenna{\y}a proqessi{\y}a, tak cto {\y}a b{\yi}l obespecen neplohim zritelyskim mestom.

Pugalu toje b{\yi}lo interesno pogl{\ia}dety. Ono torcalo zdesy s samovo utra, vprocem, ne bez dela. Vcera vecerom oduxevlenn{\yi}{\y} sper v kako{\y}-to lavke derev{\ia}nnu{\y}u marionetku, za cto {\y}emu prixlosy v{\yi}sluxaty notaqi{\y}u ot Propovednika, tak kak ``kako{\y}-to rebenok tepery lixilsa radosti". Po mne, kako{\y}-to rebenok izbejal nocn{\yi}h koxmarov, kukla v{\yi}gl{\ia}dela cudovix̨no — groteskna{\y}a kopi{\y}a celoveka s cuty v{\yi}t{\ia}nuto{\y} golovo{\y}, xirocenn{\yi}mi rukami i bockoobrazno{\y} grud{\y}u. Raskraxena ona okazalasy nicuty ne mene{\y}e bezdarno, cem v{\yi}rezana: glaza razno{\y} velicin{\yi}, gubki bantikom, soloma vmesto volos.

Pugalo pro{\y}avilo tvorcesku{\y}u jilku, cuty podpraviv vse, cto nujno, serpom. V itoge igruxka obrela harakternu{\y}u i znakomu{\y}u zlovex̨u{\y}u uhm{\yi}locku.

— Prelestno, — oqenil Propovednik. — Tepery mojex priv{\ia}zaty k ne{\y} verevocki i korcity kuklovoda iz Litavi{\y}i.

Pugalo de{\y}stvitelyno priv{\ia}zalo verevocku, no lix odnu, i za xe{\y}u. Zatem izvleklo iz karmana mundira klocok atlasno{\y} tkani, igolki, nitki i za polcasa svarganilo odejonku. Ono razve cto ne nasvist{\yi}valo ot udovolystvi{\y}a, naslajda{\y}asy rukodeli{\y}em. Kogda vs{\e} b{\yi}lo gotovo i kukla povisla pod potolkom, Propovednik ostorojno izrek:

— Sv{\ia}to{\y} Vitt i vse {\y}evo pl{\ia}ski! Mne odnomu kajetsa, cto eta xtuka pohoja na kardinala? Ono tepery v{\yi}sunet {\y}e{\y}o v okno i budet mahaty pri vse{\y} cestno{\y} kompani{\y}i?

Na stoly slojn{\yi}{\y} vopros u men{\ia} otveta ne naxlosy. I Propovednik vnes oceredno{\y}e predlojeni{\y}e:

— K cemu vs{\ia} eta su{\y}eta, Ludwig? V tvo{\y}e{\y} sumke qel{\yi}h dva prokl{\ia}t{\yi}h kamn{\ia}. Na ko{\y} cert nagrevaty kardinala na {\y}ex̨o odin?

— Nu, ``nagrety" {\y}evo v{\yi}sokopreosv{\ia}x̨enstvo {\y}ex̨o nado sumety. Vprocem, {\y}a ne nedooqeniva{\y}u sil{\yi} Valytera. On mojet provernuty necto podobno{\y}e.

Propovednik skorcil minu, stav pohojim na zamorsku{\y}u obez{\y}anku:

— T{\yi} sam sebe protivorecix, Ludwig. To ``ne sume{\y}et", to ``mojet". Pocemu b{\yi} tebe vse-taki ne otdaty im to, cto u teb{\ia} {\y}esty, i ne vput{\yi}vatsa v seryezn{\yi}{\y}e nepri{\y}atnosti? V Riapano jivut opasn{\yi}{\y}e l{\iu}di. Sto{\y}it li perebegaty im dorogu?

— A t{\yi} zadum{\yi}valsa, cto slucitsa, kogda zagovorx̨iki polucat kamni?

On razvel rukami:

— {\Y}a ne zna{\y}u.

— V tom-to i beda. B{\yi}ty mojet, oni uberut men{\ia} kak lixnevo svidetel{\ia}, b{\yi}ty mojet, Kristinu. {\Y}a horoxo uspel uznaty kolduna. Otdavaty v {\y}evo ruki to, cto on hocet, — opasno. Pot{\ia}nu vrem{\ia}.

Star{\yi}{\y} pelikan rasse{\y}anno v{\yi}ter krovy so x̨eki, otcevo ta ne stala bole{\y}e cisto{\y}:

— Eto neskolyko ne po zakonam Bojyim, i tebe, naverno{\y}e, stranno sl{\yi}xaty podobno{\y}e ot men{\ia}, no, {\y}esli on tak opasen, pocemu b{\yi} prosto vs{\e} ne zaverxity? Vz{\ia}ty pistolet i raznesti {\y}emu golovu? Hot{\ia} b{\yi} za to, cto iz-za nevo t{\yi} provel v podzemno{\y} kamere paru mes{\ia}qev i {\y}edva ne otdal Bogu duxu.

— Voobx̨e-to otdal, i, {\y}esli b{\yi} ne Sofi{\y}a, {\y}a b{\yi} s tobo{\y} se{\y}cas ne razgovarival, — napomnil {\y}a {\y}emu. — {\Y}a dumal nad etim, no men{\ia} ostanavliva{\y}et to, cto u nevo {\y}esty qenno{\y}e znani{\y}e o temnom kuzneqe. Stoly vajn{\yi}{\y}e svedeni{\y}a mogut b{\yi}ty polezn{\yi} dl{\ia} Bratstva, a pul{\ia}, drug Propovednik, raz i navsegda postavit tocku. Ot trupa oceny t{\ia}jelo cto-nibudy uznaty.

— Da jiv{\yi}m-to on toje tebe mnogo ne rasskajet.

— Posmotrim. {\Y}a duma{\y}u o tom, cto b{\yi} slucilosy, {\y}esli b{\yi} Valyter polucil glaz serafima? Otdal mineral kuznequ? V{\yi}manil {\y}evo i ubil, kak on govorit? Ili je poznakomilsa i pop{\yi}talsa ispolyzovaty mastera v svo{\y}ih qel{\ia}h? No sam{\yi}{\y} glavn{\yi}{\y} vopros, kotor{\yi}{\y} ne da{\y}et mne poko{\y}a, Propovednik, kak on sv{\ia}jetsa s etim neulovim{\yi}m ne{\y}izvestn{\yi}m celovekom?

Tot aj podskocil:

— T{\yi} nameka{\y}ex, cto on zna{\y}et, kuda sledu{\y}et otpravity vestocku, ctob{\yi} kuzneq naznacil vstrecu!

— Imenno.

— I hocex v{\yi}sledity {\y}evo sv{\ia}znovo.

— ``Sv{\ia}znovo"? Propovednik, gde t{\yi} usl{\yi}xal eto slovo?

— V bordel{\ia}h, — nebrejno mahnul on. — Tam xpionov ne menyxe, cem xl{\iu}h. Vo vs{\ia}kom sluca{\y}e, v nekotor{\yi}h. Poro{\y} taki{\y}e razgovor{\yi} vedutsa, cto {\y}a jale{\y}u o svo{\y}e{\y} smerti i o tom, cto nekomu prodaty cuji{\y}e ta{\y}n{\yi}. Da i grehovno eto. Tak nascet istori{\y}i s Urbanom. Kak {\y}a ponima{\y}u, u teb{\ia} {\y}esty plan?

— {\Y}evo nametki. — {\Y}a posmotrel na kuklu, medlenno krut{\ia}x̨u{\y}us{\ia} pod potolkom.

— Nu {\y}asno. Iz teb{\ia} slova ne v{\yi}jmex. T{\yi} huje xpiona. Oni hot{\ia} b{\yi} bolta{\y}ut.

{\Y}a rassme{\y}alsa.

— A Kristina? S ne{\y}-to kak? {\Y}a cto-to ne viju, ctob{\yi} t{\yi} spexil vernuty {\y}e{\y} propaju.

— Ona rasstalasy s kinjalom po dobro{\y} vole. Duma{\y}u, cto projivet bez nevo {\y}ex̨o kako{\y}e-to vrem{\ia}. Do teh por poka vs{\e} ne koncitsa. Inace, bo{\y}usy, vnovy pojertvu{\y}et im radi nepon{\ia}tn{\yi}h v{\yi}sxih qele{\y}, i r{\ia}dom uje ne okajetsa men{\ia} dl{\ia} tovo, ctob{\yi} vernuty {\y}evo.

— Razumno, hot{\ia} i neskolyko jestoko… O! Kajetsa, nacina{\y}etsa!

{\Y}a raspahnul okno, tak kak toje usl{\yi}xal trub{\yi} gornistov.

Obe storon{\yi} uliqi b{\yi}li zaprujen{\yi} narodom.

— Kak budto d{\ia}d{\iu}xka tvo{\y}e{\y} vedym{\yi} pri{\y}ehal, a ne kardinal, — usl{\yi}xal {\y}a vozle uha poln{\yi}{\y} skeptiqizma golos Propovednika i otvetil:

— Slava ob Urbane bejit vperedi {\y}evo. Mnogi{\y}e scita{\y}ut {\y}evo {\y}edva li ne sv{\ia}t{\yi}m, oplotom ver{\yi} i budux̨im Papo{\y}. Horoxa{\y}a reputaqi{\y}a, pravilyn{\yi}{\y}e postupki, vsenarodna{\y}a l{\iu}bovy. On lucxi{\y} pravednik iz teh, cto {\y}esty v Qerkvi na dann{\yi}{\y} moment. Nikakih skandalov, vz{\ia}tok, podkupov, ubi{\y}stv i procevo. Lix vera, a realyna{\y}a vera, kak t{\yi} pomnix iz naxe{\y} proxlo{\y} besed{\yi} na etu temu, zaraja{\y}et l{\iu}de{\y}.

— Daje takih, kak t{\yi}? — poddel on men{\ia}.

— L{\iu}b{\yi}h.

Proqessi{\y}a v{\yi}gl{\ia}dela vnuxitelyno i seryezno. Vperedi marxirovala rota kantonskih na{\y}emnikov, nahod{\ia}x̨a{\y}as{\ia} na slujbe goroda. V paradn{\yi}h zolotist{\yi}h kirasah, xlemah s pl{\iu}majami, s serebr{\ia}n{\yi}mi alebardami na cern{\yi}h drevkah. Oni v{\yi}gl{\ia}deli {\y}arko, slovno rojdestvenska{\y}a igruxka, i ih prazdnicna{\y}a odejda silyno otlicalasy ot to{\y} koji, xersti i stali, v kotor{\yi}h oni predpocitali ne radovaty tolpu, a ubivaty {\y}e{\y}e.

Za roto{\y} na{\y}emnikov sledovala kavalykada pocetn{\yi}h jitele{\y} goroda i blagorodn{\yi}h gospod. V dorogih odejdah iz barhata, mehov{\yi}h xubah, oni p{\yi}talisy perex̨egol{\ia}ty drug druga. V{\yi}sxe{\y}e duhovenstvo v{\yi}gl{\ia}delo cuty skromne{\y}e, no nenamnovo. Mestn{\yi}{\y}e kliriki dl{\ia} vstreci dorogovo gost{\ia} dostali svo{\y}i lucxi{\y}e nar{\ia}d{\yi}. Kardinal Urban na fone vstreca{\y}ux̨e{\y} delegaqi{\y}i v{\yi}gl{\ia}del nasto{\y}ax̨im skromn{\ia}go{\y} — xirokopola{\y}a ala{\y}a xl{\ia}pa, krasna{\y}a sutana iz horoxe{\y} xersti, na plecah pelerina iz meha belovo krolika.

{\Y}a vperv{\yi}{\y}e videl celoveka, kotorovo kogda-to spas. {\Y}emu okazalosy okolo semides{\ia}ti, no dl{\ia} svo{\y}evo vozrasta on otlicno derjalsa v sedle. Osanka, posadka golov{\yi} i to, kak uverenno i spoko{\y}no on pravil jerebqom, govorili o tom, cto sil etomu celoveku ne zanimaty, nesmotr{\ia} na vnexn{\iu}{\y}u hudobu, beskrovn{\yi}{\y}e gub{\yi} i zapavxi{\y}e glaza. Naverno{\y}e, v molodosti u nevo b{\yi}lo pri{\y}atno{\y}e liqo, no se{\y}cas sozdavalosy vpecatleni{\y}e, cto {\y}a smotr{\iu} na staru{\y}u hix̨nu{\y}u ptiqu.

Kardinal to i delo podnimal ruku, blagoslovl{\ia}{\y}a privetstvu{\y}ux̨ih {\y}evo jitele{\y} goroda.

{\Y}evo v{\yi}sokopreosv{\ia}x̨enstvo soprovojdala vosymerka gvarde{\y}qev-alybalandqev v paradn{\yi}h mundirah i beretah. Vse kak odin svetlovolos{\yi}{\y}e i v{\yi}socenn{\yi}{\y}e. Takje v svite prisutstvovalo neskolyko klirikov v skromn{\yi}h odejdah, sredi kotor{\yi}h {\y}a zametil odnovo kalikveqa. Za sv{\ia}x̨ennoslujitel{\ia}mi {\y}ehali slugi. Dva des{\ia}tka cel{\ia}di, na kotor{\yi}h nikto i ne smotrel. A zr{\ia}.

{\Y}a srazu uvidel tovo, kovo jdal. Celovek v belo-koricnevom plax̨e palomnika. Smugloliqi{\y}, temnoglaz{\yi}{\y}, v usah i nepokr{\yi}t{\yi}h volosah bolyxo{\y}e kolicestvo sedin{\yi}, a na pravo{\y} skule v{\yi}del{\ia}lsa zametn{\yi}{\y} izdaleka krestoobrazn{\yi}{\y} xram. On s interesom glazel po storonam i toje pocti srazu uvidel men{\ia}, napolovinu v{\yi}sunuvxegos{\ia} iz okna. V {\y}evo temn{\yi}h glazah na mgnoveni{\y}e prostupila zolotista{\y}a jeltizna, i v sledu{\y}ux̨u{\y}u sekundu on smotrel v drugu{\y}u storonu, a {\y}ex̨o cerez polminut{\yi} uje skr{\yi}lsa za povorotom so vse{\y} kavalykado{\y}.



Valyter prixel v naznacenno{\y}e vrem{\ia}, v{\yi}n{\yi}rnuv iz uzkovo, propahxevo kr{\yi}sami pereulka. On priodelsa, tocno zajitocn{\yi}{\y} gorojanin, i {\y}evo volos{\yi} i us{\yi} zametno pobeleli.

— Gde kolyqo? — sprosil u men{\ia} koldun vmesto privetstvi{\y}a.

— Gde Kristina?

— O ne{\y} ne bespoko{\y}s{\ia}. U ne{\y}o s ostalyn{\yi}mi svo{\y}a zadaca. Ot teb{\ia} mnogo{\y}e ne trebu{\y}etsa, van Norma{\y}enn. Prosto da{\y} mne podobratsa k kardinalu.

— Ostalyn{\yi}{\y}e budut tam je?

— Duma{\y}u, tebe plevaty na ostalyn{\yi}h. Kristin{\yi} na prazdnike net. Ona jdet s loxadymi v uslovlennom meste. {\Y}esli vs{\e} pro{\y}det horoxo, u{\y}edem iz goroda kak mojno b{\yi}stre{\y}e i kak mojno dalyxe. Tebe {\y}a sovetu{\y}u sdelaty to je samo{\y}e, raz uj t{\yi} idex so mno{\y} do konqa.

— Ne do konqa, — vozrazil {\y}a. — Lix do tovo mesta, kogda mo{\y}a pomox̨ bolyxe ne budet nujna Kristine.

On rassme{\y}alsa:

— {\Y}a vsegda znal, cto straji drug za druga gotov{\yi} risknuty golovo{\y}.

— I polyzu{\y}exsa etim.

Koldun otvesil legki{\y} poklon:

— Glupo otriqaty. Idem. U nas menyxe polucasa.

Uliqi b{\yi}li zaprujen{\yi} narodom, i Propovednik s utra poxutil, cto daje mertv{\yi}{\y}e, {\y}esli b{\yi} oni mogli, spolzlisy b{\yi} s{\iu}da s gorodskih pogostov. {\Y}a ne stal rasstra{\y}ivaty starovo pelikana i govority {\y}emu, cto {\y}avleni{\y}e angela — eto ne bole{\y}e cem falyxivka, sozdanna{\y}a kucko{\y} moxennikov.

M{\yi} opazd{\yi}vali, prodira{\y}asy cerez l{\iu}dsko{\y}e stolpotvoreni{\y}e. V konqe konqov Valyteru eto nado{\y}elo, i on vs{\e} je ispolyzoval kako{\y}-to fokus. Gorojane, sto{\y}avxi{\y}e pered nami, nevolyno stali delaty xag v storonu, tolka{\y}a drugih i nastupa{\y}a im na nogi, a m{\yi} vklinivalisy v otkr{\yi}va{\y}ux̨i{\y}es{\ia} i tut je zahlop{\yi}va{\y}ux̨i{\y}es{\ia} brexi. Vprocem, vskore etot effekt propal, koldun, opasa{\y}asy privlec vnimani{\y}e, ostorojnical, i nam snova prixlosy rabotaty lokt{\ia}mi, kak sam{\yi}m ob{\yi}cn{\yi}m l{\iu}d{\ia}m.

V{\yi}hod na Malu{\y}u Karetnu{\y}u okazalsa perekr{\yi}t qep{\y}u straji — ona uderjivala svobodnu{\y}u pro{\y}ezju{\y}u casty dl{\ia} bogat{\yi}h priglaxenn{\yi}h, spexivxih k bogoslujeni{\y}u.

— Prokl{\ia}tye! — zlo rugnulsa Valyter. — Pridetsa iskaty drugo{\y} puty.

— Posto{\y}, — skazal {\y}a, dostav svo{\y} ``propusk", i pokazal kinjal odnomu iz soldat. — M{\yi} iz Bratstva.

Tot daje spority ne stal i, ne trebu{\y}a u kolduna pokazaty oruji{\y}e, otkr{\yi}l nam puty na uliqu.

— I bez vs{\ia}kovo volxebstva, — probormotal sebe pod nos mo{\y} nedrug.

Tepery k gorodsko{\y} sv{\ia}t{\yi}ne prodvigatsa stalo gorazdo legce — na pr{\ia}mo{\y} doroge, svobodno{\y} ot l{\iu}de{\y}, m{\yi} lix dvajd{\yi} postoronilisy, propuska{\y}a zapozdavxih blagorodn{\yi}h gospod.

Daleko-daleko gulko udarili cas{\yi} na novo{\y} ratuxe, i ih bo{\y} podhvatili kolokola. Torjestvenna{\y}a messa nacalasy. Ohrana, puskavxa{\y}a na osnovno{\y}e de{\y}stvo, okazalasy kuda bole{\y}e vnuxitelyno{\y} — kantonski{\y}e na{\y}emniki, s nimi tro{\y}e alybalandqev i para klirikov v prost{\yi}h ser{\yi}h r{\ia}sah. Vse te, kto ne smog popasty na plox̨ady, zan{\ia}li sosedni{\y}e uliqi, kr{\yi}xi okrestn{\yi}h domov i daje derev{\y}a.

— Tolyko po priglaxeni{\y}am, — skazal mne cernoborod{\yi}{\y}, pohoji{\y} na medved{\ia} na{\y}emnik.

— M{\yi} straji, — otvetil {\y}a.

— Viju, cto straji. No prikaz puskaty tolyko po gramotam, na kotor{\yi}h pecaty burgomistra, — nicuty ne smutilsa tot. — {\Y}esty gramota?

— {\Y}esty ko{\y}e-cto polucxe, — otvetil {\y}a i dostal persteny, podarenn{\yi}{\y} mne v Vione.

— Kupity, cto li, hocex? — opexil tot.

— Pogodi, dobr{\yi}{\y} celovek. — Klirik, prisluxivavxi{\y}s{\ia} k naxemu razgovoru, prot{\ia}nul ruku. — Da{\y} posmotrety.

{\Y}a polojil bezdeluxku {\y}emu na ladony.

— {\Y}a licn{\yi}{\y} duhovnik {\y}evo v{\yi}sokopreosv{\ia}x̨enstva. I pomn{\iu} eto kolyqo. Ono de{\y}stvitelyno kogda-to prinadlejalo {\y}emu. Teb{\ia} priglasil kardinal?

— Net, — ne stal {\y}a lgaty, cuvstvu{\y}a, kak napr{\ia}gs{\ia} Valyter. — No kogda-to {\y}a okazal uslugu {\y}evo v{\yi}sokopreosv{\ia}x̨enstvu i uveren, cto on pomnit men{\ia}. {\Y}a i mo{\y} drug hotim prisutstvovaty na stoly vajnom bogoslujeni{\y}i.

Klirik kivnul l{\yi}so{\y} golovo{\y}:

— Cto je, {\y}a ponima{\y}u vaxe jelani{\y}e. Kto {\y}a tako{\y}, ctob{\yi} mexaty priobx̨itsa k cudu i Gospodu.

— No priglaxeni{\y}e… — pop{\yi}talsa zaspority podoxedxi{\y} kapitan na{\y}emnikov.

— Etot celovek sdelal horoxe{\y}e delo dl{\ia} Qerkvi i licno {\y}evo v{\yi}sokopreosv{\ia}x̨enstva. V{\yi} jela{\y}ete potom ob{\y}asn{\ia}ty kardinalu, pocemu on ne b{\yi}l dopux̨en na bogoslujeni{\y}e?

— Konecno net!

— Togda propustite ih.

Kantoneq neohotno mahnul svo{\y}im l{\iu}d{\ia}m, i te podn{\ia}li alebard{\yi}, otkr{\yi}va{\y}a dorogu.

— Vse kuda legce, cem {\y}a ojidal, — usmehnulsa koldun, kogda ohrana ostalasy daleko pozadi.

Propovednik, vs{\e} eto vrem{\ia} tocno teny sledovavxi{\y} za mno{\y}, nakoneq-to dal vol{\iu} svo{\y}im emoqi{\y}ami:

— T{\yi} mnogovo ne ojida{\y}ex, certov ubl{\iu}dok!

Po scast{\y}u, krome men{\ia}, {\y}evo nikto ne sl{\yi}xal.

— Cto tepery? — sprosil {\y}a.

— T{\yi} svo{\y}e delo sdelal, van Norma{\y}enn. Mojex naslajdatsa predstavleni{\y}em. A mne nado podobratsa k kardinalu kak mojno blije.

No {\y}a ne dal {\y}emu u{\y}ti, polojiv ruku na pleco.

— Ne tak b{\yi}stro. T{\yi} prixel vmeste so mno{\y} i u{\y}dex so mno{\y}.

— Kak zna{\y}ex. Tolyko ne mexa{\y}, — legko soglasilsa on.

— Eto b{\yi}lo b{\yi} prox̨e osux̨estvity, {\y}esli b{\yi} t{\yi} rasskazal, cto sobira{\y}exsa sdelaty.

— Povery, nikto nicevo ne po{\y}met. M{\yi} u{\y}dem prejde, cem oni zamet{\ia}t, cto cto-to slucilosy.

{\Y}evo slova ne vnuxali doveri{\y}a, no {\y}a nade{\y}alsa na koz{\yi}ri v rukave, hot{\ia} ob{\yi}graty op{\yi}tnovo kolduna ne tak prosto.

Usilenn{\yi}{\y} sv{\ia}to{\y} magi{\y}e{\y} golos kardinala raznosilsa nad plox̨ad{\y}u. On cital na staroqerkovnom, {\y}az{\yi}ke vremen Konstantina. L{\iu}di, raspolojivxi{\y}es{\ia} na plox̨adi plotno{\y} tolpo{\y}, molilisy, xevel{\ia} gubami i v obx̨em-to ne slixkom horoxo vid{\ia}, cto proishodit za spinami vperedi sto{\y}ax̨ih.

{\Y}a skore{\y}e pocuvstvoval, cem uvidel, kak k nam priso{\y}edinilsa treti{\y} celovek v kaftane slugi burgomistra i v{\yi}soko{\y} xapke. Sud{\ia} po vsemu, Cezare proniknuty s{\iu}da okazalosy gorazdo legce, cem koldunu.

— U men{\ia} vs{\e} gotovo, — xepnul on Valyteru. — Gotthod na meste i jdet tvo{\y}e{\y} komand{\yi}.

— Pozabotsa o svo{\y}e{\y} zadace, — otvetil tot.

{\Y}a uje, kajetsa, znal, v cem zakl{\iu}ca{\y}etsa rabota na{\y}emnika. Vozle ``sv{\ia}tovo" mesta postavili rasp{\ia}ti{\y}e, r{\ia}dom s nim nahodilosy nebolyxo{\y}e vozv{\yi}xeni{\y}e, s kotorovo v{\yi}stupal kardinal. Vnizu sto{\y}ali predstaviteli duhovenstva i gorodskih vlaste{\y}. Tolpa ostavila svobodn{\yi}m lix nebolyxo{\y} p{\ia}tacok plox̨adi, to samo{\y}e mesto, gde nahodilsa otpecatok ``angela".

Kondotyer vnezapno v{\yi}brosil ruku, i {\y}a, ojidavxi{\y} cevo-to podobnovo, blokiroval {\y}e{\y}o predplecyem, ne dav stiletu udarity men{\ia} v gorlo.

V sledu{\y}ux̨e{\y}e mgnoveni{\y}e v qentre Kruso razverzlasy bezdna i voqarilsa ognenn{\yi}{\y} ad.



…Kr{\iu}c{\y}a, svisavxi{\y}e s potolka, b{\yi}li v bezobraznom sosto{\y}ani{\y}i. Rjav{\yi}{\y}e, s ostatkami temno{\y} ploti na gran{\ia}h, oni smerdeli zastarelo{\y} krov{\y}u i gnil{\yi}m m{\ia}som. Tocno tak je pahla i rexetka, na kotoro{\y} zdesy l{\iu}bili podjarivaty teh, kto ne raska{\y}iva{\y}etsa v svo{\y}ih oxibkah.

Nesmotr{\ia} na to cto v bolyxo{\y} jarovne besnovalosy plam{\ia}, v podvalah qentralyno{\y} t{\iu}rym{\yi} Kruso okazalosy jutko holodno.

Master doprosa, nemolodo{\y} jilist{\yi}{\y} subyekt s {\y}arkimi lucist{\yi}mi golub{\yi}mi glazami, sn{\ia}l kojan{\yi}{\y} fartuk, percatki i peredal stalyno{\y} prut odnomu iz pomox̨nikov.

— S etim vse.

``Etot" lejal rast{\ia}nut{\yi}m na xiroko{\y} m{\ia}sniqko{\y} stolexniqe i bolyxe pohodil ne na celoveka, a na kusok otbivno{\y}. {\Y}a ne prisutstvoval na p{\yi}tke, tak cto s trudom uznal otqa Gotthoda, kanonika sobora Sv{\ia}to{\y} Mari{\y}i v Braselovette. Po perelomann{\yi}m konecnost{\ia}m i krovav{\yi}m puz{\yi}r{\ia}m, kotor{\yi}{\y}e naduvalisy i lopalisy u nevo na gubah, b{\yi}lo pon{\ia}tno, cto on ne prot{\ia}net i polucasa. Vot-vot ispustit duh. Propovednik, uvidev tako{\y}e zrelix̨e, razvernulsa na kablukah i, nicevo ne skazav, v{\yi}xel von. On predpocital ne smotrety na to, cto {\y}emu b{\yi}lo nepri{\y}atno.

Pugala {\y}a nigde ne videl s momenta sob{\yi}ti{\y} na plox̨adi. Vpolne vozmojno, cto {\y}evo sto{\y}it iskaty vozle trupovozok, kotor{\yi}{\y}e se{\y}cas v bolyxom kolicestve sobira{\y}utsa na qentralyno{\y} uliqe.

Roman, mo{\y} star{\yi}{\y} znakom{\yi}{\y}, s kotor{\yi}m {\y}a perejil napadeni{\y}e rugaru v Hrustalyn{\yi}h gorah, privalivxisy k stene, smotrel na umira{\y}ux̨evo s poln{\yi}m ravnoduxi{\y}em. Kak na tot sam{\yi}{\y} kusok otbivno{\y}, o kotorom {\y}a tolyko cto upom{\ia}nul.

— Zakanciva{\y} s nim, master.

Palac vz{\ia}l so stolika xiroki{\y} m{\ia}sniqki{\y} noj i odnim dvijeni{\y}em prekratil stradani{\y}a umira{\y}ux̨evo.

— Kardinalyska{\y}a milosty, — ob{\y}asnil mne qigan, tocno opravd{\yi}va{\y}asy.

— T{\yi} privel men{\ia} uvidety {\y}e{\y}e?

— {\Y}evo v{\yi}sokopreosv{\ia}x̨enstvo platit svo{\y}i dolgi. Hot{\ia} b{\yi} casty ih.

— Stranno. — {\Y}a posmotrel, kak dva pomox̨nika palaca razreza{\y}ut verevki, st{\ia}giva{\y}ux̨i{\y}e ruki i nogi trupa, a zatem sbras{\yi}va{\y}ut okrovavlenno{\y}e telo v telejku. — Somneva{\y}usy, cto mne nujna b{\yi}la smerty klirika.

— T{\yi} ne pon{\ia}l. T{\yi} mog okazatsa na etom okrovavlennom stole. Nekotor{\yi}{\y}e iz okrujeni{\y}a Urbana scita{\y}ut, cto imenno tam ono i doljno b{\yi}ty. I oni b{\yi}li oceny ubeditelyn{\yi}.

— I {\y}a scastlivo izbejal eto{\y} ucasti, potomu cto… — {\Y}a predostavil {\y}emu zakoncity mo{\y}u frazu.

— Potomu cto m{\yi} znakom{\yi}, van Norma{\y}enn. Potomu cto t{\yi} snova spas jizny kardinalu, i on v dolgu pered tobo{\y}. Potomu cto te, kto hotel rast{\ia}nuty teb{\ia} na d{\yi}be, se{\y}cas uje napravl{\ia}{\y}utsa v N{\y}ugort. Im trebu{\y}etsa poka{\y}ani{\y}e za ih gluposty. A lucxe vsevo {\y}evo dobitsa, izuca{\y}a vereskov{\yi}{\y}e pustoxi.

Telo uvezli, ostalsa lix gusto{\y}, lipki{\y} zapah krovi, napolnivxi{\y} holodno{\y}e pomex̨eni{\y}e.

— Nu, cto je. Peredava{\y} kardinalu mo{\y}i blagodarnosti. Hot{\ia} {\y}a i ne rad, cto celovek umer.

— Vzdumal seb{\ia} vinity?

— Net. {\Y}a postupil pravilyno. A oni znali, cem risku{\y}ut.

— Vmesto p{\ia}ti soten trupov v Kruso segodn{\ia} b{\yi}lo b{\yi} tri-cet{\yi}re t{\yi}s{\ia}ci, {\y}esli b{\yi} nikto iz nas ne okazalsa gotov k podobnomu povorotu sob{\yi}ti{\y}.

{\Y}a s sodrogani{\y}em vspomnil vzr{\yi}v, {\y}arku{\y}u vsp{\yi}xku i potoki jivovo, pohojevo na zme{\y}, zolotovo ogn{\ia}, rinuvxegos{\ia} vo vse storon{\yi}. Za sekundu on projeg v r{\ia}dah molivxihs{\ia} brex, jgucim molotom udaril po klirikam i otpr{\ia}nul ot si{\y}a{\y}ux̨e{\y} svetom pregrad{\yi}. Kupol stremitelyno rvanul vverh i v storon{\yi}, nakr{\yi}v snacala plox̨ady, zatem okrestn{\yi}{\y}e doma, a potom i uliqi, poglox̨a{\y}a v seb{\ia} stranno{\y}e, oslepitelyno-zoloto{\y}e plam{\ia}, ne dava{\y}a {\y}emu rasprostranitsa i navredity {\y}ex̨o komu-nibudy. No vs{\e} ravno etovo okazalosy nedostatocno, ctob{\yi} spasti vseh.

Paniku{\y}ux̨a{\y}a tolpa, begux̨a{\y}a proc, vo{\y}ux̨a{\y}a ot ujasa iz-za tovo, cto koneq sveta, obex̨ann{\yi}{\y} angelom, uje nastupil, razlucila men{\ia} s Cezare v tot ``cudesn{\yi}{\y}" dl{\ia} nas obo{\y}ih moment, kogda m{\yi} sobiralisy ubity drug druga.

Master doprosa v{\yi}sluxal xepot pomox̨nika i soobx̨il:

— Vtoro{\y} gotov k razgovoru. Smotrety budete?

— Vtoro{\y}? — udivilsa {\y}a. — V{\yi} smogli vz{\ia}ty dvo{\y}ih?

— K sojaleni{\y}u, tolyko dvo{\y}ih. Ih, v otlici{\y}e ot ostalyn{\yi}h, po{\y}maty okazalosy ne tak uj i slojno. {\Y}a pristavil k nim l{\iu}de{\y} srazu posle naxevo s tobo{\y} razgovora.

M{\yi} proxli v sosedni{\y} zal, zerkalynu{\y}u kopi{\y}u pred{\yi}dux̨evo, i {\y}a uvidel starovo aptekar{\ia}. On sidel za stolom, {\y}evo ruki i nogi b{\yi}li zafiksirovan{\yi} xirokimi kojan{\yi}mi remn{\ia}mi, a golova zasunuta v mehanizm kra{\y}ne neprigl{\ia}dnovo vida. Tiski plotno ohvat{\yi}vali nijn{\iu}{\y}u cel{\iu}sty i tem{\ia}.

— Horoxo, — pohvalil Roman palaca i sklonilsa nad Filippom. — Udobno li vam, metr?

Tot ne smog nicevo otvetity, lix prom{\yi}cal cto-to neclenorazdelyno{\y}e i umol{\ia}{\y}ux̨e{\y}e.

— Duma{\y}u, cto ne slixkom. Naverno{\y}e, stranno spraxivaty tako{\y}e u celoveka, nahod{\ia}x̨egos{\ia} v cerepodrobilke. Dava{\y}te {\y}a nemnogo rasskaju vam, cto eto tako{\y}e. Na samom dele, metr, vs{\e} predelyno prosto. Palac krutit vint, i tiski nacina{\y}ut sdavlivaty vaxu golovu. Sperva loma{\y}utsa zub{\yi}, a b{\yi}ty mojet, nijn{\ia}{\y}a cel{\iu}sty. {\Y}esli cestno, {\y}a nikogda ne vnikal v detali. Zatem lopnet cerep, no, poveryte, posle etovo v{\yi} projivete dostatocno dolgo, ctob{\yi} pojalety o glupost{\ia}h, kotor{\yi}{\y}e ucinili.

Filipp zaplakal, zam{\yi}cal aktivne{\y}e.

— Mo{\y} vam sovet, aptekary. {\Y}esli hotite izbejaty lixne{\y}… golovno{\y} boli i zaslujity prox̨eni{\y}e kardinala, na kotorovo v{\yi} tak bezdarno pokuxalisy, poka{\y}tesy, sozna{\y}tesy i nacina{\y}te sotrudnicaty. — Roman pohlopal uznika po plecu. — {\Y}a vernusy cerez polcasa i s radost{\y}u usl{\yi}xu pravilyn{\yi}{\y} otvet. Idem, Ludwig.

M{\yi} pokinuli p{\yi}tocnu{\y}u, podn{\ia}vxisy na dva etaja vverh, peresekli pusto{\y} t{\iu}remn{\yi}{\y} dvor, minovali ohranu i v{\yi}xli na uliqu. Eto b{\yi}la okra{\y}ina goroda, v dvuh xagah ot vnexne{\y} sten{\yi}.

— Pogovorim. — On zabralsa na goru kirpica, svalennovo rabocimi, sobiravximis{\ia} remontirovaty ukrepleni{\y}e.

— Zdesy? — udivilsa {\y}a, no priso{\y}edinilsa k nemu.

— Vidno vs{\iu} okrugu. Uj lucxe, cem kabinet nacalynika t{\iu}rym{\yi}, gde kajd{\yi}{\y} durak mojet podsluxaty.

— U vas {\y}esty ofiqialyna{\y}a versi{\y}a pro{\y}izoxedxevo v gorode?

— A kogda {\y}e{\y}o ne b{\yi}lo? — On usmehnulsa v us{\yi}. — L{\iu}di uje rabota{\y}ut {\y}az{\yi}kami, sluhi odin huje drugovo let{\ia}t po dorogam da mnojatsa v traktirah. Za mes{\ia}q uzna{\y}ut vse i vezde. Eto slucilosy oceny ne vovrem{\ia}. Hot{\ia} kogda podobn{\yi}{\y}e sob{\yi}ti{\y}a voobx̨e mogut b{\yi}ty k mestu? Nam pridetsa prilojity massu usili{\y}, ctob{\yi} sgladity posledstvi{\y}a pro{\y}isxedxevo.

— Nam?

— {\Y}a sluju kardinalu, a on — Qerkvi. Tak cto v dannom kontekste daje koldunu i rugaru mojno govority ``nam". Naskolyko {\y}a pon{\ia}l, vseh sobak poves{\ia}t na zlobnovo d{\y}avolopoklonnika, kotor{\yi}{\y} hotel isportity radosty veru{\y}ux̨im, oskvernity sv{\ia}t{\yi}n{\iu}, popraty zakon{\yi} Bojyi i proca{\y}a, proca{\y}a, proca{\y}a.

{\Y}a ojidal cevo-to podobnovo:

— V obx̨em, skazka o cudovix̨e, rexivxem ustro{\y}ity bessm{\yi}slenno{\y}e ubi{\y}stvo.

— Ili kako{\y}-nibudy jertvenn{\yi}{\y} ritual. L{\iu}d{\ia}m soverxenno nezacem znaty nasto{\y}ax̨ih pricin.

— A m{\yi} ih zna{\y}em? Eti sam{\yi}{\y}e pricin{\yi}?

Roman ne otvetil, i eto govorilo o tom, cto on, kak i {\y}a, ter{\ia}{\y}etsa v dogadkah.

— Scita{\y}ex, cto koldun urovn{\ia} Valytera sposoben ustro{\y}ity ognenn{\yi}{\y} vulkan v qentre goroda? — zadal {\y}a {\y}ex̨o odin vopros.

On posmotrel na men{\ia} dolgim, t{\ia}jel{\yi}m vzgl{\ia}dom:

— I t{\yi} duma{\y}ex tocno tak je. Zapomni eto, {\y}esli dorojix svo{\y}e{\y} xkuro{\y}. — I, cuty sm{\ia}gcivxisy, pro{\y}iznes: — {\Y}esty vex̨i, o kotor{\yi}h ne sto{\y}it boltaty, Ludwig. Naprimer, {\y}a stara{\y}usy pomalkivaty, cto polna{\y}a luna ne slixkom horoxo vli{\y}a{\y}et na men{\ia} v posledne{\y}e vrem{\ia}.

M{\yi} oba usmehnulisy tolyko nam pon{\ia}tno{\y} xutke.

— Tak cto vinovat Valyter. Poka ne budet dokazano obratno{\y}e. A ono, kak t{\yi} ponima{\y}ex, dokazano, skore{\y}e vsevo, ne budet. Vo vs{\ia}kom sluca{\y}e, tolpe.

— No t{\yi} ne verix, cto koldun markgrafa Valentina ime{\y}et stolyko sil.

— Razve eto tak vajno? — On ustalo poter veki.

— Dl{\ia} men{\ia} — vajno.

Q{\yi}gan sdalsa:

— Ne ver{\iu}. {\Y}a zna{\y}u o temnom iskusstve ne ponasl{\yi}xke. Na mo{\y} vzgl{\ia}d, podobno{\y}e ne mojet provernuty nikto iz teh, s kem {\y}a znakom. Zdesy nujna mox̨ velefa ili kakovo-nibudy legendarnovo carode{\y}a iz Temnoles{\y}a. Ili oceny seryeznovo demona. Takovo, kto odnim x̨elckom palyqev sjiga{\y}et p{\ia}ty soten dux i {\y}edva ne prolam{\yi}va{\y}et x̨it des{\ia}ti gotov{\yi}h k otrajeni{\y}u ataki klirikov. Koldun{\yi} na tako{\y}e ne sposobn{\yi}. Inace mirom pravili b{\yi} oni, a ne kn{\ia}z{\y}a i qerkovniki.

— {\Y}esli on b{\yi}l nastolyko silen, cto ne bo{\y}alsa klirikov, to pocemu otstupil? Pocemu ne ubil vseh, kto nahodilsa na plox̨adi?

— {\Y}a ne zna{\y}u.

{\Y}a zadumalsa.

— No {\y}esli eto b{\yi}l ne Valyter, a kto-to ino{\y}, to slucivxe{\y}es{\ia} na plox̨adi — sovpadeni{\y}e? Komanda zagovorx̨ikov, p{\yi}tavxihs{\ia} ubity kardinala, i ne{\y}izvestn{\yi}{\y}, rexivxi{\y} ustro{\y}ity lokalyn{\yi}{\y} apokalipsis?

— Inovo ob{\y}asneni{\y}a u men{\ia} net. I ne tolyko u men{\ia}. Inkviziqi{\y}a cexet v zat{\yi}lke, razvodit rukami i lihoradocno lista{\y}et grimuar{\yi}. Tolyko duma{\y}etsa mne, cto bez tolku eto vse. Nikovo m{\yi} ne na{\y}dem. No im nujen kozel otpux̨eni{\y}a. Lucxe vsevo tot hagjit, o kotorom t{\yi} govoril. Otlicna{\y}a kandidatura dl{\ia} kostra. Ob{\yi}vateli ne jalu{\y}ut cujezemqev i inoverqev, s radost{\y}u sojgut zlobno{\y}e cudovix̨e, voshit{\ia}tsa tem, cto vozmezdi{\y}e Qerkvi nastiglo prestupnika, i uspoko{\y}atsa. Nu a {\y}esli ne po{\y}ma{\y}em etovo hagjita, na{\y}dem drugovo. Ili je pridetsa ispolyzovaty bedn{\ia}gu-aptekar{\ia}. Morx̨ixsa? Ne sto{\y}it. Eto zvucit jestoko, no inace prosto nelyz{\ia}. Simvol{\yi} ver{\yi} i sil{\yi} doljn{\yi} ostavatsa nez{\yi}blem{\yi}. Inace nacnetsa haos.

— Kogda v{\yi} pon{\ia}li, cto sled angela — falyxivka?

Roman slojil uzlovat{\yi}{\y}e palyqi v kulak, poter im zat{\yi}lok:

— Slixkom pozdno dl{\ia} tovo, ctob{\yi} vs{\e} ostanovity. Sluhi poleteli, a palomniki i mestn{\yi}{\y}e kliriki-tupiqi uje skolacivali krest da jgli sveci. Perv{\yi}{\y} je inkvizitor s magi{\y}e{\y} vs{\e} raskusil.

— Valytera nelyz{\ia} nazvaty na{\y}ivn{\yi}m. On ne veril, cto {\y}evo obman proderjitsa dolgo, znacit, znal, cto v{\yi} ne ostanovite predstavleni{\y}e.

— U nas ne b{\yi}lo v{\yi}bora. Cto m{\yi} doljn{\yi} b{\yi}li delaty? Ob{\y}avity vo vseusl{\yi}xani{\y}e, cto kucka moxennikov rexila naduty veru{\y}ux̨ih? Cto nikakovo cuda net, a poslednevo angela videli poltor{\yi} t{\yi}s{\ia}ci let nazad? Zacem rubity suk, na kotorom sidix, Ludwig? L{\iu}di hot{\ia}t verity v cudesa, i kto m{\yi} taki{\y}e, ctob{\yi} razruxaty ih ill{\iu}zi{\y}i?

{\Y}a lix neveselo rassme{\y}alsa:

— Mne povezlo, Roman. Mo{\y}a rabota gorazdo prox̨e.

— A mo{\y}a ne jdet. — On vstal i prot{\ia}nul ruku. — T{\yi} spas mne jizny v teh prokl{\ia}t{\yi}h gorah, i {\y}a ne l{\iu}bl{\iu} b{\yi}ty doljen. {\Y}a sobira{\y}u sluhi i informaqi{\y}u. Stara{\y}a ferma v cetverti ligi za gorodom, {\y}esli {\y}ehaty cerez Koxacyi vorota. {\Y}a smogu priderjivaty etu novosty paru-tro{\y}ku casov, ne bolyxe. Uvezi jenx̨inu kak mojno dalyxe, {\y}esli ne hocex, ctob{\yi} ona sgnila v podvale.

— Oni budut preduprejden{\yi}. Vse.

— I cto s tovo? M{\yi} ih po{\y}ma{\y}em. No {\y}esli straja ne budet sredi nih, {\y}e{\y}o nikto ne stanet iskaty. Obex̨a{\y}u.

— Spasibo, Roman. M{\yi} v rascete.

— Bo{\y}usy, cto net. Eto b{\yi}la usluga za uslugu.



Ferma v{\yi}gl{\ia}dela zabroxenno{\y}, no v {\y}e{\y}o okoxkah gorel prigluxenn{\yi}{\y} svet.

Nebo b{\yi}stro temnelo.

{\Y}a napravilsa k vorotam, duma{\y}a, cto snova de{\y}stvu{\y}u naobum, pr{\yi}ga{\y}u v omut golovo{\y} i Valyteru v obx̨em-to nicevo ne sto{\y}it sdelaty to, cto ne polucilosy u na{\y}emnika.

{\Y}a zametil, kak drognula zanaveska — nabl{\iu}davxi{\y} za dorogo{\y} uvidel men{\ia}. Tem lucxe.

Vorota b{\yi}li ne zapert{\yi}. {\Y}a voxel vo dvor — gr{\ia}zn{\yi}{\y}, s dvum{\ia} ogromn{\yi}mi lujami i telego{\y}, perevernuto{\y} nabok. Dvery v dom otkr{\yi}lasy, i na poroge po{\y}avilsa Cezare. Posmotrel na men{\ia} t{\ia}jel{\yi}m vzgl{\ia}dom, ne ubira{\y}a ruki s vis{\ia}x̨e{\y} na po{\y}ase dagi:

— T{\yi} odin?

— Kak vidix.

On postoronilsa i, kogda {\y}a voxel, ostalsa na uliqe, rexiv proverity, ne prixel li za mno{\y} kto-to {\y}ex̨e.

Vnutri pahlo star{\yi}m zaplesnevevxim domom, pol b{\yi}l zeml{\ia}no{\y}. Bolyxu{\y}u casty {\y}edinstvenno{\y} komnat{\yi} zanimal ost{\yi}vxi{\y} ocag. Skudna{\y}a krest{\y}anska{\y}a mebely, a takje pr{\ia}lka okazalisy sdvinut{\yi} k dalyne{\y} stene. V uglu, zavernuvxisy v tonko{\y}e ode{\y}alo, spal hagjit. Koldun, do etovo cto-to pisavxi{\y}, tepery smotrel na men{\ia}.

Kristina por{\yi}visto brosilasy ko mne, krepko obn{\ia}la:

— {\Y}a dumala, t{\yi} pogib! Valyter skazal, cto tam, gde t{\yi} sto{\y}al, nikto ne v{\yi}jil!

B{\yi}vxi{\y} sluga markgrafa Valentina vstretil mo{\y} krasnoreciv{\yi}{\y} vzgl{\ia}d s ponima{\y}ux̨e{\y} ul{\yi}bko{\y}:

— Kak vidno, {\y}a oxibalsa. Etot Cezare vecno vs{\e} puta{\y}et.

— I nedodel{\yi}va{\y}et. Oni p{\yi}talisy izbavitsa ot men{\ia}.

— Cto?! — Ona gnevno nahmurila brovi, rezko povernuvxisy k Valyteru. — T{\yi} je obex̨al!

— E{\y}! E{\y}! Uspoko{\y}s{\ia}.

— Uspoko{\y}itsa?! — razozlilasy Kristina. — T{\yi} sukin s{\yi}n, Valyter! Ne tolyko zavalil delo, no i hotel ubity tovo, kto soglasilsa pomoc nam!

— Eto b{\yi}la iniqiativa Cezare. {\Y}a ne daval {\y}emu takih raspor{\ia}jeni{\y}. Kl{\ia}nusy!

Ona xagnula k nemu s {\y}avn{\yi}m namereni{\y}em udarity, no {\y}a vz{\ia}l {\y}e{\y}o za predplecye:

— M{\yi} uhodim. Pr{\ia}mo se{\y}cas.

— Neujeli? — vkradcivo skazal on, polojil pero na stol i vstal.

{\Y}a vosprin{\ia}l eto kak ugrozu i opustil ruku na kinjal.

— Duma{\y}ex, amulet tvo{\y}e{\y} vedym{\yi} spaset teb{\ia} ot magi{\y}i?

— Proverim?

— Prekratite! Oba! — kriknula Kristina, razbudiv Adil{\ia}. — I tak vs{\e} ploho. Stolyko mes{\ia}qev rabot{\yi} nasmarku! M{\yi} ne zna{\y}em, gde oteq Gotthod i Filipp, oni do sih por ne prixli. Xans v{\yi}{\y}ti na kuzneqa poter{\ia}n! A v{\yi} duma{\y}ete tolyko o tom, kak pustity drug drugu krovy!

Valyter mirol{\iu}bivo podn{\ia}l ruki vverh i vnovy ugnezdilsa na stule:

— Prejde cem t{\yi} primex rexeni{\y}e, uzna{\y} u van Norma{\y}enna, kak tot naxel nas. On, naverno{\y}e, veliki{\y} volxebnik, raz okazalsa zdesy. Vedy t{\yi} {\y}emu ne govorila o naxem ubejix̨e?

— Ne govorila. — Ona voprositelyno posmotrela na men{\ia}. — Tak kak, Ludwig?

Etot ubl{\iu}dok cudesno perevel razgovor na drugu{\y}u temu.

— Nevajno. {\Y}a zdesy. A u vas malo vremeni. Nado v{\yi}vezti teb{\ia} ots{\iu}da, Kristina. Ostanexsa s nim — propadex.

— Pff! — Koldun provorno nacal sobiraty vex̨i, a britogolov{\yi}{\y} hagjit zastegnul na s{\iu}rtuke po{\y}as s krivo{\y} sable{\y}.

{\Y}a pocuvstvoval, kak zat{\yi}lok ukololo cto-to holodno{\y}e.

— Cezare! — kriknula Kristina. — Sto{\y}!

— {\Y}a ne sluxa{\y}u tvo{\y}ih prikazov, jenx̨ina. Valyter? — Kondotyer podobralsa ko mne soverxenno nezametno.

— Ostavy {\y}evo, — poprosil koldun, okoncatelyno sobrav sumku. — Nam on ne nujen. Zacem rasstra{\y}ivaty Kristinu? Pora u{\y}ezjaty. I nacinaty vs{\e} snacala.

— Ne budet nikakovo nacala. T{\yi} uje pro{\y}igral.

— T{\yi} vse-taki prosto tupo{\y} gromila, alybalandeq. U men{\ia} ne b{\yi}lo v{\yi}bora. Jizny qelovo mira postavlena na kartu. {\Y}a poxel b{\yi} daje v pasty drakona, {\y}esli b{\yi} eto priblizilo men{\ia} k temnomu kuznequ. Kak t{\yi} ne po{\y}mex taku{\y}u prostu{\y}u vex̨: vse, cto {\y}a govoril tebe o temn{\yi}h kinjalah, — pravda. {\Y}a na{\y}du nov{\yi}{\y} glaz serafima. I poprobu{\y}u snova. A t{\yi} mojex bejaty v Ardenau, zar{\yi}ty golovu v pesok i dumaty, cto {\y}a lgu, raz tebe tak legce.

— Tak i postupl{\iu}. No snacala zaberu {\y}e{\y}e.

— {\Y}a ne po{\y}edu s tobo{\y}, Ludwig, — tiho skazala Kristina.

— Hocex ostatsa s nimi? Zna{\y}ex, cto slucitsa, kogda teb{\ia} po{\y}ma{\y}ut? Straj budet obvinen v zagovore protiv Riapano. Eto udarit po Bratstvu. Po kajdomu iz nas, gde b{\yi} m{\yi} ni nahodilisy. Uedem, poka ne pozdno. Ostavy ih. T{\yi} doljna jity, a ne umerety na d{\yi}be. Eto posledni{\y} xans.

Kristina vz{\ia}la iz ruk Valytera svo{\y}u kurtku, nadela, drojax̨imi palyqami, nemnogo nelovko, zastegnula pugoviqi:

— Tebe {\y}a tocno nicevo ne doljna, van Norma{\y}enn. Vse, cto b{\yi}lo mejdu nami v dalekom proxlom, tepery ne ime{\y}et znaceni{\y}a. U men{\ia} svo{\y} puty, a u teb{\ia} svo{\y}. Uhodi, Ludwig. Pr{\ia}mo se{\y}cas. I bolyxe ne ix̨i men{\ia}.

{\Y}a posmotrel {\y}e{\y} v glaza, pon{\ia}l, cto vs{\e} bespolezno, cto {\y}a ne smogu pereubedity {\y}e{\y}e, cto Kristinu, celoveka, s kotor{\yi}m {\y}a kogda-to ucilsa, ros, jil i srajalsa plecom k plecu, uje ne vernex. I v{\yi}xel iz doma, plotno prikr{\yi}v za sobo{\y} dvery…

{\Y}a xel po temnomu pustomu traktu k Kruso, a na duxe u men{\ia} skrebli koxki.

Cto je. {\Y}a hot{\ia} b{\yi} pop{\yi}talsa. Ona prin{\ia}la rexeni{\y}e. V{\yi}brala svo{\y}u sudybu, svo{\y}u jizny, svo{\y}u qely. I bessm{\yi}slenno lovity rukami uskolyza{\y}ux̨u{\y}u teny. Tratity vrem{\ia} i sil{\yi}. {\Y}a uznal otvet{\yi} na vopros{\yi}, kotor{\yi}{\y}e men{\ia} volnovali, i tepery sledu{\y}et dvigatsa dalyxe. Idti vpered i ne ogl{\ia}d{\yi}vatsa.

Vperedi pokazalisy dva znakom{\yi}h silueta. Odin v{\yi}socenn{\yi}{\y} i dolgov{\ia}z{\yi}{\y}, drugo{\y} nev{\yi}soki{\y} i suhonyki{\y}.

— Ne v{\yi}xlo? — negromko sprosil Propovednik. — Pocemu?

— {\Y}a ne mogu spasti {\y}e{\y}o ot samo{\y} seb{\ia}, drujix̨e.

— T{\ia}jelo vvesti l{\iu}de{\y} v Qarstvi{\y}e Nebesno{\y}e, {\y}esli oni ne jela{\y}ut spaseni{\y}a, — probormotal on i skazal kuda gromce: — No tepery Qerkovy na{\y}det {\y}e{\y}o i nakajet kak zagovorx̨iqu.

— Ili ne na{\y}det, {\y}esli udaca budet na {\y}e{\y}o storone.

{\Y}a sunul ruki v karman{\yi}, xaga{\y}a dalyxe, i oni pristro{\y}ilisy r{\ia}dom.

— I cto tepery? — ne v{\yi}derjal Propovednik.

— Za{\y}musy delami. V mire polno temn{\yi}h dux. Syezju v Ardenau. {\Y}a ne b{\yi}l na rodine neskolyko let. Vstrecusy s Gertrudo{\y}. Rasskaju obo vsem, cto zdesy pro{\y}izoxlo, stare{\y}xinam. Bratstvo doljno b{\yi}ty gotovo k nepri{\y}atnost{\ia}m.

M{\yi} so star{\yi}m pelikanom sdelali {\y}ex̨o neskolyko xagov, prejde cem pon{\ia}li, cto Pugalo nas ne soprovojda{\y}et. Ono sto{\y}alo na doroge, v{\yi}t{\ia}nuvxisy v strunku, tocno teryer, pocu{\y}avxi{\y} lisu, i smotrelo tuda, otkuda {\y}a prixel.

— E{\y}, Solomenna{\y}a golova! — okliknul {\y}evo Propovednik. — Zab{\yi}lo, kuda nado idti? E-e{\y}! M{\yi} zdesy. Iisuse Hriste, t{\yi} ne tolyko onemelo, no i oglohlo?!

— Pogodi, — nahmurilsa {\y}a i podoxel k Pugalu.

Ono melko drojalo, i v uzkih glazah to zagoralisy, to gasli dva malenykih ugolyka.

— Cto tam? Cto t{\yi} vidix?

Ono polojilo mne na pleco t{\ia}jelu{\y}u, kostl{\ia}vu{\y}u ruku i razvernulo, predlaga{\y}a smotrety ne na nevo, a na mracnu{\y}u dorogu, barhatno{\y}e zvezdno{\y}e nebo i temn{\yi}{\y}e siluet{\yi} derevyev, v{\yi}stupa{\y}ux̨i{\y}e na etom fone. Vokrug b{\yi}la posledn{\ia}{\y}a noc zim{\yi}, stranna{\y}a to{\y} zlovex̨e{\y} tixino{\y}, kotora{\y}a zastiga{\y}et odinokovo putnika na pust{\yi}nnom trakte. {\Y}a, kajetsa, ne d{\yi}xal, vmeste s Pugalom smotr{\ia} vo mrak. I tot otvetil mne.

Zolotisto{\y} iskro{\y}. Zoloto{\y} vsp{\yi}xko{\y}. Zolot{\yi}m svetom.

Ogony qveta jidkovo zolota podn{\ia}lsa v{\yi}xe drevesn{\yi}h kron i tut je opal, ostaviv v nebe zoloto{\y}e zarevo.

— O, Gospodi! — ahnul Propovednik.

No {\y}a uje ne sluxal {\y}evo. Bejal obratno.



Zolot{\yi}{\y}e kostr{\yi}, taki{\y}e tepl{\yi}{\y}e, prekrasn{\yi}{\y}e, pohoji{\y}e ne na ob{\yi}cn{\yi}{\y} ogony, a na rasplavlenn{\yi}{\y} dragoqenn{\yi}{\y} metall, goreli povs{\iu}du. V lesu, na ogromnom pustom pole i tam, gde {\y}ex̨o sovsem nedavno sto{\y}ala stara{\y}a ferma.

Ih b{\yi}lo neskolyko des{\ia}tkov, haoticn{\yi}h, razbrosann{\yi}h po okruge, soverxenno nevero{\y}atn{\yi}h. Volxebn{\yi}h. I smertelyno opasn{\yi}h.

Eto b{\yi}lo to je plam{\ia}, cto svirepstvovalo na plox̨adi v Kruso, puska{\y} i mene{\y}e {\y}arostno{\y}e. Ono gorelo, popira{\y}a vse zakon{\yi} mirozdani{\y}a, samo po sebe, ne nujda{\y}asy v toplive i ne zavis{\ia} ot kaprizov vetra.

Pugalo ne stalo podhodity k blija{\y}xemu kostru, a ostanovilosy kak vkopanno{\y}e i, kazalosy, n{\iu}halo vozduh, pahnux̨i{\y} t{\ia}jelo{\y} gar{\y}u i, cuty ulovimo, perejarenn{\yi}m m{\ia}som. Zatem ono opustilo golovu, ssutulilosy i s nekotor{\yi}m razocarovani{\y}em selo na zeml{\iu}. Na {\y}evo vzgl{\ia}d, tut uje ne b{\yi}lo nicevo interesnovo.

Propovednik ne poxel so mno{\y} po ino{\y} pricine — on bo{\y}alsa, hot{\ia} ni odin ogony ne mog pricinity duxe vreda.

— Mojet, ne sto{\y}it tebe tuda lezty?! — kriknul on mne v spinu.

No {\y}a ne mog postupity inace.

Pervo{\y}e telo, obuglenno{\y}e do golovexek, vs{\e} {\y}ex̨o d{\yi}m{\ia}x̨e{\y}es{\ia}, {\y}a naxel r{\ia}dom s mertv{\yi}mi loxadymi. Lix po krivo{\y} polose metalla, v kotoro{\y} trudno b{\yi}lo opoznaty sabl{\iu}, {\y}a pon{\ia}l, cto eto hagjit.

V dom {\y}a za{\y}ti ne smog, tot vs{\e} {\y}ex̨o pol{\yi}hal, poetomu napravilsa ot kostra k kostru, po v{\yi}jjenno{\y} zemle.

I {\y}edva ne spotknulsa o trup Cezare. On lejal na jivote, i v {\y}evo spine b{\yi}la projjena skvozna{\y}a d{\yi}ra velicino{\y} s dva mo{\y}ih kulaka. Glaza okazalisy raspahnut{\yi}, na liqe zast{\yi}li udivleni{\y}e i obida.

{\Y}a vs{\e} dalyxe othodil ot ferm{\yi}, prodolja{\y}a iskaty, i v glazah postepenno nacinalo dvo{\y}itsa ot zolot{\yi}h ogne{\y}. Ih b{\yi}lo kuda bolyxe, cem mne pokazalosy vnacale.

{\Y}a b{\yi} proxel mimo, {\y}esli b{\yi} ona men{\ia} ne okliknula. {\y}e{\y}o liqo pocernelo ot kopoti, prava{\y}a ruka napominala obgorevxu{\y}u vetku, a na to, cto b{\yi}lo nije grudi, nelyz{\ia} smotrety bez slez — odin sploxno{\y} ojog.

Ona pop{\yi}talasy ul{\yi}bnutsa, pokazaty, cto vs{\e} horoxo, no polucilosy eto nevajno. Kristina spl{\iu}nula temno-koricnevu{\y}u sl{\iu}nu, {\y}a ox̨util pr{\ia}n{\yi}{\y}, {\y}edki{\y} zapah i pon{\ia}l, cto ona tolyko cto syela koreny zolotovo lyva, silyn{\yi}{\y} hagjitski{\y} narkotik, izbavl{\ia}{\y}ux̨i{\y} ot l{\iu}bo{\y} boli.

— Ne povezlo, — tolyko i skazala ona. — M{\yi} iskali {\y}evo, a on naxel nas.

B{\yi}lo pon{\ia}tno, o kom ona govorit.

— T{\yi} videla temnovo kuzneqa?

— Izdali. — Straj uronila golovu na zeml{\iu}. — {\Y}a ne smogu tebe pomoc. Poobex̨a{\y} mne sdelaty ko{\y}e-cto.

— Obex̨a{\y}u.

— Otpravl{\ia}{\y}s{\ia} v Klagenfurt. Tam jivet doc Valytera. Uliqa Sten{\yi}. U ne{\y}o dar. {\Y}a pokl{\ia}lasy {\y}emu, cto Bratstvo {\y}e{\y}o primet. Ne perebiva{\y}. Sluxa{\y}.

Ona vz{\ia}la iz raspotroxenno{\y} sumki {\y}ex̨o odin koreny, otpravila {\y}evo sebe za x̨eku:

— Pod polom v {\y}evo dome {\y}esty kniga. Sojgi {\y}e{\y}e. Eto vajno. Sdela{\y}ex?

— Da.

— Vozymi sebe V{\y}una. Bolyxe {\y}a nikomu {\y}evo ne dover{\iu}.

— Horoxo.

{\Y}a videl, kak mutne{\y}ut {\y}e{\y}o glaza ot narkotika, i predstavl{\ia}l, kaku{\y}u boly ona doljna isp{\yi}t{\yi}vaty se{\y}cas.

— Tretye. Ne ix̨i temnovo kuzneqa. Inace on pridet i za tobo{\y}. Kak prixel za nami. Pokl{\ia}nisy!

— Kl{\ia}nusy. — Na etot raz {\y}a lgal.

Ona ustalo zakr{\yi}la glaza i skazala cuty zapleta{\y}ux̨imsa {\y}az{\yi}kom:

— I skaji Miriam: mne ujasno jaly, cto {\y}a {\y}e{\y}o podvela.

— Eto ne tak. No {\y}a peredam.

{\Y}a dostal iz sumki kinjal, vlojil v {\y}e{\y}o ruku:

— Prosti, cto ne smog sdelaty etovo ranyxe.

Po {\y}e{\y}o x̨eke sbejala odinoka{\y}a slezinka:

— {\Y}a ne ponima{\y}u…

— Uje nevajno, Kristina. Glavno{\y}e, cto tvo{\y} klinok tepery s tobo{\y}. T{\yi} ne poter{\ia}la {\y}evo.

Ona ul{\yi}bnulasy prizrakom svo{\y}e{\y} proxlo{\y} ul{\yi}bki, xepnula:

— Spasibo. V tom monast{\yi}re, gde pogib Gans… Tam kuzneq, cto ku{\y}et nam kinjal{\yi}. Etu ta{\y}nu on uznal, i poetomu {\y}evo ubili. Ne govori Miriam, horoxo? {\Y}e{\y} ne sto{\y}it znaty. — Kristina prervalasy, provaliva{\y}asy v zab{\yi}tye, no s usili{\y}em zakoncila: — Inace budet beda. Dl{\ia} vseh nas. {\Y}a pospl{\iu} nemnogo. Razbudix men{\ia} k utru?

— Konecno. Ni o cem ne volnu{\y}s{\ia}, — skazal {\y}a, no ne b{\yi}l uveren, cto ona men{\ia} {\y}ex̨o sl{\yi}xit.

{\Y}a sidel r{\ia}dom s ne{\y}, ox̨ux̨a{\y}a tocno taku{\y}u je zlu{\y}u bespomox̨nosty, kak kogda umirala Hanna. {\Y}a nicevo ne mog dl{\ia} ne{\y}o sdelaty.

Tolyko b{\yi}ty r{\ia}dom…

\end{document}
