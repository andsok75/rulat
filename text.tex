\documentclass[10pt]{book}
\usepackage{fontspec}
\setmainfont{Linux Libertine O}
\begin{document}

\newcommand{\e}{ë}
%\newcommand{\e}{e}
%\newcommand{\e}{é}
%\newcommand{\e}{ó}

\renewcommand{\i}{ı}
%\renewcommand{\i}{i}

\newcommand{\yi}{yı}
%\newcommand{\yi}{yi}
%\newcommand{\yi}{ǝ}

\newcommand{\ia}{ıa}
%\newcommand{\ia}{ia}
%\newcommand{\ia}{ía}
%\newcommand{\ia}{á}

\newcommand{\iu}{ıo}
%\newcommand{\iu}{ıu}
%\newcommand{\iu}{iu}
%\newcommand{\iu}{io}
%\newcommand{\iu}{ío}
%\newcommand{\iu}{íu}
%\newcommand{\iu}{ú}

\newcommand{\y}{y̆}
%\newcommand{\y}{y}

\newcommand{\Y}{Y̆}
%\newcommand{\Y}{Y}

\newcommand{\C}{C}
\renewcommand{\c}{c}

\newcommand{\X}{X̨}
\newcommand{\x}{x̨}
\newcommand{\Q}{Q}
\newcommand{\q}{q}

% % ogonek
% \newcommand{\X}{X̨}
% \newcommand{\x}{x̨}
% \newcommand{\Q}{C̨}
% \newcommand{\q}{c̨}
% 
% % retroflex hook
% \newcommand{\X}{X̢}
% \newcommand{\x}{x̢}
% \newcommand{\Q}{C̢}
% \renewcommand{\q}{c̢}
% 
% % cedilla
% \newcommand{\X}{X̧}
% \newcommand{\x}{x̧}
% \newcommand{\Q}{Ç}
% \renewcommand{\q}{ç}
% 
% % hook
% \newcommand{\X}{X̡}
% \newcommand{\x}{x̡}
% \newcommand{\Q}{C̡}
% \renewcommand{\q}{c̡}
% 
% % acute accent
% \newcommand{\X}{X̗}
% \newcommand{\x}{x̗}
% \newcommand{\Q}{C̗}
% \renewcommand{\q}{c̗}

Pugalo sidelo na kr{\yi}xe fligel{\ia}, nabl{\iu}da{\y}a rassvet. Nebo, kak eto {\c}astenyko b{\yi}va{\y}et v fevrale nedaleko ot mor{\ia}, neskolyko minut napominalo {\q}vetom morsku{\y}u rakovinu, takim nejno-rozov{\yi}m ono b{\yi}lo, a zatem srazu potusknelo, nalilosy svin{\q}om, v glubine kotorovo, kazalosy, raspleskali lu{\c}xi{\y}e vosto{\c}n{\yi}{\y}e {\c}ernila. Po{\c}ti srazu na{\c}al nakrap{\yi}vaty dojdy, i oduxevl{\e}nnovo s kr{\yi}xi kak vetrom sdulo.

Dojdy Pugalo l{\iu}bilo daje menyxe, {\c}em otsutstvi{\y}e razvle{\c}eni{\y}. A poslednih ne slu{\c}alosy uje dovolyno davno.

{\Y}a zanimalsa tem, {\c}to zakan{\c}ival ranni{\y} zavtrak — var{\e}n{\yi}{\y}e {\y}a{\y}{\q}a, jarena{\y}a sardina i {\c}esno{\c}n{\yi}{\y} sup, kotor{\yi}{\y} okazalsa por{\ia}dkom peresolen.

Vs{\e} telo {\c}esalosy, no{\c}{\y}u {\y}a otrazil {\c}et{\yi}re ataki klopov, no propustil organizovann{\yi}{\y} udar s levovo flanga i tepery m{\yi}slenno proklinal etot posto{\y}al{\yi}{\y} dvor i to, {\c}to nasto{\y}ka dl{\ia} otpugivani{\y}a nasekom{\yi}h zakon{\c}ilasy tak ne vovrem{\ia}.

Vladele{\q} zavedeni{\y}a, vid{\ia} mo{\y}u hmuru{\y}u roju, ne rexalsa prosity ras{\c}ot i toptalsa vozle kladovki, pogl{\ia}d{\yi}va{\y}a to na men{\ia}, to na obla{\c}onn{\yi}h v belo{\y}e.[39]

Stranniki v odejdah, kotor{\yi}{\y}e za vrem{\ia} putexestvi{\y}a davno uje priobreli ser{\yi}{\y} {\q}vet, s prost{\yi}mi posohami, {\y}eli postnu{\y}u gre{\c}ku. Ustavxi{\y}e ot beskone{\c}no{\y} dorogi bogomoly{\q}i xli ot m{\yi}sa Del Sur, samo{\y} {\y}ujno{\y} to{\c}ki Narar{\yi}, v Kruso.

Ne perv{\yi}{\y}e piligrim{\yi} na mo{\y}om puti. I vse kak odin tverd{\ia}t, {\c}to devo{\c}ka, jivu{\x}a{\y}a v Kruso, uzrela {\c}udo. Mol, priletel k ne{\y} kr{\yi}lat{\yi}{\y} vestnik, i kr{\yi}l{\y}a {\y}evo b{\yi}li podobn{\yi} d{\yi}mu. Soob{\x}il on, razume{\y}etsa, {\c}to Straxn{\yi}{\y} sud ne za gorami i nado poka{\y}atsa, prejde {\c}em vostrubit rog i podnimetsa iz zemli prah.

— T{\yi} verix v eto? — sprosil u men{\ia} Propovednik, kogda m{\yi} tolyko usl{\yi}xali novosty.

— {\C}to angel sletel s nebes? Posledni{\y} angel, kotorovo, kak govor{\ia}t, videli, po{\y}avl{\ia}lsa, kogda rasp{\ia}li Christa, i izvestil namestnika, deda imperatora Konstantina, o tom, {\c}to gr{\ia}dut bolyxi{\y}e peremen{\yi}. No v to{\y} istori{\y}i hot{\ia} b{\yi} b{\yi}l ser{\y}ozn{\yi}{\y} povod.

— Kone{\q} sveta, po-tvo{\y}emu, ne povod?

— {\Y}a nemnogo ustal ot kon{\q}ov sveta, Propovednik. Kajd{\yi}{\y} god vs{\ia}ki{\y}, kto s{\c}ita{\y}et, {\c}to videl angela ili sl{\yi}xal boga, za{\y}avl{\ia}{\y}et o tom, {\c}to mir na kra{\y}u gibeli, {\c}to vot-vot slu{\c}itsa treti{\y} potop, {\c}etv{\e}rta{\y}a velika{\y}a epidemi{\y}a {\y}ustirskovo pota i v kajdom gorode na meste domov grexnikov v{\yi}rastut ogned{\yi}xa{\x}i{\y}e gor{\yi}, kotor{\yi}{\y}e budut plevatsa sero{\y} i jabami.

— To {\y}esty t{\yi} ne jd{\e}x Apokalipsisa?

— Ne somneva{\y}usy, {\c}to rano ili pozdno m{\yi} dostanem nebesa i te provedut pokazatelynu{\y}u {\c}istku parxiv{\yi}h ove{\q}, no uveren, eto slu{\c}itsa ne pri mo{\y}e{\y} jizni.

— {\Y}esli {\c}estno, {\y}a toje ne ver{\iu} v etu istori{\y}u. Na ko{\y} {\c}ort, prosti Gospodi, angelu priletaty k kako{\y}-to des{\ia}tiletne{\y} dev{\c}onke, kogda v {\y}evo raspor{\ia}jeni{\y}i kuda bole{\y}e interesn{\yi}{\y}e predstaviteli {\c}elove{\c}estva?

Odin iz piligrimov, uje davno pogl{\ia}d{\yi}va{\y}u{\x}i{\y} na men{\ia}, otodvinul pustu{\y}u misku, nespexno v{\yi}ter gub{\yi} rukavom i podoxol k mo{\y}emu stolu:

— Bog v pomo{\x}. Napravl{\ia}{\y}etesy v Kruso?

{\Y}a podumal, sto{\y}it li delitsa svo{\y}imi planami s perv{\yi}m vstre{\c}n{\yi}m, rexil, {\c}to huje ne budet, i prosto kivnul.

— Nas vosemnad{\q}aty. M{\yi} mirn{\yi}{\y}e l{\iu}di, a dorogi vdoly poberej{\y}a b{\yi}va{\y}ut opasn{\yi}. Zaplatim za za{\x}itu.

Vot tolyko etovo mne ne hvatalo. Plestisy dva s polovino{\y} dn{\ia} vmeste s raspeva{\y}u{\x}imi sv{\ia}t{\yi}{\y}e gimn{\yi} bogomoly{\q}ami, kogda na loxadi mojno okazatsa v gorode uje k ve{\c}eru. U men{\ia} prosto net lixnevo vremeni.

— T{\yi} oxibsa, dobr{\yi}{\y} {\c}elovek, — l{\iu}bezno otvetil {\y}a {\y}emu. — {\Y}a ne na{\y}omnik i ne vo{\y}in.

{\C}tob{\yi} ne b{\yi}lo bolyxe voprosov, pokazal ruko{\y}atku kinjala:

— U men{\ia} svo{\y}i {\q}eli. {\Y}esli ho{\c}ex za{\x}it{\yi}, shodi na kupe{\c}eski{\y} post. Oni ob{\yi}{\c}no proda{\y}ut uslugi ohrannikov.

On, kajetsa, udovletvorilsa mo{\y}im otvetom i vernulsa k sputnikam. Propovednik pokrutil paly{\q}em u viska:

— Otdaty denygi odnomu, {\c}tob{\yi} on ohran{\ia}l vosemnad{\q}aty. Voistinu Bojyi l{\iu}di. Takovo idiotizma {\y}a ne vstre{\c}al s teh por, kak rexil ostanovity na{\y}omnikov na kr{\yi}ly{\q}e mo{\y}e{\y} {\q}erkvi. Lu{\c}xe b{\yi} sideli doma, {\c}em xl{\ia}tsa po dorogam.

— Kak t{\yi} surov s utra. — {\Y}a s usmexko{\y} otlojil lojku, okon{\c}iv zavtrak. Sledovalo rasplatitsa i otpravl{\ia}tsa v dorogu. Pod merzkim dojd{\e}m.

— {\Y}a istinu govor{\iu}, Ludwig. A uj {\y}esli doma ne siditsa i v zadni{\q}e sverbit, nau{\c}isy strel{\ia}ty iz arbaleta. Vosemnad{\q}aty {\c}elovek s arbaletami. Oni l{\iu}b{\yi}h razbo{\y}nikov udela{\y}ut.

— Otpravl{\ia}{\y}u{\x}imsa k sv{\ia}t{\yi}n{\ia}m ne pristalo nosity pri sebe {\c}to-to t{\ia}jele{\y}e bibli{\y}i.

— Vot potomu ih i razuva{\y}ut vse komu ne leny.

On {\y}e{\x}o {\c}to-to vor{\c}al po etomu povodu, no {\y}a uje napravilsa k hoz{\ia}{\y}inu posto{\y}alovo dvora, predostaviv staromu pelikanu v{\yi}skaz{\yi}vaty svo{\y}i m{\yi}sli Pugalu v namokxe{\y} solomenno{\y} xl{\ia}pe.



Narara, nesmotr{\ia} na to {\c}to eto ne sama{\y}a {\y}ujna{\y}a strana kontinenta, zimo{\y} otli{\c}alasy kuda bole{\y}e m{\ia}gkim klimatom, {\c}em tot je Leserberg ili Vitilyska. Sneg v primorskih oblast{\ia}h padal obilyno, no moroz{\yi} slu{\c}alisy redko, a k kon{\q}u fevral{\ia} dovolyno {\c}asto teplelo nastolyko, {\c}to na{\c}inal idti dojdy.

Kone{\c}no je holodn{\yi}{\y} i nepri{\y}atn{\yi}{\y}, no, {\y}esli sravnivaty s ubi{\y}stvenn{\yi}m morozom, {\c}to se{\y}{\c}as, po sluham, sobira{\y}et {\x}edru{\y}u jatvu iz putnikov v Firvaldene, — zdesy, mojno skazaty, b{\yi}l ra{\y} zemno{\y}. Vpro{\c}em, {\c}elovek nikogda ne b{\yi}va{\y}et dovolen i {\c}astenyko pen{\ia}{\y}et na sudybu. K poludn{\iu} {\y}a voznenavidel dojdy, kotor{\yi}{\y} xol ne perestava{\y}a.

Sto{\y}ilo mne podumaty o tom, {\c}to lu{\c}xe uj xol b{\yi} sneg, kak kapli obernulisy krupn{\yi}mi bel{\yi}mi hlop{\y}ami i slu{\c}ilasy ``{\c}udesna{\y}a" metely. Ona, to{\c}no soly {\Y}adovitovo mor{\ia}, ukr{\yi}la r{\yi}je-krasnu{\y}u zeml{\iu} bel{\yi}m nal{\e}tom, kotor{\yi}{\y} ne proderjalsa i {\c}asa — iz-za oblakov v{\yi}gl{\ia}nulo soln{\q}e i rastopilo vs{\iu} etu krasotu.

{\Y}a b{\yi}stro pon{\ia}l, {\c}to oxibsa v ras{\c}etah i k ve{\c}eru v Kruso ne popadu. Okajisy zeml{\ia} zamerzxe{\y}, eto b{\yi}lo b{\yi} vpolne vozmojno, no doroga razmokla, i gnaty po ne{\y} loxady ne imelo nikakovo sm{\yi}sla.

Propovednik eto toje pon{\ia}l, no pomalkival, pogl{\ia}d{\yi}val na soln{\q}e. I nakone{\q}, uje k ve{\c}eru predlojil:

— Derevenyki po puti vstre{\c}a{\y}utsa. Pereno{\c}u{\y}em v odno{\y} iz nih? Mestn{\yi}{\y}e dovolyno drujel{\iu}bn{\yi}. Vedy uje pon{\ia}tno — t{\yi} okajexsa v gorode ne ranyxe seredin{\yi} zavtraxnevo dn{\ia}.

— Predpo{\c}ita{\y}u posto{\y}al{\yi}{\y} dvor, a ne krest{\y}anski{\y} dom.

— Vse dvor{\yi} zabit{\yi} palomnikami, piligrimami i sumasxedximi. V{\c}era m{\yi} {\y}edva naxli mesto dl{\ia} no{\c}lega. {\C}to t{\yi} ime{\y}ex protiv krest{\y}an?

— {\Y}a ver{\iu} v dobrotu l{\iu}de{\y}, Propovednik, no gorazdo menyxe, {\c}em prejde. Za mo{\y}u jizny {\c}et{\yi}rejd{\yi} men{\ia} p{\yi}talisy ubity vo vrem{\ia} takih vot no{\c}legov. Odin raz, potomu {\c}to {\y}a straj, v drugo{\y} — potomu {\c}to ponravilsa mo{\y} kony, v treti{\y} — iz-za pr{\ia}jki na remne i dvuh serebr{\ia}n{\yi}h monet v koxelyke.

— A v {\c}etv{\e}rt{\yi}{\y}? — uto{\c}nil on, kogda {\y}a zamol{\c}al. — T{\yi} skazal, {\c}to {\c}et{\yi}re raza.

— Ne zna{\y}u pri{\c}in{\yi}. Tot umnik umer prejde, {\c}em uspel povedaty mne {\y}e{\y}o. V ob{\x}em, {\y}a ne slixkom jajdu nastupity na te je grabli v p{\ia}t{\yi}{\y} raz. Eto uje slixkom. Daje dl{\ia} men{\ia}.

— A {\y}esli ne budet posto{\y}al{\yi}h dvorov?

— {\C}to-nibudy priduma{\y}u. Les pod bokom.

{\Y}a obognal neskolyko grupp strannikov — ustavxih, izmojd{\e}nn{\yi}h, no vdohnovenno xaga{\y}u{\x}ih v Kruso, to{\c}no okoldovann{\yi}{\y}e.

— Vera tvorit {\c}udesa. — Propovednik s jadn{\yi}m l{\iu}bop{\yi}tstvom rassmatrival ih li{\q}a.

— Vera v slova malenyko{\y} devo{\c}ki i sluhi, kotor{\yi}{\y}e ih preumnoja{\y}ut. Skolyko etih blajenn{\yi}h ostanutsa lejaty na obo{\c}ine iz-za holoda, bolezne{\y}, pereutomleni{\y}a i vstre{\c}i s durn{\yi}mi l{\iu}dymi? Po mne, eto bolyxe napomina{\y}et sumasxestvi{\y}e, a ne veru.

— Ne soglasen s tobo{\y}. — On ostorojno potrogal prolomlenn{\yi}{\y} visok, zatem gl{\ia}nul na pale{\q}. — Vera na to i vera. {\Y}esli bo{\y}atsa za svo{\y}u jizny, to kone{\c}no je nado sidety doma. No sledu{\y}et {\c}to-to sdelaty, {\c}tob{\yi} popasty v ra{\y}. Ne vsem otkr{\yi}va{\y}utsa eti vrata i pro{\x}a{\y}utsa grehi.

— To {\y}esty, po logike, lu{\c}xe pogibnuty v puti i obresti ve{\c}no{\y}e blajenstvo na nebesah?

— A razve net?

{\Y}a poka{\c}al golovo{\y}:

— Propovednik, {\y}a kak nikto ino{\y} ver{\iu} v {\c}udesa, ad, ra{\y}, demonov, angelov, duxi i voskrexeni{\y}e Christovo. S{\iu}da mojex dobavity potop, ishod iz hagjitskih zemely, znameni{\y}a, ognenn{\yi}{\y}e dojdi i {\c}to tam {\y}e{\x}o napisano v bibli{\y}i po drugim vajn{\yi}m povodam. No {\y}a gotov pospority, kak spe{\q}ialist po duxam, {\c}to nelyz{\ia} polu{\c}ity kl{\iu}{\c} ot ra{\y}a, sdohnuv v puti ot tifa, {\y}esli t{\yi} nasluxalsa basen, kotor{\yi}{\y}e ne ime{\y}ut ni{\c}evo ob{\x}evo s vero{\y}.

— L{\iu}ba{\y}a basn{\ia} po{\y}avl{\ia}{\y}etsa po vole {\Y}evo.

— Aga. Tak mojno skazaty obo vs{\e}m. Vot eta luja toje po vole {\y}evo. I vot eta kanava zdesy ne slu{\c}a{\y}na. I von ta viseli{\q}a na perekr{\e}stke po{\y}avilasy iskl{\iu}{\c}itelyno po prikazu boga, a ne mestnovo zemlevladely{\q}a, kaznivxevo razbo{\y}nikov ili prosto kakih-to bedolag.

— Nax teologi{\c}eski{\y} spor zahodit v tupik, — zametil on. — Potomu {\c}to {\y}a ime{\y}u v rukave odin i tot je koz{\yi}ry, uklad{\yi}va{\y}u{\x}i{\y} l{\iu}bo{\y} tvo{\y} argument na obe lopatki. {\Y}emu uje bez malovo poltor{\yi} t{\yi}s{\ia}{\c}i let, no on otli{\c}no de{\y}stvu{\y}et. Ho{\c}ex usl{\yi}xaty eti volxebn{\yi}{\y}e slova?

{\Y}a pri{\x}urilsa:

— Udivi men{\ia}.

— M{\yi} prosto ne v sosto{\y}ani{\y}i posti{\c} {\Y}evo zam{\yi}slov, — nevinno izr{\e}k on. — Ibo kto m{\yi} pered Nim? I vozmojno, eta kanava s{\yi}gra{\y}et roly v {\Y}evo planah. Kak i luja. I viseli{\q}a s {\y}e{\y}o gruzom. Tolyko m{\yi} ob etom nikogda ne uzna{\y}em.

— Da, eto o{\c}eny udobno{\y}e zaklinani{\y}e. I {\y}evo mojno primenity k l{\iu}bo{\y} situa{\q}i{\y}i. K primeru, tvo{\y}a smerty b{\yi}la {\y}evo zam{\yi}slom.

On hihiknul:

— {\Y}a ni na minutu v etom ne somneva{\y}usy.

— I poetomu poro{\y} nedel{\ia}mi {\y}a sl{\yi}xu ot teb{\ia} potoki bogohulystv?

— Odno drugomu ne mexa{\y}et. {\Y}esli mo{\y}a smerty nujna {\Y}emu, to {\y}a gotov v{\yi}polnity svo{\y}e prednazna{\c}eni{\y}e.

{\Y}a sn{\ia}l s golov{\yi} kap{\iu}xon, i vlajn{\yi}{\y} veter s mor{\ia} vzyeroxil mo{\y}i otrosxi{\y}e volos{\yi}:

— I t{\yi} zna{\y}ex, kakovo ono?

— M{\yi} prosto ne v sosto{\y}ani{\y}i posti{\c} {\Y}evo zam{\yi}slov, — terpelivo povtoril star{\yi}{\y} pelikan. — B{\yi}ty mojet, On jelal, {\c}tob{\yi} {\y}a skraxival tvo{\y}o odino{\c}estvo? A zatem otpravl{\iu}sy v ra{\y}.

— T{\yi} uje mojex tuda otpravitsa. Hoty se{\y}{\c}as, — napomnil {\y}a {\y}emu.

— Poka {\y}a ne gotov. No vozvra{\x}a{\y}asy k naxe{\y} besede o vere i veru{\y}u{\x}ih. S{\c}ita{\y}u, {\c}to nevajno, naskolyko pravdiv{\yi} sluhi i devo{\c}ka, blagodar{\ia} kotoro{\y} oni po{\y}avilisy. Vajna lix vera. Daje {\y}esli u ne{\y}o net pri{\c}in{\yi}. Ibo ona — propusk v ra{\y}. Ne soglasen?

Vopros b{\yi}l obra{\x}on k dolgov{\ia}zomu Pugalu, kotoro{\y}e, to{\c}no aist, razmerenno xagalo po drugu{\y}u storonu ot mo{\y}e{\y} loxadi. To lix uhm{\yi}lynulosy.

— Nu da, — provor{\c}al Propovednik. — Kuda uj tebe o duhovn{\yi}h ve{\x}ah rassujdaty.

— Vera ne {\y}avl{\ia}{\y}etsa propuskom. T{\yi} oxiba{\y}exsa. — M{\yi} po{\c}ti dobralisy do viseli{\q}i, i {\y}a popravil palax, visevxi{\y} r{\ia}dom s sedelyn{\yi}mi sumkami, tak kak v blija{\y}xih pridorojn{\yi}h kustah mne po{\c}udilosy dvijeni{\y}e. — Krome ne{\y}o doljn{\yi} b{\yi}ty i horoxi{\y}e postupki. Otsutstvi{\y}e grehov. Slepa{\y}a vera ne pomoga{\y}et, drug Propovednik, a vredit. Eto vs{\e} ravno {\c}to neupravl{\ia}{\y}ema{\y}a kareta, nesu{\x}a{\y}asa pod gorku. Ugrobit i teh, kto sidit v ne{\y}, i teh, kto popad{\e}t pod kolesa.

Star{\yi}{\y} pelikan skrivilsa:

— {\Y}a ponima{\y}u tvo{\y}u analogi{\y}u, Ludwig. Daje prizna{\y}u, {\c}to t{\yi} prav. M{\yi}, l{\iu}di, iskaja{\y}em vs{\e}, do {\c}evo dot{\ia}giva{\y}emsa. Ili v{\yi}vora{\c}iva{\y}em naiznanku, {\c}to odno i to je. No kl{\ia}nusy krov{\y}u Christovo{\y}, tak b{\yi}ty ne doljno. Vera doljna spasaty, a ne ubivaty.

— I ne razob{\x}aty, ne stra{\x}aty, ne sudity i ne kaznity. No ot{\c}evo-to imenno tak i proishodit. Odni jgut vedym, drugi{\y}e — katzerov iz Vitilyska, tretyi — teh, kto zab{\yi}l pomolitsa pered obedom. Uveren, {\c}to pom{\yi}sl{\yi} Gospoda v etih slu{\c}a{\y}ah soverxenno ni pri {\c}em. Eto uj m{\yi} sami, voplo{\x}eni{\y}e ruk {\y}evo, dodumalisy. No vsegda gotov{\yi} spihnuty svo{\y}i ne slixkom pravedn{\yi}{\y}e postupki na {\c}uju{\y}u vol{\iu}, lixiv {\y}e{\y}o seb{\ia}. Mol, ne {\y}a srubil golovu tomu nechrist{\i}u-hagjitu, eto bog tak velel.

M{\yi} vplotnu{\y}u podyehali k viseli{\q}e — perekladine mejdu dvuh stolbov. Na ne{\y} boltalisy dva trupa. Sud{\ia} po vnexnemu vidu, vstre{\c}ali putnikov oni uje o{\c}eny davno. Poko{\y}niki v{\yi}gl{\ia}deli stoly jalko, {\c}to ne zainteresovali daje Pugalo.

— Zabavno, — izr{\e}k Propovednik s takim vidom, slovno {\y}evo zastavili proglotity tarelku jel{\c}i. — M{\yi} jiv{\e}m i m{\yi}slim, verim, jela{\y}em, l{\iu}bim i nenavidim. M{\yi} vse, sozdani{\y}a Bojyi s gor{\ia}{\c}e{\y} krov{\y}u, v kon{\q}e puti prevra{\x}a{\y}emsa vot v eto. V bezduxn{\yi}{\y} kusok m{\ia}sa na radosty {\c}erv{\ia}m i voronam.

— {\C}to eto na teb{\ia} naxlo?

On otvernulsa ot viselynikov:

— Umiraty ne straxno, Ludwig. Prosto obidno. Nikogda ne uspeva{\y}ex sdelaty vs{\e}, {\c}to hotel.

— M{\yi} ne umira{\y}em posle smerti, Propovednik. T{\yi} — tomu {\y}avno{\y}e dokazatelystvo.

— {\Y}a uznal ob etom, lix kogda men{\ia} ubili. Do tovo momenta — veril i somnevalsa. Somnevalsa i veril. Pri vseh {\c}udesah i dokazatelystvah ne vsegda mojno iskrenne b{\yi}ty ubejd{\e}nn{\yi}m do kon{\q}a, {\c}to {\y}esty jizny posle smerti.

— {\Y}e{\x}o odna {\c}elove{\c}eska{\y}a {\c}erta, — usmehnulsa {\y}a. — M{\yi} sklonn{\yi} somnevatsa daje v o{\c}evidn{\yi}h faktah. Sploxn{\yi}{\y}e protivore{\c}i{\y}a.

On neojidanno ul{\yi}bnulsa.

— Inogda t{\yi} govorix zame{\c}atelyn{\yi}{\y}e ve{\x}i, mo{\y} maly{\c}ik. V taki{\y}e minut{\yi} {\y}a uzna{\y}u {\c}to-to novo{\y}e o samom sebe, — proronil on i faly{\q}etom zapel {\q}erkovn{\yi}{\y} gimn vo slavu blagodati.



No{\c}evaty pod otkr{\yi}t{\yi}m nebom ili v kakom-nibudy zabroxennom sara{\y}e ne prixlosy. Po{\c}tova{\y}a stan{\q}i{\y}a s posto{\y}al{\yi}m dvorom okazalasy kak nelyz{\ia} kstati. I, nesmotr{\ia} na to {\c}to v krohotnom zale b{\yi}lo narodu stolyko je, skolyko v bo{\c}ke alybalandskih sel{\e}dok, svobodna{\y}a komnata naxlasy.

— Da neujeli? — izumilsa Propovednik i tknul suhim paly{\q}em v {\c}ern{\ia}vu{\y}u hoz{\ia}{\y}ku. — Sovetu{\y}u sprosity u ne{\y}o, v {\c}em zdesy podvoh. S tem koli{\c}estvom jela{\y}u{\x}ih priob{\x}itsa k sv{\ia}tomu mestu komnatu nelyz{\ia} na{\y}ti daje za florin. A zdesy svobodna{\y}a!

{\Y}a sprosil. Jen{\x}ina ne stala skr{\yi}vaty:

— Tri mes{\ia}{\q}a nazad zarezali tam odnovo putnika. Sam vinovat. Pustil {\c}ujakov, mnogo boltal, pil i soril denygami. {\Y}a i opomnitsa ne uspela, kak {\y}evo v{\yi}potroxili.

— S kakih eto por ubi{\y}stvo puga{\y}et goste{\y}?

Ona nahmurilasy, zatem rexilasy:

— Horoxo. Ne budu {\y}ulity, gospodin. Mo{\y} s{\yi}n videl vax kinjal, kogda zavodil loxady v sto{\y}lo. V{\yi} straj, a zna{\c}it, ne bo{\y}itesy prizrakov. I ne potrebu{\y}ete platu nazad.

— A vot s etovo momenta popodrobne{\y}e, — zainteresovanno poprosil {\y}a.

Ob{\yi}{\c}no prizraki i privideni{\y}a ne bole{\y}e {\c}em mif. Tak naz{\yi}va{\y}ut duxu, kotoru{\y}u vnezapno vid{\ia}t vse, komu ona jela{\y}et pokazatsa na glaza. O takom pixut v knigah, no v realynosti podobno{\y}e {\y}avleni{\y}e straji vstre{\c}a{\y}ut reje govor{\ia}{\x}evo kozla za obedenn{\yi}m stolom Pap{\yi}.

— Otpeli i zakopali, vs{\e} kak polojeno. A on, svolo{\c}, vs{\e} poko{\y}a ne da{\y}et. — {\y}e{\y}o li{\q}o stalo zl{\yi}m. — Zan{\ia}l komnatu, puga{\y}et l{\iu}de{\y}. Te uje na{\c}ali boltaty, a mne, kak ponima{\y}ete, ni k {\c}emu razgovor{\yi}. Se{\y}{\c}as narodu mnogo i mest net, a kak potok shl{\yi}net, nikto ko mne ne id{\e}t, krome pridurkov, kotor{\yi}m ohota poglazety na mertve{\q}a. Takih, kak v{\yi} ponima{\y}ete, gorazdo menyxe normalyn{\yi}h l{\iu}de{\y}.

— Kto-nibudy posle {\y}evo po{\y}avleni{\y}a zdesy umiral?

— Net.

— Bolel? Kale{\c}ilsa?

— Net, upasi Gospody. Ni{\c}evo takovo. I s dohodami poka vs{\e} horoxo. Da i ne zlo{\y} on. Prosto pugaty l{\iu}bit. Slujanki uje tuda i ne zahod{\ia}t. Komnatu ne ubirali. Izbavyte men{\ia} ot nevo, gospodin. A {\y}a besplatno pu{\x}u. I {\y}edu lu{\c}xu{\y}u, i vino. I loxadke vaxe{\y} pxeni{\q}i otborno{\y}.

— Soblaznitelyno, — bez osob{\yi}h emo{\q}i{\y} proizn{\e}s {\y}a. — Nu pokaz{\yi}va{\y}, gde u teb{\ia} ploha{\y}a komnata.

Ona okazalasy na pervom etaje, v dalynem kon{\q}e doma, s {\y}edinstvenn{\yi}m oknom i vidom na skotn{\yi}{\y} dvor.

— Nasto{\y}a{\x}i{\y} dvore{\q}. — Propovednik dal svo{\y}u kriti{\c}esku{\y}u o{\q}enku ubogomu interyeru i krovati s solomenn{\yi}m matrasom.

— Nade{\y}usy, prizrak smog napugaty ne tolyko l{\iu}de{\y}, no i klopov. — {\Y}a brosil sumku v t{\e}mn{\yi}{\y} ugol i, ne uderjavxisy, otkr{\yi}l malenyku{\y}u forto{\c}ku. Zdesy davno sledovalo provetrity.

— Vrode b{\yi} ob{\yi}{\c}n{\yi}{\y}e l{\iu}di nas videty ne doljn{\yi}. — {\Y}evo pelikanye sv{\ia}te{\y}xestvo pl{\iu}hnulsa na mo{\y}u krovaty.

— Vsegda {\y}esty iskl{\iu}{\c}eni{\y}a iz pravil.

V dvery postu{\c}ali.

— A vot i on. Kako{\y} vejliv{\yi}{\y}, — hihiknul mo{\y} sputnik.

Razume{\y}etsa, eto b{\yi}la nikaka{\y}a ne duxa, a sama hoz{\ia}{\y}ka. Ne vhod{\ia}, ona prot{\ia}nula mne {\c}isto{\y}e postelyno{\y}e belye, stara{\y}asy ne smotrety v komnatu:

— Ujinaty budete v zale ili sobraty vam zdesy, gospodin?

— V zale, — k {\y}e{\y}o {\y}avnomu obleg{\c}eni{\y}u otvetil {\y}a.

Poka {\y}a sidel v tol{\c}e{\y}e, opustoxa{\y}a tarelku, v komnate po{\y}avilsa gosty. No ne tot, kotorovo {\y}a jdal. Eto b{\yi}lo vsevo lix Pugalo. Ono bes{\q}eremonno izvleklo iz mo{\y}e{\y} ob{\y}omno{\y} sumki glavno{\y}e {\y}e{\y}o soderjimo{\y}e — t{\ia}jolu{\y}u tolstu{\y}u tetrady, pereplet{\e}nnu{\y}u v xerxavu{\y}u svinu{\y}u koju.

{\Y}a unes vs{\e}, {\c}to naxol na stole poko{\y}novo burggrafa, no lix eta ve{\x} opravdala mo{\y}i ojidani{\y}a. V mo{\y}i ruki popalo ne{\c}to vrode dnevnika, buhgaltersko{\y} knigi i {\y}ejednevnika za posledni{\y}e polgoda — {\y}evo milosty otli{\c}alsa pedanti{\c}nost{\y}u i dover{\ia}l bumage vse svo{\y}i dela.

Ote{\q} un Nomanna, k sojaleni{\y}u, ne b{\yi}l nastolyko naiven, {\c}tob{\yi} ne ispolyzovaty xifr. Posledni{\y} okazalsa slojen — izobreteni{\y}e flotoli{\y}skih bankirov. Pro{\c}itaty {\y}evo bez kl{\iu}{\c}a ne predstavl{\ia}losy vozmojn{\yi}m. Poetomu {\y}a sledu{\y}u{\x}im je utrom otpravil dnevnik {\c}erez ``Fabien Clemence i s{\yi}nov{\y}a" Gertrude, zna{\y}a, {\c}to s {\y}e{\y}o sv{\ia}z{\ia}mi i znakomstvami, v tom {\c}isle i v Riapano, gde oboja{\y}ut ne tolyko sozdavaty, no i raskr{\yi}vaty {\c}uji{\y}e sekret{\yi}, uznaty, {\c}to napisano, polu{\c}itsa gorazdo b{\yi}stre{\y}e.

Rovno {\c}erez dve nedeli {\y}a polu{\c}il tetrady obratno v drugom otdeleni{\y}i ``Fabien Clemence", nahod{\ia}{\x}emsa za sto lig ot pervovo, i sredi strani{\q} lejal kl{\iu}{\c} s pravilyno{\y} kombina{\q}i{\y}e{\y} i provo{\x}onn{\yi}{\y} trafaret, v kotor{\yi}{\y} trebovalosy podstavl{\ia}ty nujn{\yi}{\y}e bukv{\yi}.

{\Y}a na{\c}al s sam{\yi}h poslednih zapise{\y} i ne oxibsa. Uje na tretye{\y} strani{\q}e s kon{\q}a, mejdu otmetkami o v{\yi}plate jalovan{\y}a slugam i sove{\x}ani{\y}i u burgomistra, naxlosy ne{\c}to l{\iu}bop{\yi}tno{\y}e:

``Interesn{\yi}{\y} kinjal v kollek{\q}i{\y}u po proxlo{\y} dogovor{\e}nnosti. Polu{\c}en {\c}erez ``Fabien Clemence i s{\yi}nov{\y}a". Otpravitely iz Kruso. {\Q}erkovy Sv{\ia}tovo Michaela. Avans v s{\c}ot {\c}ornovo kamn{\ia}. Rasplatitsa. Pos{\yi}lku pros{\ia}t peredaty li{\c}no. Kuryer pri{\y}edet v na{\c}ale fevral{\ia}".



Propovednik, uznav, {\c}to {\y}a sobira{\y}usy v Kruso, daje rukami vsplesnul:

— Gospodi Iesuse, Ludwig! A po{\c}emu ne k hagjitam? Ili srazu k adskim vratam na vosto{\c}no{\y} okra{\y}ine mira?! Do Narar{\yi} puty neblizki{\y}, i tebe tam sovsem ne{\c}evo delaty.

— Krome kak razobratsa s tem, {\c}to slu{\c}ilosy s Kristino{\y}, po sledu kotoro{\y} {\y}a idu s samovo na{\c}ala oseni. Kto-to iz Kruso otpravil {\y}e{\y}o kinjal burggrafu. Tot, komu nujn{\yi} b{\yi}li kamni serafima. I predpolaga{\y}u, dl{\ia} tovo, {\c}tob{\yi} v{\yi}kovaty t{\e}mno{\y}e oruji{\y}e.

— Da-da! T{\e}mn{\yi}{\y} kuzne{\q} jiv{\e}t v {\q}erkvi Sv{\ia}tovo Michaela i tolyko i dela{\y}et, {\c}to jd{\e}t teb{\ia}. {\C}tob{\yi} t{\yi} pri{\y}ehal i zadal {\y}emu svo{\y}i vopros{\yi}! Tovo, kto otpravil kinjal, uje mojet tam ne b{\yi}ty!

— On tam, — s uverennost{\y}u proizn{\e}s {\y}a. — V kon{\q}e fevral{\ia} burggraf doljen b{\yi}l otpravity {\y}emu kamni.

— Nu, mojet, prejde {\c}em {\y}ehaty sotni lig, t{\yi} prosto za{\y}d{\e}x v ``Fabien Clemence" i po{\y}interesu{\y}exsa, ot kovo b{\yi}la pos{\yi}lka?

— S kako{\y} stati im otve{\c}aty? Oni ne raskr{\yi}va{\y}ut postoronnim ta{\y}n{\yi} svo{\y}ih kli{\y}entov.

— T{\yi} toje ih kli{\y}ent. I s dovolyno vnuxitelyn{\yi}m s{\c}otom.

— Eto ni{\c}evo ne men{\ia}{\y}et. Oni ne stanut riskovaty reputa{\q}i{\y}e{\y} i ot{\c}it{\yi}vatsa o {\c}ujih ta{\y}nah.

On b{\yi}l nedovolen i ne jelal {\y}ehaty na zapad. No, sobstvenno govor{\ia}, kogda b{\yi}lo ina{\c}e? Propovednik, vs{\iu} jizny provedxi{\y} v svo{\y}e{\y} derevne, nesmotr{\ia} na to {\c}to mota{\y}etsa za mno{\y} ne odin god, tak i ne priv{\yi}k k {\c}ast{\yi}m pere{\y}ezdam.

Odnako vern{\e}msa k nasto{\y}a{\x}emu. Tepery Pugalo rexilo zan{\ia}tsa {\c}teni{\y}em ili delalo vid, {\c}to {\c}ita{\y}et dnevnik burggrafa. Ono nespexno perevora{\c}ivalo strani{\q}i ser{\yi}mi kogtist{\yi}mi paly{\q}ami, naklonivxisy k samo{\y} sve{\c}e, kotora{\y}a {\y}edva ne podjigala {\y}evo xl{\ia}pu, sdelannu{\y}u iz ploho{\y} solom{\yi}.

— Kak {\y}a ponima{\y}u, teb{\ia} ne smu{\x}a{\y}et xifr.

Ono daje golov{\yi} ne povernulo.

— Interesno, {\c}to ono ho{\c}et tam na{\y}ti. — Propovednik, podperev {\x}oku, s nekotoro{\y} zavist{\y}u sledil za oduxevl{\e}nn{\yi}m.

Nelepo{\y}e Pugalo s pol{\ia} delalo to, {\c}evo ne mog selyski{\y} sv{\ia}{\x}ennik — ono prekrasno umelo {\c}itaty.

{\Y}a stal gotovitsa ko snu, kogda po{\y}avilsa nov{\yi}{\y} gosty — bledn{\yi}{\y} {\c}elovek s izurodovann{\yi}m li{\q}om i v zalito{\y} krov{\y}u odejde. Nad nim horoxenyko porabotali nojami i nanesli tako{\y}e koli{\c}estvo ran, {\c}to vporu b{\yi}lo lix pojalety {\y}evo.

V{\yi}tara{\x}iv glaza, on zaskrejetal zubami i medlenno dvinulsa ko mne.

— Ne nado{\y}elo? — s u{\c}asti{\y}em sprosil {\y}a.

On ostanovilsa kak vkopann{\yi}{\y}, posmotrev nedover{\c}ivo, i ostorojno sprosil s silyn{\yi}m litavskim ak{\q}entom:

— {\C}to, sovsem ne straxno?

— Uv{\yi}, — s sojaleni{\y}em razv{\e}l {\y}a rukami.

— I vam ne straxno? — sprosil ne{\y}izvestn{\yi}{\y} u Propovednika.

— {\Y}a m{\e}rtv, kak i t{\yi}, poludurok, — provor{\c}al tot. — Napugaty mertve{\q}a mertve{\q}om eto nado umudritsa. Kl{\ia}nusy Devo{\y} Mari{\y}e{\y}, bole{\y}e glupo{\y} zate{\y}i {\y}a {\y}e{\x}o ne vidal.

— Naverno{\y}e, sto{\y}ilo podkrastsa szadi, — probormotal tot i doveritelyno skazal mne: — Ob{\yi}{\c}no vse ubegali s voplem i {\y}edva dvery ne snosili.

— Nekotor{\yi}{\y}e l{\iu}di ne taki{\y}e, kak vse. — {\Y}a polojil na stol obnajenn{\yi}{\y} klinok.

— Bezdna! — proizn{\e}s ubit{\yi}{\y} i rvanul pro{\c}, no ne tut-to b{\yi}lo. Figura, kotoru{\y}u {\y}a kinul {\y}emu pod nogi, b{\yi}la ni{\c}uty ne huje silka, kotor{\yi}{\y}e stav{\ia}t na krolika.

On dernulsa raz, drugo{\y}, no lix zaputalsa {\y}e{\x}o silyne{\y}e.

— E{\y}, pri{\y}ately! T{\yi} ne ime{\y}ex nikakovo prava men{\ia} trogaty! — Na {\y}evo li{\q}e b{\yi}l strah. — {\Y}a ne t{\e}mn{\yi}{\y}.

— Eto t{\yi} tak s{\c}ita{\y}ex. — {\Y}a vstal, vz{\ia}vxisy za kinjal. — T{\yi} ostalsa na meste svo{\y}evo ubi{\y}stva, i ot{\c}evo-to teb{\ia} kto-to uvidel. Razume{\y}etsa, on ispugalsa. I tebe eto ponravilosy.

— Vsevo lix malenyka{\y}a xalosty, — proskulil on.

— {\C}ujo{\y} strah dobavil tebe sil. A oni dali vozmojnosty uvidety teb{\ia} {\y}e{\x}o komu-to. I t{\yi} snova napugal. I op{\ia}ty podpitalsa ujasom.

— No {\y}a… — On zatknulsa, kogda {\y}a podn{\ia}l ruku s klinkom, priz{\yi}va{\y}a {\y}evo k mol{\c}ani{\y}u.

— {\Y}a rasskaju tebe o posledstvi{\y}ah. Pitani{\y}e strahom privedet k tomu, {\c}to tvo{\y}a svetla{\y}a su{\x}nosty stanet t{\e}mno{\y}. Ne pr{\ia}mo se{\y}{\c}as. B{\yi}ty mojet, {\c}erez mes{\ia}{\q}, a mojet, i {\c}erez god — smotr{\ia} skolykih l{\iu}de{\y} t{\yi} napuga{\y}ex i kak silyno im budet straxno. No povery mne, rano ili pozdno podobno{\y}e proizo{\y}d{\e}t. Zna{\y}ex, {\c}to togda slu{\c}itsa?

— T{\yi} prid{\e}x za mno{\y}? — xopotom sprosil tot.

— {\Y}a uje prixol za tobo{\y}. I ne budu ojidaty to{\y} por{\yi}, kogda t{\yi} pererodixsa iz-za svo{\y}ih glup{\yi}h zabav i na{\c}n{\e}x ubivaty l{\iu}de{\y}. Poka pered tobo{\y} otkr{\yi}t{\yi} vrata ra{\y}a. No, kogda t{\yi} naber{\e}xsa tym{\yi}, otpravixsa ne naverh, a vniz. Zagremix v {\c}istili{\x}e. Grubo govor{\ia}, sobstvenn{\yi}mi rukami otpravix seb{\ia} tuda, kuda nikto ne ho{\c}et. Ne slixkom prekrasna{\y}a perspektiva, na mo{\y} vzgl{\ia}d. {\Y}a da{\y}u tebe v{\yi}bor. U{\y}d{\e}x sam ili mne v{\yi}polnity svo{\y}u rabotu?

— U{\y}du sam, — b{\yi}stro otvetil on. — Nikaki{\y}e xutki ne sto{\y}at ada. {\Y}a prosto snova hotel po{\c}uvstvovaty jizny.

{\Y}a razruxil figuru, uderjiva{\y}a nagotove znak. On vzdohnul, zakr{\yi}l glaza, a dalyxe slu{\c}ilosy to, {\c}to {\y}a videl uje mnogo raz. {\Y}evo siluet stal blednety, poka ne ostalosy {\y}edva zametnovo kontura. Tot na mgnoveni{\y}e zasi{\y}al solne{\c}n{\yi}m svetom, kotor{\yi}{\y} ozaril vs{\iu} komnatu, i vokrug vnovy nastupila polutyma, razgon{\ia}{\y}ema{\y}a lix sve{\c}ami na stole.

{\C}to prime{\c}atelyno, Pugalo daje golov{\yi} ne povernulo, prodolja{\y}a {\c}itaty dnevnik burggrafa.

— Nu, tepery hoz{\ia}{\y}ka posto{\y}alovo dvora to{\c}no skajet tebe spasibo. — Propovednik v{\yi}gl{\ia}del zadum{\c}iv{\yi}m, {\y}avno razm{\yi}xl{\ia}{\y}a o tom, {\c}to kogda-nibudy ne{\c}to podobno{\y}e predsto{\y}it sdelaty i {\y}emu. — Skaji, t{\yi} b{\yi} i vpravdu zabral {\y}evo kinjalom? On vedy vs{\e}-taki svetl{\yi}{\y}.

— Zabral b{\yi}. Potomu {\c}to tako{\y} svetl{\yi}{\y} b{\yi}stro stanovilsa t{\e}mn{\yi}m, a eto otnositsa k pr{\ia}mo{\y} ugroze l{\iu}d{\ia}m.

— Interesno, {\c}to on vidit se{\y}{\c}as? Raspahnut{\yi}{\y}e vrata? Sv{\ia}tovo Petera s kl{\iu}{\c}ami? Ili arhangela Michaela?

— Bo{\y}usy, ne smogu udovletvority tvo{\y}o l{\iu}bop{\yi}tstvo. Prid{\e}tsa tebe proverity samomu. Dava{\y} spaty. — I, povernuvxisy k Pugalu, dobavil: — Do{\c}ita{\y}ex, ne zabudy pogasity sve{\c}u.

I {\y}a usnul pod tihi{\y} xelest perelist{\yi}va{\y}em{\yi}h strani{\q}.



— Vot sukin s{\yi}n! — sgor{\ia}{\c}a proizn{\e}s {\y}a.

Ot dnevnika burggrafa ostalasy lix odna oblojka. Strani{\q}i b{\yi}li akkuratno v{\yi}rezan{\yi} i raskle{\y}en{\yi} po potolku. {\C}ernila na nih namokli i raspolzlisy, tak {\c}to pro{\c}itaty bolyxe ni{\c}evo b{\yi}lo nelyz{\ia}.

— {\Y}a… — proble{\y}al Propovednik, tak i ne zakon{\c}iv predlojeni{\y}e.

Vse b{\yi}lo pon{\ia}tno. Kogda oduxevl{\e}nn{\yi}{\y} eto prodelal, star{\yi}{\y} pelikan gde-to brodil, poetomu ne smog razbudity men{\ia}. {\Y}a mol{\c}a na{\c}al odevatsa. Uje rassvelo, i pora b{\yi}lo otpravl{\ia}tsa v dorogu.

— Tam soderjalosy {\c}to-to {\q}enno{\y}e? — ostorojno po{\y}interesovalsa Propovednik.

— {\Y}e{\x}o v{\c}era {\y}a b{\yi} skazal, {\c}to net. Tepery uje ne uveren. — {\Y}a zastegnul po{\y}as s kinjalom.

— Mojet, eto odna iz {\y}evo nepon{\ia}tn{\yi}h xutok? Mojet, ono prosto razvleka{\y}etsa?

— Pojivem — uvidim.

— To {\y}esty t{\yi} ni{\c}evo ne budex delaty?

— V sm{\yi}sle begaty po okrestnost{\ia}m i iskaty oduxevl{\e}nnovo, kotor{\yi}{\y} odnim {\x}el{\c}kom paly{\q}ev mojet peremestitsa na t{\yi}s{\ia}{\c}u lig, na pole, gde nahoditsa {\y}evo obolo{\c}ka? Pugalo vernetsa, ono vsegda vozvra{\x}a{\y}etsa. Mo{\y}ih planov eto nikak ne naruxalo.

— No tetrady…

— {\Y}esli {\c}estno, {\y}a sobiralsa sje{\c} {\y}e{\y}o {\y}e{\x}o neskolyko dne{\y} nazad, no ruki nikak ne dohodili. Tak {\c}to plevaty na tetrady. Kruso. {\Q}erkovy Sv{\ia}tovo Michaela. Vot mo{\y}a {\q}ely na segodn{\ia}.

Kak tolyko {\y}a okazalsa v zale, hoz{\ia}{\y}ka tut je kinulasy ko mne. V {\y}e{\y}o glazah {\c}italsa vopros.

— On bolyxe ne pobespoko{\y}it nikovo, — skazal {\y}a, i ona rass{\yi}palasy v iskrennih blagodarnost{\ia}h.

Kogda {\y}a v{\yi}xel na uli{\q}u, maly{\c}ixka tut je podvel mne loxady.

Po sravneni{\y}u s proxl{\yi}m dnem segodn{\ia} b{\yi}lo {\y}asno i o{\c}eny teplo. Trakt kone{\c}no je okazalsa zabit telegami, vsadnikami i pexehodami. Vse xli v gorod, {\c}tob{\yi} poklonitsa novo{\y} sv{\ia}t{\yi}ne i uvidety sled boso{\y} stupni, kotor{\yi}{\y} {\y}akob{\yi} angel ostavil pered domom devo{\c}ki.

Kruso — sploxn{\yi}{\y}e sten{\yi} i baxni iz jeltovo kamn{\ia}. Gorod, ranyxe b{\yi}vxi{\y} stoli{\q}e{\y} korolevstva, narodom okazalsa zaprujen ni{\c}uty ne menyxe, {\c}em doroga. U R{\yi}bn{\yi}h vorot {\y}a popal v nesusvetnu{\y}u davku. Vokrug kri{\c}ali molitv{\yi}, peli, ponosili drug druga, vizjali svinyi i orali te, kto poter{\ia}l v tol{\c}e{\y}e svo{\y}i koxelyki. To i delo melykali bel{\yi}{\y}e pla{\x}i palomnikov, {\q}vetn{\yi}{\y}e lento{\c}ki na posohah. Kaka{\y}a-to gruppa krest{\y}an nesla krest, obhod{\ia} gorodski{\y}e sten{\yi} po krugu. K nim kajdu{\y}u minutu priso{\y}edin{\ia}lisy nov{\yi}{\y}e mol{\ia}{\x}i{\y}es{\ia}, raspeva{\y}a ``Veli{\c}it duxa mo{\y}a Gospoda".

Pod kop{\yi}ta mo{\y}e{\y} loxadi brosilsa ni{\x}i{\y}, vop{\ia}, {\c}to gr{\ia}det kone{\q} sveta i {\y}a doljen poka{\y}atsa i otdaty {\y}emu vse denygi. Odnako, pon{\ia}v, {\c}to {\y}a ne otli{\c}a{\y}usy osobo{\y} nabojnost{\y}u, on tut je zab{\yi}l obo mne i pristal k dvum dorodn{\yi}m kup{\q}am, kotor{\yi}{\y}e b{\yi}li neskolyko perepugan{\yi} tem bezumi{\y}em, {\c}to pro{\y}ishodilo vokrug.

U sledu{\y}u{\x}ih vorot b{\yi}lo ni{\c}uty ne lu{\c}xe. Usilenn{\yi}{\y} otr{\ia}d straji sderjival tolpu. L{\iu}de{\y} nabralosy stolyko, {\c}to mnogi{\y}e, poter{\ia}v nadejdu probratsa v gorod segodn{\ia}, razbivali ogromn{\yi}{\y} palato{\c}n{\yi}{\y} lagery na golom pole.

— Kuda prex? — po-nararski zaoral na men{\ia} ustal{\yi}{\y} strajnik v polosatom berete. — Vali nazad!

{\Y}a pokazal {\y}emu kinjal, i men{\ia}, nesmotr{\ia} na rugany o{\c}eredi, propustili.

— {\Q}erkovy Michaela. Kak mne {\y}e{\y}o na{\y}ti? — sprosil {\y}a u soldata.

— Sprosi {\c}evo poleg{\c}e! — otmahnulsa tot. — Ih tut do {\c}erta, kak i bogomoly{\q}ev!

— Ne t{\yi} odin ne l{\iu}bix palomnikov, — hihiknul Propovednik.

Prixlosy rasspraxivaty na uli{\q}ah. Kako{\y}-to pareny so znakom gilydi{\y}i portn{\yi}h na kamzole po{\c}esal v zat{\yi}lke:

— Eto ta, kotora{\y}a vozle kolod{\q}a, {\c}to ly?

— Znal b{\yi}, ne spraxival.

— {\Y}episkopska{\y}a, na Malo{\y} Zlotinke, po puti k vnutrenne{\y} stene.

{\Y}a poblagodaril {\y}evo i napravilsa k {\q}entru goroda. Kruso do etovo {\y}a nikogda ne pose{\x}al, no rexil, {\c}to zdesy, skore{\y}e vsevo, budet tak je, kak i v drugih mestah pri prazdnestvah, {\y}armarkah, sv{\ia}t{\yi}h palomni{\c}estvah i svadybah kn{\ia}ze{\y} — vs{\e} dexevo{\y}e jilye rashvatano, i lezty tuda ne ime{\y}et nikakovo sm{\yi}sla. A vot v bogat{\yi}h ra{\y}onah, gde poro{\y} mogut za no{\c} sodraty i {\c}etverty florina, {\y}esli sovesty otsutstvu{\y}et, krovaty dl{\ia} putexestvennika vsegda na{\y}detsa.

Mo{\y} op{\yi}t men{\ia} ne obmanul. Posto{\y}al{\yi}{\y} dvor ``Pod korono{\y} kn{\ia}z{\ia}", raspolojenn{\yi}{\y} naprotiv star{\yi}h korolevskih kon{\iu}xen, {\q}enami raspugal vseh jela{\y}u{\x}ih i prinimal lix l{\iu}de{\y}, kotor{\yi}{\y}e ne o{\c}eny-to s{\c}itali denygi. Ostaviv loxady i sprosiv u hoz{\ia}{\y}ina dalyne{\y}xu{\y}u dorogu, {\y}a otpravilsa pexkom. Tak v{\yi}hodilo gorazdo b{\yi}stre{\y}e.

{\Q}erkovy — sera{\y}a gromada, stisnuta{\y}a s dvuh storon jil{\yi}mi domami tak, {\c}to predstavl{\ia}la s nimi {\y}edino{\y}e {\q}elo{\y}e i otli{\c}alasy ot nih lix xpilem, tor{\c}a{\x}im nad {\c}erepi{\c}n{\yi}mi kr{\yi}xami. Na stupenykah sideli dvo{\y}e {\c}umaz{\yi}h maly{\c}ixek let des{\ia}ti, oni bez vs{\ia}kovo entuziazma prosili milost{\yi}n{\iu}. {\Y}a podergal dvery, no bezrezulytatno, hot{\ia} sl{\yi}xal, {\c}to vnutri igra{\y}et organ.

— Zakr{\yi}to, d{\ia}de{\c}ka, — skazal mne odin.

— No tam kto-to {\y}esty.

— {\C}to s tovo? Sv{\ia}{\x}ennika-to net.

{\Y}a dostal neskolyko med{\ia}kov, kinul im v xapku.

— Spraxiva{\y}te, — stepenno pozvolil vtoro{\y} i v{\yi}ter rukavom sopliv{\yi}{\y} nos.

— Gde on i po{\c}emu zakr{\yi}ta dvery?

— Vse sv{\ia}{\x}enniki tepery vozle {\c}asovni na to{\y} storone krut{\ia}tsa. Gde devi{\q}a videla angela.

— Da ne mogla ona ni{\c}evo videty! — vozmutilsa {\y}evo tovari{\x}. — Ona ot rojdeni{\y}a slepa{\y}a!

— Vot potomu i videla, {\c}to ne videla! — zasporil tot. — Tak ote{\q} Seliko govoril! A on-to pobole, {\c}em t{\yi}, zna{\y}et!

— A po{\c}emu muz{\yi}ka igra{\y}et?

— Muz{\yi}kant repetiru{\y}et. On {\c}asto s{\iu}da prihodit. No {\q}erkovy otkro{\y}etsa tolyko posle voskresen{\y}a.

— A {\c}to budet v voskresenye?

Maly{\c}ixka mnogozna{\c}itelyno posmotrel v xapku:

— {\Y}esli uj vam leny u drugih uznavaty, gospodin, to v{\yi} nam {\y}e{\x}o med{\ia}k na hlebuxek podkinyte.

{\Y}a rassme{\y}alsa {\y}evo nahalystvu, kinul dva.

— Slujba torjestvenna{\y}a. Kardinal pri{\y}edet. Iz Riapano, govor{\ia}t. {\C}tob{\yi} provesti messu dl{\ia} uvaja{\y}em{\yi}h jitele{\y} goroda. V {\c}esty {\y}avleni{\y}a.

— Kak mne popasty v {\q}erkovy?

Maly{\c}ixki peregl{\ia}nulisy.

— Ne zna{\y}em, — otvetil tot, {\c}to v{\yi}gl{\ia}del postarxe.

— Vraty t{\yi} ne ume{\y}ex, pri{\y}ately. — Mejdu ukazatelyn{\yi}m i srednim paly{\q}em {\y}a derjal serebr{\ia}nu{\y}u monetku.

Deti naklonilisy drug k drugu, poxuxukalisy.

— Ladno, d{\ia}de{\c}ka. Provedu.

Maly{\c}ik zabral denejku, otdal {\y}e{\y}o svo{\y}emu pri{\y}atel{\iu}, kotor{\yi}{\y} tut je spr{\ia}tal sokrovi{\x}e za pazuhu.

— Idemte, d{\ia}de{\c}ka.

On otvel men{\ia} v pereulok, ogl{\ia}delsa i to{\c}no kotenok {\y}urknul v raspahnuto{\y}e sluhovo{\y}e okoxko, nahod{\ia}{\x}e{\y}es{\ia} na urovne mostovo{\y}. Nado priznatsa, tuda b{\yi} {\y}a ne prolez pri vsem jelani{\y}i.

Li{\q}o maly{\c}ixki po{\y}avilosy v okoxke:

— Idite k podvalu, von tomu. {\Y}a {\x}as dvery otkro{\y}u.

Spusk v podval toje b{\yi}l na uli{\q}e, zakr{\yi}t{\yi}{\y} stalyn{\yi}m {\x}itom. Kla{\q}nula zadvijka, {\y}a podn{\ia}l nelegki{\y} l{\iu}k, spustilsa vniz. Maly{\c}ixka provorno za{\x}elknul zamok:

— {\Y}esli d{\ia}dyka Mikely uzna{\y}et, {\c}to {\y}a snova zdesy laza{\y}u, on uxi otorvet. Dava{\y}te b{\yi}stre{\y}e, d{\ia}de{\c}ka.

Podvalyno{\y}e pome{\x}eni{\y}e pod domom, s nizkim potolkom i zat{\ia}nut{\yi}mi pautino{\y} uglami pohodilo na labirint. Lestni{\q}a v{\yi}vela nas v polut{\e}mn{\yi}{\y} koridor. Zdesy silyno pahlo kvaxeno{\y} kapusto{\y} i koxkami. Gde-to za dver{\y}u, nadr{\yi}va{\y}asy, kri{\c}al mladene{\q}. Xustr{\yi}{\y} maly{\c}ixka bejal vpered, tak {\c}to mne ostavalosy lix pospevaty za nim i ne vrezatsa golovo{\y} v v{\ia}zanki luka, svisa{\y}u{\x}i{\y}e s potolka.

{\C}ern{\yi}{\y} hod v{\yi}vel nas v malenyki{\y} vnutrenni{\y} dvor doma — gr{\ia}zn{\yi}{\y}, neuhojenn{\yi}{\y}, s pokosivxe{\y}s{\ia} golub{\ia}tne{\y} vozle zabora.

— {\C}erez zabor vam, — skazal maly{\c}ixka i, bolyxe ni{\c}evo ne ob{\y}asn{\ia}{\y}a, skr{\yi}lsa v zdani{\y}i.

{\Y}a tak i postupil, blago perebratsa {\c}erez pregradu b{\yi}lo neslojno. {\Q}erkovn{\yi}{\y} dvor okazalsa {\y}e{\x}o menyxe — tako{\y} tesn{\yi}{\y}, {\c}to napominal komnatu v kako{\y}-nibudy provin{\q}ialyno{\y} taverne. Organ prodoljal igraty, i daje tolst{\yi}{\y}e kamenn{\yi}{\y}e sten{\yi} ne mogli prigluxity {\y}evo veli{\c}estvenn{\yi}{\y}e zvuki.

Malenyka{\y}a kalitka b{\yi}la poluotkr{\yi}ta, tak {\c}to {\y}a voxel. Krome zvuka organa {\y}a sl{\yi}xal, kak nahod{\ia}{\x}i{\y}es{\ia} vnizu podsobn{\yi}{\y}e rabo{\c}i{\y}e razduva{\y}ut mehi muz{\yi}kalynovo instrumenta. Uzkimi zakutkami v{\yi}xel na balkon, otkuda otkr{\yi}valsa vid na kolonnadu, pust{\yi}{\y}e skamyi i {\y}arko-jelt{\yi}{\y} uzor na polu ot vitraje{\y}, v kotor{\yi}{\y}e svetilo soln{\q}e. Spustilsa vniz po vito{\y} lestni{\q}e, rexiv ne mexaty nevidimomu organistu, i sel na pervu{\y}u skam{\y}u.

Zakr{\yi}l glaza, sluxa{\y}a muz{\yi}ku. Ona b{\yi}la grandiozno{\y}, obyemno{\y} i, kazalosy, pronzala teb{\ia} naskvozy.

— Potr{\ia}sa{\y}u{\x}e, — prozvu{\c}al u men{\ia} nad uhom golos Propovednika. — V ko{\y}i-to veki t{\yi} dovolen, nahod{\ia}sy v {\q}erkvi.

— {\C}udesna{\y}a muz{\yi}ka, — v otvet proizn{\e}s {\y}a. — Ne pobo{\y}usy etovo slova — bojestvenna{\y}a.

— I mnitsa mne, {\c}to {\y}a sl{\yi}xu {\y}e{\y}o v perv{\yi}{\y} raz. — On b{\yi}l nemnogo raster{\ia}n. — K kako{\y} eto molitve?

— Ne ime{\y}u pon{\ia}ti{\y}a.

— Togda {\c}emu t{\yi} ul{\yi}ba{\y}exsa?

— Tomu, {\c}to mo{\y} dolgi{\y} puty okon{\c}en.

On gl{\ia}nul na men{\ia} kak na sumasxedxevo. Hm{\yi}knul i pristro{\y}ilsa na lavke, ne jela{\y}a bolyxe ni{\c}evo spraxivaty. Tak m{\yi} i sideli, poka zvuki ne stihli pod svodami.

Organist voxel v zal, i okazalosy, {\c}to eto jen{\x}ina. V rukah ona derjala stopku ispisann{\yi}h not i na hodu {\c}to-to {\c}erkala v nih grifelem, ne zame{\c}a{\y}a men{\ia}. Tak {\c}to {\y}a otli{\c}no smog {\y}e{\y}o rassmotrety. O{\c}eny malenyka{\y}a, huda{\y}a i tonenyka{\y}a, kak devo{\c}ka. Iz-pod barhatnovo bereta gilydi{\y}i muzi{\q}irovani{\y}a vo vse storon{\yi} tor{\c}ali vihrast{\yi}{\y}e {\c}ern{\yi}{\y}e volos{\yi}. Oni silyno otrosli i padali {\y}e{\y} na ple{\c}i. Milovidno{\y}e li{\q}o b{\yi}lo sosredoto{\c}eno, lob nahmuren, krasiv{\yi}{\y}e gub{\yi} sjat{\yi}, a v uglah nemnogo raskos{\yi}h vosto{\c}n{\yi}h glaz, haraktern{\yi}h dl{\ia} teh, u kovo predki jili v Iliate, po{\y}avilisy mor{\x}inki.

— Redko mojno vstretity v {\q}erkvi jen{\x}inu-muz{\yi}kanta, — gromko skazal Propovednik.

Razume{\y}etsa, skazal dl{\ia} men{\ia}, ne duma{\y}a, {\c}to kto-to drugo{\y} {\y}evo usl{\yi}xit.

No ona usl{\yi}xala i, vzdrognuv, {\y}edva ne uronila not{\yi}, po{\y}mav ih v posledni{\y} moment pokale{\c}enno{\y} ruko{\y}. Pri{\x}urivxisy, devuxka s podozreni{\y}em gl{\ia}nula na Propovednika, hotela {\c}to-to skazaty i nakone{\q} uvidela men{\ia}.

— Zdravstvu{\y}, Kristina, — negromko proizn{\e}s {\y}a.

— Privet, Sineglaz{\yi}{\y}, — otvetila ta, kovo {\y}a tak dolgo iskal.

Vo{\q}arilosy mol{\c}ani{\y}e. Porajenn{\yi}{\y} Propovednik tara{\x}ilsa na nas, kak palomnik na snizoxedxevo na {\y}evo molitv{\yi} sv{\ia}tovo.

— Tvo{\y} kony sku{\c}a{\y}et.

{\Y}ee ple{\c}i rasslabilisy, i ona vzdohnula:

— {\Y}a toje o{\c}eny sku{\c}a{\y}u po V{\y}unu. No se{\y}{\c}as {\y}emu lu{\c}xe b{\yi}ty s Miriam, {\c}em so mno{\y}. Kak t{\yi} men{\ia} naxol?

— {\C}ereda slu{\c}a{\y}noste{\y} i vezeni{\y}e. Ho{\c}u vernuty tebe ko{\y}e-{\c}to.

{\Y}a prot{\ia}nul svo{\y}e{\y} b{\yi}vxe{\y} naparni{\q}e braslet iz d{\yi}m{\c}at{\yi}h rauhtopazov. Vot tepery straj de{\y}stvitelyno b{\yi}la porajena. {\y}e{\y}o not{\yi} — muz{\yi}ka, v kotoro{\y} devuxka duxi ne {\c}a{\y}ala, — upali nam pod nogi. {\Y}a videl, kak drojat {\y}e{\y}o paly{\q}i, kogda ona zabirala svo{\y} braslet.

— Nam nado seryezno pogovority, Ludwig. — {\y}e{\y}o golos sel i zvu{\c}al hriplo, no glaz ona ne opustila.

— Imenno eto {\y}a i hotel predlojity.



Komnat{\yi}, kotor{\yi}{\y}e ona snimala, nahodilisy nad bolyxo{\y} apteko{\y}, na vtorom etaje. Vhod b{\yi}l {\c}erez torgov{\yi}{\y} zal. Sedovlas{\yi}{\y} i sedoborod{\yi}{\y} aptekary, malenyki{\y} i nelep{\yi}{\y}, posmotrel na men{\ia} poverh uveli{\c}itelyn{\yi}h stekol, zakreplenn{\yi}h u nevo na nosu, no ni{\c}evo ne skazal, vernuvxisy k vesam, na kotor{\yi}h otmer{\ia}l kako{\y}e-to kori{\c}nevo{\y}e snadobye dl{\ia} pokupatel{\ia}.

Xurxa {\y}ubko{\y}, Kristina brosila not{\yi} na komod, dostala iz nevo but{\yi}lku vina, dva bokala:

— T{\yi} vs{\e} {\y}e{\x}o pyex krasno{\y}e?

— Vrem{\ia} ot vremeni.

— Otkro{\y}. — Ona sela za stol, malenykimi paly{\q}ami zdorovo{\y} ruki perebira{\y}a gladki{\y}e d{\yi}m{\c}at{\yi}{\y}e kamni. — Zna{\c}it, t{\yi} naxol {\y}evo?

Im{\ia} ne prozvu{\c}alo, no b{\yi}lo i tak pon{\ia}tno, pro kovo ona spraxiva{\y}et. Pro Gansa.

— Da. — {\Y}a v{\yi}ta{\x}il probku iz but{\yi}lki, plesnul v bokal{\yi} vina, sel naprotiv.

— I v{\yi}jil. T{\yi} vsegda b{\yi}l vezu{\c}im, Ludwig. Vezu{\c}im, kak {\c}ert. — Ona goryko usmehnulasy. — V otli{\c}i{\y}e ot nevo.

— V{\yi} b{\yi}li vmeste?

Ona ne stala otri{\q}aty:

— Kako{\y}e-to vrem{\ia}. — Pomol{\c}ala i dobavila: — O{\c}eny kratko{\y}e vrem{\ia}. T{\yi} udivlen?

— Se{\y}{\c}as? Net. Vot kogda naxol tvo{\y} braslet u nevo — udivilsa. V{\yi} ne slixkom ladili posle tovo, kak t{\yi} podderjala ide{\y}u Miriam, {\c}to u kajdovo kn{\ia}z{\ia} doljen b{\yi}ty personalyn{\yi}{\y} straj.

— {\Y}a s{\c}itala, {\c}to politi{\c}eski eto polezno dl{\ia} Bratstva. — B{\yi}lo vidno, {\c}to {\y}e{\y} nepri{\y}atn{\yi} vospominani{\y}a. — M{\yi} s Gansom rexili vse raznoglasi{\y}a. Tebe on ne hotel govority.

— Vaxe pravo i vaxi dela, — pojal {\y}a ple{\c}ami. — Men{\ia} bolyxe interesu{\y}et, {\c}to slu{\c}ilosy v gorah.

Ona nervno krutanula stakan:

— {\C}ert {\y}evo zna{\y}et, Sineglaz{\yi}{\y}. On {\y}e{\x}o v Ardenau v{\yi}gl{\ia}del vstrevojenn{\yi}m. Govoril, {\c}to naxol ne{\c}to interesno{\y}e. Zatem {\y}evo ot{\c}itali stare{\y}xin{\yi} na sovete, t{\yi} vedy pomnix, kakovo slona oni sdelali iz to{\y} muhi?

{\Y}a kivnul.

— V ob{\x}em, on u{\y}ehal iz Alybalanda, a zatem, gde-to {\c}erez mes{\ia}{\q}, m{\yi} vstretilisy v Lise{\q}ke. On skazal, {\c}to {\y}evo jdut dela na vostoke, zval s sobo{\y}, i {\y}a po{\y}ehala. V Bude m{\yi} natknulisy na temnu{\y}u duxu, zasevxu{\y}u v kolod{\q}e. Gans toropilsa, govoril, {\c}to {\y}emu vo {\c}to b{\yi} to ni stalo nado popasty v Dor{\c}-gan-To{\y}n, poprosil men{\ia} razobratsa s problemo{\y} i dojdatsa {\y}evo. Obe{\x}al vernutsa {\c}erez poltor{\yi} nedeli.

— No t{\yi} ne dojdalasy.

— Ne stala jdaty, — ul{\yi}bnulasy ona, i {\y}a vspomnil, kako{\y} upr{\ia}mo{\y} poro{\y} stanovilasy Kristina. — Prikon{\c}ila tu tvary, vz{\ia}la deneg s burgomistra, ostavila V{\y}una v horoxe{\y} kon{\iu}xne i napravilasy sledom za nim, v gor{\yi}. No ne uspela. Kalikve{\q} na vorotah, na mo{\y}e s{\c}astye, okazalsa serdobolyn{\yi}m {\c}elovekom. Skazal, {\c}to {\y}evo brat{\y}a i Orden Pravednosti ubili straja. {\C}to {\y}a ne na{\y}du telo i mne sledu{\y}et uhodity kak mojno b{\yi}stre{\y}e.

{\Y}ee golos zadrojal, i ona po staro{\y} priv{\yi}{\c}ke prilojila pokale{\c}enn{\yi}{\y} bez{\yi}m{\ia}nn{\yi}{\y} pale{\q} klevo{\y} skule, prijala do boli, tak {\c}to pobelela koja.

— M{\yi} daje ne popro{\x}alisy. I {\y}a ne uvidela {\y}evo mogilu.

— T{\yi} poverila monahu?

— O! On b{\yi}l o{\c}eny ubeditelen. {\Y}a do sih por blagodar{\iu} {\y}evo za spaseni{\y}e.

— On m{\e}rtv, — jestko skazal {\y}a. — Za to, {\c}to predupredil teb{\ia}, {\y}evo rasp{\ia}li v led{\ia}no{\y} pe{\x}ere.

Ona lix othlebnula vina:

— Pusty na nebe {\y}evo duxe budet horoxo.

Kristina ne sprosila men{\ia}, otkuda {\y}a zna{\y}u, {\c}to kalikve{\q} m{\e}rtv, a {\y}a ne stal {\y}e{\y} rasskaz{\yi}vaty, vo {\c}to on prevratilsa posle smerti.

— {\C}to b{\yi}lo dalyxe?

— {\Y}a ne mogla mstity ubl{\iu}dkam s krasn{\yi}mi verevkami na r{\ia}sah. No mne hvatilo sil na zakonnikov. — Ul{\yi}bka u ne{\y}o b{\yi}la zlo{\y} i o{\c}eny nepri{\y}atno{\y}. {\Y}a nevolyno podumal, {\c}to Kristina {\c}em-to napomina{\y}et mne Miriam v {\y}e{\y}o ne sam{\yi}{\y}e lu{\c}xi{\y}e dni.

— I t{\yi} ubila vseh tro{\y}ih.

Ona potr{\ia}senno morgnula:

— T{\yi} i eto zna{\y}ex.

— Sl{\yi}xal, hoty oni i p{\yi}talisy skr{\yi}ty, {\c}to v gorah, ne slixkom daleko ot monast{\yi}r{\ia}, naxli dva tela.

— Verno. Tretyevo {\y}a ranila iz arbaleta. Prijala {\y}evo k kamn{\ia}m, no on pr{\yi}gnul v reku, i {\y}evo unes potok. Nade{\y}usy, on ne v{\yi}pl{\yi}l.

— V{\yi}pl{\yi}l. {\Y}emu hvatilo sil, {\c}tob{\yi} minovaty u{\x}el{\y}a i v{\yi}{\y}ti v dolin{\yi} Brobergera, k objit{\yi}m mestam.

I, vid{\ia} vopros v {\y}e{\y}o glazah, po{\y}asnil:

— {\Y}a naxol {\y}evo kosti vozle odno{\y} derevuxki. Mestn{\yi}{\y}e so{\c}li, {\c}to mertve{\q} — straj. Sobstvenno govor{\ia}, imenno poetomu {\y}a okazalsa v Dor{\c}-gan-To{\y}ne i tepery siju pered tobo{\y}.

— Straj? — nedoumenno naklonilasy ona ko mne. — Kakovo {\c}erta oni tak podumali?

— U nevo b{\yi}l kinjal Gansa.

— Prokl{\ia}tye! — Ona zakr{\yi}la li{\q}o rukami i prostonala: — Prokl{\ia}tye!

Povisla tixina, {\y}a sl{\yi}xal lix {\y}e{\y}o prer{\yi}visto{\y}e d{\yi}hani{\y}e. Kogda ona ubrala ruki, {\y}e{\y}o glaza b{\yi}li soverxenno suhimi i zl{\yi}mi.

— T{\yi} sdal kinjal v Bratstvo?

— Kone{\c}no.

Ona vzdohnula.

— Horoxo. — I, slovno ubejda{\y}a seb{\ia}, dobavila: — Da. Horoxo. Tak budet lu{\c}xe. Dalyxe {\y}a zna{\y}u, {\c}to slu{\c}ilosy. T{\yi} ne sdalsa, kak b{\yi}valo i prejde. I naxol {\y}evo?

— V led{\ia}no{\y} pe{\x}ere. Gluboko pod monast{\yi}rem.

— Kak on umer?

— Srajalsa do poslednevo i zabral s sobo{\y} neskolykih. Duma{\y}u, {\c}to usnul. Ot holoda i poteri krovi.

A {\c}to {\y}a {\y}e{\x}o mog {\y}e{\y} skazaty? {\C}to {\y}evo zakololi, slovno zver{\ia}? Kak zakololi Hartviga.

— T{\yi} pohoronil {\y}evo? — proxeptala mo{\y}a b{\yi}vxa{\y}a naparni{\q}a.

— Net. No {\y}a uveren, {\c}to tepery telo Gansa nikto ne pobespoko{\y}it.

{\Y}evo ne kosnutsa ni {\c}ervi, ni trupo{\y}ed{\yi} iz in{\yi}h su{\x}estv, ni zlo, ni svet. On nave{\c}no ostanetsa vo mrake, poka svod pe{\x}er{\yi} ne obvalitsa i ne prevratitsa v savan dl{\ia} mo{\y}evo starovo druga.

Odinoka{\y}a slezinka pokatilasy po {\y}e{\y}o {\x}eke, i Kristina pospexno, to{\c}no st{\yi}d{\ia}sy, v{\yi}terla {\y}e{\y}o t{\yi}lyno{\y} storono{\y} ladoni.

— Spasibo.

— Za {\c}to?

— Za to, {\c}to naxol {\y}evo. Za to, {\c}to rasskazal mne. Za to, {\c}to {\y}a tepery zna{\y}u.

— No po{\c}emu t{\yi} ne sdelala etovo? Stolyko let, Krista. M{\yi} vse tak dolgo {\y}evo iskali, ne sdavalisy, verili. A t{\yi} vs{\e} znala. Znala s samovo na{\c}ala, no ni {\c}erta ni{\c}evo ne skazala! Nikomu iz nas!

{\Y}a {\c}uvstvoval, kak holodn{\yi}{\y} gnev pros{\yi}pa{\y}etsa u men{\ia} v grudi. On jil tam {\y}e{\x}o s oseni, s teh por kak {\y}a pon{\ia}l, {\c}to devuxka kak-to sv{\ia}zana s Gansom i {\y}evo is{\c}eznoveni{\y}em.

— Budy mo{\y}a vol{\ia}, ni{\c}evo ne govorila b{\yi} i dalyxe.

— Po{\c}emu?

— A {\c}to b{\yi}lo b{\yi}?! {\C}to b{\yi} togda slu{\c}ilosy, Ludwig?! — kriknula ona mne v li{\q}o, razom ter{\ia}{\y}a vs{\e} svo{\y}e spoko{\y}stvi{\y}e. — Skaji mne! T{\yi} b{\yi} prin{\ia}l eto?! Otoxel b{\yi} pro{\c}?! Skazal b{\yi}: nu {\c}to podelaty, raz takova {\y}evo sudyba?! Kto iz teh, kovo m{\yi} zna{\y}em, otstupil?! Kto?!

Tepery slez{\yi} lilisy iz {\y}e{\y}o glaz neprer{\yi}vno, i ona ne stesn{\ia}lasy ih.

— {\Y}a sama otve{\c}u: nikto! T{\yi}, Gertruda, Lyvenok, Xuko, Rozi ne ostalisy b{\yi} v storone, brosilisy b{\yi} spasaty to, {\c}to uje nelyz{\ia} spasti, ili tovo huje — mstity. Kto iz nas obladal osob{\yi}m razumom des{\ia}ty let nazad? V{\yi} pogibli b{\yi}, kak i on. A {\y}esli b{\yi} vmexalisy ne m{\yi}, {\y}edini{\q}i, a {\q}elo{\y}e Bratstvo? Tolyko predstavy, Ludwig, sam{\yi}{\y} seryezn{\yi}{\y} konflikt s {\Q}erkov{\y}u za vs{\iu} naxu istori{\y}u. Nas b{\yi} sm{\ia}li i uni{\c}tojili, {\y}esli b{\yi} m{\yi} tolyko posmeli podn{\ia}ty na nih ruku!

Ona b{\yi}la prava, no {\y}a vs{\e} ravno s{\c}ital, {\c}to {\y}e{\y}o mol{\c}ani{\y}e slixkom jestoko dl{\ia} teh, kto do sih por jil nadejdo{\y}.

— Duma{\y} obo mne {\c}to ho{\c}ex, no, zaklina{\y}u, sohrani ta{\y}nu. Ne sto{\y}it nikomu znaty, {\c}to Gans naxol smerty v monast{\yi}re kalikve{\q}ev. Eto slixkom opasna{\y}a informa{\q}i{\y}a.

— T{\yi} b{\yi}la ne vprave rexaty za drugih, Kristina. Kaki{\y}e b{\yi} blagi{\y}e namereni{\y}a tobo{\y} ni dvigali.

— {\Y}a ni o {\c}em ne jale{\y}u.

{\Y}a vz{\ia}l seb{\ia} v ruki, otkinuvxisy na stul:

— Po{\c}emu {\y}evo ubili? {\C}to on hotel ot monahov?

— Ne zna{\y}u.

Ona v{\yi}derjala mo{\y} vzgl{\ia}d, no {\y}a lix vzdohnul:

— Eto loj.

— Pusty tak, — legko soglasilasy ona. — No pravda o pri{\c}inah smerti Gansa tepery ni{\c}evo ne izmenit. Vse davno zakon{\c}ilosy, Ludwig. Vse v proxlom. Ostavy {\y}evo, ina{\c}e ono prosnetsa i ubyet teb{\ia}.

— T{\yi} vedy men{\ia} zna{\y}ex. {\Y}a vs{\e} ravno dokopa{\y}usy do istin{\yi}, pusty dl{\ia} etovo potrebu{\y}etsa {\y}e{\x}o des{\ia}ty let.

— Ne s mo{\y}e{\y} pomo{\x}{\y}u. Prosti, no {\y}a ne jela{\y}u braty na seb{\ia} otvetstvennosty za tvo{\y}u smerty.

Nasta{\y}ivaty ne imelo sm{\yi}sla, i {\y}a otstupil.

— Horoxo. Zabudem o pri{\c}inah, pobudivxih Gansa otpravitsa v monast{\yi}ry. Rasskaji o tom, {\c}to b{\yi}lo dalyxe. Posle tovo kak t{\yi} razobralasy s zakonnikami.

Ona vstala, zakr{\yi}la okno, {\y}ejasy ot holoda.

— {\C}to teb{\ia} interesu{\y}et?

— {\C}ern{\yi}{\y} kinjal.

Kristina hm{\yi}knula:

— {\Y}a na{\c}ina{\y}u dumaty, {\c}to t{\yi} ne Ludwig, a d{\y}avol.

— Oble{\y} men{\ia} osv{\ia}{\x}enno{\y} vodo{\y}, {\y}esli teb{\ia} {\c}to-to smu{\x}a{\y}et, — predlojil {\y}a {\y}e{\y}.

— K sojaleni{\y}u, net pod ruko{\y}, — nevolyno ul{\yi}bnulasy ona. — T{\yi} prav. Tako{\y} kinjal u men{\ia} b{\yi}l. {\C}to t{\yi} zna{\y}ex o klinke?

— T{\yi} vladela im kako{\y}e-to vrem{\ia}, zatem {\y}evo ukrali, on ob{\y}avilsa v Xossi{\y}i i pri{\c}inil nemalo nepri{\y}atnoste{\y}, poka m{\yi} s Miriam ne razobralisy s {\y}evo vladely{\q}em.

— Kinjal u ne{\y}o?

— Uni{\c}tojen v prisutstvi{\y}i kn{\ia}ze{\y} {\Q}erkvi.

Pro vtoro{\y} klinok, tot, {\c}to prinadlejal imperatoru Konstantinu, dob{\yi}t{\yi}{\y} mno{\y} i Ranse v ta{\y}nike prejnevo Bratstva, {\y}a upominaty ne stal.

— {\y}e{\x}o {\c}to-nibudy?

— Su{\x}i{\y}e melo{\c}i, Krista. Kinjal, kotor{\yi}{\y} t{\yi} v{\yi}pustila v mir, dela{\y}et duxi t{\e}mn{\yi}mi.

Ona b{\yi}la ni{\c}uty ne udivlena. Ni kapli.

— {\Y}a rada tvo{\y}im znani{\y}am. M{\yi} sekonomim ku{\c}u vremeni, Ludwig. Mne ne potrebu{\y}etsa rasskaz{\yi}vaty tebe vs{\e} s samovo na{\c}ala i ubejdaty, {\c}to eto pravda.

— Vse daje lu{\c}xe, {\c}em {\y}a rass{\c}it{\yi}val, — razdalsa {\c}uty nasmexliv{\yi}{\y} golos u men{\ia} za spino{\y}. — Mojno srazu pristupaty k delu.

{\Y}a obernulsa i uvidel v dver{\ia}h pervovo pomo{\x}nika n{\yi}ne m{\e}rtvovo markgrafa Valentina.

Koldun Valyter, s kotor{\yi}m m{\yi} rasstalisy pri ne sam{\yi}h lu{\c}xih obsto{\y}atelystvah, s ul{\yi}bko{\y} prislonilsa k kos{\ia}ku:

— Dobrovo tebe dene{\c}ka, van Norma{\y}enn.

Rassto{\y}ani{\y}e do nevo {\y}a preodolel za odno mgnoveni{\y}e. Stul uletel v protivopolojnu{\y}u {\c}asty komnat{\yi}, a {\y}a okazalsa pered nenavistn{\yi}m koldunom. On, kajetsa, ne ojidal ot men{\ia} takih skoroste{\y}. {\Y}a uvidel, kak zast{\yi}va{\y}et ul{\yi}bka na {\y}evo li{\q}e, i perv{\yi}m je udarom kulaka slomal {\y}emu nos. Broskom povalil na pol i, ne duma{\y}a, {\c}to v l{\iu}bo{\y} moment on mojet primenity magi{\y}u, na{\c}al delaty to, o {\c}em me{\c}tal bolyxe goda.

Kristina s voplem povisla u men{\ia} na ple{\c}ah:

— Ludwig! Ostavy {\y}evo! Prekrati! Nu je!

{\C}erta s dva {\y}a sobiralsa {\y}e{\y}o sluxaty. No men{\ia} i v{\q}epivxu{\y}us{\ia} Kristu otbrosilo v storonu. Potolok neskolyko raz krutanulsa pered glazami, i {\y}a o{\x}util silynu{\y}u toxnotu. Dernulsa, p{\yi}ta{\y}asy vstaty i vernutsa k koldunu. Na etot raz {\y}a sobiralsa vospolyzovatsa ne kulakami, a kinjalom, no i tut mo{\y}a b{\yi}vxa{\y}a naparni{\q}a ne razjala paly{\q}ev, povisnuv na mne, kak laska na ohotni{\c}yem pse.

— Uspoko{\y}s{\ia}, {\c}ert teb{\ia} poderi! Stop! Hvatit! On drug! On mo{\y} drug!



Valyter to i delo trogal paly{\q}ami razbit{\yi}{\y}e gub{\yi}, na kotor{\yi}h zapeklasy krovy. Nos u nevo raspuh, lev{\yi}{\y} glaz zapl{\yi}l, no koldun ne sobiralsa jdaty polojenn{\yi}h dne{\y} do svo{\y}evo v{\yi}zdorovleni{\y}a. Sidel sebe v uglu da xeptal nagovor{\yi}.

— T{\yi} v norme? — Kristina prot{\ia}nula {\y}emu vlajnu{\y}u tr{\ia}pi{\q}u, i etot {\c}ertov ubl{\iu}dok s blagodarnost{\y}u {\y}e{\y}o prin{\ia}l.

— B{\yi}valo i huje, — prognusavil on. — K zavtraxnemu dn{\iu} zajivet.

{\Y}a hotel u nevo sprosity, {\c}to je on ne zajivil sebe xram, kotor{\yi}{\y} {\y}a ostavil, kogda kinul arbalet {\y}emu v li{\q}o, no sderjalsa.

— Shodi k Filippu. On mojet pomo{\c}.

Valyter lix skrivil gub{\yi} i tut je ob etom pojalel, tak kak na{\c}ala so{\c}itsa krovy.

— Prokl{\ia}t{\yi}{\y} deny! — rugnulsa on. — {\Y}a lu{\c}xe sam. Bez {\y}evo adskih pritirok i boltuxek. Za{\y}misy svo{\y}im vsp{\yi}ly{\c}iv{\yi}m kollego{\y}.

— {\Y}a b{\yi} tebe {\y}e{\x}o dobavil, {\y}esli b{\yi} ne ona, — mra{\c}no zametil {\y}a.

— Ohotno ver{\iu}. {\Y}a b{\yi} s radost{\y}u vskip{\ia}til tvo{\y}i mozgi, {\y}esli b{\yi} ne ona, — ozlobilsa on.

— Zatknitesy oba i sidite tiho! — vsp{\yi}lila Kristina. — Konflikt{\yi} proxlovo ostanutsa v proxlom!

{\Y}a ne sobiralsa zab{\yi}vaty zastenki markgrafa Valentina, to, kak {\y}a b{\yi}l kuklo{\y} dl{\ia} bit{\y}a, i to, kak etot sid{\ia}{\x}i{\y} v p{\ia}ti {\y}ardah ot men{\ia} hm{\yi}ry {\y}edva ne ukral kinjal Natana.

— T{\yi} ranyxe takim ne b{\yi}l… — Kristina ustalo opustilasy na stul peredo mno{\y}, perekr{\yi}v puty k koldunu.

— U nas star{\yi}{\y}e s{\c}et{\yi}.

— Zna{\y}u {\y}a o vaxih s{\c}etah. On rasskazal.

— Togda ne ponima{\y}u tvo{\y}evo udivleni{\y}a. {\Y}esli b{\yi} zdesy b{\yi}la Gertruda, ona b{\yi} uje razmazala {\y}evo po stenke.

— V{\yi}hodit, {\y}a legko otdelalsa. — Valyter vnovy pop{\yi}talsa ul{\yi}bnutsa, no vspomnil o gubah, i ul{\yi}bka prevratilasy v oskal.

Ona t{\ia}jelo vzdohnula:

— Ladno. O kinjale. Posle tovo kak {\y}a {\y}evo naxla, rexila, {\c}to zakonniki pridumali {\c}to-to svo{\y}e dl{\ia} sbora dux. No s duxami kinjal ne rabotal. {\Y}a ne smogla pon{\ia}ty, dl{\ia} {\c}evo on nujen, vozila s sobo{\y} po{\c}ti polgoda.

— Odin {\c}elovek skazal mne, {\c}to, kogda im dolgo vlade{\y}ex, na{\c}ina{\y}ut pro{\y}ishodity nepri{\y}atnosti. U teb{\ia} tako{\y}e b{\yi}lo?

— Na straje{\y} pravilo ne rasprostran{\ia}{\y}etsa, — vlez v razgovor Valyter. — Klinok nikak ne vli{\y}a{\y}et na teh, u kovo uje {\y}esty kinjal{\yi}. V ostalynom — su{\x}a{\y}a pravda. Ve{\x} dovolyno opasna{\y}a.

Kristina razdrajenno dernula ple{\c}om i prodoljila:

— {\Y}a sdala {\y}evo na hraneni{\y}e v ``Fabien Clemence i s{\yi}nov{\y}a" i, sobstvenno govor{\ia}, zab{\yi}la o nem na kako{\y}e-to koli{\c}estvo let. Vspomnila, lix kogda uvidela opisani{\y}e {\c}ernovo kamn{\ia} iz knigi, {\c}to lejala na stole u Miriam. Redki{\y} foliant, horoxi{\y}e risunki. Hagjitski{\y} {\y}a zna{\y}u dovolyno poverhnostno, no pro{\c}itannovo hvatilo, {\c}tob{\yi} pon{\ia}ty — glaz serafima dostato{\c}no redka{\y}a i {\q}enna{\y}a ve{\x}i{\q}a.

— I t{\yi} zabrala oruji{\y}e. Da{\y} dogada{\y}usy — eto slu{\c}ilosy v Barburge. I v etot je deny polu{\c}ila dva udara nojom.

Kristina peregl{\ia}nulasy s Valyterom, i tot proronil:

— Govoril {\y}a tebe, on {\y}e{\x}o tot umnik.

— Vse verno. Kak {\y}a ponima{\y}u, tebe rasskazal ob etom tot, kto pohitil klinok iz mo{\y}e{\y} sumki.

— Nu t{\yi} doljna b{\yi}ty {\y}emu blagodarna. On spas tvo{\y}u jizny, oplatil lekar{\ia} i komnatu. Kinjal ne prines {\y}emu nikakovo s{\c}ast{\y}a, i on izbavilsa ot nevo. Otdal {\c}eloveku, kotorovo m{\yi} po{\y}mali v Xossi{\y}i. Kto te l{\iu}di, {\c}to napali na teb{\ia}?

— Ne ime{\y}u pon{\ia}ti{\y}a. {\Y}a podozreva{\y}u, {\c}to oni na{\y}emniki Ordena. On, — kivok v storonu kolduna, — s{\c}ita{\y}et, {\c}to storonniki {\c}eloveka, sozdavxevo kinjal.

— Interesno, — s somneni{\y}em prot{\ia}nul {\y}a.

— {\C}to ne tak? — Ona prekrasno {\c}uvstvovala, kogda men{\ia} smu{\x}a{\y}ut fakt{\yi}.

— Na ko{\y} {\c}ert eto Ordenu? Da i kak oni voob{\x}e uznali? T{\yi} vedy ne begala po uli{\q}am i ne razmahivala takim oruji{\y}em napravo i nalevo. Tro{\y}e zakonnikov, kotor{\yi}h t{\yi} vstretila v gorah, mertv{\yi}. Kalikve{\q}i, {\y}esli b{\yi} oni znali tvo{\y}o im{\ia} ili s{\c}itali, {\c}to straj v{\yi}jila, dostali b{\yi} teb{\ia} iz-pod zemli i davno uje prikon{\c}ili. Dl{\ia} nih t{\yi} — vsevo lix bez{\yi}m{\ia}nna{\y}a jen{\x}ina, kotoru{\y}u v li{\q}o videl tolyko pogibxi{\y} monah-privratnik. M{\yi} vozvra{\x}a{\y}emsa k sam{\yi}m legkim iz mo{\y}ih voprosov: kak oni uznali tvo{\y}o im{\ia}, raz t{\yi} nikomu {\y}evo ne govorila? po{\c}emu pon{\ia}li, {\c}to kinjal u teb{\ia}? otkuda dogadalisy, v kakom otdeleni{\y}i ``Fabien Clemence i s{\yi}nov{\y}a" t{\yi} {\y}evo zaberex i v kako{\y} deny, {\y}esli napali srazu je posle etovo?

— Tvo{\y}i predpolojeni{\y}a? — Valyter b{\yi}l tak l{\iu}bezen, {\c}to pozvolil mne v{\yi}skazatsa.

— Kto napal — bez pon{\ia}ti{\y}a. O tom, kak naxli, — Kristina ostavila sled{\yi}. Zadela kolokoly{\c}ik, kotor{\yi}{\y} usl{\yi}xali ne te uxi. No ona utverjda{\y}et, {\c}to ni s kem ne govorila ni o sob{\yi}ti{\y}ah v gorah, ni o temnom kinjale.

— Eto tak, — podtverdila devuxka. — No {\y}a zadavala vopros{\yi} o glazah serafima. Spraxivala u kollek{\q}ionerov kamne{\y} i u hagjitskih torgov{\q}ev.

— Vozmojno, kto-to iskal to je samo{\y}e, {\c}to i t{\yi}, i za{\y}interesovalsa {\c}elovekom, kotor{\yi}{\y} pro{\y}avl{\ia}{\y}et l{\iu}bop{\yi}tstvo v stoly spe{\q}ifi{\c}esko{\y} oblasti.

— No bolyxe nikto ne p{\yi}talsa napasty na teb{\ia} posle tovo slu{\c}a{\y}a. — Valyter rabotal nad svo{\y}im nosom, provod{\ia} si{\y}a{\y}u{\x}imi paly{\q}ami i postepenno snima{\y}a otek.

— Kako{\y} sm{\yi}sl? {\Y}a perestala b{\yi}ty interesna. U men{\ia} bolyxe ne b{\yi}lo kinjala.

— No t{\yi} vs{\e} ravno slixkom mnogo znala, — ul{\yi}bnulsa {\y}a. — Li{\c}no {\y}a b{\yi} zaverxil delo, {\c}tob{\yi} {\c}elovek ne sozdaval problem{\yi}.

— A t{\yi} izmenilsa. — Kristina vnimatelyno posmotrela na men{\ia}, zatem neohotno kivnula. — {\Y}a b{\yi} postupila to{\c}no tak je. Raz uj t{\yi} men{\ia} raz{\yi}skal, nesmotr{\ia} na to {\c}to {\y}a skr{\yi}va{\y}usy, to i ubi{\y}{\q}i mogli. Dva goda — bolyxo{\y} srok.

Valyter smotrel na men{\ia} neotr{\yi}vno. {\Y}a znal, {\c}evo on bo{\y}itsa, i proizn{\e}s to, {\c}to uje davno sidelo u men{\ia} v golove:

— Ostavity teb{\ia} jivo{\y} mojno b{\yi}lo lix po odno{\y} pri{\c}ine — eto komu-to v{\yi}godno. K primeru, t{\yi} mojex privesti k klinku. Ili je {\y}e{\x}o kak-to pomo{\c}. Vot, dopustim, tvo{\y} drug. On vpolne mog nan{\ia}ty l{\iu}de{\y}, a zatem, kogda u nih ni{\c}evo ne v{\yi}xlo, vteretsa k tebe v doveri{\y}e i vsegda nahoditsa poblizosti.

B{\yi}vxi{\y} sluga markgrafa Valentina rassme{\y}alsa i podn{\ia}lsa so svo{\y}evo mesta:

— Pojalu{\y}, {\y}a po{\y}du shoju k Filippu. Ina{\c}e {\y}a vs{\e}-taki kovo-nibudy v samom dele ub{\y}u.

On v{\yi}xel, a {\y}a, dojdavxisy, kogda {\y}evo xagi stihnut na lestni{\q}e, vstal. Raspahnul dvery, prover{\ia}{\y}a, de{\y}stvitelyno li m{\yi} ostalisy odni.

Kristina sidela s neproni{\q}a{\y}em{\yi}m li{\q}om, no {\y}a videl, kak v {\y}e{\y}o temn{\yi}h glazah buxu{\y}et bur{\ia}.

— Kak davno t{\yi} {\y}evo zna{\y}ex?

— S teh por, kak men{\ia} {\y}edva ne ubili. Kogda {\y}a prixla v seb{\ia}, on b{\yi}l r{\ia}dom.

{\Y}a neveselo hohotnul:

— O{\c}eny udobno. I vpis{\yi}va{\y}etsa v mo{\y}u teori{\y}u. Zabotlivo okazatsa podle posteli raneno{\y} v tot moment, kogda nujno, raz uj ne udalosy polu{\c}ity klinok.

Ona ne jelala verity:

— Eto vsevo lix teori{\y}a, Sineglaz{\yi}{\y}. U teb{\ia} net nikakih dokazatelystv, vpro{\c}em, kak i u men{\ia}.

— T{\yi} ne slixkom dover{\c}iva{\y}a natura, Krista. Ot{\c}evo je poverila prohodim{\q}u?

Devuxka dopila vino, podumala:

— Krome tovo {\c}to on neskolyko raz spasal mo{\y}u jizny, Valyter o{\c}eny ubeditelen. {\Y}emu nujna pomo{\x} straja. I {\y}a ver{\iu} v {\y}evo istori{\y}u. M{\yi} sto{\y}im na grani katastrof{\yi}, Ludwig. Do propasti, v kotoro{\y} buxu{\y}et plam{\ia}, vsevo odin xag. No nikto iz l{\iu}de{\y} daje ne podozreva{\y}et ob etom.

V komnate b{\yi}lo duxno, i {\y}a rasstegnul vorot rubahi.

— Katastrof{\yi} slu{\c}a{\y}utsa {\y}ejegodno. {\Y}esli ne epidemi{\y}a {\c}um{\yi}, tak {\y}ustirski{\y} pot. {\Y}esli ne o{\c}eredna{\y}a {\y}ereti{\c}eska{\y}a sekta, risu{\y}u{\x}a{\y}a na grav{\iu}rah Papu s kozlin{\yi}mi nogami, tak vo{\y}na. {\C}elovek, sozda{\y}u{\x}i{\y} kinjal{\yi}, otravl{\ia}{\y}u{\x}i{\y}e duxi, bez somneni{\y}a, opasen. No ne slixkom li rano m{\yi} kri{\c}im ``apokalipsis!"?

— Etot nekto ruxit osnov{\yi}, Ludwig. On {\c}ertovski talantliv{\yi}{\y} master, no ispolyzu{\y}et svo{\y} talant vo zlo. To, {\c}to on dela{\y}et, nepravilyno. Valyter lovit kuzne{\q}a uje ne perv{\yi}{\y} god.

— Tvo{\y} koldun lovit ne tolyko {\y}evo, no i straje{\y}. On ubi{\y}{\q}a. Takih, kak m{\yi} s tobo{\y}.

— {\Y}a zna{\y}u.

— No osta{\y}exsa s nim?

Kristina upr{\ia}mo sjala gub{\yi}.

— I dalyxe budu.

— Nesmotr{\ia} na smerty teh, kto b{\yi}l tebe dorog?

Ona s sojaleni{\y}em opustila golovu, no otvetila tverdo:

— {\Y}esli kuzne{\q} prodoljit, umret {\y}e{\x}o bolyxe takih, kak m{\yi}. Vse Bratstvo. A Valyter i {\y}evo l{\iu}di se{\y}{\c}as {\y}edinstvenn{\yi}{\y}e, kto mojet na{\y}ti i ostanovity temnovo mastera. Vse o{\c}eny seryezno, Ludwig. Valyter pokaz{\yi}val mne star{\yi}{\y}e manuskript{\yi} vremen Konstantina. Togda su{\x}estvovalo lix dva takih klinka. Govor{\ia}t, ih dostavili s vostoka, s samo{\y} grani{\q}i objit{\yi}h zemely. Zna{\y}ex, po{\c}emu imperator sozdal straje{\y}? Iz-za prokl{\ia}t{\yi}h kinjalov. On jelal jity ve{\c}no, a dl{\ia} etovo {\y}emu trebovalisy duxi.

{\Y}a kivnul:

— Uje dumal ob etom. S pomo{\x}{\y}u {\c}ern{\yi}h klinkov on ubival l{\iu}de{\y}, temnil ih duxi. A zatem zabiral svetl{\yi}m, dobavl{\ia}{\y}a sebe jizny.

— {\Y}a ne zna{\y}u, kogda eto na{\c}alosy. Po{\c}ti vse svidetelystva tovo vremeni uni{\c}tojen{\yi}. No sob{\yi}ti{\y}a sv{\ia}z{\yi}va{\y}u s tem momentom, kogda iz zemely hagjitov na nax materik pri{\y}ehala sem{\y}a svetl{\yi}h kuzne{\q}ov. Kak govor{\ia}t legend{\yi}, oni — potomki odnovo iz u{\c}enikov Christa. Oni stali kovaty kinjal{\yi} s sapfirami, no net ni odnovo podtverjdeni{\y}a, {\c}to temn{\yi}{\y}e klinki — ih ruk delo. {\Q}erkovy vz{\ia}la masterov pod svo{\y}u za{\x}itu i na prot{\ia}jeni{\y}i mnogih pokoleni{\y} ih oberegala. Imperi{\y}a Konstantina rosla, kak i {\y}evo vlasty. A jizny dlilasy i dlilasy. No temn{\yi}{\y}e kinjal{\yi} privlekali na materik temn{\yi}{\y}e duxi. Razume{\y}etsa, te zarojdalisy i ranyxe. Grehi i prestupleni{\y}a nikto ne otmen{\ia}l. No, po utverjdeni{\y}am istorikov, ranyxe ih b{\yi}lo gorazdo menyxe, {\c}em posle tovo, kak imperator stal obman{\yi}vaty smerty. Poetomu {\y}emu i potrebovalisy taki{\y}e, kak m{\yi}, — o{\c}i{\x}aty zemli ot {\y}evo oxibok.

Ni{\c}evo udivitelynovo {\y}a dl{\ia} seb{\ia} ne uznal:

— Konstantin davno prevratilsa v prah. Kinjal{\yi} {\y}emu ne pomogli. Ne pomogut i tomu, kto dela{\y}et ih se{\y}{\c}as. {\C}evo t{\yi} bo{\y}ixsa? Naxestvi{\y}a zl{\yi}h su{\x}noste{\y}? M{\yi} spravimsa. Voln{\yi} temn{\yi}h dux zahlest{\yi}vali stran{\yi} i ranyxe, no Bratstvo vsegda ih pobejdalo.

— {\Y}a ne etovo opasa{\y}usy, Ludwig. Men{\ia} puga{\y}et ne{\c}to ino{\y}e. T{\yi} zna{\y}ex, {\c}to Konstantin otpravil vosemy ekspedi{\q}i{\y} na vostok, nade{\y}asy razdob{\yi}ty {\y}e{\x}o podobnovo oruji{\y}a?

— Zapasliv{\yi}{\y} sukin s{\yi}n, — nevolyno voshitilsa {\y}a. — U nevo vedy ne polu{\c}ilosy?

— Nikto ne vernulsa s kra{\y}a objit{\yi}h zemely. No Konstantin prodoljal iskaty i s{\c}ital, {\c}to {\y}esli sobraty des{\ia}ty temn{\yi}h klinkov, to oni stanut kl{\iu}{\c}om… — Ona sdelala pauzu, vnimatelyno nabl{\iu}da{\y}a za mo{\y}im li{\q}om. — Kl{\iu}{\c}om dl{\ia} tovo, {\c}tob{\yi} otkr{\yi}ty adski{\y}e vrata.

Mne potrebovalosy neskolyko sekund, {\c}tob{\yi} perevarity {\y}e{\y}o slova i rassme{\y}atsa:

— Potr{\ia}sa{\y}u{\x}e! Eto Valyter tebe skazal? I t{\yi} {\y}emu verix?!

— Ver{\iu}, Ludwig.

— Otkr{\yi}ty dorogu v ad. Tak ne b{\yi}va{\y}et.

— A b{\yi}va{\y}et, {\c}to kinjal prevra{\x}a{\y}et {\c}istu{\y}u duxu, na kotoro{\y} net grehov, v temnu{\y}u su{\x}nosty? — privela Kristina argument. — Skaji {\y}a tebe tako{\y}e god nazad, t{\yi} b{\yi} mne poveril? Ili vot tak je sme{\y}alsa?

— Ne poveril b{\yi}, — prixlosy priznaty mne. — Zna{\c}it, des{\ia}ty {\c}ern{\yi}h kinjalov otkro{\y}ut vrata v ad? Skaji, pojalu{\y}sta, kak Valyter otz{\yi}va{\y}etsa ob umstvenn{\yi}h sposobnost{\ia}h velikovo Konstantina? Na ko{\y} {\c}ert tomu iskaty stoly slojn{\yi}{\y} sposob samoubi{\y}stva? {\Y}esli gde-nibudy v Fringbou po{\y}avitsa predstavitelystvo ada, to ploho budet ne v odnom gorode, a vo mnogih stranah. Legion{\yi} demonov, sukkubov, {\c}erte{\y}, adsko{\y}e plam{\ia}, sera s nebes i pro{\c}i{\y}e ve{\x}i. Ne s{\c}ita{\y}a gibeli t{\yi}s{\ia}{\c} l{\iu}de{\y}. Eto nemnogo nerazumno. Daje dl{\ia} Konstantina. Ne nahodix?

— Ob{\ia}zatelyno iskaty logiku, Ludwig?

— {\Y}esli ho{\c}ex pon{\ia}ty motiv{\yi} drugovo {\c}eloveka? Da. Ob{\ia}zatelyno. Osobenno kogda ne verix na slovo koldunu, ohotivxemus{\ia} za klinkami straje{\y}.

— Ad toje mojet daty silu i vlasty. I Konstantin s{\c}ital, {\c}to, raz etovo ne da{\y}ut {\y}emu nebesa, nesmotr{\ia} na to {\c}to on prin{\ia}l novu{\y}u religi{\y}u, otkazavxisy ot {\y}az{\yi}{\c}eskih bogov, sledu{\y}et zakl{\iu}{\c}ity inu{\y}u sdelku. Sozdaty prohod dl{\ia} teh, komu popasty v nax mir ne tak-to prosto. On s{\c}ital, {\c}to priobretet gorazdo bolyxe, {\c}em poter{\ia}{\y}et.

{\Y}a lix razvel rukami:

— I o takom otkroveni{\y}i zna{\y}et tolyko nax ob{\x}i{\y} drug?

— Net. V Riapano eto toje izvestno. Poetomu dva klinka Konstantina uni{\c}tojen{\yi} uje o{\c}eny davno.

Kone{\c}no. Tolyko odin. Vtoro{\y} lejit v sumke naxe{\y} ob{\x}e{\y} u{\c}itelyni{\q}i.

No skazal {\y}a sovsem drugo{\y}e:

— Polojim, vs{\e} tak, kak t{\yi} govorix. Poradu{\y}emsa, {\c}to u Konstantina ni{\c}evo ne v{\yi}xlo. No tepery v mire po{\y}avilsa {\y}e{\x}o odin bezume{\q}, kotoromu ne k {\c}emu prilojity ruki, poetomu on sozda{\y}et oruji{\y}e, bole{\y}e opasno{\y}e, {\c}em hagjitska{\y}a pes{\c}ana{\y}a kobra. No ne hranit {\y}evo. I ne sobira{\y}et, a v{\yi}bras{\yi}va{\y}et v narod. Ina{\c}e b{\yi} m{\yi} s tobo{\y} ne uvideli ni odnovo takovo klinka. Sledovatelyno, on {\y}avno ne jela{\y}et otkr{\yi}vaty nikakih mifi{\c}eskih vrat. Tak?

— Net. Ne tak. M{\yi} podozreva{\y}em, {\c}to tot, {\c}to b{\yi}l u men{\ia}, okazalsa vsevo lix probn{\yi}m ekzempl{\ia}rom. {\Y}emu nado b{\yi}lo uznaty, rabota{\y}et li kinjal.

— I poetomu oruji{\y}e kakim-to obrazom po{\y}avilosy u predstavitel{\ia} Ordena? — {\Y}a b{\yi}l polon skepti{\q}izma.

— Po{\c}emu b{\yi} i net? Oni spe{\q}ialist{\yi} v takih voprosah. Raz prover{\ia}{\y}ut naxi klinki, vid{\ia}t istori{\y}u sobrann{\yi}h dux, to, vozmojno, on nade{\y}alsa, {\c}to dadut o{\q}enku i {\y}evo rabote.

— {\Y}esty dva ``no". {\Y}a ne l{\iu}bl{\iu} Orden, no delo oni zna{\y}ut. Po{\y}avisy sredi nih podobn{\yi}{\y} {\c}elovek, oni {\y}avno b{\yi} ne okaz{\yi}vali {\y}emu uslugi, a pota{\x}ili k sebe v podval{\yi}. Ili je srazu prikon{\c}ili.

— A mojet, kuzne{\q} zakl{\iu}{\c}il sdelku tolyko s odnim iz nih. S sam{\yi}m ne{\c}istoplotn{\yi}m, — vesko vozrazila ona. — A tvo{\y}o vtoro{\y}e ``no"?

— {\y}e{\x}o odin t{\e}mn{\yi}{\y} kinjal putexestvoval po miru bez svo{\y}evo sozdatel{\ia}. Klinok naxol odin inkvizitor v sumke gon{\q}a, oderjimovo besom. Gone{\q} umer, tak ni{\c}evo i ne uspev rasskazaty. A klinok kliriki uni{\c}tojili.

— T{\yi} uveren v eto{\y} informa{\q}i{\y}i?

— {\Y}a vmeste s Gertrudo{\y} videl oblomki oruji{\y}a u kardinala di Travinno. V tvo{\y}e{\y} teori{\y}i, {\c}to kuzne{\q} rexil proverity, rabota{\y}et li {\y}evo tvoreni{\y}e, togda kak ostalyn{\yi}{\y}e klinki on derjit pri sebe, {\y}esty nekotora{\y}a nesoglasovannosty. Proverka kinjala — zvu{\c}it kra{\y}ne nat{\ia}nuto. On mog eto sdelaty i sam, raz ume{\y}et sozdavaty taki{\y}e ve{\x}i. {\Y}a duma{\y}u, {\y}edinstvenna{\y}a pri{\c}ina, po{\c}emu kuzne{\q} mog otdaty t{\e}mn{\yi}{\y} klinok komu-to iz Ordena, — eto plata. Plata za pomo{\x} i sotrudni{\c}estvo.

No Kristina s{\c}itala ina{\c}e:

— Otn{\iu}dy. Teori{\y}a lix ukrepilasy s tvo{\y}im rasskazom. Poduma{\y} sam. Pervo{\y}e oruji{\y}e ne doxlo do adresata. {\Y}evo perehvatili i uni{\c}tojili kliriki. Poetomu po{\y}avilsa tot, vtoro{\y}, v itoge popavxi{\y} mne v ruki.

Versi{\y}a zvu{\c}ala {\c}uty bole{\y}e skladno, {\c}em pred{\yi}du{\x}a{\y}a. No ne namnovo. {\Y}a pomnil, kak v Riapano govorili o tom, {\c}to ote{\q} Mart naxol klinok tri goda nazad. A Kristina zapolu{\c}ila svo{\y} na semy let ranyxe, sledovatelyno, {\y}e{\y}o klinok nikak ne mog b{\yi}ty vtor{\yi}m.

— Valyter tak uveren v podobnom razviti{\y}i sob{\yi}ti{\y}? — Po mo{\y}emu tonu b{\yi}lo pon{\ia}tno, naskolyko silyno {\y}a ``{\q}en{\iu}" mneni{\y}e kolduna.

Ona podn{\ia}la vverh ladoni:

— Sluxa{\y}. On neob{\yi}{\c}n{\yi}{\y} {\c}elovek. I jestoki{\y}. V drugo{\y}e vrem{\ia} {\y}a b{\yi} ubila {\y}evo ne zadum{\yi}va{\y}asy. Za vs{\e} zlo, {\c}to on pri{\c}inil Bratstvu. Da net! K {\c}ertu Bratstvo! Za vs{\e} to, {\c}to on sdelal tebe. No, kak {\y}a govorila, situa{\q}i{\y}a o{\c}eny silyno izmenilasy. {\Y}a mnogo{\y}e uznala za paru poslednih let, i mo{\y}e otnoxeni{\y}e k jizni perevernulosy. On — menyxe{\y}e zlo i mojet spasti vseh nas.

— Zlo ne mojet b{\yi}ty malenykim ili bolyxim. Zlo osta{\y}etsa zlom. {\Y}a podhoju k slu{\c}ivxemus{\ia} bez emo{\q}i{\y}. Vo vs{\ia}kom slu{\c}a{\y}e, se{\y}{\c}as. On ne smog otsledity zakonnikov, no naxol teb{\ia}. Ugada{\y}, kuda spexil gone{\q}, ubit{\yi}{\y} inkvizitorom? V zamok Latka, vladely{\q}em kotorovo b{\yi}l markgraf Valentin, a {\y}emu slujil tvo{\y} razl{\iu}bezn{\yi}{\y} koldun. Vse ukaz{\yi}va{\y}et na to, {\c}to on iskal t{\e}mn{\yi}{\y} kinjal.

— {\Y}a toje {\y}evo i{\x}u, kak i drugi{\y}e l{\iu}di Valytera. M{\yi} nade{\y}emsa, {\c}to artefakt privedet nas k kuzne{\q}u. — Ona podalasy vpered, nakr{\yi}la mo{\y}u ruku svo{\y}e{\y}, vkrad{\c}ivo skazav: — Sluxa{\y}. Skore{\y}e vsevo, t{\yi} prav. Za tem napadeni{\y}em de{\y}stvitelyno mog sto{\y}aty on. Slixkom mnogo sovpadeni{\y}. No eto ni{\c}evo ne men{\ia}{\y}et, Ludwig. {\Y}a nujna {\y}emu, a koldun nujen mne. U nas odna {\q}ely. I drug bez druga m{\yi} ne obo{\y}demsa. {\Y}a mnogim pojertvovala radi tovo, {\c}tob{\yi} pop{\yi}tatsa na{\y}ti temnovo mastera. Slixkom mnogim. I otstupaty se{\y}{\c}as… Povery, {\y}a prosto ne mogu tak postupity.

— On opasen, Kristina.

— {\Y}a eto zna{\y}u lu{\c}xe, {\c}em t{\yi}. U nevo t{\yi}s{\ia}{\c}a i odin nedostatok, no daje tako{\y} {\c}elovek, kak on, mojet spasti nax mir.

Eto b{\yi}lo tak smexno sl{\yi}xaty. Valyter — spasitely {\c}elove{\c}estva. Po mne, tak eto mir sledu{\y}et izbavl{\ia}ty ot nevo.

— T{\yi} is{\c}ezla i ne podavala o sebe veste{\y} po{\c}ti god. — {\Y}a smenil temu. — M{\yi} volnovalisy za teb{\ia}.

Ona otvela glaza, skazav tihim golosom:

— Prosti. U men{\ia} ne b{\yi}lo v{\yi}bora. Proxlo{\y} vesno{\y} m{\yi} s Valyterom vlipli v nepri{\y}atnosti, kogda sbili so sleda kuzne{\q}a klirikov. Uverena, v otli{\c}i{\y}e ot nas oni ne hot{\ia}t {\y}evo ubivaty. Po{\y}mav kuzne{\q}a, Riapano polu{\c}it v svo{\y}i ruki ogromnu{\y}u vlasty. Poetomu t{\yi} ponima{\y}ex, kak vajno nam na{\y}ti {\y}evo perv{\yi}mi?

— Zna{\c}it, v{\yi} hotite ubity zagado{\c}novo mastera?

Ona gor{\ia}{\c}o kivnula:

— Sam{\yi}m b{\yi}str{\yi}m sposobom iz vseh vozmojn{\yi}h. {\C}tob{\yi} nikto ne uznal {\y}evo sekretov. Oni doljn{\yi} umerety vmeste s nim.

— A sv{\ia}{\x}enniki? Vdrug v{\yi} oxiba{\y}etesy, i u nih taka{\y}a je {\q}ely, kak i u vas, — ubity {\y}evo.

— Ne oxiba{\y}emsa, — bezapell{\ia}{\q}ionno za{\y}avila ona. — Kak tolyko oni pon{\ia}li, {\c}to m{\yi} toje i{\x}em {\y}evo, poslali za nami svo{\y}ih ubi{\y}{\q}. Povery, Ludwig, eto b{\yi}lo straxno. P{\ia}tero iz naxevo otr{\ia}da pogibli. M{\yi} nasilu uxli i vot uje kotor{\yi}{\y} mes{\ia}{\q} skr{\yi}va{\y}emsa. I {\y}a ne mogu vernutsa v Ardenau. Daje pisymo napisaty komu-libo iz straje{\y} ne mogu. Ono podstavit pod udar l{\iu}bovo.

— Organist v {\q}erkvi, — ul{\yi}bnulsa {\y}a. — T{\yi} ne slixkom-to horoxo skr{\yi}va{\y}exsa.

— Lu{\c}xe pr{\ia}tatsa ot sobaki v {\y}e{\y}o je budke. Tam ona budet iskaty v posledn{\iu}{\y}u o{\c}eredy. Nikto ne smotrit na muz{\yi}kantov.

— Ubegaty ve{\c}no ne polu{\c}itsa.

— {\Y}a i ne stanu. Nujno lix uni{\c}tojity zlo. Vse ostalyno{\y}e nevajno.

{\Y}a videl, {\c}to ona oderjima ide{\y}e{\y} na{\y}ti {\c}eloveka, ku{\y}u{\x}evo temn{\yi}{\y}e kinjal{\yi}, to{\c}no tak je, kak Miriam vot uje vek ne da{\y}ut poko{\y}a po{\y}iski kuzne{\q}a, sozda{\y}u{\x}evo klinki dl{\ia} Bratstva. {\Y}a ponimal, {\c}to ne otgovor{\iu} {\y}e{\y}e, {\c}to spority bessm{\yi}slenno i to, {\c}to, pri{\y}ehav s{\iu}da {\c}erez neskolyko stran, {\y}a uznal, {\c}to ona jiva i o pro{\y}izoxedxem s ne{\y} i Gansom, eto i {\y}esty vs{\e}, {\c}evo {\y}a dostig.

— Viju, {\c}to jelani{\y}e spasti mir veliko.

— Mir? — Ona izognula brovy. — Plevaty {\y}a hotela na nevo. {\Y}a spasa{\y}u ne mir, a Bratstvo. Za{\x}i{\x}a{\y}u {\y}evo v meru otpu{\x}enn{\yi}h sil i umeni{\y}a.

— Tolyko v Bratstve ob etom ne podozreva{\y}ut.

— I horoxo. Menyxe problem i mne, i im.

— M{\yi} su{\x}estvu{\y}em po{\c}ti poltor{\yi} t{\yi}s{\ia}{\c}i let. U nas slu{\c}alisy razn{\yi}{\y}e nepri{\y}atnosti, no Bratstvo vsegda v{\yi}jivalo i ostavalosy na nogah. Ostanetsa i vpredy. Tebe neza{\c}em sklad{\yi}vaty golovu lix radi nepodtverjdenn{\yi}h slov kolduna. Po{\y}ehali v Ardenau. Pr{\ia}mo se{\y}{\c}as. Bratstvo dogovoritsa nas{\c}et teb{\ia} s Riapano. Gertruda pomojet. A di Travinno tolyko poradu{\y}etsa informa{\q}i{\y}i, kotora{\y}a tebe izvestna. M{\yi} za{\x}itim teb{\ia}.

Kristina grustno rassme{\y}alasy:

— Mne otradno znaty, {\c}to t{\yi} do sih por p{\yi}ta{\y}exsa spasti mo{\y}u golovu, Ludwig. {\Y}a b{\yi} o{\c}eny hotela, {\c}tob{\yi} vs{\e} b{\yi}lo kak dvenad{\q}aty let nazad, kogda m{\yi} ple{\c}om k ple{\c}u otrajali natisk temn{\yi}h dux. No mne uje kajetsa, {\c}to naxa {\y}unosty ne bole{\y}e {\c}em mif, kotor{\yi}{\y} {\y}a sama sebe pridumala. A se{\y}{\c}as vokrug men{\ia} realynosty, i ona o{\c}eny straxna. T{\yi} prosto poka ne mojex o{\q}enity tovo ujasa, kotor{\yi}{\y} isp{\yi}t{\yi}va{\y}u {\y}a, pon{\ia}ty vse{\y} seryeznosti problem{\yi}. I soverxa{\y}ex tu je samu{\y}u oxibku, kak togda s tem kartografom.

V golove u men{\ia} trevojno zv{\ia}knulo.

— T{\yi} kone{\c}no je vs{\e} zna{\y}ex.

— Zna{\y}u. Vedy ispravl{\ia}ty tvo{\y}i oxibki prixlosy mne.

{\Y}a pri{\x}urilsa:

— Zna{\c}it, vot kto ubil {\y}evo.

Ona daje ne p{\yi}talasy otri{\q}aty:

— A t{\yi} ostavil mne v{\yi}bor, kogda pro{\y}avil jalosty? Tebe nado b{\yi}lo privezti {\y}evo v Bratstvo, a ne otpuskaty na vse {\c}et{\yi}re storon{\yi}. Togda b{\yi} Miriam ne prosila men{\ia} spasaty situa{\q}i{\y}u!

Tepery ponima{\y}u, po{\c}emu Gertruda ni{\c}evo mne ne rasskazala. {\Y}a poka{\c}al golovo{\y}:

— T{\yi} govorila, {\c}to {\y}a izmenilsa. T{\yi} izmenilasy ne menyxe men{\ia}. I {\y}a sojale{\y}u ob etom. Prejn{\ia}{\y}a Kristina nikogda b{\yi} ne stala ubi{\y}{\q}e{\y} na pobeguxkah u magistrov.

{\Y}ee li{\q}o iskazilosy ot obid{\yi}, i ona proxipela:

— T{\yi} de{\y}stvitelyno tak i ne pon{\ia}l, {\c}to togda pro{\y}izoxlo?! Iz-za {\c}evo oni sto{\y}ali na uxah?

— Pon{\ia}l. Ispugalisy novovo messi{\y}i i tovo, {\c}to on mojet nau{\c}ity drugih l{\iu}de{\y} snimaty grehi s l{\iu}dskih dux. V perspektive Bratstvo stalo b{\yi} nikomu ne nujno.

Ona dvajd{\yi} bezzvu{\c}no hlopnula v ladoxi:

— Potr{\ia}sa{\y}u{\x}e! T{\yi} uvidel m{\yi}x, no ne zametil koxku. Nikto iz teh, kto b{\yi}l v kurse situa{\q}i{\y}i, ne bo{\y}alsa dalekovo budu{\x}evo. M{\yi} opasalisy nasto{\y}a{\x}evo. A ono takovo: kartograf po kako{\y}-to nasmexke sudyb{\yi} mog o{\c}i{\x}aty duxi straje{\y}. No on zabiral ne tolyko naxi grehi, no i nax dar. M{\yi} stanovilisy ob{\yi}{\c}n{\yi}mi l{\iu}dymi, takimi, kak t{\yi}s{\ia}{\c}i drugih ob{\yi}vatele{\y}, Ludwig. Lixivxisy dara, m{\yi} ne mogli delaty svo{\y}u rabotu. A tepery tolyko predstavy, {\c}to b{\yi} b{\yi}lo, {\y}esli b{\yi} {\y}evo, k primeru, zahvatil Orden? I ispolyzoval protiv nas. A {\y}esli b{\yi} kartograf nau{\c}il kovo-to i po{\y}avilosy neskolyko takih l{\iu}de{\y}? Des{\ia}tok? Sotn{\ia}? Armi{\y}a! M{\yi} b{\yi}li b{\yi} uni{\c}tojen{\yi}, slovno gorod, v kotor{\yi}{\y} popal {\c}elovek, zarajenn{\yi}{\y} {\y}ustirskim potom.

— I mnogih li straje{\y} Hartvig lixil ih rabot{\yi}?

— Slava bogu — ni odnovo.

— Togda prosti, no tvo{\y}i slova ne bole{\y}e {\c}em nelepa{\y}a fantazi{\y}a.

— Odna taka{\y}a fantazi{\y}a slujila poslednemu potomku imperatora Konstantina. I kogda Bratstvo ne vernulo {\y}emu kinjal, na kotor{\yi}{\y} pretendoval koroly Progansu, v delo vstupil tako{\y} je, kak tvo{\y} Hartvig. {\C}etvero magistrov i opomnitsa ne uspeli, kak lixilisy svo{\y}evo dara. Togda {\y}evo ubili vmeste s korolem, i zavertelasy vs{\ia} eta kaxa. Tepery, pomn{\ia} o proxlom, Bratstvo ne stalo jdaty na{\c}ala epidemi{\y}i, a uni{\c}tojilo bolynovo do {\y}evo po{\y}avleni{\y}a v gorode. — Ona s v{\yi}zovom posmotrela na men{\ia}. — S{\c}ita{\y}ex, {\c}to v Ardenau oxiblisy?

{\Y}a vzdohnul, vstal iz-za stola, tak i ne pritronuvxisy k bokalu vina:

— Ne ime{\y}et sm{\yi}sla ter{\ia}ty vrem{\ia} na spor{\yi}. Magistr{\yi} uvideli opasnosty. Realynu{\y}u ili mnimu{\y}u, ne mne sudity. No {\c}elovek, kotor{\yi}{\y} mog sdelaty mir {\c}uty lu{\c}xe, mertv. I mne jaly tovo, {\c}to tepery nikogda ne slu{\c}itsa.

— Bo{\y}usy, {\y}a ne smogu teb{\ia} pon{\ia}ty, Ludwig. M{\yi} stali slixkom razn{\yi}mi, — s grust{\y}u skazala ona. — Kogda t{\yi} u{\y}ezja{\y}ex iz goroda?

— {\y}e{\x}o ne rexil, — {\c}estno otvetil {\y}a.

— Ostanysa. Mne nujna tvo{\y}a pomo{\x}, i t{\yi} {\y}edinstvenn{\yi}{\y}, komu {\y}a mogu dover{\ia}ty zdesy. Obe{\x}a{\y}, {\c}to primex vzvexenno{\y}e rexeni{\y}e.

{\Y}a posmotrel v {\y}e{\y}o glaza i vopreki svo{\y}emu jelani{\y}u otkazaty kivnul…

Propovednik sidel na pervom etaje, v apteke. On diplomati{\c}no ne stal sluxaty nax razgovor s Kristino{\y} i se{\y}{\c}as zanimalsa tem, {\c}to s nenavist{\y}u pos{\yi}lal prokl{\ia}ti{\y}e za prokl{\ia}ti{\y}em na golovu Valytera, kotor{\yi}{\y}, ne zame{\c}a{\y}a svetlo{\y} duxi, negromko besedoval s sedoborod{\yi}m aptekarem.

— Van Norma{\y}enn. — Koldun vstal, zakr{\yi}va{\y}a mne v{\yi}hod. — M{\yi} ploho na{\c}ali. B{\yi}ty mojet, se{\y}{\c}as samo{\y}e vrem{\ia} vs{\e} ispravity?

— Ugolyev tebe nado v glaza napihaty, d{\y}avolysko{\y}e otrodye! — besnovalsa Propovednik.

— Ne duma{\y}u, — holodno proizn{\e}s {\y}a.

— Nu hot{\ia} b{\yi} na vrem{\ia}. {\C}tob{\yi} ne rasstra{\y}ivaty {\c}udesnu{\y}u Kristinu.

{\Y}a xagnul k nemu navstre{\c}u i skazal tak tiho, {\c}tob{\yi} sl{\yi}xal tolyko on:

— {\Y}a ne ver{\iu} ni odnomu tvo{\y}emu slovu. I t{\yi} jiv tolyko potomu, {\c}to ona men{\ia} ostanovila. Poetomu s dorogi. Poka {\y}a ne ubil teb{\ia} za to, {\c}to t{\yi} delal so strajami.

On usmehnulsa, sdelal xag v storonu i, kogda {\y}a uje v{\yi}hodil, kriknul mne v spinu:

— Poduma{\y} o tom, {\c}to {\y}a skazal! Adski{\y}e vrata! V odnom iz naxih gorodov. I kogda oni otkro{\y}utsa, poblizosti ne okajetsa angelov, kotor{\yi}{\y}e steregut poko{\y} {\c}elove{\c}estva na vostoke. M{\yi} budem predostavlen{\yi} sami sebe!



{\Y}a pisal b{\yi}stro, to i delo okuna{\y}a pero v {\c}ernilyni{\q}u, i Propovednik, uznavxi{\y} osnovno{\y}e soderjani{\y}e naxevo s Kristino{\y} razgovora, po{\y}interesovalsa:

— Dl{\ia} {\c}evo vs{\e} eto?

— Konkretiziru{\y}, — poprosil {\y}a {\y}evo.

— Pisymo. Za{\c}em ono?

— Potomu {\c}to sozdalasy opasna{\y}a situa{\q}i{\y}a. I {\y}esli so mno{\y} {\c}to-to slu{\c}itsa, hoty kto-to doljen b{\yi}ty v kurse tovo, {\c}to zdesy proishodit.

— Gertruda ne budet s{\c}astliva.

{\Y}a podn{\ia}l na nevo vzgl{\ia}d:

— Nade{\y}usy, ona ni{\c}evo ne uzna{\y}et. Ne jela{\y}u vput{\yi}vaty {\y}e{\y}o v eto.

— Togda komu je t{\yi} pixex?

— Miriam. Bratstvo doljno b{\yi}ty gotovo k nepri{\y}atnost{\ia}m, {\y}esli kliriki rexat sprosity o Kristine i {\y}e{\y}o delah.

On pomol{\c}al, sluxa{\y}a, kak skripit pero:

— Etot Valyter, on kak bexena{\y}a sobaka. {\Y}evo nado ubity.

— Pri{\y}atno znaty, {\c}to m{\yi} shodimsa vo mneni{\y}ah i ne duma{\y}em o bible{\y}skih zapoved{\ia}h. — {\Y}a dal {\c}ernilam v{\yi}sohnuty. — No beda v tom, {\c}to u nevo {\y}esty informa{\q}i{\y}a. O tom je temnom kuzne{\q}e.

— T{\yi} ne dumal, {\c}to on vret?

— Nas{\c}et adskih vrat? Vpolne vozmojno. No odno ne otmen{\ia}{\y}et drugovo. Pohoje, on de{\y}stvitelyno i{\x}et temnovo mastera. I {\y}esli ne dl{\ia} ubi{\y}stva, to dl{\ia} svo{\y}ih {\q}ele{\y}. Ili {\c}yih-to {\y}e{\x}e. Kristina uverena, {\c}to oni po{\c}ti naxli kuzne{\q}a. Razumno b{\yi}lo b{\yi} nahoditsa r{\ia}dom s nimi.

— Tvo{\y}a b{\yi}vxa{\y}a naparni{\q}a — sumasxedxa{\y}a.

— Ona b{\yi} ne vv{\ia}zalasy v etu istori{\y}u, {\y}esli b{\yi} ne verila v to, {\c}to govorit.

{\Y}a slojil bumagu, ubral {\y}e{\y}o v konvert.

— Kak Kristina lixilasy paly{\q}ev?

— Kogda na{\c}ala rabotaty odna, — neohotno otvetil {\y}a.

— To {\y}esty bez teb{\ia}?

— Da.

On pon{\ia}l, {\c}to {\y}a ne jela{\y}u prodoljaty etu temu:

— {\Y}a shodil gl{\ia}nul na mesto, gde {\y}akob{\yi} b{\yi}l angel. Tolpi{\x}a, kak pered ra{\y}skimi vratami. Soldat{\yi}, zevaki, mol{\ia}{\x}i{\y}es{\ia}, sv{\ia}{\x}enniki. Vs{\ia} eta l{\iu}dska{\y}a massa kri{\c}it, gudit, oret, po{\y}et i {\y}edva li ne la{\y}et. I vs{\e} radi odnovo otpe{\c}atka boso{\y} nogi, ostavxegos{\ia} na brus{\c}atke.

— I {\c}em je on neob{\yi}{\c}en?

— Tem, {\c}to vplavlen v {\c}ern{\yi}{\y} kameny, a sam oslepitelyno-bel. I govor{\ia}t, {\c}to blagouha{\y}et jasminom.

— Jasmin v kon{\q}e zim{\yi} — eto pohoje na {\c}udo. — {\Y}a zape{\c}atal konvert al{\yi}m surgu{\c}om.

— I za pravo prikosnutsa gubami k etomu {\c}udu derutsa. A nekotor{\yi}{\y}e proda{\y}ut svo{\y}e mesto v o{\c}eredi za des{\ia}ty dukatov.

— B{\yi}lo b{\yi} stranno, {\y}esli b{\yi} zab{\yi}li o najive, — melanholi{\c}no proizn{\e}s {\y}a.

M{\yi}, l{\iu}di, vsegda na{\y}dem {\c}to prodaty i {\c}to kupity. {\Y}edu, zemli, titul{\yi}, zvani{\y}a, sv{\ia}t{\yi}{\y}e mo{\x}i ili mesto poblije k ra{\y}skim vratam.

— Segodn{\ia} {\y}a vozblagodaril Gospoda, {\c}to umer. Pravo, budy {\y}a jiv, {\c}erta s dva smog b{\yi} dobratsa do relikvi{\y}i i uvidety sv{\ia}tu{\y}u Djuli{\y}u.

{\Y}a ubral pisymo za goleni{\x}e sapoga:

— A eto {\y}e{\x}o kto?

— Slepa{\y}a devo{\c}ka, s kotoro{\y} govoril angel.

— {\C}to, uje b{\yi}la kanoniza{\q}i{\y}a? — s ironi{\y}e{\y} proizn{\e}s {\y}a, i tak zna{\y}a otvet.

K liku sv{\ia}t{\yi}h pri{\c}isl{\ia}li ne ranyxe {\c}em {\c}erez p{\ia}ty let posle smerti pretendenta, i {\c}tob{\yi} dosti{\c} stoly v{\yi}sokovo zvani{\y}a — slov o tom, {\c}to govoril s angelom, nedostato{\c}no.

— Kone{\c}no net. Prosto tak {\y}e{\y}o naz{\yi}va{\y}et narod.

— Narod… — {\Y}a v{\yi}xel na uli{\q}u, taku{\y}u je xumnu{\y}u, kak i ranyxe. — V takih voprosah vajno to, {\c}to govorit ne narod, a kn{\ia}z{\y}a {\q}erkvi.

Tut on kone{\c}no je ne sporil. Vidno, kak i {\y}a, vspomnil pred{\yi}du{\x}evo kn{\ia}z{\ia} Lezerberga, prosivxevo, {\c}tob{\yi} v Riapano priznali {\y}evo matuxku sv{\ia}to{\y}. Taka{\y}a blaj sto{\y}ila {\y}emu po{\c}ti semysot t{\yi}s{\ia}{\c} dukatov, a razrazivxi{\y}s{\ia} skandal privel k smerti des{\ia}tka podkuplenn{\yi}h kardinalov, reforma{\q}i{\y}i {\Q}erkvi i po{\y}avleni{\y}u protestn{\yi}h dvijeni{\y} v Vitilyska, trebovavxih lixity Riapano privilegi{\y}, a vseh Pap priznaty ``ne namestnikami Boga na zemle, a vsevo lix prodajn{\yi}mi sukin{\yi}mi s{\yi}nami i posledovatel{\ia}mi d{\y}avolyskih nauk, a takje jadn{\yi}mi kol{\c}enogimi jabami".

S teh por Sv{\ia}to{\y} grad trijd{\yi} duma{\y}et, prejde {\c}em kovo-libo pri{\c}islity hot{\ia} b{\yi} k blajenn{\yi}m, ne govor{\ia} uje o sv{\ia}t{\yi}h. Oni trebu{\y}ut dokazatelystv kak minimum dvuh soverxenn{\yi}h pri jizni {\c}udes, pravednovo su{\x}estvovani{\y}a i xesti monografi{\y} s razm{\yi}xleni{\y}ami o religi{\y}i ({\y}esli, kone{\c}no, pretendent umel {\c}itaty i pisaty).

Kontora ``Fabien Clemence i s{\yi}nov{\y}a" raspolagalasy nedaleko ot {\q}erkvi, gde {\y}a naxol Kristinu. Malenyki{\y} neprimetn{\yi}{\y} klerk prin{\ia}l mo{\y}e pisymo, podslepovato {\x}ur{\ia}sy.

— Ob{\yi}{\c}na{\y}a otpravka?

— Otsro{\c}enna{\y}a, — skazal {\y}a. — Otoxlite {\y}evo adresatu, {\y}esli {\y}a ne zaberu poslani{\y}e v te{\c}eni{\y}e sledu{\y}u{\x}ih des{\ia}ti dne{\y}.

— Eto budet {\c}uty doroje. — On sdelal otmetku v tolsto{\y} knige. — {\C}to-nibudy {\y}e{\x}e?

— Net, blagodar{\iu}.

Na uli{\q}e men{\ia} jdal Valyter.

— Dvad{\q}aty p{\ia}ty klinkov straje{\y} trebu{\y}etsa dl{\ia} odnovo temnovo kinjala. — On skazal eto b{\yi}stro, prejde {\c}em {\y}a rexil, {\c}to s nim sdelaty. — Dl{\ia} des{\ia}ti trebu{\y}etsa dvesti p{\ia}tydes{\ia}t.

— {\Y}a ume{\y}u s{\c}itaty. {\Y}emu ne hvatit i vse{\y} jizni, {\c}tob{\yi} sobraty ih.

Koldun, to{\c}no pti{\q}a, sklonil golovu:

— Jizny — ve{\x} otnositelyna{\y}a, Ludwig. K primeru, {\y}esty te, kto jivet sebe posle smerti i v us ne du{\y}et. A {\y}esty taki{\y}e, kak t{\yi}. Kto mojet uveli{\c}ivaty svo{\y}u jizny hoty do beskone{\c}nosti. Ponima{\y}ex, na {\c}to {\y}a nameka{\y}u?

— {\C}to t{\e}mn{\yi}{\y} kuzne{\q} — straj.

— {\Y}esty u men{\ia} taka{\y}a teori{\y}a.

— No nikakih dokazatelystv.

— Nikakih, — priznal on. — Tot, kto sozda{\y}et temno{\y}e oruji{\y}e, oblada{\y}et ogromn{\yi}m terpeni{\y}em i beskone{\c}n{\yi}m vremenem.

— Skaji mne, koldun. Skolyko kinjalov u nevo se{\y}{\c}as?

— Sprosi {\c}to poleg{\c}e. Semy. B{\yi}ty mojet, vosemy. I {\y}e{\x}o odin dela{\y}etsa. A zna{\c}it, vremeni u nas ne tak uj mnogo.

— A otkuda tebe izvestno ob etom?

— {\Y}a ume{\y}u sluxaty. {\Y}a zna{\y}u l{\iu}de{\y}. I nel{\iu}de{\y}. {\Y}a ponima{\y}u {\y}evo lu{\c}xe, {\c}em {\q}erkovniki, no, k sojaleni{\y}u, nedostato{\c}no, {\c}tob{\yi} skazaty, kto on tako{\y}. Etot {\c}elovek rabota{\y}et uje davno, iz pokoleni{\y}a v pokoleni{\y}e sobira{\y}a klinki straje{\y} i ostava{\y}asy nezametn{\yi}m v naxem mire.

— On mojet b{\yi}ty i ne odin. Naprimer, {\q}el{\yi}{\y} klan. Togda ne sto{\y}it obra{\x}aty vnimani{\y}a na skazku o bessmerti{\y}i.

— Da nevajno, skolyko ih — odin, dvo{\y}e ili sotn{\ia}. Kogda {\y}evo delo budet zakon{\c}eno, stanet slixkom pozdno. B{\yi}ty mojet, t{\yi} pro{\y}avix blagorazumi{\y}e i m{\yi} pogovorim? Pr{\ia}mo se{\y}{\c}as?

— Govori, — skazal {\y}a, prislonivxisy k stene doma.

— Horoxo, — legko soglasilsa on, b{\yi}stro ogl{\ia}devxisy i udostoverivxisy, {\c}to nikto ne obra{\x}a{\y}et na nas vnimani{\y}a. — {\Y}a uznal obo vsem etom davno, kogda {\y}e{\x}o b{\yi}l molod{\yi}m. Malenyki{\y} sluh, broxenna{\y}a fraza na odnom iz balov vedym. {\Y}a za{\y}interesovalsa, stal raskru{\c}ivaty nito{\c}ku. Nabl{\iu}dal, rasspraxival. Daje v mo{\y}em soob{\x}estve informa{\q}i{\y}i b{\yi}lo malo, {\y}a dovolystvovalsa lix sluhami i mifami, bolyxinstvo iz kotor{\yi}h okazalisy v{\yi}dumko{\y}. No {\y}a ne sdavalsa, nahodil kollek{\q}ionerov star{\yi}h knig, pose{\x}al {\c}astn{\yi}{\y}e biblioteki i daje {\y}ezdil v Temnolesye.

— Ne le{\y} vodu, koldun. {\C}uty bolyxe konkretiki.

— Nakone{\q} {\y}a uznal, {\c}to {\y}emu trebu{\y}utsa kinjal{\yi} straje{\y}. Ne {\y}un{\q}ov, a teh, kto uje ne perv{\yi}{\y} god sobira{\y}et duxi. I stal nabl{\iu}daty za vami. P{\ia}ty let mne potrebovalosy, {\c}tob{\yi} pon{\ia}ty — vaxi umira{\y}ut regul{\ia}rno, no v osnovnom molodn{\ia}k. Te, kto stanov{\ia}tsa masterami, pogiba{\y}ut dovolyno redko, a bessledno is{\c}eza{\y}ut {\y}e{\x}o reje. I po{\c}ti vse ih kinjal{\yi} popada{\y}ut v Orden i uni{\c}toja{\y}utsa.

— Poka {\y}a ne uznal ni{\c}evo novovo.

On hotel otvetity, no uvidel mal{\ia}ra, nesu{\x}evo vedro kraski, i ne otkr{\yi}val rta, poka {\c}elovek ne skr{\yi}lsa za povorotom.

— Napr{\ia}gi mozg, straj. {\Y}a govor{\iu} o tom, {\c}to {\y}edinstvenn{\yi}{\y} sposob sobraty kinjal{\yi}, ne v{\yi}reza{\y}a vaxu brati{\y}u napravo i nalevo, eto zabiraty te klinki, {\c}to Bratstvo otda{\y}et zakonnikam na uni{\c}tojeni{\y}e.

— Nevozmojno, — vozrazil {\y}a. — Naxe oruji{\y}e uni{\c}toja{\y}etsa pri svidetel{\ia}h.

On rassme{\y}alsa, zapanibratski hlopnuv men{\ia} po ple{\c}u:

— Let semy nazad {\y}a b{\yi}l takim je na{\y}ivn{\yi}m, kak t{\yi}, van Norma{\y}enn. No zatem na{\c}al dumaty. Kto vse eti svideteli? Straji na uni{\c}tojeni{\y}i b{\yi}va{\y}ut kra{\y}ne redko — vas malo i del polno. Orden loma{\y}et kinjal, kogda r{\ia}dom predstaviteli vlasti. A tepery poduma{\y}, mnogo li tolst{\yi}{\y} burgomistr, zanos{\c}iv{\yi}{\y} graf ili {\y}edva ume{\y}u{\x}i{\y} {\c}itaty prihodsko{\y} sv{\ia}{\x}ennik ponima{\y}ut v kinjalah straje{\y}?

S etimi slovami on dostal iz sumki klinok s sapfirom na ruko{\y}ati i prot{\ia}nul mne.

— Kopi{\y}a, — posle beglovo osmotra skazal {\y}a.

— Verno. No eto opredelit lix op{\yi}tn{\yi}{\y} glaz. Vseh ostalyn{\yi}h smutit zvezd{\c}at{\yi}{\y} sapfir, kotor{\yi}{\y}, prizna{\y}emsa {\c}estno, ne taka{\y}a uj i redkosty.

{\Y}a poter {\x}etinist{\yi}{\y} podborodok:

— T{\yi} ho{\c}ex ubedity men{\ia}, {\c}to Orden pomoga{\y}et temnomu koldunu?

— Orden ili kto-to iz sosto{\y}a{\x}ih v nem. Naprimer, lu{\c}xi{\y} drug markgrafa Valentina gospodin Aleksandr, horoxo tebe znakom{\yi}{\y} po sob{\yi}ti{\y}am v Vione. {\Y}esli zakonniki ne mogut vospolyzovatsa vaximi kinjalami sami, eto ne ozna{\c}a{\y}et, {\c}to oni ne na{\y}dut kuda ih pristro{\y}ity.

— I m{\yi} vnovy ut{\yi}ka{\y}emsa v stenu, koldun. Mo{\y} vopros: ``Kaka{\y}a v etom v{\yi}goda?" — nikuda ne delsa. {\Y}a ne jalu{\y}u zakonnikov, no vs{\e} je ne pover{\iu} v ih jelani{\y}e raspahnuty vrata ada i ustro{\y}ity kone{\q} dl{\ia} vsevo sveta. S kako{\y} stati v{\yi}rvavxi{\y}es{\ia} iz pekla {\c}erti ne na{\c}nut im vredity?

Valyter po-pri{\y}atelyski pozdorovalsa s dvum{\ia} prohodivximi mimo nas strajnikami i lix posle otvetil:

— Gl{\ia}ju, t{\yi} uje verix, {\c}to vrata, kotor{\yi}{\y}e mogut otkr{\yi}ty kinjal{\yi}, ne prosto glupa{\y}a skazka.

— Ne ver{\iu}, — otrezal {\y}a. — No eto vesom{\yi}{\y} kontrargument v tvo{\y}e{\y} nelepo{\y} istori{\y}i.

On ul{\yi}bnulsa, no glaza {\y}evo stali zl{\yi}mi i razdrajenn{\yi}mi:

— Spor{\iu} na dukat, {\c}to zakonniki ne zna{\y}ut o tom, {\c}to klinki mogut otkr{\yi}ty vrata. Tot, kto splavl{\ia}{\y}et kinjal{\yi} kuzne{\q}u, dela{\y}et mal{\yi}{\y}e pakosti, daje ne podozreva{\y}a o bolyxo{\y}. K primeru, on ne protiv, {\c}tob{\yi} u Bratstva b{\yi}lo mnogo rabot{\yi}. Kak togda, v Xossi{\y}i. I nade{\y}etsa, {\c}to neskolyko temn{\yi}h kinjalov diskreditiru{\y}ut straje{\y}, kotor{\yi}{\y}e prosto ne sprav{\ia}tsa s valom temn{\yi}h dux. Razve eto ne v{\yi}godno Ordenu? Paniku{\y}u{\x}e{\y}e naseleni{\y}e, nedovolyn{\yi}{\y}e kn{\ia}z{\y}a, nov{\yi}{\y}e volynosti, usileni{\y}e, vlasty? Eto odna iz versi{\y}. Druga{\y}a — on banalyno zarabat{\yi}va{\y}et na etom. L{\iu}d{\ia}m, zna{\y}ex li, nujno zoloto. A nekotor{\yi}m l{\iu}d{\ia}m {\y}evo trebu{\y}etsa kak mojno bolyxe. I nakone{\q}, tret{\y}a — Kristina s{\c}ita{\y}et, {\c}to kuzne{\q} otdal kinjal, {\c}tob{\yi} proverity {\y}evo rabotu. {\Y}a ne soglasen s etim. Tot, kto sozda{\y}et tako{\y}e, zna{\y}et, {\c}to v{\yi}hodit iz-pod {\y}evo molota. Po mne, eto b{\yi}la plata za klinki straje{\y}, kotor{\yi}{\y}e {\y}emu nujn{\yi}.

— {\Y}a zna{\y}u o dvuh temn{\yi}h kinjalah. Odin zabrala u zakonnikov Kristina, drugo{\y} — {\q}erkovy u kuryera, napravl{\ia}vxegos{\ia} v Latku.

On v{\yi}gl{\ia}del udivlenn{\yi}m:

— Seryezno? V perv{\yi}{\y} raz sl{\yi}xu. U teb{\ia} to{\c}n{\yi}{\y}e svedeni{\y}a?

— Da.

— Kogda eto slu{\c}ilosy?

— Ne mogu skazaty.

— Vozmojno, {\y}e{\x}o pri jizni gospodina Aleksandra, l{\iu}bivxevo gostity u markgrafa, — probormotal tot. — Gde kinjal tepery?

— Uni{\c}tojen klirikami.

— Tepery mne pon{\ia}tno, kak oni v{\yi}xli na kuzne{\q}a. — On otvernulsa, sobira{\y}asy uhodity. I brosil {\c}erez ple{\c}o: — Kristina prosila peredaty, {\c}tob{\yi} t{\yi} prixel {\c}erez paru {\c}asov v apteku.

— E{\y}, koldun, — ostanovil {\y}a {\y}evo. — Ni odin {\c}elovek ne men{\ia}{\y}etsa nastolyko b{\yi}stro. S {\c}evo eto t{\yi} stal takim l{\iu}bezn{\yi}m?

— L{\iu}bezn{\yi}m? — On skrivil ugol rta. — T{\yi} men{\ia} s kem-to sputal, straj. {\Y}a govor{\iu} s tobo{\y} lix potomu, {\c}to mne mojet ponadobitsa tvo{\y}a pomo{\x}. Ina{\c}e nikakih razgovorov b{\yi} ne polu{\c}ilosy.

— Otvety mne vsevo lix na dva voprosa, a zatem mojex katitsa k {\c}ertu.

— Ve{\c}erom.

— Net. Se{\y}{\c}as.

V {\y}evo glazah b{\yi}lo {\q}elo{\y}e more gneva, {\y}a videl, kak on sjal kulaki, no tut je rasslabilsa i ne{\y}iskrenne ul{\yi}bnulsa:

— Ladno. Pro{\x}e vs{\e} zakon{\c}ity se{\y}{\c}as. Val{\ia}{\y}.

— Markgraf Valentin sobiral kinjal{\yi}. Dl{\ia} kovo?

— Aleksandr i {\y}evo ne{\y}izvestn{\yi}{\y}e mne druz{\y}a vnuxili markgrafu, {\c}to tot obretet bessmerti{\y}e. Na samom dele zakonniki prosto ispolyzovali {\y}evo vozmojnosti dl{\ia} sbora oruji{\y}a Bratstva. Vtoro{\y} vopros?

— Kinjal, kotor{\yi}{\y} t{\yi} {\y}edva ne ukral v Livette. Dl{\ia} {\c}evo nujen on?

— A… — prot{\ia}nul koldun, i b{\yi}lo vidno, {\c}to {\y}emu nepri{\y}atno vspominaty o to{\y} neuda{\c}e. — Dl{\ia} obmena. Odin kollek{\q}ioner jelal sebe taku{\y}u igruxku v obmen na bezdeluxku.

— Kaku{\y}u bezdeluxku?

Kak nazlo, navstre{\c}u xel {\y}e{\x}o odin patruly straji, i Valyter vospolyzovalsa etim, otodvinuv men{\ia} ple{\c}om:

— Prihodi k Kristine. Uzna{\y}ex.

— To {\y}esty t{\yi} polu{\c}il {\y}e{\y}e? Naxel drugo{\y} klinok?

— Mne pora, van Norma{\y}enn.

{\Y}a dal {\y}emu pro{\y}ti, potomu {\c}to i tak uje znal, {\c}e{\y} kinjal on ispolyzoval i na {\c}to {\y}evo hotel pomen{\ia}ty.



— Poraja{\y}usy tvo{\y}emu terpeni{\y}u. Drugo{\y}, ne zna{\y}a teb{\ia}, nazval b{\yi} eto slaboharakternost{\y}u. Nu posle tovo, {\c}to sdelal etot hm{\yi}ry. — Propovednik posmotrel na men{\ia} po-starikovski hitro.

— No t{\yi} men{\ia} zna{\y}ex i… — {\Y}a predostavil {\y}emu zakon{\c}ity frazu.

— T{\yi} obuzdal emo{\q}i{\y}i i rexil ne pugaty kr{\yi}su, poka ona ne privedet teb{\ia} k zernohranili{\x}u.

— Skore{\y}e uj zme{\y}u, poka ta ne pokajet, gde {\y}e{\y}o kladka.

Propovednik pogrozil mne paly{\q}em:

— Zme{\y}a mojet i ukusity. A {\y}e{\y}o {\y}ad opasen. Pomn{\iu}, v mo{\y}e{\y} derevne odin pastuh…

— Men{\ia} bolyxe interesu{\y}et ta vajna{\y}a novosty, kotora{\y}a vsevo minutu nazad zanimala vs{\e} tvo{\y}o voobrajeni{\y}e.

— Teb{\ia} ona udivit. Zna{\y}ex, kak zovut kardinala, kotor{\yi}{\y} prib{\yi}va{\y}et v Kruso, {\c}tob{\yi} provesti torjestvenno{\y}e bogoslujeni{\y}e? Tvo{\y} star{\yi}{\y} drug Urban.

{\Y}a daje ostanovilsa:

— Nepri{\y}atno{\y}e izvesti{\y}e, {\y}esli {\c}estno. Kardinal Urban v gorode, a r{\ia}dom Valyter. Kak b{\yi} ne slu{\c}ilosy vtorovo Viona.

— Tot {\c}elovek iz Ordena, Aleksandr, mertv.

— I vs{\e} ravno mne eto ne nravitsa.

{\Y}a v{\yi}xel na bolyxu{\y}u kruglu{\y}u plo{\x}ady, na kotoro{\y} letom vo vrem{\ia} prazdnestva ustra{\y}ivali znamenit{\yi}{\y}e p{\ia}timinutn{\yi}{\y}e ska{\c}ki. Se{\y}{\c}as zdesy goreli kostr{\yi} i b{\yi}li razbit{\yi} palatki. Palomniki, te s{\c}astliv{\c}iki, kovo pustili v gorod, jili pr{\ia}mo na uli{\q}e, ojida{\y}a svo{\y}e{\y} o{\c}eredi prikosnutsa gubami k sv{\ia}tomu sledu.

Vozle odno{\y} iz v{\ia}zanok hvorosta, v{\yi}t{\ia}nuv nogi, sidelo Pugalo.

— Vot eto vstre{\c}a, — probubnil Propovednik. — Ne ho{\c}ex pro{\c}itaty {\y}emu nota{\q}i{\y}u za to, {\c}to ono razodralo dnevnik burgomistra? Ina{\c}e v sledu{\y}u{\x}i{\y} raz ono ukradet u teb{\ia} ispodne{\y}e. I spalit na ogne.

— A {\y}esli ne pro{\c}itaty {\y}emu nota{\q}i{\y}u, {\y}a sekonoml{\iu} {\q}elu{\y}u minutu vremeni. Potomu {\c}to itogov{\yi}{\y} rezulytat budet odin i tot je — ono vs{\e} ravno propustit mo{\y}i slova mimo uxe{\y}. K tomu je mne sledu{\y}et zagl{\ia}nuty v apteku.

— T{\yi} vs{\e} ravno ni{\c}evo ot nih ne dobyexsa.

— No hot{\ia} b{\yi} uzna{\y}u, kakim obrazom oni hot{\ia}t po{\y}maty kuzne{\q}a.

— Taka{\y}a je bespolezna{\y}a trata vremeni, kak ubejdaty Pugalo ostavatsa pa{\y}inyko{\y}.

Ni tot, ni drugo{\y} ne zahoteli b{\yi}ty mo{\y}imi soprovojda{\y}u{\x}imi, tak {\c}to {\y}a ostavil ih na plo{\x}adi sluxaty l{\iu}dsku{\y}u boltovn{\iu}.

Apteka okazalasy zakr{\yi}ta, stavni opu{\x}en{\yi}, no v okne vtorovo etaja gorel svet. {\Y}a postu{\c}al, i mne otkr{\yi}l sedoborod{\yi}{\y} aptekary.

— A, gospodin straj. M{\yi} uje dumali, v{\yi} ne pridete. — On blagosklonno kivnul, vpuska{\y}a men{\ia}.

Starik v{\yi}gl{\ia}del nervn{\yi}m i napr{\ia}jenn{\yi}m. Za vse{\y} eto{\y} l{\iu}beznost{\y}u skr{\yi}valsa kako{\y}-to su{\y}etliv{\yi}{\y} strah. Eto ni{\c}uty ne vnuxilo mne doveri{\y}a.

— Ludwig! Horoxo, {\c}to t{\yi} vernulsa! — Kristina sto{\y}ala na lestni{\q}e i ul{\yi}balasy, ne skr{\yi}va{\y}a, {\c}to rada men{\ia} videty. — Idem, {\y}a teb{\ia} poznakoml{\iu} s ostalyn{\yi}mi.

V komnatah, kotor{\yi}{\y}e ona snimala, goreli sve{\c}i. Dva stola okazalisy sdvinut{\yi}, i za nimi razmestilisy l{\iu}di. Kogda {\y}a voxel, na mne sosredoto{\c}ilosy vnimani{\y}e vseh prisutstvu{\y}u{\x}ih.

— Pozvolyte poznakomity vas s gospodinom van Norma{\y}ennom, druz{\y}a, — obratilasy Kristina k {\c}etver{\yi}m neznakom{\q}am. — On — straj, kak i {\y}a. Odin iz lu{\c}xih v mo{\y}em pokoleni{\y}i. A eto l{\iu}di, kotor{\yi}{\y}e, kak i m{\yi} s Valyterom, jela{\y}ut raz i navsegda pokon{\c}ity s t{\e}mn{\yi}m kuzne{\q}om.

{\Q}ela{\y}a kompani{\y}a sumasxedxih me{\c}tatele{\y}, jela{\y}u{\x}ih spasti mir.

— Metr Filipp, — predstavila ona aptekar{\ia}. — On zanima{\y}etsa alhimi{\y}e{\y} i b{\yi}l nastolyko l{\iu}bezen, {\c}to okazal nam gostepri{\y}imstvo.

Starik su{\y}etlivo poklonilsa i, pl{\iu}hnuvxisy na stul, stal pomexivaty loje{\c}ko{\y} v stakane s kakim-to varevom, to i delo gromko zv{\ia}ka{\y}a o tonku{\y}u stekl{\ia}nnu{\y}u stenku.

— Adily aly Djuma — predstavitely Lavenduzzskovo so{\y}uza v svo{\y}ih zeml{\ia}h.

T{\iu}rban na brito{\y} golove delal to{\x}evo hagjita pohojim na strann{\yi}{\y} lesno{\y} grib. Glaza b{\yi}li podveden{\yi} surymo{\y}.

— On okazal nam neo{\q}enimu{\y}u uslugu.

— V{\yi} priukraxiva{\y}ete mo{\y}i dostijeni{\y}a, ba{\y}an[40] Kristina. — On ul{\yi}bnulsa, i {\y}a uvidel, {\c}to dvuh {\q}entralyn{\yi}h verhnih zubov u nevo net. — {\Y}a vsevo lix skromn{\yi}{\y} sluga pust{\yi}nn{\yi}h mudre{\q}ov, i ih prikaz{\yi} priveli men{\ia} s{\iu}da.

— {\C}ezare Motto. Kondotyer.

V{\yi}soki{\y} i ple{\c}ist{\yi}{\y} {\c}elovek s {\x}etinist{\yi}m podborodkom i gust{\yi}mi, {\c}uty r{\yi}jevat{\yi}mi brov{\ia}mi neohotno pripodn{\ia}l dva paly{\q}a v privetstvennom jeste na{\y}emnikov Kavarzere. {\Y}a ne znal, {\c}to on zdesy dela{\y}et, no soldat uda{\c}i kazalsa takim je lixnim, kak {\c}ert, zagl{\ia}nuvxi{\y} na voskresnu{\y}u messu.

— I ote{\q} Gotthod, kanonik sobora Sv{\ia}to{\y} Mari{\y}i v Braselovette.

Borodat{\yi}{\y} tolst{\ia}k v {\c}erno{\y} r{\ia}se, krugloli{\q}i{\y}, s ospinami na {\x}ekah i lbu, pripodn{\ia}lsa nad stulom:

— Master.

— S {\c}evo mne na{\c}aty rasskaz, Ludwig? — Valyter sidel na podokonnike, na rukah u nevo dremala to{\x}a{\y}a p{\ia}tnista{\y}a koxka. Li{\q}o kolduna uje zajilo, slovno {\y}a i ne kasalsa {\y}evo svo{\y}imi kulakami.

— Na{\c}ni s tovo, za{\c}em {\y}a zdesy. — {\Y}a pro{\y}ignoriroval stul, vstav tak, {\c}tob{\yi} videty ih vseh. Razume{\y}etsa, eto ne ostalosy nezame{\c}enn{\yi}m, no nikto, krome usmehnuvxegos{\ia} kondotyera, ne podal vida.

— Delo ne v tom, {\c}to t{\yi} straj… — Koldun ne otr{\yi}val vzgl{\ia}da ot koxki.

— Sam Gospody pos{\yi}la{\y}et vas nam, — vajno kivnul ote{\q} Gotthod. — Ne ina{\c}e eto {\y}evo jelani{\y}e.

— Nam de{\y}stvitelyno nujna tvo{\y}a pomo{\x}, Ludwig, — podhvatila Kristina. — M{\yi} s Valyterom segodn{\ia} pogovorili i pon{\ia}li, {\c}to, {\y}esli t{\yi} budex s nami, vs{\e} pro{\y}det legko i ne budet nikako{\y} krovi.

— {\Y}a, pojalu{\y}, na{\c}nu s samovo na{\c}ala. — Koldun posmotrel na Kristinu, i ta obodr{\ia}{\y}u{\x}e kivnula. — Dnem t{\yi} zadaval vopros, za{\c}em mne b{\yi}l nujen kinjal tvo{\y}evo druga…

— Natana, — podskazala straj.

— Tvo{\y}evo druga Natana. Pred{\yi}stori{\y}a takova. Dostopo{\c}tim{\yi}{\y} Adily aly Djuma, blagodar{\ia} svo{\y}im sv{\ia}z{\ia}m v torgovle, mnogo{\y}e sl{\yi}xit. Daje to, {\c}to p{\yi}ta{\y}utsa skr{\yi}ty ot {\y}evo uxe{\y}. Do nevo doxel sluh o tom, {\c}to v Veliko{\y} pust{\yi}ne lovki{\y}e l{\iu}di ot{\yi}skali dva {\c}ern{\yi}h kamn{\ia} i privezli na nax kontinent.

— Re{\c} o glazah serafima?

— Verno, van Norma{\y}enn. Ih dostavili po osobomu zakazu. Etot kameny o{\c}eny redok i {\y}avl{\ia}{\y}etsa ob{\ia}zatelyn{\yi}m materialom dl{\ia} izgotovleni{\y}a temnovo klinka. I trebu{\y}etsa kuzne{\q}u.

On sdelal zna{\c}itelynu{\y}u pauzu, no, ne dojdavxisy nikakih kommentari{\y}ev ot men{\ia}, prodoljil:

— Nam prixlosy pob{\yi}vaty v xestnad{\q}ati portah, prejde {\c}em udalosy napasty na sled prodav{\q}a. I {\y}e{\x}o neskolyko mes{\ia}{\q}ev, {\c}tob{\yi} uznaty o pokupatele. On priobrel oba glaza serafima, tak kak sobira{\y}et mineral{\yi} — u nevo dovolyno obxirna{\y}a kollek{\q}i{\y}a, kak {\y}a sl{\yi}xal. M{\yi} pop{\yi}talisy v{\yi}kupity hot{\ia} b{\yi} odin kameny, no boga{\c}u ne nujn{\yi} denygi.

— Pop{\yi}talisy ukrasty, — prodoljila Kristina. — No eto okazalosy ne tak-to prosto. M{\yi} daje ne smogli uznaty, gde on ih hranit.

— Tvo{\y}a reputa{\q}i{\y}a pod ugrozo{\y}, koldun, — s usmexko{\y} skazal {\y}a Valyteru. — Neujeli t{\yi} ne isproboval sam{\yi}{\y} vern{\yi}{\y} sposob merzav{\q}ev — nasili{\y}e?

— Povery, mne o{\c}eny hotelosy. — On vernul usmexku. — No u nevo mnogo druze{\y}. I eto izvestn{\yi}{\y} {\c}elovek. {\Y}evo is{\c}eznoveni{\y}e, ne govor{\ia} uje o smerti, vzbudorajilo b{\yi} vlasti. A {\y}a v posledne{\y}e vrem{\ia} i tak privlek slixkom mnogo nezdorovovo vnimani{\y}a k po{\y}iskam temnovo kuzne{\q}a.

— Na samom dele eto {\y}a otgovoril {\y}evo ot pospexn{\yi}h de{\y}stvi{\y}. — Aptekary nervno s{\q}epil paly{\q}i. — Nasili{\y}e — ne v{\yi}hod. Osobenno {\y}esli ono mojet privesti k {\y}e{\x}o bolyxomu nasili{\y}u i provalu vajno{\y} missi{\y}i. Da i so smert{\y}u kollek{\q}ionera m{\yi} b{\yi} ne naxli ta{\y}nik. Poetomu rexili de{\y}stvovaty ina{\c}e. Adily v{\yi}stupil kak predstavitely drugovo l{\iu}bitel{\ia} mineralov, predlojil, razume{\y}etsa, denygi. Zatem obmen. Uznal, {\c}evo ho{\c}et bur… tot {\c}elovek.

— I {\c}to {\y}emu nujno? — sprosil {\y}a, hot{\ia} znal otvet.

— Kinjal straja. {\Y}evo interesoval zvezd{\c}at{\yi}{\y} sapfir v neob{\yi}{\c}nom ispolneni{\y}i. Za taku{\y}u relikvi{\y}u on gotov b{\yi}l ustupity odin iz dvuh svo{\y}ih kamne{\y}.

— T{\yi} v kurse slu{\c}ivxegos{\ia}? — po{\y}interesovalsa {\y}a u Kristin{\yi}.

— Net. V Livette men{\ia} ne b{\yi}lo. {\Y}a b{\yi} ne dala {\y}emu vz{\ia}ty kinjal Natana, t{\yi} je zna{\y}ex.

{\Y}esli {\c}estno, {\y}a uje ne znal, kto ona i na {\c}to sposobna radi tovo, {\c}tob{\yi} po{\y}maty temnovo mastera.

— I kak je v{\yi} postupili, kogda u tvo{\y}evo druga ni{\c}evo ne polu{\c}ilosy i klinok vernulsa k zakonnomu vladely{\q}u?

— {\Y}a izgotovil poddelku, — ojivilsa Filipp. — Otmenna{\y}a ve{\x}, nasto{\y}a{\x}i{\y} zvezd{\c}at{\yi}{\y} sapfir, i staly podhod{\ia}{\x}a{\y}a. Mojno obmanuty po{\c}ti vseh, no ne nasto{\y}a{\x}evo znatoka.

— On provozilsa do dekabr{\ia}, m{\yi} poter{\ia}li po{\c}ti xesty mes{\ia}{\q}ev. — {\C}ezare prenebrejitelyn{\yi}m {\x}el{\c}kom otpravil {\c}erez stol nevidimu{\y}u sorinku. — A v itoge kollek{\q}ioner podn{\ia}l nas na smeh. Obmanuty {\y}evo ne udalosy.

— I?.. — podstegnul {\y}a ih.

Gluboka{\y}a tixina razlilasy po pome{\x}eni{\y}u, vse tepery smotreli na Kristinu, kak b{\yi} otstran{\ia}{\y}asy ot tovo, {\c}to slu{\c}ilosy dalyxe.

— {\Y}a otdala {\y}emu svo{\y} kinjal! — nabrav vozduha v grudy, v{\yi}palila ona.

Mne prihodilosy igraty, i {\y}a ne b{\yi}l uveren, {\c}to akter iz men{\ia} horoxi{\y}.

— {\C}to?!

— Mne prixlosy, Ludwig.

{\Y}a s kamenn{\yi}m li{\q}om pomol{\c}al, vid{\ia}, {\c}to ona to krasne{\y}et, to bledne{\y}et, i spoko{\y}no proizn{\e}s:

— {\Y}a ho{\c}u {\y}evo uvidety.

Li{\q}o u Kristin{\yi} stalo raster{\ia}nn{\yi}m:

— T{\yi} o kinjale? {\Y}a je govor{\iu}…

— K {\c}ertu tvo{\y} kinjal, Kristina. Raz on tebe ne nujen i t{\yi} rasstalasy s nim dobrovolyno, {\y}a ne tot {\c}elovek, kotor{\yi}{\y} budet ubejdaty teb{\ia} v tvo{\y}e{\y} gluposti! — rezko otvetil {\y}a, i mo{\y}i slova b{\yi}li dl{\ia} ne{\y}o kak po{\x}e{\c}ina. — Pokajite mne kameny, radi kotorovo v{\yi} ustro{\y}ili vs{\e} eto.

— Em… — Filipp poter perenosi{\q}u. — Ponima{\y}ete, u nas {\y}evo net. I kak b{\yi}… {\y}a polaga{\y}u, {\c}to uje i ne budet. Gospodin {\C}ezare nedel{\iu} nazad vernulsa s plohimi novost{\ia}mi. Kollek{\q}ioner mertv, kamni tak i ne na{\y}den{\yi}. V delo vmexalasy inkvizi{\q}i{\y}a, i m{\yi} ne mojem se{\y}{\c}as vernuty daje oruji{\y}e Kristin{\yi}. Ne zna{\y}em, gde ono.

— A {\y}a govoril, {\c}to klinok nado men{\ia}ty na kameny srazu. — Kondotyer {\q}edil slova zlo. — V{\yi} je poxli na povodu u eto{\y} svolo{\c}i. Mirol{\iu}bi{\y}e, ne nado nasili{\y}a, ne sto{\y}it privlekaty k sebe dopolnitelyno{\y}e vnimani{\y}e…

— Posle tovo kak m{\yi} p{\yi}talisy podsunuty {\y}emu poddelku, on perestal nam dover{\ia}ty, — vinovato razvel rukami aptekary. — On potreboval pereslaty {\y}emu kinjal {\c}erez ``Fabien Clemence i s{\yi}nov{\y}a".

— No obmanul vas i ne peredal glaz serafima?

— Net, — gluho otvetila Kristina. — Valyter ne hotel polyzovatsa posrednikami. M{\yi} rexili zabraty kameny li{\c}no, no ne uspeli.

— V itoge u vas net ni kur, ni lis{\yi}, — otvetil {\y}a staro{\y} pogovorko{\y}.

Na Kristinu b{\yi}lo jalko smotrety, tepery ona v{\yi}gl{\ia}dela nastolyko podavlenno{\y}, {\c}to {\y}a s trudom poborol v sebe jelani{\y}e otkr{\yi}ty visevxu{\y}u {\c}erez ple{\c}o sumku i vernuty {\y}e{\y}o oruji{\y}e. No {\y}a sderjalsa. Ne se{\y}{\c}as. I ne pri etih l{\iu}d{\ia}h.

— Obrazno govor{\ia}, v{\yi} soverxenno prav{\yi}, — podtverdil Filipp.

— {\Y}esli v{\yi} nade{\y}etesy, {\c}to tepery {\y}a dam vam svo{\y} kinjal, {\c}tob{\yi} v{\yi} {\y}evo poter{\ia}li tak je bezdarno, kak i {\y}e{\y}o klinok, to obra{\x}a{\y}etesy ne po adresu.

— Gospody s vami! — vsplesnul rukami ote{\q} Gotthod. — Ni{\c}evo podobnovo! Vam ne nado budet s nim rasstavatsa. {\Y}esli {\c}estno, to on nam sovsem ne nujen. Ne hotite vs{\e} je prisesty?

— Net. Tak {\c}to je vam nujno?

— Odna ve{\x}, kotora{\y}a prinadlejit tebe. — Koldun ostorojno opustil koxku na podokonnik, podoxel k Kristine, polojil ruku {\y}e{\y} na ple{\c}o, i mne ne ponravilsa etot jest — sobstvennika, za{\y}avl{\ia}{\y}u{\x}evo prava na svo{\y}u ve{\x}. — Koly{\q}o, kotoro{\y}e tebe podaril {\y}episkop Urban, posle tovo kak t{\yi} spas {\y}emu jizny v Vione. Ono {\y}e{\x}o u teb{\ia}?

Neojidann{\yi}{\y} povorot. Priznatsa, {\y}a ne b{\yi}l gotov k takomu voprosu.

— Ne priv{\yi}k taskaty s sobo{\y} goru bezdeluxek.

— No t{\yi} i ne prodal {\y}evo. — Kristina ne spraxivala, utverjdala. — T{\yi} slixkom umen, {\c}tob{\yi} razbras{\yi}vatsa podobn{\yi}mi podarkami i ostavl{\ia}ty ih v lombarde. Uverena, {\c}to kak vsegda hranix na depozite v ``Fabien Clemence", {\c}tob{\yi} vz{\ia}ty v l{\iu}bo{\y} moment.

Ona slixkom horoxo znala mo{\y}i priv{\yi}{\c}ki, i se{\y}{\c}as men{\ia} eto ne radovalo.

— Podrobne{\y}e, Krista. {\Y}esli vam nujno koly{\q}o klirika, {\y}a ho{\c}u znaty dl{\ia} {\c}evo. Vax kuzne{\q} kl{\iu}{\y}et na l{\iu}bu{\y}u pobr{\ia}kuxku?

Valyter poka{\c}al golovo{\y}:

— Vse gorazdo slojne{\y}e i pro{\x}e. Tot slu{\c}a{\y} v Vione imel nekotor{\yi}{\y}e predpos{\yi}lki. Aleksandr i markgraf Valentin hoteli izbavitsa ot kardinala, to{\c}ne{\y}e togda {\y}e{\x}o {\y}episkopa, po neskolykim pri{\c}inam. Razume{\y}etsa, vse videli lix politi{\c}eski{\y}e — on mexal razvernutsa Ordenu v kn{\ia}jestve i ne daval jizni markgrafu, obli{\c}a{\y}a {\y}evo prestupleni{\y}a. No b{\yi}lo {\y}e{\x}o odno ``no". {\Y}a o nem uznal pozje, primerno za nedel{\iu} do tovo, kak t{\yi} prikon{\c}il {\y}evo milosty v Latke. On mne sam priznalsa, {\c}to Aleksandr jajdal polu{\c}ity {\c}ern{\yi}{\y} kameny {\y}episkopa. Mol, tot {\y}emu nujen ne menyxe, {\c}em kinjal{\yi} straje{\y}, i horoxo b{\yi} etu xtuku privezti v Latku, kak tolyko po{\y}avitsa taka{\y}a vozmojnosty.

— Hm… — {\Y}a gl{\ia}del na nevo ispodlob{\y}a. — U {\y}evo v{\yi}sokopreosv{\ia}{\x}enstva ime{\y}etsa glaz serafima?

— Imenno. {\Y}a posluxal veter, i tot dones do men{\ia} interesn{\yi}{\y}e sluhi. {\C}el{\ia}dy, kak t{\yi} zna{\y}ex, redko hranit ta{\y}n{\yi}. U kardinala {\y}esty mineral, kotor{\yi}{\y} on nosit na xe{\y}e, r{\ia}dom s rasp{\ia}ti{\y}em. Kameny ploho{\y} i zlo{\y}. Urban vrode kak dal obet mnogo let nazad, i eto {\y}evo krest, kotor{\yi}{\y} on ta{\x}it vo slavu Gospoda, ist{\ia}za{\y}a svo{\y}u ploty takim obrazom.

— U glaza serafima {\y}esty podobn{\yi}{\y}e svo{\y}stva?

— M{\yi} tolkom ne uveren{\yi}, — otvetil aptekary. — Traktat{\yi} govor{\ia}t razno{\y}e, v alhimi{\y}i kameny s{\c}ita{\y}etsa t{\e}mn{\yi}m i sposobn{\yi}m prinosity vred {\c}eloveku nesto{\y}komu. U hagjitov na se{\y} s{\c}et voob{\x}e mnojestvo legend.

— Oni razn{\ia}tsa, — ul{\yi}bnulsa britogolov{\yi}{\y} torgove{\q}. — Pust{\yi}nn{\yi}{\y}e star{\q}i, da prodl{\ia}tsa ih goda ve{\c}no, naz{\yi}va{\y}ut {\y}evo ubi{\y}{\q}e{\y} sveta. I skazok o nem de{\y}stvitelyno mnogo. Vse kak odna s plohim kon{\q}om. Ne skaju, skolyko v nih pravd{\yi}, no mo{\y} narod stara{\y}etsa ne derjaty taki{\y}e ve{\x}i podolgu, osobenno blizko k domu.

— Predpo{\c}ita{\y}et sbagrivaty ih nam za ku{\c}u florinov, — poddel {\y}evo {\C}ezare, i hagjit ul{\yi}bnulsa, no, sud{\ia} po {\y}evo li{\q}u, iskl{\iu}{\c}itelyno iz vejlivosti.

— I raz ne polu{\c}ilosy s kollek{\q}ionerom, v{\yi} rexili poza{\y}imstvovaty mineral u kardinala. Vmesto tovo {\c}tob{\yi} vz{\ia}ty lopatu i otpravitsa v bezvodnu{\y}u pust{\yi}n{\iu}. Po mne — posledni{\y} variant b{\yi}l b{\yi} gorazdo bole{\y}e razumen. V plane v{\yi}jivani{\y}a.

— {\Y}a ponima{\y}u vaxu ironi{\y}u, gospodin van Norma{\y}enn. M{\yi} pr{\ia}{\c}emsa ot {\Q}erkvi, zna{\y}em, na {\c}to ona sposobna. A tut sami lezem v vol{\c}{\y}u pasty. No m{\yi} ob{\ia}zan{\yi}. Vo im{\ia} l{\iu}de{\y} i vo blago vsevo mira, — pospexno uto{\c}nil aptekary.

— Razume{\y}etsa, — ehom otkliknulsa {\y}a. — A koly{\q}o vam trebu{\y}etsa…

— {\C}tob{\yi} podobratsa k Urbanu. Eto propusk, Ludwig. Prosto otda{\y} nam {\y}evo. Tebe neza{\c}em u{\c}astvovaty v ostalynom.

{\Y}a pot{\ia}nulsa:

— V{\yi}, l{\iu}bezn{\yi}{\y}e gospoda, kone{\c}no, bolyxi{\y}e fantazer{\yi}, no, kak mne kajetsa, ne idiot{\yi}. I ne stanete ubivaty kardinala. Klirikov takovo ranga ubiva{\y}ut lix drugi{\y}e kliriki, no ne prost{\yi}{\y}e smertn{\yi}{\y}e vrode nas.

— Nikto ne govorit ob ubi{\y}stve. Da{\y} mne vozmojnosty podobratsa k nemu, vs{\e} ostalyno{\y}e — delo tehniki. — Valyter nehoroxo ul{\yi}bnulsa.

{\Y}a znal, kak nekotor{\yi}{\y}e koldun{\yi} ume{\y}ut us{\yi}pl{\ia}ty, rasse{\y}ivaty vnimani{\y}e ili pritormajivaty vrem{\ia}. Videl, {\c}to prodel{\yi}va{\y}et Gertruda.

— Nu tak i sdela{\y} vs{\e} sam, — pojal {\y}a ple{\c}ami. — Dl{\ia} etovo ne ob{\ia}zatelyno koly{\q}o.

— U kardinala seryezna{\y}a ohrana. {\Q}erkovniki s magi{\y}e{\y}. So vsemi {\y}a prosto ne spravl{\iu}sy. — On legko raspisalsa v svo{\y}e{\y} bespomo{\x}nosti.

— A kogda u teb{\ia} budet pobr{\ia}kuxka, oni {\c}to? Rastvor{\ia}tsa v vozduhe, {\c}to li?

— {\Q}erkovniki budut mene{\y}e bditelyn{\yi}. {\Y}a smogu podobratsa blizko, ogluxity ih. Bez propuska, izdali, eto nevozmojno.

— Kak t{\yi} ob{\y}asnix, otkuda ono u teb{\ia}?

— Ne budu ob{\y}asn{\ia}ty. M{\yi} planiru{\y}em vs{\e} provernuty vo vrem{\ia} torjestvennovo bogoslujeni{\y}a. {\Y}esli i budet proverka, to ne nastolyko seryezna{\y}a. A potom, uveren, im stanet ne do nas.

{\C}ezare zarjal, i Filipp, podderjiva{\y}a na{\y}emnika, ul{\yi}bnulsa.

— {\Y}esty pri{\c}ina dl{\ia} smeha? — nahmurilsa {\y}a.

— Nebolyxa{\y}a. — Aptekary ponizil golos. — M{\yi} s Valyterom pridumali i osu{\x}estvili grandioznu{\y}u aferu, so{\c}iniv, {\c}to angel snizoxel v etot gorod.

— I proxlo, {\c}ert men{\ia} deri! — hlopnul ladon{\y}u po stolu kondotyer. — Slu{\c}ilosy {\c}udo!

— Sv{\ia}totat{\q}i, — so smireni{\y}em poka{\c}al golovo{\y} kanonik. — Nade{\y}usy, Gospody po{\y}met, {\c}to ne radi zla m{\yi} eto sdelali, i prostit nas.

— To {\y}esty l{\iu}di, {\c}to sid{\ia}t na uli{\q}ah i p{\yi}ta{\y}utsa popasty v gorod, proxli sotni lig radi nesu{\x}estvu{\y}u{\x}evo {\c}uda? Da, smexno. Kak v{\yi} eto ustro{\y}ili?

Valyter skromno razvel rukami:

— Nemnovo ne{\y}tralyno{\y} magi{\y}i, kotoru{\y}u ne srazu opredel{\ia}t kliriki, nemnogo alhimi{\c}eskih smese{\y} Filippa, odin slepo{\y} rebenok i umeni{\y}e rasprostran{\ia}ty sluhi. Vot re{\q}ept bojestvennovo {\c}uda v naxi dni.

— V{\yi} zate{\y}ali eto, {\c}tob{\yi} v{\yi}manity Urbana, tak kak Kruso pod {\y}evo pokrovitelystvom, i on ne mog pro{\y}ignorirovaty tako{\y}e sob{\yi}ti{\y}e i ne pri{\y}ehaty s{\iu}da.

— T{\yi} pravilyno ponima{\y}ex.

Da {\c}to uj tut ponimaty? I idiotu {\y}asno.

— Zna{\c}it, v{\yi} vs{\e} splanirovali davno. {\y}e{\x}o do tovo, kak pon{\ia}li, {\c}to s kollek{\q}ionera kameny ne polu{\c}ity.

— Rezervn{\yi}{\y} variant. Odno rovn{\yi}m s{\c}etom ne mexalo drugomu, i, kak vidix, {\y}a okazalsa prav. {\Y}esli b{\yi} t{\yi} ne po{\y}avilsa, m{\yi} spravilisy b{\yi} bez teb{\ia}. Prosto vs{\e} stalo b{\yi} gorazdo slojne{\y}e. Kogda na{\c}netsa bogoslujeni{\y}e, lu{\c}xi{\y}e mesta budut otdan{\yi} po{\c}etn{\yi}m jitel{\ia}m goroda i blagorodn{\yi}m, ostalyn{\yi}m pridetsa dovolystvovatsa li{\q}ezreni{\y}em {\c}ujih spin — kardinalyska{\y}a ohrana ne ime{\y}et priv{\yi}{\c}ki puskaty kovo ni popad{\ia}. Persteny — mo{\y} propusk na sam{\yi}{\y} verh. {\C}eloveku s koldovskim darom t{\ia}jelo prohodity {\c}erez stro{\y} klirikov, v otli{\c}i{\y}e ot ob{\yi}{\c}no{\y} straji. A tvo{\y}o koly{\q}o mne v etom pomojet. I kogda {\y}a zaberu kameny, to razve{\y}u volxebstvo — sled angela v kamne is{\c}eznet, i sv{\ia}toxam, pravo, {\y}e{\x}o dolgo budet {\c}em zan{\ia}tsa. Poka oni hvat{\ia}tsa propaji, m{\yi} budem uje daleko.

S tolku on men{\ia} ne sbil. {\Y}a videl, on nedogovariva{\y}et, i znal, {\c}to imenno. Propaja ``{\c}uda", b{\yi}ty mojet, na kako{\y}e-to vrem{\ia} otodvinet obnarujeni{\y}e vnezapnovo is{\c}eznoveni{\y}a kamn{\ia} kardinala, no i tolyko. Vse ravno na{\c}nut iskaty i r{\yi}ty. Ibo sv{\ia}{\x}enniki ne nastolyko kretin{\yi}, {\c}tob{\yi} ne po{\c}uvstvovaty ostatki {\c}ujo{\y} magi{\y}i. A zna{\c}it, ime{\y}etsa lix odin sposob zamesti sled{\yi} — ubity vseh pri{\c}astn{\yi}h.

Teh, kto propustit {\y}evo k kardinalu. Teh, kto uvidit {\y}evo. Nu i men{\ia} zaodno.

{\Y}a znal, {\c}to Valyter opasen, no ne dumal, {\c}to nastolyko. {\Y}evo hladnokrovi{\y}e, besst{\yi}dstvo i umeni{\y}e manipulirovaty l{\iu}dymi b{\yi}li potr{\ia}sa{\y}u{\x}imi. Imenno se{\y}{\c}as {\y}a okon{\c}atelyno utverdilsa v m{\yi}sli, {\c}to kolduna sledu{\y}et ubity srazu, kak tolyko predstavitsa taka{\y}a vozmojnosty, nevajno, kaki{\y}e {\q}eli on presledu{\y}et. Etot {\c}elovek pro{\y}det po golovam i uni{\c}tojit vseh.

On vedet svo{\y}u igru, no vmexiva{\y}et v ne{\y}o Bratstvo. {\Y}esli {\y}a pomogu {\y}emu, {\y}esli u nevo vs{\e} polu{\c}itsa, kliriki budut zl{\yi}. Da net. {\C}to tam! Oni budut v {\y}arosti. I po{\y}dut po lojnomu sledu, kotor{\yi}{\y} privedet ih v Ardenau.

No skazal {\y}a sovsem ino{\y}e:

— {\Y}a pomogu vam.

{\Y}a uvidel, kak oni ojivilisy. Vse, krome Kristin{\yi}, v glazah kotoro{\y} {\c}italosy somneni{\y}e.

— Na neskolykih uslovi{\y}ah.

— Nazovi ih, — predlojila mo{\y}a b{\yi}vxa{\y}a naparni{\q}a.

— {\Y}esli on ho{\c}et polu{\c}ity koly{\q}o, to pusty vernet to, {\c}to sn{\ia}l s mo{\y}evo paly{\q}a v Latke.

Valyter poka{\c}al golovo{\y}:

— Izvini, van Norma{\y}enn, no eto nevozmojno. {\Y}a srazu je {\y}evo uni{\c}tojil — rabota vedym{\yi}, i ono moglo navesti {\y}e{\y}o na nas. Viju, {\c}to se{\y}{\c}as u teb{\ia} tako{\y}e je. B{\yi}ty mojet, {\y}a kak-to mogu kompensirovaty tvo{\y}u poter{\iu}?

Da. Raspahnuty okno i siganuty golovo{\y} vniz.

— Zabudem. — Skrep{\ia} serd{\q}e {\y}a otkazalsa ot svo{\y}e{\y} me{\c}t{\yi}. — Dale{\y}e. {\Y}a idu s tobo{\y}.

— Za{\c}em tebe riskovaty?

{\C}tob{\yi} t{\yi} ne ubil men{\ia} srazu posle tovo, kak vozymex jela{\y}emo{\y}e.

— Potomu {\c}to, {\y}esli kto-to iz vas predstavitsa mno{\y}, {\y}a ne sobira{\y}usy potom rashleb{\yi}vaty {\y}evo oxibki.

— {\C}to-nibudy {\y}e{\x}e?

— Nikako{\y} krovi.

— Nu, razume{\y}etsa. — On skazal eto tak {\c}estno, {\c}to daje {\y}a b{\yi} poveril {\y}emu, {\y}esli b{\yi} ne znal kolduna slixkom horoxo i ne pob{\yi}val v zastenkah zamka Latka.

— Togda net problem. {\C}as pozdni{\y}, gospoda. S vaxevo pozvoleni{\y}a, {\y}a otpravl{\iu}sy domo{\y}.

— T{\yi} legko soglasilsa, van Norma{\y}enn. {\Y}a dumal, pridetsa ubejdaty teb{\ia} do utra.

{\Y}a zametil legku{\y}u usmexku kondotyera. Interesno, kak oni planirovali men{\ia} ubejdaty?

— T{\yi} mne vs{\e} {\y}e{\x}o ne nravixsa, koldun. Kak i vs{\e}, {\c}to proishodit. No {\y}a eto dela{\y}u tolyko radi ne{\y}o. Zapomni eto.

— Vesom{\yi}{\y} argument, — kivnul on. — I {\y}a v nevo ver{\iu}. Uvidimsa zavtra, van Norma{\y}enn.

— {\Y}a provoju, — v{\yi}zvalasy Kristina.

M{\yi} vmeste spustilisy na perv{\yi}{\y} etaj i, ne sgovariva{\y}asy, v{\yi}xli na uli{\q}u.

— {\y}e{\x}o ne pozdno u{\y}ehaty. Pr{\ia}mo se{\y}{\c}as, — vnovy predlojil {\y}a.

Ona izbegala smotrety na men{\ia} i otri{\q}atelyno poka{\c}ala golovo{\y}:

— {\Y}a poter{\ia}la svo{\y} kinjal. I {\c}to samo{\y}e ujasno{\y}e — ni{\c}uty ne jale{\y}u. {\Y}a bolyxe ne straj, Ludwig. Mne nekuda vozvra{\x}atsa.

— Propaja oruji{\y}a ne lixa{\y}et teb{\ia} dara. Daje bez nevo t{\yi} — straj.

Ona neojidanno prislonilasy lbom k mo{\y}e{\y} grudi:

— {\Y}a ustala, Sineglaz{\yi}{\y}. Ustala b{\yi}ty strajem, ustala opravd{\yi}vaty ojidani{\y}a Miriam, ustala spasaty Bratstvo, ot{\c}it{\yi}vatsa pered magistrami, vrajdovaty s zakonnikami i v{\yi}{\c}i{\x}aty vse te gor{\yi} deryma, {\c}to ostavl{\ia}{\y}ut l{\iu}di posle svo{\y}e{\y} smerti, ne jela{\y}a otpravl{\ia}tsa v {\c}istili{\x}e. Vse mo{\y}e su{\x}estvovani{\y}e, skolyko {\y}a seb{\ia} pomn{\iu}, posv{\ia}{\x}eno imenno etomu. Zna{\y}ex, {\c}evo {\y}a ho{\c}u? Poko{\y}a. Sbejaty daleko-daleko, tuda, gde net temn{\yi}h dux i l{\iu}de{\y} s prokl{\ia}ti{\y}em dara, magi{\y}i, i ostavity vs{\e} za spino{\y}, jity svo{\y}e{\y} malenyko{\y} jizn{\y}u, pisaty muz{\yi}ku i rastity dete{\y}. No samo{\y}e straxno{\y}e v etom to, {\c}to mo{\y}i jelani{\y}a ni{\c}evo ne zna{\c}at. V men{\ia} vbili to je samo{\y}e, {\c}to i v teb{\ia}, — spasaty l{\iu}de{\y} i za{\x}i{\x}aty Bratstvo. Temn{\yi}{\y} kuzne{\q} — samo{\y}e opasno{\y}e, s {\c}em m{\yi} stalkivalisy. {\Y}a ob{\ia}zana razobratsa s nim.

B{\yi}lo obidno, {\c}to m{\yi} ponima{\y}em za{\x}itu Bratstva soverxenno po-raznomu. Ona namerena podstavity {\y}evo, {\c}tob{\yi} potom p{\yi}tatsa spasti to, {\c}to uje budet uni{\c}tojeno. {\Y}a gotov predaty {\y}e{\y}o nadejd{\yi}, {\c}tob{\yi} xkola v Ardenau su{\x}estvovala i dalyxe.

— {\C}to je. Eto tvo{\y} v{\yi}bor, — s sojaleni{\y}em skazal {\y}a {\y}e{\y}.

— Posto{\y}! — Ona shvatila men{\ia} za ruku. — Tebe ne ob{\ia}zatelyno u{\c}astvovaty. Pravda. Prosto da{\y} nam koly{\q}o i uhodi. M{\yi} {\c}to-nibudy priduma{\y}em.

— {\Y}a doljen {\c}to-to {\y}e{\x}o znaty?

Ona sdelala xag nazad:

— Mogu govority tolyko za seb{\ia}. Ne ponima{\y}u, kak pri ogromnom ste{\c}eni{\y}i naroda i klirikah mojno ukrasty ve{\x} u kardinala… I bo{\y}usy za tvo{\y}u jizny. Vozmojno, {\y}a oxiba{\y}usy, no skazaty ob etom — pravilyno, i t{\yi} doljen b{\yi}ty v kurse mo{\y}ih razm{\yi}xleni{\y}.

— Do zavtra, Krista. Budy ostorojna s nimi.

Ona ul{\yi}bnulasy:

— Eto im sledu{\y}et b{\yi}ty ostorojn{\yi}mi.

I vernulasy obratno v apteku.

Vstre{\c}aty kardinala Urbana {\y}a planiroval ne v{\yi}bira{\y}asy iz doma. Okna mo{\y}ih komnat v{\yi}hodili na {\q}entralynu{\y}u gorodsku{\y}u uli{\q}u, po kotor{\yi}m doljna pro{\y}ehaty torjestvenna{\y}a pro{\q}essi{\y}a, tak {\c}to {\y}a b{\yi}l obespe{\c}en neplohim zritelyskim mestom.

Pugalu toje b{\yi}lo interesno pogl{\ia}dety. Ono tor{\c}alo zdesy s samovo utra, vpro{\c}em, ne bez dela. V{\c}era ve{\c}erom oduxevl{\e}nn{\yi}{\y} sper v kako{\y}-to lavke derev{\ia}nnu{\y}u marionetku, za {\c}to {\y}emu prixlosy v{\yi}sluxaty nota{\q}i{\y}u ot Propovednika, tak kak ``kako{\y}-to rebenok tepery lixilsa radosti". Po mne, kako{\y}-to rebenok izbejal no{\c}n{\yi}h koxmarov, kukla v{\yi}gl{\ia}dela {\c}udovi{\x}no — groteskna{\y}a kopi{\y}a {\c}eloveka s {\c}uty v{\yi}t{\ia}nuto{\y} golovo{\y}, xiro{\c}enn{\yi}mi rukami i bo{\c}koobrazno{\y} grud{\y}u. Raskraxena ona okazalasy ni{\c}uty ne mene{\y}e bezdarno, {\c}em v{\yi}rezana: glaza razno{\y} veli{\c}in{\yi}, gubki bantikom, soloma vmesto volos.

Pugalo pro{\y}avilo tvor{\c}esku{\y}u jilku, {\c}uty podpraviv vs{\e}, {\c}to nujno, serpom. V itoge igruxka obrela harakternu{\y}u i znakomu{\y}u zlove{\x}u{\y}u uhm{\yi}lo{\c}ku.

— Prelestno, — o{\q}enil Propovednik. — Tepery mojex priv{\ia}zaty k ne{\y} verevo{\c}ki i kor{\c}ity kuklovoda iz Litavi{\y}i.

Pugalo de{\y}stvitelyno priv{\ia}zalo verevo{\c}ku, no lix odnu, i za xe{\y}u. Zatem izvleklo iz karmana mundira klo{\c}ok atlasno{\y} tkani, igolki, nitki i za pol{\c}asa svarganilo odejonku. Ono razve {\c}to ne nasvist{\yi}valo ot udovolystvi{\y}a, naslajda{\y}asy rukodeli{\y}em. Kogda vs{\e} b{\yi}lo gotovo i kukla povisla pod potolkom, Propovednik ostorojno izrek:

— Sv{\ia}to{\y} Vitt i vse {\y}evo pl{\ia}ski! Mne odnomu kajetsa, {\c}to eta xtuka pohoja na kardinala? Ono tepery v{\yi}sunet {\y}e{\y}o v okno i budet mahaty pri vse{\y} {\c}estno{\y} kompani{\y}i?

Na stoly slojn{\yi}{\y} vopros u men{\ia} otveta ne naxlosy. I Propovednik vnes o{\c}eredno{\y}e predlojeni{\y}e:

— K {\c}emu vs{\ia} eta su{\y}eta, Ludwig? V tvo{\y}e{\y} sumke {\q}el{\yi}h dva prokl{\ia}t{\yi}h kamn{\ia}. Na ko{\y} {\c}ert nagrevaty kardinala na {\y}e{\x}o odin?

— Nu, ``nagrety" {\y}evo v{\yi}sokopreosv{\ia}{\x}enstvo {\y}e{\x}o nado sumety. Vpro{\c}em, {\y}a ne nedoo{\q}eniva{\y}u sil{\yi} Valytera. On mojet provernuty ne{\c}to podobno{\y}e.

Propovednik skor{\c}il minu, stav pohojim na zamorsku{\y}u obez{\y}anku:

— T{\yi} sam sebe protivore{\c}ix, Ludwig. To ``ne sume{\y}et", to ``mojet". Po{\c}emu b{\yi} tebe vs{\e}-taki ne otdaty im to, {\c}to u teb{\ia} {\y}esty, i ne vput{\yi}vatsa v seryezn{\yi}{\y}e nepri{\y}atnosti? V Riapano jivut opasn{\yi}{\y}e l{\iu}di. Sto{\y}it li perebegaty im dorogu?

— A t{\yi} zadum{\yi}valsa, {\c}to slu{\c}itsa, kogda zagovor{\x}iki polu{\c}at kamni?

On razvel rukami:

— {\Y}a ne zna{\y}u.

— V tom-to i beda. B{\yi}ty mojet, oni uberut men{\ia} kak lixnevo svidetel{\ia}, b{\yi}ty mojet, Kristinu. {\Y}a horoxo uspel uznaty kolduna. Otdavaty v {\y}evo ruki to, {\c}to on ho{\c}et, — opasno. Pot{\ia}nu vrem{\ia}.

Star{\yi}{\y} pelikan rasse{\y}anno v{\yi}ter krovy so {\x}eki, ot{\c}evo ta ne stala bole{\y}e {\c}isto{\y}:

— Eto neskolyko ne po zakonam Bojyim, i tebe, naverno{\y}e, stranno sl{\yi}xaty podobno{\y}e ot men{\ia}, no, {\y}esli on tak opasen, po{\c}emu b{\yi} prosto vs{\e} ne zaverxity? Vz{\ia}ty pistolet i raznesti {\y}emu golovu? Hot{\ia} b{\yi} za to, {\c}to iz-za nevo t{\yi} provel v podzemno{\y} kamere paru mes{\ia}{\q}ev i {\y}edva ne otdal Bogu duxu.

— Voob{\x}e-to otdal, i, {\y}esli b{\yi} ne Sofi{\y}a, {\y}a b{\yi} s tobo{\y} se{\y}{\c}as ne razgovarival, — napomnil {\y}a {\y}emu. — {\Y}a dumal nad etim, no men{\ia} ostanavliva{\y}et to, {\c}to u nevo {\y}esty {\q}enno{\y}e znani{\y}e o temnom kuzne{\q}e. Stoly vajn{\yi}{\y}e svedeni{\y}a mogut b{\yi}ty polezn{\yi} dl{\ia} Bratstva, a pul{\ia}, drug Propovednik, raz i navsegda postavit to{\c}ku. Ot trupa o{\c}eny t{\ia}jelo {\c}to-nibudy uznaty.

— Da jiv{\yi}m-to on toje tebe mnogo ne rasskajet.

— Posmotrim. {\Y}a duma{\y}u o tom, {\c}to b{\yi} slu{\c}ilosy, {\y}esli b{\yi} Valyter polu{\c}il glaz serafima? Otdal mineral kuzne{\q}u? V{\yi}manil {\y}evo i ubil, kak on govorit? Ili je poznakomilsa i pop{\yi}talsa ispolyzovaty mastera v svo{\y}ih {\q}el{\ia}h? No sam{\yi}{\y} glavn{\yi}{\y} vopros, kotor{\yi}{\y} ne da{\y}et mne poko{\y}a, Propovednik, kak on sv{\ia}jetsa s etim neulovim{\yi}m ne{\y}izvestn{\yi}m {\c}elovekom?

Tot aj podsko{\c}il:

— T{\yi} nameka{\y}ex, {\c}to on zna{\y}et, kuda sledu{\y}et otpravity vesto{\c}ku, {\c}tob{\yi} kuzne{\q} nazna{\c}il vstre{\c}u!

— Imenno.

— I ho{\c}ex v{\yi}sledity {\y}evo sv{\ia}znovo.

— ``Sv{\ia}znovo"? Propovednik, gde t{\yi} usl{\yi}xal eto slovo?

— V bordel{\ia}h, — nebrejno mahnul on. — Tam xpionov ne menyxe, {\c}em xl{\iu}h. Vo vs{\ia}kom slu{\c}a{\y}e, v nekotor{\yi}h. Poro{\y} taki{\y}e razgovor{\yi} vedutsa, {\c}to {\y}a jale{\y}u o svo{\y}e{\y} smerti i o tom, {\c}to nekomu prodaty {\c}uji{\y}e ta{\y}n{\yi}. Da i grehovno eto. Tak nas{\c}et istori{\y}i s Urbanom. Kak {\y}a ponima{\y}u, u teb{\ia} {\y}esty plan?

— {\Y}evo nametki. — {\Y}a posmotrel na kuklu, medlenno krut{\ia}{\x}u{\y}us{\ia} pod potolkom.

— Nu {\y}asno. Iz teb{\ia} slova ne v{\yi}jmex. T{\yi} huje xpiona. Oni hot{\ia} b{\yi} bolta{\y}ut.

{\Y}a rassme{\y}alsa.

— A Kristina? S ne{\y}-to kak? {\Y}a {\c}to-to ne viju, {\c}tob{\yi} t{\yi} spexil vernuty {\y}e{\y} propaju.

— Ona rasstalasy s kinjalom po dobro{\y} vole. Duma{\y}u, {\c}to projivet bez nevo {\y}e{\x}o kako{\y}e-to vrem{\ia}. Do teh por poka vs{\e} ne kon{\c}itsa. Ina{\c}e, bo{\y}usy, vnovy pojertvu{\y}et im radi nepon{\ia}tn{\yi}h v{\yi}sxih {\q}ele{\y}, i r{\ia}dom uje ne okajetsa men{\ia} dl{\ia} tovo, {\c}tob{\yi} vernuty {\y}evo.

— Razumno, hot{\ia} i neskolyko jestoko… O! Kajetsa, na{\c}ina{\y}etsa!

{\Y}a raspahnul okno, tak kak toje usl{\yi}xal trub{\yi} gornistov.

Obe storon{\yi} uli{\q}i b{\yi}li zaprujen{\yi} narodom.

— Kak budto d{\ia}d{\iu}xka tvo{\y}e{\y} vedym{\yi} pri{\y}ehal, a ne kardinal, — usl{\yi}xal {\y}a vozle uha poln{\yi}{\y} skepti{\q}izma golos Propovednika i otvetil:

— Slava ob Urbane bejit vperedi {\y}evo. Mnogi{\y}e s{\c}ita{\y}ut {\y}evo {\y}edva li ne sv{\ia}t{\yi}m, oplotom ver{\yi} i budu{\x}im Papo{\y}. Horoxa{\y}a reputa{\q}i{\y}a, pravilyn{\yi}{\y}e postupki, vsenarodna{\y}a l{\iu}bovy. On lu{\c}xi{\y} pravednik iz teh, {\c}to {\y}esty v {\Q}erkvi na dann{\yi}{\y} moment. Nikakih skandalov, vz{\ia}tok, podkupov, ubi{\y}stv i pro{\c}evo. Lix vera, a realyna{\y}a vera, kak t{\yi} pomnix iz naxe{\y} proxlo{\y} besed{\yi} na etu temu, zaraja{\y}et l{\iu}de{\y}.

— Daje takih, kak t{\yi}? — poddel on men{\ia}.

— L{\iu}b{\yi}h.

Pro{\q}essi{\y}a v{\yi}gl{\ia}dela vnuxitelyno i seryezno. Vperedi marxirovala rota kantonskih na{\y}emnikov, nahod{\ia}{\x}a{\y}as{\ia} na slujbe goroda. V paradn{\yi}h zolotist{\yi}h kirasah, xlemah s pl{\iu}majami, s serebr{\ia}n{\yi}mi alebardami na {\c}ern{\yi}h drevkah. Oni v{\yi}gl{\ia}deli {\y}arko, slovno rojdestvenska{\y}a igruxka, i ih prazdni{\c}na{\y}a odejda silyno otli{\c}alasy ot to{\y} koji, xersti i stali, v kotor{\yi}h oni predpo{\c}itali ne radovaty tolpu, a ubivaty {\y}e{\y}e.

Za roto{\y} na{\y}emnikov sledovala kavalykada po{\c}etn{\yi}h jitele{\y} goroda i blagorodn{\yi}h gospod. V dorogih odejdah iz barhata, mehov{\yi}h xubah, oni p{\yi}talisy pere{\x}egol{\ia}ty drug druga. V{\yi}sxe{\y}e duhovenstvo v{\yi}gl{\ia}delo {\c}uty skromne{\y}e, no nenamnovo. Mestn{\yi}{\y}e kliriki dl{\ia} vstre{\c}i dorogovo gost{\ia} dostali svo{\y}i lu{\c}xi{\y}e nar{\ia}d{\yi}. Kardinal Urban na fone vstre{\c}a{\y}u{\x}e{\y} delega{\q}i{\y}i v{\yi}gl{\ia}del nasto{\y}a{\x}im skromn{\ia}go{\y} — xirokopola{\y}a ala{\y}a xl{\ia}pa, krasna{\y}a sutana iz horoxe{\y} xersti, na ple{\c}ah pelerina iz meha belovo krolika.

{\Y}a vperv{\yi}{\y}e videl {\c}eloveka, kotorovo kogda-to spas. {\Y}emu okazalosy okolo semides{\ia}ti, no dl{\ia} svo{\y}evo vozrasta on otli{\c}no derjalsa v sedle. Osanka, posadka golov{\yi} i to, kak uverenno i spoko{\y}no on pravil jereb{\q}om, govorili o tom, {\c}to sil etomu {\c}eloveku ne zanimaty, nesmotr{\ia} na vnexn{\iu}{\y}u hudobu, beskrovn{\yi}{\y}e gub{\yi} i zapavxi{\y}e glaza. Naverno{\y}e, v molodosti u nevo b{\yi}lo pri{\y}atno{\y}e li{\q}o, no se{\y}{\c}as sozdavalosy vpe{\c}atleni{\y}e, {\c}to {\y}a smotr{\iu} na staru{\y}u hi{\x}nu{\y}u pti{\q}u.

Kardinal to i delo podnimal ruku, blagoslovl{\ia}{\y}a privetstvu{\y}u{\x}ih {\y}evo jitele{\y} goroda.

{\Y}evo v{\yi}sokopreosv{\ia}{\x}enstvo soprovojdala vosymerka gvarde{\y}{\q}ev-alybaland{\q}ev v paradn{\yi}h mundirah i beretah. Vse kak odin svetlovolos{\yi}{\y}e i v{\yi}so{\c}enn{\yi}{\y}e. Takje v svite prisutstvovalo neskolyko klirikov v skromn{\yi}h odejdah, sredi kotor{\yi}h {\y}a zametil odnovo kalikve{\q}a. Za sv{\ia}{\x}ennoslujitel{\ia}mi {\y}ehali slugi. Dva des{\ia}tka {\c}el{\ia}di, na kotor{\yi}h nikto i ne smotrel. A zr{\ia}.

{\Y}a srazu uvidel tovo, kovo jdal. {\C}elovek v belo-kori{\c}nevom pla{\x}e palomnika. Smugloli{\q}i{\y}, temnoglaz{\yi}{\y}, v usah i nepokr{\yi}t{\yi}h volosah bolyxo{\y}e koli{\c}estvo sedin{\yi}, a na pravo{\y} skule v{\yi}del{\ia}lsa zametn{\yi}{\y} izdaleka krestoobrazn{\yi}{\y} xram. On s interesom glazel po storonam i toje po{\c}ti srazu uvidel men{\ia}, napolovinu v{\yi}sunuvxegos{\ia} iz okna. V {\y}evo temn{\yi}h glazah na mgnoveni{\y}e prostupila zolotista{\y}a jeltizna, i v sledu{\y}u{\x}u{\y}u sekundu on smotrel v drugu{\y}u storonu, a {\y}e{\x}o {\c}erez polminut{\yi} uje skr{\yi}lsa za povorotom so vse{\y} kavalykado{\y}.



Valyter prixel v nazna{\c}enno{\y}e vrem{\ia}, v{\yi}n{\yi}rnuv iz uzkovo, propahxevo kr{\yi}sami pereulka. On priodelsa, to{\c}no zajito{\c}n{\yi}{\y} gorojanin, i {\y}evo volos{\yi} i us{\yi} zametno pobeleli.

— Gde koly{\q}o? — sprosil u men{\ia} koldun vmesto privetstvi{\y}a.

— Gde Kristina?

— O ne{\y} ne bespoko{\y}s{\ia}. U ne{\y}o s ostalyn{\yi}mi svo{\y}a zada{\c}a. Ot teb{\ia} mnogo{\y}e ne trebu{\y}etsa, van Norma{\y}enn. Prosto da{\y} mne podobratsa k kardinalu.

— Ostalyn{\yi}{\y}e budut tam je?

— Duma{\y}u, tebe plevaty na ostalyn{\yi}h. Kristin{\yi} na prazdnike net. Ona jdet s loxadymi v uslovlennom meste. {\Y}esli vs{\e} pro{\y}det horoxo, u{\y}edem iz goroda kak mojno b{\yi}stre{\y}e i kak mojno dalyxe. Tebe {\y}a sovetu{\y}u sdelaty to je samo{\y}e, raz uj t{\yi} idex so mno{\y} do kon{\q}a.

— Ne do kon{\q}a, — vozrazil {\y}a. — Lix do tovo mesta, kogda mo{\y}a pomo{\x} bolyxe ne budet nujna Kristine.

On rassme{\y}alsa:

— {\Y}a vsegda znal, {\c}to straji drug za druga gotov{\yi} risknuty golovo{\y}.

— I polyzu{\y}exsa etim.

Koldun otvesil legki{\y} poklon:

— Glupo otri{\q}aty. Idem. U nas menyxe polu{\c}asa.

Uli{\q}i b{\yi}li zaprujen{\yi} narodom, i Propovednik s utra poxutil, {\c}to daje mertv{\yi}{\y}e, {\y}esli b{\yi} oni mogli, spolzlisy b{\yi} s{\iu}da s gorodskih pogostov. {\Y}a ne stal rasstra{\y}ivaty starovo pelikana i govority {\y}emu, {\c}to {\y}avleni{\y}e angela — eto ne bole{\y}e {\c}em falyxivka, sozdanna{\y}a ku{\c}ko{\y} moxennikov.

M{\yi} opazd{\yi}vali, prodira{\y}asy {\c}erez l{\iu}dsko{\y}e stolpotvoreni{\y}e. V kon{\q}e kon{\q}ov Valyteru eto nado{\y}elo, i on vs{\e} je ispolyzoval kako{\y}-to fokus. Gorojane, sto{\y}avxi{\y}e pered nami, nevolyno stali delaty xag v storonu, tolka{\y}a drugih i nastupa{\y}a im na nogi, a m{\yi} vklinivalisy v otkr{\yi}va{\y}u{\x}i{\y}es{\ia} i tut je zahlop{\yi}va{\y}u{\x}i{\y}es{\ia} brexi. Vpro{\c}em, vskore etot effekt propal, koldun, opasa{\y}asy privle{\c} vnimani{\y}e, ostorojni{\c}al, i nam snova prixlosy rabotaty lokt{\ia}mi, kak sam{\yi}m ob{\yi}{\c}n{\yi}m l{\iu}d{\ia}m.

V{\yi}hod na Malu{\y}u Karetnu{\y}u okazalsa perekr{\yi}t {\q}ep{\y}u straji — ona uderjivala svobodnu{\y}u pro{\y}ezju{\y}u {\c}asty dl{\ia} bogat{\yi}h priglaxenn{\yi}h, spexivxih k bogoslujeni{\y}u.

— Prokl{\ia}tye! — zlo rugnulsa Valyter. — Pridetsa iskaty drugo{\y} puty.

— Posto{\y}, — skazal {\y}a, dostav svo{\y} ``propusk", i pokazal kinjal odnomu iz soldat. — M{\yi} iz Bratstva.

Tot daje spority ne stal i, ne trebu{\y}a u kolduna pokazaty oruji{\y}e, otkr{\yi}l nam puty na uli{\q}u.

— I bez vs{\ia}kovo volxebstva, — probormotal sebe pod nos mo{\y} nedrug.

Tepery k gorodsko{\y} sv{\ia}t{\yi}ne prodvigatsa stalo gorazdo leg{\c}e — na pr{\ia}mo{\y} doroge, svobodno{\y} ot l{\iu}de{\y}, m{\yi} lix dvajd{\yi} postoronilisy, propuska{\y}a zapozdavxih blagorodn{\yi}h gospod.

Daleko-daleko gulko udarili {\c}as{\yi} na novo{\y} ratuxe, i ih bo{\y} podhvatili kolokola. Torjestvenna{\y}a messa na{\c}alasy. Ohrana, puskavxa{\y}a na osnovno{\y}e de{\y}stvo, okazalasy kuda bole{\y}e vnuxitelyno{\y} — kantonski{\y}e na{\y}emniki, s nimi tro{\y}e alybaland{\q}ev i para klirikov v prost{\yi}h ser{\yi}h r{\ia}sah. Vse te, kto ne smog popasty na plo{\x}ady, zan{\ia}li sosedni{\y}e uli{\q}i, kr{\yi}xi okrestn{\yi}h domov i daje derev{\y}a.

— Tolyko po priglaxeni{\y}am, — skazal mne {\c}ernoborod{\yi}{\y}, pohoji{\y} na medved{\ia} na{\y}emnik.

— M{\yi} straji, — otvetil {\y}a.

— Viju, {\c}to straji. No prikaz puskaty tolyko po gramotam, na kotor{\yi}h pe{\c}aty burgomistra, — ni{\c}uty ne smutilsa tot. — {\Y}esty gramota?

— {\Y}esty ko{\y}e-{\c}to polu{\c}xe, — otvetil {\y}a i dostal persteny, podarenn{\yi}{\y} mne v Vione.

— Kupity, {\c}to li, ho{\c}ex? — opexil tot.

— Pogodi, dobr{\yi}{\y} {\c}elovek. — Klirik, prisluxivavxi{\y}s{\ia} k naxemu razgovoru, prot{\ia}nul ruku. — Da{\y} posmotrety.

{\Y}a polojil bezdeluxku {\y}emu na ladony.

— {\Y}a li{\c}n{\yi}{\y} duhovnik {\y}evo v{\yi}sokopreosv{\ia}{\x}enstva. I pomn{\iu} eto koly{\q}o. Ono de{\y}stvitelyno kogda-to prinadlejalo {\y}emu. Teb{\ia} priglasil kardinal?

— Net, — ne stal {\y}a lgaty, {\c}uvstvu{\y}a, kak napr{\ia}gs{\ia} Valyter. — No kogda-to {\y}a okazal uslugu {\y}evo v{\yi}sokopreosv{\ia}{\x}enstvu i uveren, {\c}to on pomnit men{\ia}. {\Y}a i mo{\y} drug hotim prisutstvovaty na stoly vajnom bogoslujeni{\y}i.

Klirik kivnul l{\yi}so{\y} golovo{\y}:

— {\C}to je, {\y}a ponima{\y}u vaxe jelani{\y}e. Kto {\y}a tako{\y}, {\c}tob{\yi} mexaty priob{\x}itsa k {\c}udu i Gospodu.

— No priglaxeni{\y}e… — pop{\yi}talsa zaspority podoxedxi{\y} kapitan na{\y}emnikov.

— Etot {\c}elovek sdelal horoxe{\y}e delo dl{\ia} {\Q}erkvi i li{\c}no {\y}evo v{\yi}sokopreosv{\ia}{\x}enstva. V{\yi} jela{\y}ete potom ob{\y}asn{\ia}ty kardinalu, po{\c}emu on ne b{\yi}l dopu{\x}en na bogoslujeni{\y}e?

— Kone{\c}no net!

— Togda propustite ih.

Kantone{\q} neohotno mahnul svo{\y}im l{\iu}d{\ia}m, i te podn{\ia}li alebard{\yi}, otkr{\yi}va{\y}a dorogu.

— Vse kuda leg{\c}e, {\c}em {\y}a ojidal, — usmehnulsa koldun, kogda ohrana ostalasy daleko pozadi.

Propovednik, vs{\e} eto vrem{\ia} to{\c}no teny sledovavxi{\y} za mno{\y}, nakone{\q}-to dal vol{\iu} svo{\y}im emo{\q}i{\y}ami:

— T{\yi} mnogovo ne ojida{\y}ex, {\c}ertov ubl{\iu}dok!

Po s{\c}ast{\y}u, krome men{\ia}, {\y}evo nikto ne sl{\yi}xal.

— {\C}to tepery? — sprosil {\y}a.

— T{\yi} svo{\y}e delo sdelal, van Norma{\y}enn. Mojex naslajdatsa predstavleni{\y}em. A mne nado podobratsa k kardinalu kak mojno blije.

No {\y}a ne dal {\y}emu u{\y}ti, polojiv ruku na ple{\c}o.

— Ne tak b{\yi}stro. T{\yi} prixel vmeste so mno{\y} i u{\y}dex so mno{\y}.

— Kak zna{\y}ex. Tolyko ne mexa{\y}, — legko soglasilsa on.

— Eto b{\yi}lo b{\yi} pro{\x}e osu{\x}estvity, {\y}esli b{\yi} t{\yi} rasskazal, {\c}to sobira{\y}exsa sdelaty.

— Povery, nikto ni{\c}evo ne po{\y}met. M{\yi} u{\y}dem prejde, {\c}em oni zamet{\ia}t, {\c}to {\c}to-to slu{\c}ilosy.

{\Y}evo slova ne vnuxali doveri{\y}a, no {\y}a nade{\y}alsa na koz{\yi}ri v rukave, hot{\ia} ob{\yi}graty op{\yi}tnovo kolduna ne tak prosto.

Usilenn{\yi}{\y} sv{\ia}to{\y} magi{\y}e{\y} golos kardinala raznosilsa nad plo{\x}ad{\y}u. On {\c}ital na staro{\q}erkovnom, {\y}az{\yi}ke vremen Konstantina. L{\iu}di, raspolojivxi{\y}es{\ia} na plo{\x}adi plotno{\y} tolpo{\y}, molilisy, xevel{\ia} gubami i v ob{\x}em-to ne slixkom horoxo vid{\ia}, {\c}to proishodit za spinami vperedi sto{\y}a{\x}ih.

{\Y}a skore{\y}e po{\c}uvstvoval, {\c}em uvidel, kak k nam priso{\y}edinilsa treti{\y} {\c}elovek v kaftane slugi burgomistra i v{\yi}soko{\y} xapke. Sud{\ia} po vsemu, {\C}ezare proniknuty s{\iu}da okazalosy gorazdo leg{\c}e, {\c}em koldunu.

— U men{\ia} vs{\e} gotovo, — xepnul on Valyteru. — Gotthod na meste i jdet tvo{\y}e{\y} komand{\yi}.

— Pozabotsa o svo{\y}e{\y} zada{\c}e, — otvetil tot.

{\Y}a uje, kajetsa, znal, v {\c}em zakl{\iu}{\c}a{\y}etsa rabota na{\y}emnika. Vozle ``sv{\ia}tovo" mesta postavili rasp{\ia}ti{\y}e, r{\ia}dom s nim nahodilosy nebolyxo{\y}e vozv{\yi}xeni{\y}e, s kotorovo v{\yi}stupal kardinal. Vnizu sto{\y}ali predstaviteli duhovenstva i gorodskih vlaste{\y}. Tolpa ostavila svobodn{\yi}m lix nebolyxo{\y} p{\ia}ta{\c}ok plo{\x}adi, to samo{\y}e mesto, gde nahodilsa otpe{\c}atok ``angela".

Kondotyer vnezapno v{\yi}brosil ruku, i {\y}a, ojidavxi{\y} {\c}evo-to podobnovo, blokiroval {\y}e{\y}o predple{\c}yem, ne dav stiletu udarity men{\ia} v gorlo.

V sledu{\y}u{\x}e{\y}e mgnoveni{\y}e v {\q}entre Kruso razverzlasy bezdna i vo{\q}arilsa ognenn{\yi}{\y} ad.



…Kr{\iu}{\c}{\y}a, svisavxi{\y}e s potolka, b{\yi}li v bezobraznom sosto{\y}ani{\y}i. Rjav{\yi}{\y}e, s ostatkami t{\e}mno{\y} ploti na gran{\ia}h, oni smerdeli zastarelo{\y} krov{\y}u i gnil{\yi}m m{\ia}som. To{\c}no tak je pahla i rexetka, na kotoro{\y} zdesy l{\iu}bili podjarivaty teh, kto ne raska{\y}iva{\y}etsa v svo{\y}ih oxibkah.

Nesmotr{\ia} na to {\c}to v bolyxo{\y} jarovne besnovalosy plam{\ia}, v podvalah {\q}entralyno{\y} t{\iu}rym{\yi} Kruso okazalosy jutko holodno.

Master doprosa, nemolodo{\y} jilist{\yi}{\y} subyekt s {\y}arkimi lu{\c}ist{\yi}mi golub{\yi}mi glazami, sn{\ia}l kojan{\yi}{\y} fartuk, per{\c}atki i peredal stalyno{\y} prut odnomu iz pomo{\x}nikov.

— S etim vse.

``Etot" lejal rast{\ia}nut{\yi}m na xiroko{\y} m{\ia}sni{\q}ko{\y} stolexni{\q}e i bolyxe pohodil ne na {\c}eloveka, a na kusok otbivno{\y}. {\Y}a ne prisutstvoval na p{\yi}tke, tak {\c}to s trudom uznal ot{\q}a Gotthoda, kanonika sobora Sv{\ia}to{\y} Mari{\y}i v Braselovette. Po perelomann{\yi}m kone{\c}nost{\ia}m i krovav{\yi}m puz{\yi}r{\ia}m, kotor{\yi}{\y}e naduvalisy i lopalisy u nevo na gubah, b{\yi}lo pon{\ia}tno, {\c}to on ne prot{\ia}net i polu{\c}asa. Vot-vot ispustit duh. Propovednik, uvidev tako{\y}e zreli{\x}e, razvernulsa na kablukah i, ni{\c}evo ne skazav, v{\yi}xel von. On predpo{\c}ital ne smotrety na to, {\c}to {\y}emu b{\yi}lo nepri{\y}atno.

Pugala {\y}a nigde ne videl s momenta sob{\yi}ti{\y} na plo{\x}adi. Vpolne vozmojno, {\c}to {\y}evo sto{\y}it iskaty vozle trupovozok, kotor{\yi}{\y}e se{\y}{\c}as v bolyxom koli{\c}estve sobira{\y}utsa na {\q}entralyno{\y} uli{\q}e.

Roman, mo{\y} star{\yi}{\y} znakom{\yi}{\y}, s kotor{\yi}m {\y}a perejil napadeni{\y}e rugaru v Hrustalyn{\yi}h gorah, privalivxisy k stene, smotrel na umira{\y}u{\x}evo s poln{\yi}m ravnoduxi{\y}em. Kak na tot sam{\yi}{\y} kusok otbivno{\y}, o kotorom {\y}a tolyko {\c}to upom{\ia}nul.

— Zakan{\c}iva{\y} s nim, master.

Pala{\c} vz{\ia}l so stolika xiroki{\y} m{\ia}sni{\q}ki{\y} noj i odnim dvijeni{\y}em prekratil stradani{\y}a umira{\y}u{\x}evo.

— Kardinalyska{\y}a milosty, — ob{\y}asnil mne {\q}igan, to{\c}no opravd{\yi}va{\y}asy.

— T{\yi} privel men{\ia} uvidety {\y}e{\y}e?

— {\Y}evo v{\yi}sokopreosv{\ia}{\x}enstvo platit svo{\y}i dolgi. Hot{\ia} b{\yi} {\c}asty ih.

— Stranno. — {\Y}a posmotrel, kak dva pomo{\x}nika pala{\c}a razreza{\y}ut verevki, st{\ia}giva{\y}u{\x}i{\y}e ruki i nogi trupa, a zatem sbras{\yi}va{\y}ut okrovavlenno{\y}e telo v telejku. — Somneva{\y}usy, {\c}to mne nujna b{\yi}la smerty klirika.

— T{\yi} ne pon{\ia}l. T{\yi} mog okazatsa na etom okrovavlennom stole. Nekotor{\yi}{\y}e iz okrujeni{\y}a Urbana s{\c}ita{\y}ut, {\c}to imenno tam ono i doljno b{\yi}ty. I oni b{\yi}li o{\c}eny ubeditelyn{\yi}.

— I {\y}a s{\c}astlivo izbejal eto{\y} u{\c}asti, potomu {\c}to… — {\Y}a predostavil {\y}emu zakon{\c}ity mo{\y}u frazu.

— Potomu {\c}to m{\yi} znakom{\yi}, van Norma{\y}enn. Potomu {\c}to t{\yi} snova spas jizny kardinalu, i on v dolgu pered tobo{\y}. Potomu {\c}to te, kto hotel rast{\ia}nuty teb{\ia} na d{\yi}be, se{\y}{\c}as uje napravl{\ia}{\y}utsa v N{\y}ugort. Im trebu{\y}etsa poka{\y}ani{\y}e za ih gluposty. A lu{\c}xe vsevo {\y}evo dobitsa, izu{\c}a{\y}a vereskov{\yi}{\y}e pustoxi.

Telo uvezli, ostalsa lix gusto{\y}, lipki{\y} zapah krovi, napolnivxi{\y} holodno{\y}e pome{\x}eni{\y}e.

— Nu, {\c}to je. Peredava{\y} kardinalu mo{\y}i blagodarnosti. Hot{\ia} {\y}a i ne rad, {\c}to {\c}elovek umer.

— Vzdumal seb{\ia} vinity?

— Net. {\Y}a postupil pravilyno. A oni znali, {\c}em risku{\y}ut.

— Vmesto p{\ia}ti soten trupov v Kruso segodn{\ia} b{\yi}lo b{\yi} tri-{\c}et{\yi}re t{\yi}s{\ia}{\c}i, {\y}esli b{\yi} nikto iz nas ne okazalsa gotov k podobnomu povorotu sob{\yi}ti{\y}.

{\Y}a s sodrogani{\y}em vspomnil vzr{\yi}v, {\y}arku{\y}u vsp{\yi}xku i potoki jivovo, pohojevo na zme{\y}, zolotovo ogn{\ia}, rinuvxegos{\ia} vo vse storon{\yi}. Za sekundu on projeg v r{\ia}dah molivxihs{\ia} brex, jgu{\c}im molotom udaril po klirikam i otpr{\ia}nul ot si{\y}a{\y}u{\x}e{\y} svetom pregrad{\yi}. Kupol stremitelyno rvanul vverh i v storon{\yi}, nakr{\yi}v sna{\c}ala plo{\x}ady, zatem okrestn{\yi}{\y}e doma, a potom i uli{\q}i, poglo{\x}a{\y}a v seb{\ia} stranno{\y}e, oslepitelyno-zoloto{\y}e plam{\ia}, ne dava{\y}a {\y}emu rasprostranitsa i navredity {\y}e{\x}o komu-nibudy. No vs{\e} ravno etovo okazalosy nedostato{\c}no, {\c}tob{\yi} spasti vseh.

Paniku{\y}u{\x}a{\y}a tolpa, begu{\x}a{\y}a pro{\c}, vo{\y}u{\x}a{\y}a ot ujasa iz-za tovo, {\c}to kone{\q} sveta, obe{\x}ann{\yi}{\y} angelom, uje nastupil, razlu{\c}ila men{\ia} s {\C}ezare v tot ``{\c}udesn{\yi}{\y}" dl{\ia} nas obo{\y}ih moment, kogda m{\yi} sobiralisy ubity drug druga.

Master doprosa v{\yi}sluxal xepot pomo{\x}nika i soob{\x}il:

— Vtoro{\y} gotov k razgovoru. Smotrety budete?

— Vtoro{\y}? — udivilsa {\y}a. — V{\yi} smogli vz{\ia}ty dvo{\y}ih?

— K sojaleni{\y}u, tolyko dvo{\y}ih. Ih, v otli{\c}i{\y}e ot ostalyn{\yi}h, po{\y}maty okazalosy ne tak uj i slojno. {\Y}a pristavil k nim l{\iu}de{\y} srazu posle naxevo s tobo{\y} razgovora.

M{\yi} proxli v sosedni{\y} zal, zerkalynu{\y}u kopi{\y}u pred{\yi}du{\x}evo, i {\y}a uvidel starovo aptekar{\ia}. On sidel za stolom, {\y}evo ruki i nogi b{\yi}li zafiksirovan{\yi} xirokimi kojan{\yi}mi remn{\ia}mi, a golova zasunuta v mehanizm kra{\y}ne neprigl{\ia}dnovo vida. Tiski plotno ohvat{\yi}vali nijn{\iu}{\y}u {\c}el{\iu}sty i tem{\ia}.

— Horoxo, — pohvalil Roman pala{\c}a i sklonilsa nad Filippom. — Udobno li vam, metr?

Tot ne smog ni{\c}evo otvetity, lix prom{\yi}{\c}al {\c}to-to ne{\c}lenorazdelyno{\y}e i umol{\ia}{\y}u{\x}e{\y}e.

— Duma{\y}u, {\c}to ne slixkom. Naverno{\y}e, stranno spraxivaty tako{\y}e u {\c}eloveka, nahod{\ia}{\x}egos{\ia} v {\c}erepodrobilke. Dava{\y}te {\y}a nemnogo rasskaju vam, {\c}to eto tako{\y}e. Na samom dele, metr, vs{\e} predelyno prosto. Pala{\c} krutit vint, i tiski na{\c}ina{\y}ut sdavlivaty vaxu golovu. Sperva loma{\y}utsa zub{\yi}, a b{\yi}ty mojet, nijn{\ia}{\y}a {\c}el{\iu}sty. {\Y}esli {\c}estno, {\y}a nikogda ne vnikal v detali. Zatem lopnet {\c}erep, no, poveryte, posle etovo v{\yi} projivete dostato{\c}no dolgo, {\c}tob{\yi} pojalety o glupost{\ia}h, kotor{\yi}{\y}e u{\c}inili.

Filipp zaplakal, zam{\yi}{\c}al aktivne{\y}e.

— Mo{\y} vam sovet, aptekary. {\Y}esli hotite izbejaty lixne{\y}… golovno{\y} boli i zaslujity pro{\x}eni{\y}e kardinala, na kotorovo v{\yi} tak bezdarno pokuxalisy, poka{\y}tesy, sozna{\y}tesy i na{\c}ina{\y}te sotrudni{\c}aty. — Roman pohlopal uznika po ple{\c}u. — {\Y}a vernusy {\c}erez pol{\c}asa i s radost{\y}u usl{\yi}xu pravilyn{\yi}{\y} otvet. Idem, Ludwig.

M{\yi} pokinuli p{\yi}to{\c}nu{\y}u, podn{\ia}vxisy na dva etaja vverh, peresekli pusto{\y} t{\iu}remn{\yi}{\y} dvor, minovali ohranu i v{\yi}xli na uli{\q}u. Eto b{\yi}la okra{\y}ina goroda, v dvuh xagah ot vnexne{\y} sten{\yi}.

— Pogovorim. — On zabralsa na goru kirpi{\c}a, svalennovo rabo{\c}imi, sobiravximis{\ia} remontirovaty ukrepleni{\y}e.

— Zdesy? — udivilsa {\y}a, no priso{\y}edinilsa k nemu.

— Vidno vs{\iu} okrugu. Uj lu{\c}xe, {\c}em kabinet na{\c}alynika t{\iu}rym{\yi}, gde kajd{\yi}{\y} durak mojet podsluxaty.

— U vas {\y}esty ofi{\q}ialyna{\y}a versi{\y}a pro{\y}izoxedxevo v gorode?

— A kogda {\y}e{\y}o ne b{\yi}lo? — On usmehnulsa v us{\yi}. — L{\iu}di uje rabota{\y}ut {\y}az{\yi}kami, sluhi odin huje drugovo let{\ia}t po dorogam da mnojatsa v traktirah. Za mes{\ia}{\q} uzna{\y}ut vse i vezde. Eto slu{\c}ilosy o{\c}eny ne vovrem{\ia}. Hot{\ia} kogda podobn{\yi}{\y}e sob{\yi}ti{\y}a voob{\x}e mogut b{\yi}ty k mestu? Nam pridetsa prilojity massu usili{\y}, {\c}tob{\yi} sgladity posledstvi{\y}a pro{\y}isxedxevo.

— Nam?

— {\Y}a sluju kardinalu, a on — {\Q}erkvi. Tak {\c}to v dannom kontekste daje koldunu i rugaru mojno govority ``nam". Naskolyko {\y}a pon{\ia}l, vseh sobak poves{\ia}t na zlobnovo d{\y}avolopoklonnika, kotor{\yi}{\y} hotel isportity radosty veru{\y}u{\x}im, oskvernity sv{\ia}t{\yi}n{\iu}, popraty zakon{\yi} Bojyi i pro{\c}a{\y}a, pro{\c}a{\y}a, pro{\c}a{\y}a.

{\Y}a ojidal {\c}evo-to podobnovo:

— V ob{\x}em, skazka o {\c}udovi{\x}e, rexivxem ustro{\y}ity bessm{\yi}slenno{\y}e ubi{\y}stvo.

— Ili kako{\y}-nibudy jertvenn{\yi}{\y} ritual. L{\iu}d{\ia}m soverxenno neza{\c}em znaty nasto{\y}a{\x}ih pri{\c}in.

— A m{\yi} ih zna{\y}em? Eti sam{\yi}{\y}e pri{\c}in{\yi}?

Roman ne otvetil, i eto govorilo o tom, {\c}to on, kak i {\y}a, ter{\ia}{\y}etsa v dogadkah.

— S{\c}ita{\y}ex, {\c}to koldun urovn{\ia} Valytera sposoben ustro{\y}ity ognenn{\yi}{\y} vulkan v {\q}entre goroda? — zadal {\y}a {\y}e{\x}o odin vopros.

On posmotrel na men{\ia} dolgim, t{\ia}jel{\yi}m vzgl{\ia}dom:

— I t{\yi} duma{\y}ex to{\c}no tak je. Zapomni eto, {\y}esli dorojix svo{\y}e{\y} xkuro{\y}. — I, {\c}uty sm{\ia}g{\c}ivxisy, proizn{\e}s: — {\Y}esty ve{\x}i, o kotor{\yi}h ne sto{\y}it boltaty, Ludwig. Naprimer, {\y}a stara{\y}usy pomalkivaty, {\c}to polna{\y}a luna ne slixkom horoxo vli{\y}a{\y}et na men{\ia} v posledne{\y}e vrem{\ia}.

M{\yi} oba usmehnulisy tolyko nam pon{\ia}tno{\y} xutke.

— Tak {\c}to vinovat Valyter. Poka ne budet dokazano obratno{\y}e. A ono, kak t{\yi} ponima{\y}ex, dokazano, skore{\y}e vsevo, ne budet. Vo vs{\ia}kom slu{\c}a{\y}e, tolpe.

— No t{\yi} ne verix, {\c}to koldun markgrafa Valentina ime{\y}et stolyko sil.

— Razve eto tak vajno? — On ustalo poter veki.

— Dl{\ia} men{\ia} — vajno.

{\Q}{\yi}gan sdalsa:

— Ne ver{\iu}. {\Y}a zna{\y}u o temnom iskusstve ne ponasl{\yi}xke. Na mo{\y} vzgl{\ia}d, podobno{\y}e ne mojet provernuty nikto iz teh, s kem {\y}a znakom. Zdesy nujna mo{\x} velefa ili kakovo-nibudy legendarnovo {\c}arode{\y}a iz Temnoles{\y}a. Ili o{\c}eny seryeznovo demona. Takovo, kto odnim {\x}el{\c}kom paly{\q}ev sjiga{\y}et p{\ia}ty soten dux i {\y}edva ne prolam{\yi}va{\y}et {\x}it des{\ia}ti gotov{\yi}h k otrajeni{\y}u ataki klirikov. Koldun{\yi} na tako{\y}e ne sposobn{\yi}. Ina{\c}e mirom pravili b{\yi} oni, a ne kn{\ia}z{\y}a i {\q}erkovniki.

— {\Y}esli on b{\yi}l nastolyko silen, {\c}to ne bo{\y}alsa klirikov, to po{\c}emu otstupil? Po{\c}emu ne ubil vseh, kto nahodilsa na plo{\x}adi?

— {\Y}a ne zna{\y}u.

{\Y}a zadumalsa.

— No {\y}esli eto b{\yi}l ne Valyter, a kto-to ino{\y}, to slu{\c}ivxe{\y}es{\ia} na plo{\x}adi — sovpadeni{\y}e? Komanda zagovor{\x}ikov, p{\yi}tavxihs{\ia} ubity kardinala, i ne{\y}izvestn{\yi}{\y}, rexivxi{\y} ustro{\y}ity lokalyn{\yi}{\y} apokalipsis?

— Inovo ob{\y}asneni{\y}a u men{\ia} net. I ne tolyko u men{\ia}. Inkvizi{\q}i{\y}a {\c}exet v zat{\yi}lke, razvodit rukami i lihorado{\c}no lista{\y}et grimuar{\yi}. Tolyko duma{\y}etsa mne, {\c}to bez tolku eto vse. Nikovo m{\yi} ne na{\y}dem. No im nujen kozel otpu{\x}eni{\y}a. Lu{\c}xe vsevo tot hagjit, o kotorom t{\yi} govoril. Otli{\c}na{\y}a kandidatura dl{\ia} kostra. Ob{\yi}vateli ne jalu{\y}ut {\c}ujezem{\q}ev i inover{\q}ev, s radost{\y}u sojgut zlobno{\y}e {\c}udovi{\x}e, voshit{\ia}tsa tem, {\c}to vozmezdi{\y}e {\Q}erkvi nastiglo prestupnika, i uspoko{\y}atsa. Nu a {\y}esli ne po{\y}ma{\y}em etovo hagjita, na{\y}dem drugovo. Ili je pridetsa ispolyzovaty bedn{\ia}gu-aptekar{\ia}. Mor{\x}ixsa? Ne sto{\y}it. Eto zvu{\c}it jestoko, no ina{\c}e prosto nelyz{\ia}. Simvol{\yi} ver{\yi} i sil{\yi} doljn{\yi} ostavatsa nez{\yi}blem{\yi}. Ina{\c}e na{\c}netsa haos.

— Kogda v{\yi} pon{\ia}li, {\c}to sled angela — falyxivka?

Roman slojil uzlovat{\yi}{\y}e paly{\q}i v kulak, poter im zat{\yi}lok:

— Slixkom pozdno dl{\ia} tovo, {\c}tob{\yi} vs{\e} ostanovity. Sluhi poleteli, a palomniki i mestn{\yi}{\y}e kliriki-tupi{\q}i uje skola{\c}ivali krest da jgli sve{\c}i. Perv{\yi}{\y} je inkvizitor s magi{\y}e{\y} vs{\e} raskusil.

— Valytera nelyz{\ia} nazvaty na{\y}ivn{\yi}m. On ne veril, {\c}to {\y}evo obman proderjitsa dolgo, zna{\c}it, znal, {\c}to v{\yi} ne ostanovite predstavleni{\y}e.

— U nas ne b{\yi}lo v{\yi}bora. {\C}to m{\yi} doljn{\yi} b{\yi}li delaty? Ob{\y}avity vo vseusl{\yi}xani{\y}e, {\c}to ku{\c}ka moxennikov rexila naduty veru{\y}u{\x}ih? {\C}to nikakovo {\c}uda net, a poslednevo angela videli poltor{\yi} t{\yi}s{\ia}{\c}i let nazad? Za{\c}em rubity suk, na kotorom sidix, Ludwig? L{\iu}di hot{\ia}t verity v {\c}udesa, i kto m{\yi} taki{\y}e, {\c}tob{\yi} razruxaty ih ill{\iu}zi{\y}i?

{\Y}a lix neveselo rassme{\y}alsa:

— Mne povezlo, Roman. Mo{\y}a rabota gorazdo pro{\x}e.

— A mo{\y}a ne jdet. — On vstal i prot{\ia}nul ruku. — T{\yi} spas mne jizny v teh prokl{\ia}t{\yi}h gorah, i {\y}a ne l{\iu}bl{\iu} b{\yi}ty doljen. {\Y}a sobira{\y}u sluhi i informa{\q}i{\y}u. Stara{\y}a ferma v {\c}etverti ligi za gorodom, {\y}esli {\y}ehaty {\c}erez Koxa{\c}yi vorota. {\Y}a smogu priderjivaty etu novosty paru-tro{\y}ku {\c}asov, ne bolyxe. Uvezi jen{\x}inu kak mojno dalyxe, {\y}esli ne ho{\c}ex, {\c}tob{\yi} ona sgnila v podvale.

— Oni budut preduprejden{\yi}. Vse.

— I {\c}to s tovo? M{\yi} ih po{\y}ma{\y}em. No {\y}esli straja ne budet sredi nih, {\y}e{\y}o nikto ne stanet iskaty. Obe{\x}a{\y}u.

— Spasibo, Roman. M{\yi} v ras{\c}ete.

— Bo{\y}usy, {\c}to net. Eto b{\yi}la usluga za uslugu.



Ferma v{\yi}gl{\ia}dela zabroxenno{\y}, no v {\y}e{\y}o okoxkah gorel prigluxenn{\yi}{\y} svet.

Nebo b{\yi}stro temnelo.

{\Y}a napravilsa k vorotam, duma{\y}a, {\c}to snova de{\y}stvu{\y}u naobum, pr{\yi}ga{\y}u v omut golovo{\y} i Valyteru v ob{\x}em-to ni{\c}evo ne sto{\y}it sdelaty to, {\c}to ne polu{\c}ilosy u na{\y}emnika.

{\Y}a zametil, kak drognula zanaveska — nabl{\iu}davxi{\y} za dorogo{\y} uvidel men{\ia}. Tem lu{\c}xe.

Vorota b{\yi}li ne zapert{\yi}. {\Y}a voxel vo dvor — gr{\ia}zn{\yi}{\y}, s dvum{\ia} ogromn{\yi}mi lujami i telego{\y}, perevernuto{\y} nabok. Dvery v dom otkr{\yi}lasy, i na poroge po{\y}avilsa {\C}ezare. Posmotrel na men{\ia} t{\ia}jel{\yi}m vzgl{\ia}dom, ne ubira{\y}a ruki s vis{\ia}{\x}e{\y} na po{\y}ase dagi:

— T{\yi} odin?

— Kak vidix.

On postoronilsa i, kogda {\y}a voxel, ostalsa na uli{\q}e, rexiv proverity, ne prixel li za mno{\y} kto-to {\y}e{\x}e.

Vnutri pahlo star{\yi}m zaplesnevevxim domom, pol b{\yi}l zeml{\ia}no{\y}. Bolyxu{\y}u {\c}asty {\y}edinstvenno{\y} komnat{\yi} zanimal ost{\yi}vxi{\y} o{\c}ag. Skudna{\y}a krest{\y}anska{\y}a mebely, a takje pr{\ia}lka okazalisy sdvinut{\yi} k dalyne{\y} stene. V uglu, zavernuvxisy v tonko{\y}e ode{\y}alo, spal hagjit. Koldun, do etovo {\c}to-to pisavxi{\y}, tepery smotrel na men{\ia}.

Kristina por{\yi}visto brosilasy ko mne, krepko obn{\ia}la:

— {\Y}a dumala, t{\yi} pogib! Valyter skazal, {\c}to tam, gde t{\yi} sto{\y}al, nikto ne v{\yi}jil!

B{\yi}vxi{\y} sluga markgrafa Valentina vstretil mo{\y} krasnore{\c}iv{\yi}{\y} vzgl{\ia}d s ponima{\y}u{\x}e{\y} ul{\yi}bko{\y}:

— Kak vidno, {\y}a oxibalsa. Etot {\C}ezare ve{\c}no vs{\e} puta{\y}et.

— I nedodel{\yi}va{\y}et. Oni p{\yi}talisy izbavitsa ot men{\ia}.

— {\C}to?! — Ona gnevno nahmurila brovi, rezko povernuvxisy k Valyteru. — T{\yi} je obe{\x}al!

— E{\y}! E{\y}! Uspoko{\y}s{\ia}.

— Uspoko{\y}itsa?! — razozlilasy Kristina. — T{\yi} sukin s{\yi}n, Valyter! Ne tolyko zavalil delo, no i hotel ubity tovo, kto soglasilsa pomo{\c} nam!

— Eto b{\yi}la ini{\q}iativa {\C}ezare. {\Y}a ne daval {\y}emu takih raspor{\ia}jeni{\y}. Kl{\ia}nusy!

Ona xagnula k nemu s {\y}avn{\yi}m namereni{\y}em udarity, no {\y}a vz{\ia}l {\y}e{\y}o za predple{\c}ye:

— M{\yi} uhodim. Pr{\ia}mo se{\y}{\c}as.

— Neujeli? — vkrad{\c}ivo skazal on, polojil pero na stol i vstal.

{\Y}a vosprin{\ia}l eto kak ugrozu i opustil ruku na kinjal.

— Duma{\y}ex, amulet tvo{\y}e{\y} vedym{\yi} spaset teb{\ia} ot magi{\y}i?

— Proverim?

— Prekratite! Oba! — kriknula Kristina, razbudiv Adil{\ia}. — I tak vs{\e} ploho. Stolyko mes{\ia}{\q}ev rabot{\yi} nasmarku! M{\yi} ne zna{\y}em, gde ote{\q} Gotthod i Filipp, oni do sih por ne prixli. Xans v{\yi}{\y}ti na kuzne{\q}a poter{\ia}n! A v{\yi} duma{\y}ete tolyko o tom, kak pustity drug drugu krovy!

Valyter mirol{\iu}bivo podn{\ia}l ruki vverh i vnovy ugnezdilsa na stule:

— Prejde {\c}em t{\yi} primex rexeni{\y}e, uzna{\y} u van Norma{\y}enna, kak tot naxol nas. On, naverno{\y}e, veliki{\y} volxebnik, raz okazalsa zdesy. Vedy t{\yi} {\y}emu ne govorila o naxem ubeji{\x}e?

— Ne govorila. — Ona voprositelyno posmotrela na men{\ia}. — Tak kak, Ludwig?

Etot ubl{\iu}dok {\c}udesno perevel razgovor na drugu{\y}u temu.

— Nevajno. {\Y}a zdesy. A u vas malo vremeni. Nado v{\yi}vezti teb{\ia} ots{\iu}da, Kristina. Ostanexsa s nim — propadex.

— Pff! — Koldun provorno na{\c}al sobiraty ve{\x}i, a britogolov{\yi}{\y} hagjit zastegnul na s{\iu}rtuke po{\y}as s krivo{\y} sable{\y}.

{\Y}a po{\c}uvstvoval, kak zat{\yi}lok ukololo {\c}to-to holodno{\y}e.

— {\C}ezare! — kriknula Kristina. — Sto{\y}!

— {\Y}a ne sluxa{\y}u tvo{\y}ih prikazov, jen{\x}ina. Valyter? — Kondotyer podobralsa ko mne soverxenno nezametno.

— Ostavy {\y}evo, — poprosil koldun, okon{\c}atelyno sobrav sumku. — Nam on ne nujen. Za{\c}em rasstra{\y}ivaty Kristinu? Pora u{\y}ezjaty. I na{\c}inaty vs{\e} sna{\c}ala.

— Ne budet nikakovo na{\c}ala. T{\yi} uje pro{\y}igral.

— T{\yi} vs{\e}-taki prosto tupo{\y} gromila, alybalande{\q}. U men{\ia} ne b{\yi}lo v{\yi}bora. Jizny {\q}elovo mira postavlena na kartu. {\Y}a poxel b{\yi} daje v pasty drakona, {\y}esli b{\yi} eto priblizilo men{\ia} k temnomu kuzne{\q}u. Kak t{\yi} ne po{\y}mex taku{\y}u prostu{\y}u ve{\x}: vs{\e}, {\c}to {\y}a govoril tebe o temn{\yi}h kinjalah, — pravda. {\Y}a na{\y}du nov{\yi}{\y} glaz serafima. I poprobu{\y}u snova. A t{\yi} mojex bejaty v Ardenau, zar{\yi}ty golovu v pesok i dumaty, {\c}to {\y}a lgu, raz tebe tak leg{\c}e.

— Tak i postupl{\iu}. No sna{\c}ala zaberu {\y}e{\y}e.

— {\Y}a ne po{\y}edu s tobo{\y}, Ludwig, — tiho skazala Kristina.

— Ho{\c}ex ostatsa s nimi? Zna{\y}ex, {\c}to slu{\c}itsa, kogda teb{\ia} po{\y}ma{\y}ut? Straj budet obvinen v zagovore protiv Riapano. Eto udarit po Bratstvu. Po kajdomu iz nas, gde b{\yi} m{\yi} ni nahodilisy. Uedem, poka ne pozdno. Ostavy ih. T{\yi} doljna jity, a ne umerety na d{\yi}be. Eto posledni{\y} xans.

Kristina vz{\ia}la iz ruk Valytera svo{\y}u kurtku, nadela, droja{\x}imi paly{\q}ami, nemnogo nelovko, zastegnula pugovi{\q}i:

— Tebe {\y}a to{\c}no ni{\c}evo ne doljna, van Norma{\y}enn. Vse, {\c}to b{\yi}lo mejdu nami v dalekom proxlom, tepery ne ime{\y}et zna{\c}eni{\y}a. U men{\ia} svo{\y} puty, a u teb{\ia} svo{\y}. Uhodi, Ludwig. Pr{\ia}mo se{\y}{\c}as. I bolyxe ne i{\x}i men{\ia}.

{\Y}a posmotrel {\y}e{\y} v glaza, pon{\ia}l, {\c}to vs{\e} bespolezno, {\c}to {\y}a ne smogu pereubedity {\y}e{\y}e, {\c}to Kristinu, {\c}eloveka, s kotor{\yi}m {\y}a kogda-to u{\c}ilsa, ros, jil i srajalsa ple{\c}om k ple{\c}u, uje ne vernex. I v{\yi}xel iz doma, plotno prikr{\yi}v za sobo{\y} dvery…

{\Y}a xel po temnomu pustomu traktu k Kruso, a na duxe u men{\ia} skrebli koxki.

{\C}to je. {\Y}a hot{\ia} b{\yi} pop{\yi}talsa. Ona prin{\ia}la rexeni{\y}e. V{\yi}brala svo{\y}u sudybu, svo{\y}u jizny, svo{\y}u {\q}ely. I bessm{\yi}slenno lovity rukami uskolyza{\y}u{\x}u{\y}u teny. Tratity vrem{\ia} i sil{\yi}. {\Y}a uznal otvet{\yi} na vopros{\yi}, kotor{\yi}{\y}e men{\ia} volnovali, i tepery sledu{\y}et dvigatsa dalyxe. Idti vpered i ne ogl{\ia}d{\yi}vatsa.

Vperedi pokazalisy dva znakom{\yi}h silueta. Odin v{\yi}so{\c}enn{\yi}{\y} i dolgov{\ia}z{\yi}{\y}, drugo{\y} nev{\yi}soki{\y} i suhonyki{\y}.

— Ne v{\yi}xlo? — negromko sprosil Propovednik. — Po{\c}emu?

— {\Y}a ne mogu spasti {\y}e{\y}o ot samo{\y} seb{\ia}, druji{\x}e.

— T{\ia}jelo vvesti l{\iu}de{\y} v {\Q}arstvi{\y}e Nebesno{\y}e, {\y}esli oni ne jela{\y}ut spaseni{\y}a, — probormotal on i skazal kuda grom{\c}e: — No tepery {\Q}erkovy na{\y}det {\y}e{\y}o i nakajet kak zagovor{\x}i{\q}u.

— Ili ne na{\y}det, {\y}esli uda{\c}a budet na {\y}e{\y}o storone.

{\Y}a sunul ruki v karman{\yi}, xaga{\y}a dalyxe, i oni pristro{\y}ilisy r{\ia}dom.

— I {\c}to tepery? — ne v{\yi}derjal Propovednik.

— Za{\y}musy delami. V mire polno temn{\yi}h dux. Syezju v Ardenau. {\Y}a ne b{\yi}l na rodine neskolyko let. Vstre{\c}usy s Gertrudo{\y}. Rasskaju obo vsem, {\c}to zdesy pro{\y}izoxlo, stare{\y}xinam. Bratstvo doljno b{\yi}ty gotovo k nepri{\y}atnost{\ia}m.

M{\yi} so star{\yi}m pelikanom sdelali {\y}e{\x}o neskolyko xagov, prejde {\c}em pon{\ia}li, {\c}to Pugalo nas ne soprovojda{\y}et. Ono sto{\y}alo na doroge, v{\yi}t{\ia}nuvxisy v strunku, to{\c}no teryer, po{\c}u{\y}avxi{\y} lisu, i smotrelo tuda, otkuda {\y}a prixel.

— E{\y}, Solomenna{\y}a golova! — okliknul {\y}evo Propovednik. — Zab{\yi}lo, kuda nado idti? E-e{\y}! M{\yi} zdesy. Iesuse Hriste, t{\yi} ne tolyko onemelo, no i oglohlo?!

— Pogodi, — nahmurilsa {\y}a i podoxel k Pugalu.

Ono melko drojalo, i v uzkih glazah to zagoralisy, to gasli dva malenykih ugolyka.

— {\C}to tam? {\C}to t{\yi} vidix?

Ono polojilo mne na ple{\c}o t{\ia}jelu{\y}u, kostl{\ia}vu{\y}u ruku i razvernulo, predlaga{\y}a smotrety ne na nevo, a na mra{\c}nu{\y}u dorogu, barhatno{\y}e zvezdno{\y}e nebo i temn{\yi}{\y}e siluet{\yi} derevyev, v{\yi}stupa{\y}u{\x}i{\y}e na etom fone. Vokrug b{\yi}la posledn{\ia}{\y}a no{\c} zim{\yi}, stranna{\y}a to{\y} zlove{\x}e{\y} tixino{\y}, kotora{\y}a zastiga{\y}et odinokovo putnika na pust{\yi}nnom trakte. {\Y}a, kajetsa, ne d{\yi}xal, vmeste s Pugalom smotr{\ia} vo mrak. I tot otvetil mne.

Zolotisto{\y} iskro{\y}. Zoloto{\y} vsp{\yi}xko{\y}. Zolot{\yi}m svetom.

Ogony {\q}veta jidkovo zolota podn{\ia}lsa v{\yi}xe drevesn{\yi}h kron i tut je opal, ostaviv v nebe zoloto{\y}e zarevo.

— O, Gospodi! — ahnul Propovednik.

No {\y}a uje ne sluxal {\y}evo. Bejal obratno.



Zolot{\yi}{\y}e kostr{\yi}, taki{\y}e tepl{\yi}{\y}e, prekrasn{\yi}{\y}e, pohoji{\y}e ne na ob{\yi}{\c}n{\yi}{\y} ogony, a na rasplavlenn{\yi}{\y} drago{\q}enn{\yi}{\y} metall, goreli povs{\iu}du. V lesu, na ogromnom pustom pole i tam, gde {\y}e{\x}o sovsem nedavno sto{\y}ala stara{\y}a ferma.

Ih b{\yi}lo neskolyko des{\ia}tkov, haoti{\c}n{\yi}h, razbrosann{\yi}h po okruge, soverxenno nevero{\y}atn{\yi}h. Volxebn{\yi}h. I smertelyno opasn{\yi}h.

Eto b{\yi}lo to je plam{\ia}, {\c}to svirepstvovalo na plo{\x}adi v Kruso, puska{\y} i mene{\y}e {\y}arostno{\y}e. Ono gorelo, popira{\y}a vse zakon{\yi} mirozdani{\y}a, samo po sebe, ne nujda{\y}asy v toplive i ne zavis{\ia} ot kaprizov vetra.

Pugalo ne stalo podhodity k blija{\y}xemu kostru, a ostanovilosy kak vkopanno{\y}e i, kazalosy, n{\iu}halo vozduh, pahnu{\x}i{\y} t{\ia}jelo{\y} gar{\y}u i, {\c}uty ulovimo, perejarenn{\yi}m m{\ia}som. Zatem ono opustilo golovu, ssutulilosy i s nekotor{\yi}m razo{\c}arovani{\y}em selo na zeml{\iu}. Na {\y}evo vzgl{\ia}d, tut uje ne b{\yi}lo ni{\c}evo interesnovo.

Propovednik ne poxel so mno{\y} po ino{\y} pri{\c}ine — on bo{\y}alsa, hot{\ia} ni odin ogony ne mog pri{\c}inity duxe vreda.

— Mojet, ne sto{\y}it tebe tuda lezty?! — kriknul on mne v spinu.

No {\y}a ne mog postupity ina{\c}e.

Pervo{\y}e telo, obuglenno{\y}e do golovexek, vs{\e} {\y}e{\x}o d{\yi}m{\ia}{\x}e{\y}es{\ia}, {\y}a naxol r{\ia}dom s mertv{\yi}mi loxadymi. Lix po krivo{\y} polose metalla, v kotoro{\y} trudno b{\yi}lo opoznaty sabl{\iu}, {\y}a pon{\ia}l, {\c}to eto hagjit.

V dom {\y}a za{\y}ti ne smog, tot vs{\e} {\y}e{\x}o pol{\yi}hal, poetomu napravilsa ot kostra k kostru, po v{\yi}jjenno{\y} zemle.

I {\y}edva ne spotknulsa o trup {\C}ezare. On lejal na jivote, i v {\y}evo spine b{\yi}la projjena skvozna{\y}a d{\yi}ra veli{\c}ino{\y} s dva mo{\y}ih kulaka. Glaza okazalisy raspahnut{\yi}, na li{\q}e zast{\yi}li udivleni{\y}e i obida.

{\Y}a vs{\e} dalyxe othodil ot ferm{\yi}, prodolja{\y}a iskaty, i v glazah postepenno na{\c}inalo dvo{\y}itsa ot zolot{\yi}h ogne{\y}. Ih b{\yi}lo kuda bolyxe, {\c}em mne pokazalosy vna{\c}ale.

{\Y}a b{\yi} proxel mimo, {\y}esli b{\yi} ona men{\ia} ne okliknula. {\y}e{\y}o li{\q}o po{\c}ernelo ot kopoti, prava{\y}a ruka napominala obgorevxu{\y}u vetku, a na to, {\c}to b{\yi}lo nije grudi, nelyz{\ia} smotrety bez slez — odin sploxno{\y} ojog.

Ona pop{\yi}talasy ul{\yi}bnutsa, pokazaty, {\c}to vs{\e} horoxo, no polu{\c}ilosy eto nevajno. Kristina spl{\iu}nula temno-kori{\c}nevu{\y}u sl{\iu}nu, {\y}a o{\x}util pr{\ia}n{\yi}{\y}, {\y}edki{\y} zapah i pon{\ia}l, {\c}to ona tolyko {\c}to syela koreny zolotovo lyva, silyn{\yi}{\y} hagjitski{\y} narkotik, izbavl{\ia}{\y}u{\x}i{\y} ot l{\iu}bo{\y} boli.

— Ne povezlo, — tolyko i skazala ona. — M{\yi} iskali {\y}evo, a on naxol nas.

B{\yi}lo pon{\ia}tno, o kom ona govorit.

— T{\yi} videla temnovo kuzne{\q}a?

— Izdali. — Straj uronila golovu na zeml{\iu}. — {\Y}a ne smogu tebe pomo{\c}. Poobe{\x}a{\y} mne sdelaty ko{\y}e-{\c}to.

— Obe{\x}a{\y}u.

— Otpravl{\ia}{\y}s{\ia} v Klagenfurt. Tam jivet do{\c} Valytera. Uli{\q}a Sten{\yi}. U ne{\y}o dar. {\Y}a pokl{\ia}lasy {\y}emu, {\c}to Bratstvo {\y}e{\y}o primet. Ne perebiva{\y}. Sluxa{\y}.

Ona vz{\ia}la iz raspotroxenno{\y} sumki {\y}e{\x}o odin koreny, otpravila {\y}evo sebe za {\x}eku:

— Pod polom v {\y}evo dome {\y}esty kniga. Sojgi {\y}e{\y}e. Eto vajno. Sdela{\y}ex?

— Da.

— Vozymi sebe V{\y}una. Bolyxe {\y}a nikomu {\y}evo ne dover{\iu}.

— Horoxo.

{\Y}a videl, kak mutne{\y}ut {\y}e{\y}o glaza ot narkotika, i predstavl{\ia}l, kaku{\y}u boly ona doljna isp{\yi}t{\yi}vaty se{\y}{\c}as.

— Tretye. Ne i{\x}i temnovo kuzne{\q}a. Ina{\c}e on pridet i za tobo{\y}. Kak prixel za nami. Pokl{\ia}nisy!

— Kl{\ia}nusy. — Na etot raz {\y}a lgal.

Ona ustalo zakr{\yi}la glaza i skazala {\c}uty zapleta{\y}u{\x}imsa {\y}az{\yi}kom:

— I skaji Miriam: mne ujasno jaly, {\c}to {\y}a {\y}e{\y}o podvela.

— Eto ne tak. No {\y}a peredam.

{\Y}a dostal iz sumki kinjal, vlojil v {\y}e{\y}o ruku:

— Prosti, {\c}to ne smog sdelaty etovo ranyxe.

Po {\y}e{\y}o {\x}eke sbejala odinoka{\y}a slezinka:

— {\Y}a ne ponima{\y}u…

— Uje nevajno, Kristina. Glavno{\y}e, {\c}to tvo{\y} klinok tepery s tobo{\y}. T{\yi} ne poter{\ia}la {\y}evo.

Ona ul{\yi}bnulasy prizrakom svo{\y}e{\y} proxlo{\y} ul{\yi}bki, xepnula:

— Spasibo. V tom monast{\yi}re, gde pogib Gans… Tam kuzne{\q}, {\c}to ku{\y}et nam kinjal{\yi}. Etu ta{\y}nu on uznal, i poetomu {\y}evo ubili. Ne govori Miriam, horoxo? {\Y}e{\y} ne sto{\y}it znaty. — Kristina prervalasy, provaliva{\y}asy v zab{\yi}tye, no s usili{\y}em zakon{\c}ila: — Ina{\c}e budet beda. Dl{\ia} vseh nas. {\Y}a pospl{\iu} nemnogo. Razbudix men{\ia} k utru?

— Kone{\c}no. Ni o {\c}em ne volnu{\y}s{\ia}, — skazal {\y}a, no ne b{\yi}l uveren, {\c}to ona men{\ia} {\y}e{\x}o sl{\yi}xit.

{\Y}a sidel r{\ia}dom s ne{\y}, o{\x}u{\x}a{\y}a to{\c}no taku{\y}u je zlu{\y}u bespomo{\x}nosty, kak kogda umirala Hanna. {\Y}a ni{\c}evo ne mog dl{\ia} ne{\y}o sdelaty.

Tolyko b{\yi}ty r{\ia}dom…

\end{document}
