\documentclass[10pt]{article}
\usepackage[a4paper, margin=2cm]{geometry}
\usepackage{fontspec}
\setmainfont{Linux Libertine O}
\begin{document}

\newcommand{\e}{ë}
\newcommand{\yi}{\mbox{y\hspace{-0.55pt}ı}}
\newcommand{\ia}{\mbox{ı\hspace{-0.55pt}a}}
\newcommand{\io}{\mbox{ı\hspace{-0.55pt}o}}
\newcommand{\y}{y̆}
\newcommand{\Y}{Y̆}

\newcommand{\yf}{y̆}

\newcommand{\X}{X̹}
\newcommand{\x}{x̹}
\newcommand{\C}{C̹}
\renewcommand{\c}{c̹}

Sootvetstviye s cyrilli{\c}ey:

\setlength{\tabcolsep}{2pt}
\begin{tabular}{c c c @{\hspace{1cm}} c c c @{\hspace{1cm}} c c c @{\hspace{1cm}} c c c @{\hspace{1cm}} c c c @{\hspace{1cm}} c c c @{\hspace{1cm}} c c c }
\textit{c} &---& \textit{ч}    & \textit{x} &---& \textit{ш}    & \textit{{\y}} &---& \textit{й} & \textit{{\io}} &---& \textit{ю} & \textit{{\ia}} &---& \textit{я} & \textit{e} &---& \textit{э} & \textit{{\e}}  &---& \textit{ё} \\
\textit{{\c}} &---& \textit{ц} & \textit{{\x}} &---& \textit{щ} & \textit{y} &---& \textit{ь}    & \textit{{\yi}} &---& \textit{ы} & \textit{j} &---& \textit{ж}     & \textit{h} &---& \textit{х} \\
\end{tabular}

\noindent V ostalynom sootvetstviye standartnoye.

Glasnyiye \textit{{\e}}, \textit{{\io}}, \textit{{\ia}}, \textit{{\yi}} upotrebliayutsa tolyko posle soglasnyih.
V otliciye ot ih cyrilliceskih ekvivalentov bukvyi \textit{{\e}}, \textit{{\io}}, \textit{{\ia}}
ispolyzuyutsa tolyko dlia oboznaceniya miagkosti predidu{\x}ih soglasnyih.
V nacale slov i posle glasnyih pixem \textit{{\y}o}, \textit{{\y}u}, \textit{{\y}a}.
Takje i bukva \textit{e} ne upotrebliayetsa v nacale slov i posle soglasnyih za redkimi isklioceniyami
(\textit{eto, poetomu}), vmesto etovo pixem \textit{{\y}e}.
Bukva \textit{ъ} ne imeyet ekvivalenta vovse tak kak yeyo funk{\c}iya ne trebuyetsa.
Naprimer: 
\textit{{\y}abloko, pri{\y}atn{\yi}{\y}, {\y}olka, {\y}ujn{\yi}{\y}, {\y}esli, hoz{\ia}{\y}in, v zdani{\y}i, ot{\y}ehaty, ob{\y}om, v{\y}uga}.

Dopuskayetsa sleduyu{\x}eye upro{\x}eniye:
    \textit{{\y}}  $\rightarrow$ \textit{y},
    \textit{{\e}}  $\rightarrow$ \textit{e},
    \textit{{\ia}} $\rightarrow$ \textit{ia},
    \textit{{\io}} $\rightarrow$ \textit{io},
    \textit{{\yi}} $\rightarrow$ \textit{yi}.
Naprimer, sravnite

\noindent\textit{Pod kop{\yi}ta mo{\y}e{\y} loxadi brosilsa ni{\x}i{\y}, vop{\ia}, cto gr{\ia}d{\e}t kone{\c} sveta i {\y}a doljen poka{\y}atsa i otdaty {\y}emu vse denygi. Odnako, pon{\ia}v, cto {\y}a ne otlica{\y}usy osobo{\y} nabojnost{\y}u, on tut je zab{\yi}l obo mne i pristal k dvum dorodn{\yi}m kup{\c}am, kotor{\yi}{\y}e b{\yi}li neskolyko perepugan{\yi} tem bezumi{\y}em, cto proishodilo vokrug.}

\noindent s

\noindent\textit{Pod kopyita moyey loxadi brosilsa ni{\x}iy, vopia, cto griadet kone{\c} sveta i ya doljen pokayatsa i otdaty yemu vse denygi. Odnako, poniav, cto ya ne otlicayusy osoboy nabojnostyu, on tut je zabyil obo mne i pristal k dvum dorodnyim kup{\c}am, kotoryiye byili neskolyko perepuganyi tem bezumiyem, cto proishodilo vokrug.}


% "А", "Б", "В", "Г", "Д", "Е", "Ё", "Ж", "З", "И", "Й", "К", "Л", "М", "Н", "О", "П", "Р", "С", "Т", "У", "Ф", "Х", "Ц", "Ч", "Ш", "Щ", "Ъ", "Ы", "Ь", "Э", "Ю", "Я"
% "а", "б", "в", "г", "д", "е", "ё", "ж", "з", "и", "й", "к", "л", "м", "н", "о", "п", "р", "с", "т", "у", "ф", "х", "ц", "ч", "ш", "щ", "ъ", "ы", "ь", "э", "ю", "я"

\end{document}