\documentclass[12pt]{book}
\usepackage{fontspec}
\setmainfont{Linux Libertine O}
\begin{document}

\newcommand{\X}{X̹}
\newcommand{\x}{x̹}
\newcommand{\C}{C̹}
\renewcommand{\c}{c̹}

\newcommand{\e}{ë}
\newcommand{\yi}{\mbox{y\hspace{-0.55pt}ı}}
\newcommand{\ia}{\mbox{ı\hspace{-0.55pt}a}}
\newcommand{\io}{\mbox{ı\hspace{-0.55pt}o}}
\newcommand{\y}{y̆}
\newcommand{\Y}{Y̆}

Superm{\yi}x metnulasy vverh, za{\c}epilasy zadnimi lapami za potolocnu{\y}u balku i povisla vniz go\-lo\-vo{\y} rovno v {\c}entre poli{\c}e{\y}skovo ucastka. Ona vsegda bez\-oxi\-boc\-no opredel{\ia}la {\c}entr l{\io}bovo prostranstva i predpocitala visety imenno v n{\e}m. Xesty casov. Rassvet. Vrem{\ia} utrenne{\y} raboce{\y} p{\ia}\-ti\-mi\-nut\-ki. A potom mojno budet pozvolity sebe paru casov sna. Dvuh casov sna v sutki Superm{\yi}xi b{\yi}lo vpolne dostatocno.

Ona b{\yi}stro prot{\e}rla kra{\y}exkom rukokr{\yi}la echolokator{\yi} v nosu i vo rtu — vne vs{\ia}kovo somneni{\y}a, oni b{\yi}li bezuprecno cist{\yi}mi, no nekotor{\yi}{\y}e ve{\x}i dela{\y}ex prosto reflektorno, dl{\ia} por{\ia}dka. Postucala kogotkom po uxn{\yi}m ra{\c}i{\y}am, skorcila grimasu — rabota{\y}ut. Ogl{\ia}dela podcin{\e}nn{\yi}h — sotrudnikov poli{\c}i{\y}i Dalynevo Lesa. Neopr{\ia}tn{\yi}{\y}e, nem{\yi}t{\yi}{\y}e, nepovorotliv{\yi}{\y}e, vecno sonn{\yi}{\y}e zveri. Nicevo. Ona {\y}e{\x}o naved{\e}t v etom ucastke por{\ia}dok.

Barsuk Starxi{\y}, Skvorconok i Grif Sterv{\ia}tnik rasselisy po rabocim mestam i zadrali golov{\yi}. Barsuk Starxi{\y} pot{\ia}nul nosom. Polumrak ucastka b{\yi}l propitan gust{\yi}m i kisl{\yi}m zapahom limonov, potno{\y} xersti, sval{\ia}vxihsa per{\y}ev i zverskovo neu{\y}uta. Neu{\y}utom pahlo ot samo{\y} Superm{\yi}xi, tak dumal Barsuk. Vse drugi{\y}e svo{\y}i prirodn{\yi}{\y}e zapahi ona t{\x}atelyno unictojala, obrabat{\yi}vala seb{\ia} kon{\c}entratom cistotela do polne{\y}xe{\y} sterilynosti — ctob{\yi} b{\yi}ty nezametno{\y}, neulovimo{\y}, nepobedimo{\y}… No vot etot {\y}e{\y}o neu{\y}ut cistotelom, ocevidno, ne v{\yi}vodilsa. Etot zapah prixol vmeste s ne{\y} v perv{\yi}{\y} je deny ih sovmestno{\y} rabot{\yi}.

V tot sam{\yi}{\y} deny, kogda on {\y}e{\x}o ne podozreval, cto {\y}emu tepery prid{\e}tsa sjiraty po d{\io}jine limonov v sutki, ctob{\yi} ne spaty, potomu cto eta fanaticka ne spala pocti nikogda.

V tot sam{\yi}{\y} deny, kogda u nih v ucastke votsarilsa etot vecn{\yi}{\y} nocno{\y} polumrak — na solnecn{\yi}{\y} svet u Superm{\yi}xi b{\yi}la allergi{\y}a.

V tot sam{\yi}{\y} deny, kogda on {\y}e{\x}o naivno scital, cto ona stanet {\y}evo nov{\yi}m naparnikom. Ona b{\yi}stro postavila {\y}evo na mesto: ``{\Y}a — nacalynik, v{\yi} — podcin{\e}nn{\yi}{\y}, takovo raspor{\ia}jeni{\y}e sverhu".

Ili, mojet b{\yi}ty, neu{\y}utom pahlo ot plotn{\yi}h xtor iz medvej{\y}e{\y} xkur{\yi}, ko\-to\-r{\yi}\-mi ona zavesila okna. Oni toje po{\y}avilisy vmeste s ne{\y} v perv{\yi}{\y} deny…

Superm{\yi}x raskacivalasy pod potolkom, ritmicno jestikuliru{\y}a ru\-ko\-kr{\yi}\-l{\y}a\-mi i bezzvucno xevel{\ia} xiroko razinut{\yi}m rtom. Kajetsa, ona b{\yi}la nedovolyna, zlilasy i cto-to im v{\yi}govarivala. V{\yi}gl{\ia}delo eto bezmolvno{\y}e v{\yi}stupleni{\y}e jutko, kak ob{\yi}cno. I, kak ob{\yi}cno, Barsuk Starxi{\y} pocustvoval, cto nacina{\y}et bolety zat{\yi}lok.

Barsuk Starxi{\y} prikr{\yi}l glaza. On vdrug vspomnil ih prejni{\y}e utrenni{\y}e p{\ia}\-ti\-mi\-nut\-ki. Zdesy b{\yi}l svet. Zdesy b{\yi}li solnecn{\yi}{\y}e luci. I {\y}e{\x}o zdesy b{\yi}l Barsukot, kotor{\yi}{\y} igral s lucami… A tepery — nicevo. Ni Barsukota, ni soln{\c}a, ni radosti.

Boly v za\-t{\yi}l\-ke usililasy.

Barsuk Starxi{\y} t{\ia}jelo vzdohnul i rexilsa:

— Izvinite, spe{\c}agent Superm{\yi}x. Nicevo ne sl{\yi}xno.

Superm{\yi}x na sekundu zast{\yi}la s razinut{\yi}m rtom. Potom skorcila pre\-zri\-tely\-nu\-{\y}u grimasu — na li{\c}e {\y}e{\y}o pocti vsegda b{\yi}la kaka{\y}a-nibudy grimasa, — i skazala svo{\y}im pronzitelyn{\yi}m golosom:

— {\Y}a op{\ia}ty zab{\yi}la, cto v{\yi} ne vosprinima{\y}ete rec na v{\yi}sokih castotah.

Barsuk Starxi{\y} hotel otvetity, cto {\y}e{\y}o golos i na dostupn{\yi}h im castotah prakticeski nev{\yi}nosim, no promolcal.

— Povtor{\ia}{\y}u dl{\ia} nepon{\ia}tliv{\yi}h, — skazala Superm{\yi}x. — Spisok vseh pti{\c} Dalynevo Lesa mne na stol!

— Tak vedy m{\yi} uje vcera… — nacal b{\yi}lo Barsuk.

— Vcera spisok b{\yi}l ne po forme! — srezala {\y}evo Superm{\yi}x. — {\Y}a je cotko skazala. Nomer pti{\c}i — poroda pti{\c}i — dlina mahov{\yi}h per{\y}ev — okras — im{\ia}.

— No, pozvolyte, u pti{\c} Dalynevo Lesa net nomerov, — skazal Grif Sterv{\ia}tnik.

— Ne pozvol{\io}, — holodno otozvalasy Superm{\yi}x. — Net nomerov — pronumeru{\y}te.

— Ne pozvol{\io}, — pecalyno otozvalsa Skvorconok.

— I {\y}e{\x}o m{\yi} ne zna{\y}em dlinu vseh mahov{\yi}h per{\y}ev u vseh pti{\c}, potomu cto…

— Neiteresno, — oborvala {\y}evo Superm{\yi}x. — Ne zna{\y}ete — izmeryte i budete znaty. I {\y}e{\x}o. Ot kajdo{\y} pti{\c}i mne nujno pero. Pero — sveje{\y}e, ne obron{\e}nno{\y}e god nazad, a tolyko cto v{\yi}rvanno{\y}e. Eto {\y}asno?

— V{\yi} predlaga{\y}ete nam v{\yi}diraty iz pti{\c} per{\y}a? — izumilsa Barsuk Starxi{\y}. — Pti{\c}i i tak napugan{\yi} tem, cto…

— Da, predlaga{\y}u. Eto v interesah sledstvi{\y}a. V{\yi} je hotite po{\y}maty opasnovo man{\y}aka, o{\x}ip{\yi}va{\y}u{\x}evo pti{\c} i sjiga{\y}u{\x}evo ih per{\y}a, ne tak li? Teoreticeski l{\io}bu{\y}u pti{\c}u mogut o{\x}ipaty. Pod ugrozo{\y} — vse, daje v{\yi}, — ona strelynula glaza\-mi v Sterv{\ia}tnika i Skvorconka. — Znacit, ot kajdo{\y} pti{\c}i trebu{\y}etsa pero, na sluca{\y} opoznani{\y}a. Vse per{\y}a doljn{\yi} b{\yi}ty pronumerovan{\yi} i vnesen{\yi} v registr.

— Pod ugrozo{\y} — vse, — pisknul Skvorconok.

— Proxu pro{\x}eni{\y}a. V{\yi} vedy ne ime{\y}ete v vidu, cto iz men{\ia}, sotrudnika poli{\c}i{\y}i Dalynevo Lesa, toje nujno v{\yi}rvaty pero? — utocnil Grif Sterv{\ia}tnik.

— Iz men{\ia}? Iz men{\ia}? — zavolnovalsa Skvorconok.

— Imenno eto {\y}a i ime{\y}u v vidu. Iz vas i iz sotrudnika Skvor{\c}a. Net i ne mojet b{\yi}ty nikakih iskl{\io}ceni{\y}. Nikakih slujebn{\yi}h privilegi{\y} i jalkovo kumovstva. Dale{\y}e. Do sih por ne ustanovleno mestonahojdeni{\y}e glavnovo podozreva{\y}emovo, b{\yi}vxevo sotrudnika poli{\c}i{\y}i Dalynevo Lesa, mladxevo Barsuka Poli{\c}i{\y}i — Barsukota.

— M{\yi} pereda{\y}om orientirovku po kornevizoru kajd{\yi}{\y} deny, — neohotno proizn{\e}s Barsuk Starxi{\y}. — M{\yi} obe{\x}a{\y}em voznagrajdeni{\y}e v t{\yi}s{\ia}cu xixe{\y} tomu, kto na{\y}d{\e}t i sdast v poli{\c}i{\y}u {\X}ipaca. M{\yi} obe{\x}a{\y}em oplatity obed v bare ``Sucok" l{\io}bomu, kto soob{\x}it nam informa{\c}i{\y}u, kasa{\y}u{\x}u{\y}usa {\X}ipaca.

— No effekt — nulevo{\y}. Skolyko raz novosti o mestonahojdeni{\y}i Barsukota prinosila nam na hvoste eta vaxa Soroka?

— Tri raza.

— Skolyko raz ona za eto poobedala v bare ``Sucok"?

— Tri raza.

— I skoly\-ko raz v ukazann{\yi}h Soroko{\y} mestah obnarujivalsa Barsukot?

— Ni razu…

— O com eto govorit?

— O tom, cto Soroka — nenad{\e}jn{\yi}{\y} istocnik? — predpolojil Barsuk Starxi{\y}.

— Libo o tom, cto Barsukot — vsegda na xag vperedi. V l{\io}bom sluca{\y}e, Barsukota do sih por ne naxli. Eto vozmutitelyno i nedopustimo. Etovo ne po{\y}mut naverhu. 

Naverhu. Barsuka Starxevo pered{\e}rnulo. ``Verhom" Superm{\yi}x uporno ime\-no\-va\-la svo{\y}o nacalystvo — Lasku, kotora{\y}a god nazad stala vojakom So{\y}uza Smexann{\yi}h Lesov, i {\y}e{\y}o sekretar{\ia} Golubcika. S Lasko{\y} Barsuk Starxi{\y} znakom ne b{\yi}l — ona b{\yi}la dl{\ia} nevo zverem, kak skazala b{\yi} Superm{\yi}x, ``slixkom v{\yi}\-so\-ko\-vo pol{\e}ta". Pogovarivali, cto Laska oceny laskova{\y}a, ul{\yi}bciva{\y}a i puxista{\y}a, no za nepovinoveni{\y}e inogda jr{\e}t svo{\y}ih podcin{\e}nn{\yi}h, spe{\c}agentov kr{\yi}s i letucih m{\yi}xe{\y}, poetomu, ne\-smo\-tr{\ia} na {\y}e{\y}o m{\ia}gki{\y} character, oni {\y}e{\y}o uvaja{\y}ut. A vot s Golubcikom Starxi{\y} imel od\-naj\-d{\yi} besedu. Merzki{\y}, samodovolyn{\yi}{\y}, nadut{\yi}{\y} tip s hoholkom. Za spino{\y} u nevo bezliko{\y} sero{\y} ten{\y}u — telohranitely: malenyki{\y}e bessm{\yi}slenn{\yi}{\y}e glazki, ostr{\yi}{\y}e rjav{\yi}{\y}e kl{\yi}ki, izviva{\y}u{\x}i{\y}sa cer\-v{\e}m hvost… Barsuk Starxi{\y} togda p{\yi}talsa ugovority Golubcika prislaty kakovo-to drugovo spe{\c}agenta vzamen Superm{\yi}xi. Potomu cto Superm{\yi}x srazu, v perv{\yi}{\y} je deny ih sovmestno{\y} rabot{\yi}, naznacila Barsukota glavn{\yi}m po\-do\-zre\-va\-{\y}e\-m{\yi}m po delu o Serom {\X}ipace, otkaz{\yi}valasy sluxaty argument{\yi}, a glavno{\y}e, soverxenno ne interesovalasy vnutrennim zverskim cut{\y}om Barsuka Starxevo. A vnutrenne{\y}e zversko{\y}e cut{\y}o podskaz{\yi}valo {\y}emu, cto b{\yi}vxi{\y} {\y}evo naparnik Barsukot — ne {\X}ipac. Ne man{\y}ak. Hot{\ia} on i ugrojal o{\x}ipaty star{\y}ov{\x}ika s{\yi}ca Uga. Hot{\ia} on i sbejal, kak tolyko stalo izvestno, cto s{\yi}ca de{\y}stvitelyno o{\x}ipali. ``Barsukot — on zvery, bezuslovno, derzki{\y}, emo{\c}ionalyn{\yi}{\y}, por{\yi}vist{\yi}{\y}, — ob{\y}asn{\ia}l Barsuk Starxi{\y} Golubciku. — On sposoben na neobdumann{\yi}{\y}e postupki. On sposoben na ugroz{\yi} i hamstvo. No on dobr{\yi}{\y} i smel{\yi}{\y}. U nevo {\y}esty pocotna{\y}a gramota za otvagu, podpisanna{\y}a samim Tsar{\e}m Zvere{\y}. I sodraty s jivo{\y} pti{\c}i vse per{\y}a — na eto Barsukot ne sposoben!". — ``A cevo eto v{\yi} {\y}evo tak v{\yi}gorajiva{\y}ete? — prokurl{\yi}kal v otvet Golubcik. — Podozritelyno. Vesyma podozritelyno. Spe{\c}agent Rukokr{\yi}la{\y}a Superm{\yi}x osta{\y}otsa na etom dele. A v{\yi} — v {\y}e{\y}o podcineni{\y}i. {\Y}e{\y}o lini{\y}a rassledovani{\y}a odobrena zdesy u nas, naverhu. Sama Laska {\y}e{\y}o odobrila". — ``V takom sluca{\y}e, {\y}a hotel b{\yi} po\-go\-vo\-rity s Lasko{\y}", — skazal Barsuk Starxi{\y}. Golubcik zalilsa kurl{\yi}ka{\y}u{\x}im smehom i zatr{\ia}s go\-lo\-vo{\y}, budto podavilsa hlebno{\y} gorbuxko{\y}: ``Gospoja Laska — oceny zan{\ia}to{\y} zvery. Ona ne stanet govority s kakim-to tam barsukom. Skajite spasibo, cto vas udostoil besed{\yi} {\y}a, {\y}e{\y}o sekretary". — ``{\Y}a ne kako{\y}-to tam barsuk, a Starxi{\y} Barsuk Poli{\c}i{\y}i Dalynevo Lesa. {\Y}esli Laska ne mojet so mno{\y} govority, {\y}a trebu{\y}u vstreci s Tsar{\e}m Zvere{\y}!" — ``Eto so Lyvom, cto li? Da na zdorov{\y}e. On star{\yi}{\y}, bolyno{\y} i v marazme. Tsar Zvere{\y} — davno uje pusta{\y}a formalynosty. Vsemi lesami pravit cestno izbrann{\yi}{\y} Vojak — Laska. V prejni{\y}e vremena takih vot dr{\ia}hl{\yi}h tsare{\y}-zvere{\y} stalkivali so skal{\yi} ili zakl{\e}v{\yi}vali nasmerty. No gospoja Laska laskova i dobra. Tak cto Lev u nas s pocotom otstran{\e}n ot del i otpravlen na pensi{\y}u". — ``Ne tako{\y} uj on dr{\ia}hl{\yi}{\y}, — vozrazil Barsuk Starxi{\y}. — M{\yi} so Lyvom odnovo vozrasta!" — ``Znacit, vam toje pora so skal{\yi}. To {\y}esty, prostite, na pensi{\y}u. A uj naskolyko s pocest{\ia}mi, budet zavisety ot rezulytatov n{\yi}nexnevo rassledovani{\y}a. A se{\y}cas — izvinite, {\y}a oceny zan{\ia}t. Mo{\y} ohrannik pomojet vam na{\y}ti v{\yi}hod". Navsegda zapomnit Barsuk nejivo{\y}, t{\ia}jol{\yi}{\y} vzgl{\ia}d golubcikova telohranitel{\ia} i prikosnoveni{\y}e {\y}evo kogtisto{\y}, holodno{\y} lap{\yi}…

— Ne spim! — istericeski vzvizgnula Superm{\yi}x i ot vozmu{\x}eni{\y}a perexla na ultrazvuk.

Barsuk Starxi{\y} vzdrognul i ocnulsa ot zadumcivosti. Treugolyna{\y}a pasty Superm{\yi}xi bezzvucno xevelilasy i kla{\c}ala, zaostr{\e}nn{\yi}{\y} narost na razdvo{\y}ennom nosu ritmicno podragival. R{\yi}lokr{\yi}la{\y}a. Tak oni so Skvorconkom i Grifom {\y}e{\y}o prozvali.

— Nicevo ne sl{\yi}xno, spe{\c}agent Rukokr{\yi}la{\y}a Superm{\yi}x, — ustalo skazal Barsuk.

Superm{\yi}x rezko zahlopnula rot, razjala paly{\c}i zadnih lap i nacala padaty. Vot k etomu Barsuk Starxi{\y} nikak ne mog priv{\yi}knuty. K eto{\y} {\y}e{\y}o manere vzletaty ne vverh, a vniz, a potom, v posledni{\y} moment, kogda kajetsa, cto cerez sekundu ona votkn{\e}tsa svo{\y}im razdvo{\y}enn{\yi}m r{\yi}ly{\c}em v zeml{\ia}no{\y} pol, vzm{\yi}vaty pod kakim-to soverxenno nepredskazu{\y}em{\yi}m i dikim uglom.

Superm{\yi}x s minutu pometalasy po ucastku bezzvucno{\y} sero{\y} ten{\y}u, potom uspoko{\y}ilasy, snova povisla v {\c}entre potolka i prodoljila utrenn{\io}{\y}u p{\ia}\-ti\-mi\-nut\-ku:

— Cto tam u nas po pervomu postradavxemu?

— O{\x}ipann{\yi}{\y} s{\yi}c-star{\y}ov{\x}ik po-prejnemu lejit v kome, — otcitalsa Barsuk.

— V kome cevo?

— V{\yi} je zna{\y}ete.

— Otveca{\y}em po forme! — vzvizgnula Superm{\yi}x.

— S{\yi}c Chuck lejit v kome sobstvenn{\yi}h casticno sojjonn{\yi}h per{\y}ev uje dva mes{\ia}{\c}a, s teh por, kak {\y}evo o{\x}ipali, — spo\-ko{\y}\-no otvetil Barsuk Starxi{\y}. — Grac Vrac do sih por uporno boretsa za {\y}evo jizny.

— Pusty boretsa dalyxe. I pusty kajd{\yi}{\y} deny massiru{\y}et {\y}emu kl{\io}v, ctob{\yi} on smog govority, {\y}esli prid{\e}t v seb{\ia}. Pokazani{\y}a pervovo postradavxevo budut crezv{\yi}ca{\y}no vajn{\yi}… Cto po ostalyn{\yi}m postradavxim?

— Brat-blizne{\c} s{\yi}ca Chucka, s{\yi}c Ug, kotor{\yi}{\y} takje podvergsa napadeni{\y}u {\X}ipaca, no smog otbitsa i sbejaty, po-prejnemu nahoditsa v Ohotkah pod za{\x}ito{\y} ohotnic{\y}ih psov. O{\x}ipanno{\y}e kr{\yi}lo uje nacalo pokr{\yi}vatsa nov{\yi}m opereni{\y}em, kotoro{\y}e daje...

— Neiteresno. Cto po kukuxke?

— Madam Kuku, o{\x}ipanna{\y}a dve nedeli nazad, vcera, na\-ko\-ne{\c}, soglasilasy daty pokazani{\y}a.

— Po\-ce\-mu v{\yi} mne ne soob{\x}ili? — vzvizgnula Superm{\yi}x.

— Potomu cto pokazani{\y}a madam Kuku dala… nemnojko ne po delu. Ona {\y}avno do sih por v glubokom xoke.

— Cto ona skazala?

— Vs{\e} vrem{\ia}, cto m{\yi} s ne{\y} vcera proveli, ona posto{\y}anno otscit{\yi}vala, skoly\-ko ostalosy jity mne i Grifu Sterv{\ia}tniku. Na drugi{\y}e tem{\yi} s ne{\y} po\-go\-vo\-rity ne udalosy.

— I skoly\-ko je ostalosy jity?

— Sterv{\ia}tniku {\y}e{\x}o dolgo. Tocno otscitaty ona ne smogla, sbivalasy, no {\y}avno do gluboko{\y} starosti.

— A vam?

— A mne, po {\y}e{\y}o mneni{\y}u, ostalosy ne tak mnogo. Neskolyko dne{\y}.

Skvorconok vstopor{\x}il per{\y}a i trevojno ustavilsa na Barsuka Starxevo.

— Neskolyko — eto skoly\-ko konkretno? — ravnoduxno utocnila Superm{\yi}x.

— Nu… konkretno — tri dn{\ia}. No m{\yi} je s vami zveri obrazovann{\yi}{\y}e? — golos Barsuka zvucal ne slixkom uverenno. — M{\yi} je ne verim vo vse eti kukuxkin{\yi} skazki?

— Po statistike predskazani{\y}a kukuxek otnositelyno prodoljitelynosti jizni sb{\yi}va{\y}utsa v dev{\ia}nosta dev{\ia}ti sluca{\y}ah iz sta, — holodno soob{\x}ila Superm{\yi}x. — I, mejdu procim, net nikakih ``nas s vami". {\Y}esty v{\yi} — ob{\yi}cn{\yi}{\y}e lesn{\yi}{\y}e zveri, i {\y}esty {\y}a — Rukokr{\yi}la{\y}a Superm{\yi}x, leta{\y}u{\x}i{\y} spe{\c}ialyn{\yi}{\y} agent s v{\yi}\-da\-{\y}u\-{\x}i\-misa supersposobnost{\ia}mi… Tak, v{\yi} men{\ia} sbili s m{\yi}sli. Vern{\e}msa k postradavxe{\y} kukuxke. S rodstvennikami ona toje ne id{\e}t na kontakt? Tolyko predskaz{\yi}va{\y}et?

— U Madam Kuku net rodstvennikov.

— Cto, sovsem?

— Ona podkid{\yi}x. Pri{\y}omna{\y}a maty, strijiha, davno uje umerla. {\Y}esty svod\-n{\yi}{\y} brat, Strij-parikmaher, no on s detstva {\y}e{\y}o ne prizna{\y}ot.

— Ne uzna{\y}ot?

— Ne prizna{\y}ot. Ne sci\-ta\-{\y}et svo{\y}e{\y} sestro{\y} i ne hocet delity s ne{\y} rodovo{\y}e gnezdo. Posle soverxonnovo na kukuxku napadeni{\y}a on ne iz{\y}avil jelani{\y}a {\y}e{\y}o navestity. Ona toje o n{\e}m ne spraxivala.

— L{\io}bop{\yi}tno, — Superm{\yi}x pokacalasy iz storon{\yi} i v storonu, {\y}e{\y} tak legce dumalosy. — Oceny l{\io}bop{\yi}tno. Svodn{\yi}{\y} brat… Starxi{\y} Barsuk Poli{\c}i{\y}i! Ot\-prav\-l{\ia}{\y}\-tesy k Striju-parikmaheru. Oprosite {\y}evo. Vozymite pod nabl{\io}deni{\y}e. Predostavyte {\y}emu kruglosutocnovo ohrannovo psa — po vozmojnosti ne samovo tupovo, ume{\y}u{\x}evo derjatsa v teni i ne la{\y}aty popustu. Bolyxinstvo man{\y}akov, s ko\-to\-r{\yi}\-mi {\y}a imela delo, isp{\yi}t{\yi}vali ostr{\yi}{\y} interes k rodn{\yi}m i blizkim svo{\y}ih jertv. Nax {\X}ipac, sud{\ia} po vsemu, ne iskl{\io}ceni{\y}e. On uje od\-naj\-d{\yi} napal na brata-blizne{\c}a o{\x}ipannovo s{\yi}ca. Skore{\y}e vsevo, on {\y}avitsa i k bratu Madam Kuku. Daje {\y}esli eto ne krovn{\yi}{\y} brat. Grif Sterv{\ia}tnik i Skvore{\c} Poli{\c}i{\y}i v eto vrem{\ia} za{\y}mutsa spiskom pti{\c}. Zadani{\y}e vsem pon{\ia}tno?

— Tak tocno, — un{\yi}lo otozvalisy Grif i Skvorconok.

— {\Y}eju pon{\ia}tno! — kivnul Barsuk Starxi{\y}. On znal, cto tako{\y} otvet vzbesit Superm{\yi}x, i ne smog otkazaty sebe v nebolyxom udovolystvi{\y}i.

— Otvecaty po forme! — zavizjala Superm{\yi}x. — {\Y}a spraxiva{\y}u pro vas, a ne pro {\y}eja! Zadani{\y}e vam pon{\ia}tno?

— Tak tocno, — ispravilsa Barsuk Starxi{\y}. — Zadani{\y}e mne pon{\ia}tno.

— Togda otpravl{\ia}{\y}tesy k Striju nemedlenno. I, kstati, privedite seb{\ia} v nor\-maly\-nu\-{\y}u zversku{\y}u formu. U vas sonn{\yi}{\y} vid i obvisxi{\y} jivot. Eto {\y}asno?

— Nikak net, — Barsuk Starxi{\y} zevnul.

— Cto ne {\y}asno?!

— Nemedlenno otpravl{\ia}tsa k Striju — ili snacala privesti seb{\ia} v nor\-maly\-nu\-{\y}u zversku{\y}u formu?

— Snacala v formu! — zavizjala Superm{\yi}x. — Potom nemedlenno k Striju!
\end{document}
